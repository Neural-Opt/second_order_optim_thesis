%*******************************************************
% Abstract
%*******************************************************
%\renewcommand{\abstractname}{Abstract}
\pdfbookmark[1]{Introduction}{Introduction}
% \addcontentsline{toc}{chapter}{\tocEntry{Abstract}}
\begingroup


\chapter*{Introduction}\label{ch:introduction}

Optimization techniques play a crucial role in science and engineering,
as they enable the refinement of models and solutions by iteratively minimizing or maximizing
objective functions. This process ensures that solutions are as effective and efficient as possible,
impacting a wide range of applications,
from designing engineering systems to refining algorithms in computational research.\\
In the field of numerical optimization, we differentiate between so-called \emph{first-order} and \emph{second-order} optimization methods,
where the former refers to solely utilizing \emph{first-order}, i.e., gradient information, for the task of optimization.

Second-order optimization methods utilize both the gradient and Hessian information,
providing deeper insights into the nature of the loss landscape. \\
In traditional numerical optimization problems, second-order methods are essential for efficient algorithm design as they
have provably better convergence properties than first-order methods.
Techniques such as Newton's method utilize the Hessian for rapid convergence.\\
Algorithms in this family often use a preconditioning matrix to transform the gradient before applying each step.
Classically, the preconditioner is the matrix of second-order derivatives (the Hessian) in the context of
exact deterministic optimization.\cite{anil2021scalable}\\
In machine learning, the direct application of second-order information is unfortunately limited due to the computational 
intensity and storage requirements of handling full Hessian matrices,
particularly since today's models often involve billions of trainable parameters.
While first-order methods like (stochastic) gradient descent (SGD) are preferred for the training of today's models
because of their simplicity and reduced computational demands,
they often fall short in convergence speed and sensitivity to hyperparameter settings.\\ 
Recent practice of training large models even suggests
that the utility of common first-order methods is quickly reaching a plateau, as their time-per-step is already negligible.
Consequently, the only way to accelerate training is by reducing the number of steps taken by the optimizers to reach a solution.\cite{anil2021scalable}\\
Therefore integrating Hessian-based information can potentially improve optimization efficiency by drastically increasing training convergence.
In this thesis, we will focus on two novel approaches for neural network optimization: \emph{AdaHessian} \cite{yao2021adahessian} and \emph{Apollo},
as they are optimizers that incorporate second-order information by estimating diagonal Hessian elements to adjust learning rates.
This work will provide an in-depth analysis of both AdaHessian and Apollo, evaluating their performance in comparison to other optimizers, with focus on the quality of Hessian approximation.
Additionally, we aim to make the following contributions:
\begin{enumerate}
    \item Examining whether the potential benefits of these optimizers, in terms of faster convergence and improved generalization, justify the increased computational costs.
    \item Investigating whether the claimed advantage of these optimizers in providing a better approximation of the Hessian diagonal holds up when compared to simpler approximations made by first-order methods.
    \item If the claim in 2 does not hold, investigate the reasons behind the optimizer's failure and propose a solution.
\end{enumerate}
We begin by providing a general overview of the theoretical foundations of optimization and neural network training in Chapter 1.
This includes a detailed explanation of well-known first-order optimizers, as well as Newton's Method, and the motivation behind \emph{Apollo} and \emph{AdaHessian}.
The chapter concludes with an overview of the neural network training process, along with manual methods for calculating the Hessian matrix using forward and backward propagation.
In Chapter 2, we evaluate each optimizer on multiple datasets, including tasks in both vision and translation.
Additionally, we measure the resource footprint of each optimizer to assess their real-world applicability.
In Chapter 3, we assess the quality of the Hessian approximation for both second-order and selected first-order optimizers using
a small convolutional neural network.
We conclude our work by introducing the \emph{SApollo} optimizer, which aims to address issues related to the implementation of \emph{Apollo}.


\vfill

\endgroup

\vfill
