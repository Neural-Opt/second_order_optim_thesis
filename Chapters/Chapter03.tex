%************************************************
\chapter{The Hessian in Neural Networks}\label{ch:mathtest} % $\mathbb{ZNR}$
%************************************************
In this chapter, we will closely examine the behavior of \emph{Apollo} and \emph{AdaHessian} to understand how they differ from established first-order methods and whether their resource overhead is justified.
As both \emph{Apollo} and \emph{AdaHessian} claim to provide more accurate second-order estimates
than their predecessors, we therefore compare the calculated batch Hessian
diagonal with the approximations generated by \emph{Apollo}, \emph{AdaHessian}, \emph{Adam}, and \emph{AdaBelief}
across different batch sizes.

\section{Hessian Approximization Quality}
Before comparing the quality of the Hessian approximations provided by the previously mentioned optimizers,
 we need to address some additional details.
Recalling their definitions, \emph{Adam}, \emph{AdaBelief}, and \emph{AdaHessian} all use an estimate of the
absolute curvature in their second-moment computations.
The diagonal elements of the Hessian matrix of the loss function consist of the second-order partial derivatives with respect to each parameter $H_{ii} = \frac{\partial^2 L}{\partial \theta_i^2},$
where \( L \) is the loss function and \( \theta_i \) is the \( i \)-th parameter.
The entries \( H_{ii} \) are negative when the loss function is locally concave with respect to \( \theta_i \) and positive when it is locally convex.
Consequently, we will compare the \emph{absolute} values of the Hessian diagonal elements with the optimizers' approximations,
 since they only consider the \emph{absolute} curvature—that is, they ignore the sign of the Hessian diagonal entries, 
 as they all use the squared approaximations in their second moment.
To do so, we first calculate the Hessian diagonal for the current batch using \texttt{torch.autograd}.
Since PyTorch does not natively support the calculation of the Hessian diagonal in a fully vectorized form,
we implemented this by iterating over each gradient element, performing a second backward pass, and extracting the corresponding second-order derivative.
Given that this process is inherently slow, we limited our investigation to a relatively small CNN model with 13.5K parameters, consisting of two convolutional layers and two fully connected layers.
For training, we used the MNIST dataset, which contains 60,000 images of handwritten digits, each of size 32 $\times$ 32.
To measure the similarity between the approximated Hessian diagonal and the actual batch Hessian diagonal, we needed a metric that is independent of the magnitude of the vectors.
This is crucial because the scale of the second moments can be adjusted through the learning rate.
Therefore, we utilized the cosine similarity measure, which is defined as follows:
\[
\cos(\theta) = \frac{\mathbf{A} \cdot \mathbf{B}}{\|\mathbf{A}\| \|\mathbf{B}\|} = \frac{\sum_{i=1}^{n} A_i B_i}{\sqrt{\sum_{i=1}^{n} A_i^2} \sqrt{\sum_{i=1}^{n} B_i^2}}, \quad \mathbf{A}, \mathbf{B} \in \mathbb{R}^n.
\]
This provides a measure of the angle between the two vectors, where \( \cos(\theta) = 1 \) indicates perfect directional alignment, and \( \cos(\theta) = 0 \) indicates orthogonality.
We record the measured cosine similarity and plot its development throughout the training process.
This is done for both batch sizes of 128 and 1024 to observe whether the increased stochastic variance in the gradient significantly impacts the quality of the approximation results.
In Figures \ref{fig:cosine-small-batch} and \ref{fig:cosine-big-batch}, we observe the angle between the approximations
and the \emph{absolute} batch Hessian diagonal for both small and large batch sizes.
For improved visualization, a moving average with a window size of 10 is applied to the small batch plots.
The results indicate that \emph{Apollo}'s approximations are much worse than those of the other optimizers across most layers and for both batch sizes.

Although the approximation quality of \emph{Apollo} improves significantly with larger batch sizes, even approaching that of \emph{Adam} and \emph{AdaBelief},
it still fails to match their performance.
\emph{AdaHessian} is able to produces more accurate approximations than \emph{Adam} and \emph{AdaBelief} both in the small batch setting,
as well as in the large batch setting.
Finally, we see that \emph{AdaBelief} performs similarly to \emph{Adam}, but is able to provide a noticeably better approximation of the batch Hessian when less noisy gradient estimates are available.
This brings us to the conclusion that, although we could only evaluate \emph{Apollo} on a small model,
it does not seem to live up to its claim of providing a better curvature approximation than traditional first-order methods.
\emph{AdaHessian}, on the other hand, is able to significantly outperform both traditional first-order methods, as well as \emph{Apollo},
in terms of curvature approximation.
However, due to the nature of \emph{AdaHessian}'s stochastic approximation, it will always require a warm-up period for its approximation to become accurate,
as $ \text{diag}(H) = \mathbb{E}[z \odot (Hz)]$ (see \ref{sec:adahessian}) needs several evaluations of $z \odot (Hz)$ . We can see this in Figure \ref{fig:cosine-small-batch},
were it starts of with a much higher degree until arriving at better approximates in later steps.


The good approximation performance of \emph{AdaBelief},in the \emph{big} batch setting,  may be explained by regarding the \emph{EMA} of the belief term, $(g_t - m_t)^2$ (see \ref{sec:adabelief}), 
as a approximation for the diagonal entries of the gradient variance, as $Var(g)_{ii} = E[(g_i - E[g_i])^2] \quad g \in \mathbb{R}^n, i \leq n$.
It can be shown that if we model our loss function as the negative log-likelihood, $\mathcal{L}(y,x,\theta) = - \log p(y|x,\theta)$, with $y \in \mathbb{R}^m$ beeing the lables of the input $x\in \mathbb{R}^n$, then
\begin{equation}
    \text{Var}(\nabla \log p(y|x,\theta)) = \mathbb{E}[\nabla^2 \log p(y|x,\theta)] = - \mathbb{E}[H_{\log p(y|x,\theta)}] \quad \text{\cite{jaketae_fisher}}.
\end{equation}
In this way, the belief term of \emph{AdaBelief} directly approximates the diagonal entries of the expected negative Hessian.
This might provide a new perspective on the superior approximation abilities of \emph{AdaBelief},
as we have not seen this connection in the literature yet, to the best of our knowledge.
Finally, the most interesting observation arises when we examine Figure \ref{fig:loss-big-batch},
where the evolution of the loss is depicted for both batch sizes.
We can see that, despite tuning each optimizer for optimal performance, with Hyperparameter in \ref{tab:curve-approx-params}, \emph{AdaHessian} is unable to achieve the same loss as \emph{Apollo},
even though it is capable of much more accurate curvature approximations. \emph{Apollo} on the other hand is 
able to achive basicly the same convergence behavior as first-oder methods, although providing much less accurate curvature
approaximations.
This raises the question of whether very accurate curvature estimates might actually hinder
the model's performance. It is conceivable that such precision could lead to an
earlier discovery of local minima, which are suboptimal, whereas optimizers like
\emph{Apollo} may explore more of the loss surface while still accounting for curvature.
However, as we observed in the benchmarks on \emph{CIFAR-10} and \emph{WMT-14}, both \emph{AdaHessian} and \emph{Apollo} are able to discover minima that generalize well.
This leads to the hypothesis that the better generalization performance of the tested second-order
methods may not be solely based on superior Hessian approximation capabilities,
but rather on other mechanisms that have yet to be uncovered. In conclusion, our analysis shows that \emph{Apollo} is not competitive with standard first-order methods in terms of curvature approximation ability,
at least within the context of our small testing network. However, \emph{Apollo} still manages to outperform first-order optimizers with respect to generalization performance.
Given that the increase in resource consumption for \emph{Apollo} is relatively modest, it may be a suitable option when maximal test performance is prioritized over faster convergence.
Furthermore, since \emph{AdaHessian} requires 2 to 3 times the memory of \emph{SGD} (see Table \ref{tab:optimizer_comparison_perf})
and due to PyTorch's current inability to work with gradient graphs on \emph{DDP}, training large, distributed models becomes infeasible.
Therefore, one might conclude that the approximation of \emph{AdaHessian} is unlikely to find widespread application over existing first-order methods.
\begin{figure}[h!]
    \centering
    \begin{tabular}{cc}
        % This file was created with tikzplotlib v0.10.1.
\begin{tikzpicture}[scale=0.75]

    \definecolor{crimson2143940}{RGB}{214,39,40}
    \definecolor{darkgrey176}{RGB}{176,176,176}
    \definecolor{darkorange25512714}{RGB}{255,127,14}
    \definecolor{forestgreen4416044}{RGB}{44,160,44}
    \definecolor{lightgrey204}{RGB}{204,204,204}
    \definecolor{steelblue31119180}{RGB}{31,119,180}
    
    \begin{groupplot}[group style={group size=2 by 2,
        horizontal sep=1cm,  % Adjust horizontal spacing
        vertical sep=1.5cm,}]
    \nextgroupplot[
    tick align=outside,
    tick pos=left,
    title={conv\_layer\_0},
    x grid style={darkgrey176},
    xmin=-8.8, xmax=184.8,
    xtick style={color=black},
    y grid style={darkgrey176},
    ymin=12.6035497155118, ymax=62.3421159265199,
    ytick style={color=black}
    ]
    \addplot [semithick, steelblue31119180]
    table {%
    0 31.9352386325363
    1 37.6931702388668
    2 24.5405829206317
    3 27.7001623951388
    4 27.1457281656991
    5 23.607782241006
    6 24.0268832120314
    7 21.8261134157249
    8 17.6509987235188
    9 15.8692890977915
    10 14.864393634194
    11 17.7790691374316
    12 22.6694123227228
    13 21.0049021223852
    14 21.975105906837
    15 25.5016997078295
    16 26.1263000610692
    17 27.8276302917791
    18 29.8135651674402
    19 30.529166833311
    20 33.013486847594
    21 36.7134049330333
    22 37.6446624113845
    23 38.0896646904384
    24 38.8356051173444
    25 41.5423367821689
    26 43.0755746971042
    27 46.6168696328042
    28 48.2473886575591
    29 50.5561134860378
    30 49.389121541134
    31 50.2770823268647
    32 50.2133904036319
    33 51.2615827290646
    34 52.6435722221975
    35 55.304571348266
    36 55.8487597724189
    37 54.5120254058702
    38 54.6056447311995
    39 56.0751618779923
    40 55.5102587504313
    41 55.5052202385923
    42 56.7183335397176
    43 58.1709803638609
    44 58.7026712428651
    45 58.6818942139938
    46 57.1889103850586
    47 58.6456613458843
    48 57.5051245155741
    49 57.9351823865913
    50 57.9335543098425
    51 58.7012044502051
    52 58.6109467450801
    53 58.1687213612525
    54 58.7126184523564
    55 58.581117698229
    56 59.8168129405212
    57 60.0812720078377
    58 59.6865582455762
    59 58.9478650385982
    60 58.8139650703843
    61 59.5060194026364
    62 59.7680212569076
    63 58.6508840257536
    64 58.1022368913992
    65 57.5512268763781
    66 57.2426401496363
    67 57.2865114516498
    68 58.1244224779707
    69 57.3997925876226
    70 57.3010492588158
    71 57.5094851931698
    72 56.8198919811883
    73 57.2816771667963
    74 56.6823361679825
    75 56.6219285098655
    76 57.0507259435118
    77 57.4485704082526
    78 56.85534742112
    79 57.6820342682961
    80 56.6072045333832
    81 56.8887581266554
    82 56.3572353940062
    83 56.4682713141642
    84 56.8351186486927
    85 56.9359225426494
    86 56.9006708443435
    87 56.2333917691601
    88 57.3369980458981
    89 57.3538962813566
    90 56.8456598938676
    91 57.5272212962725
    92 57.1779345847912
    93 57.0967991700644
    94 56.7391366000299
    95 56.8417356309867
    96 56.228235954606
    97 56.5516563206856
    98 56.1758467818014
    99 56.4236978324017
    100 55.5504730673827
    101 56.3970009170095
    102 55.8372619751529
    103 55.9926389294354
    104 55.0120592250314
    105 55.41748222817
    106 55.7987250918378
    107 54.9581559633748
    108 54.5434843240661
    109 55.2556226109613
    110 54.4243182117756
    111 54.6974691221041
    112 54.6975569984516
    113 53.7729519337413
    114 54.4915004425276
    115 54.0755044322376
    116 53.9194171952054
    117 54.6640440786635
    118 53.6382949057739
    119 53.6953851582389
    120 53.4585729882219
    121 54.3431854340031
    122 53.6047088081113
    123 52.9597105963706
    124 52.9340527946986
    125 52.3300142822327
    126 49.7077454637169
    127 50.0729113493165
    128 50.3037845028166
    129 49.8346656001665
    130 49.6786769864135
    131 49.6060742706298
    132 49.8831227806516
    133 49.9886780546649
    134 49.6259402280865
    135 49.518322218515
    136 49.4186257808481
    137 49.9889901719654
    138 49.3715207378631
    139 48.716615245186
    140 49.1242946523123
    141 49.4510024995358
    142 49.6103788344744
    143 48.9611004565068
    144 48.4249233310148
    145 47.1793891734845
    146 47.7253083538769
    147 47.40014544036
    148 47.1752219294641
    149 47.4427720607265
    150 46.6903731900225
    151 47.1121947660813
    152 46.5823694595425
    153 47.1486939775827
    154 46.1971965962682
    155 45.8942849907705
    156 46.360982654093
    157 47.1575580768122
    158 47.0907921702708
    159 46.531683421849
    160 46.9592203280775
    161 45.686895842653
    162 45.7169090706357
    163 46.3693109765468
    164 44.6713779400492
    165 46.7236956745986
    166 45.580140959819
    167 45.1693738300318
    168 46.2406002090921
    169 45.804318156412
    170 45.6002102717345
    171 45.1183748108562
    172 45.5418811843489
    173 45.5797680012777
    174 44.953569003976
    175 44.7024290134788
    176 46.1721125706464
    };
    \addplot [semithick, darkorange25512714]
    table {%
    0 18.226925634473
    1 43.5400150874125
    2 24.760114520267
    3 25.0123971601117
    4 24.3181318978041
    5 22.0019490199761
    6 20.8691744232079
    7 20.9970119768464
    8 22.8600215652928
    9 24.7617942085533
    10 25.5397000602191
    11 27.0671497580233
    12 27.7103138843854
    13 27.1589737231103
    14 27.0301335483464
    15 26.489918900619
    16 24.7104584027233
    17 27.1590560214711
    18 27.4822764419193
    19 26.9067759051759
    20 26.3295014451034
    21 29.2048804287666
    22 28.4296051310721
    23 28.0305549515788
    24 27.7369984652955
    25 28.9550949136718
    26 28.6869310342617
    27 27.5341155832234
    28 26.9040892269086
    29 27.46009658433
    30 27.6038539749628
    31 27.625514981969
    32 27.982503700677
    33 28.5784045400337
    34 30.0162877726778
    35 29.1907531471437
    36 28.264154313805
    37 28.0702435632492
    38 30.038317443601
    39 30.4855404471111
    40 29.6870427151821
    41 29.9371795102787
    42 28.787389569544
    43 29.6337105261315
    44 30.0080809079577
    45 31.3998498488096
    46 31.8489694364225
    47 30.927748006718
    48 32.0585659138465
    49 32.2613282864293
    50 32.3236288016611
    51 31.7021235703504
    52 31.9202957692666
    53 32.4135792817287
    54 32.4598361061133
    55 32.8708961888859
    56 33.3853866899545
    57 33.051402142863
    58 32.914381427658
    59 32.3223513941183
    60 32.9428534418
    61 32.3347786585517
    62 32.4235742047285
    63 34.6409630841105
    64 33.6855329521273
    65 33.5448113918164
    66 33.2564614490206
    67 34.3862601637097
    68 33.8180685310327
    69 34.1414793354796
    70 34.536479583948
    71 34.2047354418737
    72 33.7481268566051
    73 34.7481435342024
    74 34.7835866602355
    75 34.2701451642181
    76 34.2908511650932
    77 35.0681265029606
    78 35.324711525898
    79 34.9801728285331
    80 35.0714312014078
    81 34.9619999200517
    82 35.3342550282851
    83 34.1203710719785
    84 35.8191207951301
    85 34.3375221999383
    86 35.1260323626986
    87 36.1616885874992
    88 35.883669316459
    89 36.5054607406594
    90 36.4762710941888
    91 36.4020780334126
    92 35.3967166986286
    93 37.0163028106546
    94 35.0907665673881
    95 36.3473963415161
    96 36.48501918689
    97 36.2427183295151
    98 36.8567293000754
    99 37.0147541858747
    100 36.7540653679706
    101 37.3680158404061
    102 36.7614270485452
    103 37.663445946206
    104 37.5366642699908
    105 37.3248368736865
    106 37.8577176748551
    107 36.4915316956635
    108 37.3711048518351
    109 37.5160546088239
    110 37.3612856541629
    111 37.3728265012738
    112 37.092955616497
    113 37.1828560807915
    114 37.5733912545714
    115 37.6269728752915
    116 37.2939605569157
    117 37.4274958535361
    118 37.2805101351294
    119 39.2277385305773
    120 37.3258619535501
    121 37.9437878049167
    122 37.0804963807487
    123 37.4502195398737
    124 36.9297726393329
    125 38.5092489000076
    126 37.2671006829291
    127 36.723863514592
    128 37.1394422459867
    129 36.7011152561475
    130 37.037030886444
    131 37.1280600628328
    132 37.4676433597846
    133 37.4650383684329
    134 37.6506673019773
    135 38.2836686240716
    136 38.2771693079225
    137 37.3296804459874
    138 37.6247688769386
    139 37.0372009800057
    140 38.0799535820698
    141 37.5625024176037
    142 38.1838377060653
    143 38.4764815762928
    144 37.9794303338581
    145 37.9330895003288
    146 37.1994943700137
    147 36.6112491241461
    148 38.0287056996453
    149 38.1320224247977
    150 38.098521339025
    151 37.6706721274009
    152 38.7098716512768
    153 37.8831523930546
    154 37.7478956577038
    155 38.5356289000113
    156 37.4058004601059
    157 37.0014439342621
    158 37.1712309228139
    159 37.7395773281813
    160 38.3328875779946
    161 36.3272356057336
    162 37.4133388595208
    163 36.8546454031935
    164 35.2008440165262
    165 36.9497128555399
    166 36.4585740894993
    167 36.7493623602424
    168 38.0867803672518
    169 38.0966726578497
    170 37.2578052530237
    171 36.7357184294066
    172 36.0464135071027
    173 37.1924556535856
    174 36.5717017597986
    175 37.1518506283892
    176 35.6816940756201
    };
    \addplot [semithick, forestgreen4416044]
    table {%
    0 22.8945802079211
    1 25.218689242585
    2 26.9646151385345
    3 28.8306760445654
    4 28.8536911815348
    5 29.3147527437701
    6 28.9295202724724
    7 28.8572152419509
    8 29.2701225112971
    9 30.9685886950643
    10 31.9838726519345
    11 33.7815904751143
    12 36.8451242381652
    13 38.8521251944271
    14 40.4325197031838
    15 41.8284483146979
    16 43.3485080806559
    17 45.072530589161
    18 46.1302034645977
    19 47.171729609194
    20 47.3312718193939
    21 47.2733593235602
    22 45.6999002772708
    23 45.5025717296205
    24 41.4320045177456
    25 37.6871487712207
    26 40.0006327026874
    27 39.0942283462916
    28 37.891683012791
    29 36.2633121983936
    30 34.6774443910472
    31 33.5140587715948
    32 33.203777048235
    33 37.8531543562202
    34 36.1635868933222
    35 35.9707408426113
    36 36.016519863536
    37 35.5803661624693
    38 36.7758899194539
    39 37.2826920234658
    40 38.3149448904337
    41 38.4979377283868
    42 40.3259198291569
    43 42.2055501460516
    44 41.8221493067163
    45 42.2353871434634
    46 41.8340911590695
    47 43.9218319329662
    48 43.3579500391967
    49 43.7128207179512
    50 45.3942702193059
    51 45.5081971948653
    52 45.5658903395542
    53 46.9708494645333
    54 45.5807864593646
    55 45.1424154001802
    56 45.9956868330688
    57 46.3179023118123
    58 45.5510286928519
    59 45.3544576996583
    60 45.4409049754771
    61 46.3820395195096
    62 45.2027738088535
    63 46.1711752650772
    64 46.6716629280136
    65 45.7390012215801
    66 46.3994248652181
    67 47.4435138612554
    68 46.8437999036392
    69 46.5025630095716
    70 47.9063753367437
    71 48.0825486243629
    72 47.347711489339
    73 47.9781113305138
    74 47.9536549458222
    75 47.113798146571
    76 45.9872919778104
    77 48.3250574560503
    78 48.0160945626236
    79 48.8549121979505
    80 48.1308487562794
    81 40.0422994612868
    82 39.9254338934282
    83 37.8271679154175
    84 38.4812675580531
    85 37.6563303852256
    86 43.2650805518516
    87 47.2819081026787
    88 44.5375186840509
    89 37.6027116465938
    90 37.574024113506
    91 38.595712448998
    92 38.3117278928454
    93 38.8542156461523
    94 37.2946650959043
    95 50.5906296888646
    96 49.6424705522862
    97 46.7406783335694
    98 41.7750768448627
    99 44.4380323534863
    100 43.7795999159002
    101 43.706425482177
    102 42.1545675550133
    103 53.5021894814146
    104 54.526083217844
    105 53.1540367110287
    106 50.8151410093114
    107 45.9772431790684
    108 49.0846908299179
    109 48.5050688259365
    110 46.0343979581376
    111 49.4485574845531
    112 49.9653900773549
    113 49.1541061258515
    114 47.9090767944016
    115 38.9498940621785
    116 44.3784678437663
    117 46.0000025038461
    118 45.7078303753539
    119 42.5526370616298
    120 42.8316685901449
    121 47.9969610170792
    122 48.2525933016252
    123 48.3027916560933
    124 46.5941929169035
    125 39.2092284673236
    126 36.2732416975042
    127 40.1978462768595
    128 40.718700780943
    129 39.4680453715506
    130 38.5662886149135
    131 42.5191302771959
    132 51.0109110561165
    133 50.5374622258944
    134 48.3631735881237
    135 43.9047064495009
    136 43.5181625759568
    137 44.7667692024995
    138 43.142478469982
    139 42.6292693250824
    140 42.353322531313
    141 43.6013852971629
    142 38.5064734795447
    143 39.7018045039485
    144 38.8441653683217
    145 38.5147114870577
    146 39.1035369140353
    147 38.4736547567501
    148 37.3945775912607
    149 38.9722923090288
    150 37.813812507965
    151 37.270557761295
    152 38.4225276374238
    153 36.7400461390259
    154 34.8884913003955
    155 36.8091502518931
    156 35.2222787851219
    157 39.17474898998
    158 38.7605804753466
    159 38.7273321506266
    160 38.2284636050826
    161 37.732412606669
    162 37.5427567366735
    163 38.9688279146591
    164 38.129467152114
    165 38.0233337465105
    166 36.2358322042783
    167 36.3197059962258
    168 38.1933551976707
    169 36.5392480844163
    170 35.3965280301578
    171 35.3064923267628
    172 33.998231482831
    173 33.8429730271164
    174 32.614955178338
    175 35.2082251615428
    176 34.1916840677188
    };
    \addplot [semithick, crimson2143940]
    table {%
    0 32.0403980338568
    1 39.9492669542008
    2 23.5733922271837
    3 26.1083561241106
    4 26.6583394897346
    5 22.1362352659167
    6 20.5671288519668
    7 19.2119732838443
    8 15.9480057177207
    9 15.8061837702646
    10 16.5382639463969
    11 19.858702473256
    12 23.3640780159232
    13 22.9328673414231
    14 20.6516439509732
    15 24.9725394873345
    16 26.2532560042425
    17 30.5062338499938
    18 31.8357255054874
    19 32.0546409378989
    20 34.4112862869015
    21 40.5664172765841
    22 39.5923438619903
    23 41.1301107809292
    24 39.3090000106047
    25 42.2050062026217
    26 42.713569330343
    27 44.8983378812176
    28 47.1313067732044
    29 49.5294203844747
    30 49.8829173561844
    31 51.2189256884348
    32 50.2284191532921
    33 50.9885917913897
    34 50.2695337031393
    35 51.2561447173458
    36 51.1769707738845
    37 50.9211825484028
    38 51.1888402014994
    39 52.9602154494137
    40 50.7352643731594
    41 50.6389466604175
    42 51.9607058192121
    43 52.4228080920301
    44 52.3947678646519
    45 52.034630095744
    46 51.3659238000459
    47 51.2527424656013
    48 51.4138455031709
    49 51.3330134398326
    50 51.3866260569259
    51 51.7683791772813
    52 51.8393666839539
    53 52.1969659709967
    54 52.6422532204022
    55 52.4487658422252
    56 53.2389476135543
    57 52.4502045633588
    58 52.6665586593501
    59 50.840400130825
    60 49.754655881828
    61 51.1910402990595
    62 51.75976175511
    63 50.7928830823433
    64 50.2787251531179
    65 50.3749487381802
    66 49.93910908506
    67 50.356616824791
    68 50.7604375945218
    69 49.8512075590027
    70 50.0146413362879
    71 50.695942893533
    72 50.4598357296661
    73 49.8862085387355
    74 50.0187239488638
    75 49.2973276999113
    76 50.1281505398592
    77 50.0176587492351
    78 49.3356073686956
    79 50.7225464308983
    80 49.7963113538975
    81 49.9694265755318
    82 50.0066269396071
    83 50.2076259938952
    84 50.4598623001209
    85 49.6094282669636
    86 49.9755901372985
    87 49.9347316528484
    88 49.8454303959687
    89 50.1682132861438
    90 49.791321025226
    91 50.7026377524277
    92 50.0788963736312
    93 50.5521996566708
    94 50.065798999469
    95 50.1853949368503
    96 49.1910786459003
    97 49.6492554176844
    98 49.5944909369856
    99 48.6280027387532
    100 48.4059525351904
    101 48.7391981981926
    102 49.3976593200034
    103 48.5816315042738
    104 48.8789979186089
    105 48.1857476638029
    106 49.0141303725812
    107 45.9605990774018
    108 46.7321664911454
    109 46.5417054670827
    110 47.3056283998493
    111 46.2147396882202
    112 46.8903066737352
    113 46.2564193690159
    114 46.4699146966999
    115 46.7614628635731
    116 46.7118786511483
    117 46.6799579272138
    118 45.8763330361347
    119 46.0607636517073
    120 45.3432676577996
    121 46.0725954752602
    122 45.4660097340576
    123 45.5753305683916
    124 44.5431763339553
    125 45.2703330499082
    126 42.2972833342645
    127 43.391162890768
    128 43.2174736582615
    129 42.5051868177396
    130 43.0991414630625
    131 42.3157313335295
    132 43.1791332288849
    133 43.0907188090628
    134 42.7454685947272
    135 41.9404364366727
    136 42.6739629181227
    137 43.6310200262936
    138 42.7112483589038
    139 41.1916128088273
    140 42.3379049462834
    141 42.4691208235136
    142 42.3446434117381
    143 42.5278411657797
    144 41.5133515591145
    145 40.7842530109389
    146 40.9784115988666
    147 40.502929569878
    148 40.6822876954657
    149 40.7584348794023
    150 40.7779631435178
    151 39.7847956738925
    152 40.2207094507433
    153 38.7354650145054
    154 38.5972179055056
    155 38.1580535985234
    156 38.6647209316494
    157 38.6095990899383
    158 38.9651460330265
    159 36.8947455210408
    160 37.4965910328756
    161 37.6380135872528
    162 36.9128767509719
    163 37.921099465588
    164 36.5117692187228
    165 37.7334170766572
    166 36.60429104073
    167 37.2137773162566
    168 37.984552306235
    169 39.329809333825
    170 37.4769953529353
    171 37.5812481425372
    172 36.7016123967759
    173 37.5217966813896
    174 38.0865423054303
    175 36.7816223684233
    176 35.3897176921395
    };
    
    \nextgroupplot[
    tick align=outside,
    tick pos=left,
    title={conv\_layer\_1},
    x grid style={darkgrey176},
    xmin=-8.8, xmax=184.8,
    xtick style={color=black},
    y grid style={darkgrey176},
    ymin=16.2947477111354, ymax=73.0710106508133,
    ytick style={color=black}
    ]
    \addplot [semithick, steelblue31119180]
    table {%
    0 36.2159762789249
    1 32.1068851598069
    2 29.162381744018
    3 27.7012056148922
    4 35.3868870865941
    5 45.5021791237024
    6 37.3398614614916
    7 37.502571249985
    8 40.2270766321906
    9 39.046773912123
    10 32.5650688584905
    11 33.5087329585819
    12 33.6915913338273
    13 26.0918139632053
    14 24.7008412181261
    15 22.1897010351886
    16 21.9699307166567
    17 21.1583498369548
    18 21.5252054368096
    19 23.8957350961791
    20 25.0871508043362
    21 23.479119109218
    22 22.3909692823651
    23 23.3866127391173
    24 24.328744807172
    25 27.3643842016133
    26 26.83799875283
    27 30.4703908862131
    28 33.2274050577189
    29 31.9859228857743
    30 32.0639765292939
    31 31.8604098819887
    32 31.2926260172328
    33 34.1677330139969
    34 34.2516915034391
    35 36.8749562415498
    36 36.8949958158757
    37 35.1921045620683
    38 31.2371483288111
    39 33.4025120319981
    40 34.0700435285338
    41 34.7952224637024
    42 34.9970156303085
    43 36.2339140555678
    44 38.5424366417888
    45 36.8319437347177
    46 33.0972599290037
    47 34.7605204756489
    48 34.4241437484689
    49 33.9463320352639
    50 36.246796324377
    51 36.2592589031278
    52 35.6798087334074
    53 37.0354092942218
    54 34.6774684005986
    55 35.6317325854713
    56 35.2791422659429
    57 36.198932911236
    58 36.9613295024995
    59 33.739064651008
    60 33.5688693606764
    61 35.6110693935492
    62 34.445915701825
    63 33.751464699675
    64 32.8075862596108
    65 34.4753867598095
    66 32.4103171192059
    67 33.4646193453437
    68 33.3645652160751
    69 33.1122791204141
    70 33.1751042006576
    71 33.2344344798468
    72 32.410616587726
    73 31.8711157928518
    74 31.0096328305229
    75 32.1634932225336
    76 30.4560497633895
    77 31.6920816333806
    78 31.4250046458893
    79 33.3514978163376
    80 32.5527643625756
    81 31.7236602525729
    82 31.8903395428799
    83 31.7018571220981
    84 31.3497353336921
    85 32.1608949376719
    86 31.7374005026838
    87 30.2881363973471
    88 31.0513166892844
    89 31.6201903995834
    90 31.6446868131726
    91 31.0519522893414
    92 33.4354874731513
    93 31.2647896584848
    94 32.2285240497425
    95 32.0775348694148
    96 32.3326843306715
    97 30.117320882622
    98 30.0560983778011
    99 31.8281692072322
    100 32.8085569106389
    101 34.1476732981079
    102 32.4369347027344
    103 32.6490275247271
    104 30.8632869225297
    105 31.5130710925297
    106 30.640939033719
    107 31.1347054159779
    108 31.6651850993878
    109 32.0299681000266
    110 31.1212617605107
    111 33.1016124892953
    112 32.7299081786457
    113 30.8532796630108
    114 32.9991363782392
    115 32.5155909390607
    116 33.0889411253114
    117 32.4780616504542
    118 31.2455635203099
    119 32.7852100930056
    120 31.5190031476662
    121 33.1737935592975
    122 31.2224003673394
    123 31.9070137136304
    124 33.5834672375608
    125 30.2994222425269
    126 30.7492184931838
    127 34.8259177228508
    128 35.1514753465777
    129 32.1894080778767
    130 34.5455022480553
    131 33.449133051186
    132 29.9659220726477
    133 32.8320338542967
    134 28.5450253985275
    135 34.8817259900586
    136 33.236204091493
    137 34.488536894776
    138 31.9631380566697
    139 30.4503628185487
    140 31.2140849998522
    141 30.5862098364606
    142 30.6315499427084
    143 29.627176030356
    144 33.793086464158
    145 34.2893599722468
    146 32.6230961531476
    147 32.3541709880712
    148 29.1752395591269
    149 30.9740835412227
    150 31.8135635834744
    151 32.211556575974
    152 30.2698218048023
    153 30.719602162098
    154 29.7338987048455
    155 31.0512239966309
    156 33.6118442016913
    157 33.1452101702796
    158 35.1211650373672
    159 30.0279595909526
    160 30.499512409147
    161 32.9536911917157
    162 32.4515250320353
    163 34.1522481650717
    164 33.1302979348515
    165 29.1345533218228
    166 29.8506022014404
    167 27.9181007420586
    168 32.4978163761182
    169 33.9148846148615
    170 35.0370216901585
    171 33.9664962791659
    172 30.2390559131647
    173 28.1128957949947
    174 30.7888115084867
    175 28.5193220398618
    176 31.8609015834488
    };
    \addplot [semithick, darkorange25512714]
    table {%
    0 36.4814523888615
    1 61.7458644422132
    2 46.5657704131588
    3 25.6326632674313
    4 47.3100749468881
    5 42.0155399742267
    6 38.0077575391586
    7 36.223270204453
    8 34.5704393598752
    9 34.7501147572161
    10 33.4153323423279
    11 30.7940817714667
    12 31.0658266395278
    13 28.7941401637987
    14 29.1659768473006
    15 28.4031814792134
    16 28.4344754757358
    17 28.0050078393651
    18 29.1034077870876
    19 28.8717036306133
    20 27.1725274019246
    21 28.7561918146849
    22 27.2125610457959
    23 27.5593186776409
    24 28.0758096422383
    25 28.1848947115
    26 26.789258174653
    27 27.0602742475868
    28 25.7826888356685
    29 27.5477055936173
    30 25.7444726007493
    31 26.1599537973501
    32 27.1422773537128
    33 27.9827730046508
    34 29.1712181060439
    35 27.8628790475517
    36 28.3635612026613
    37 28.2063039656619
    38 28.2401582494968
    39 28.9567596548702
    40 28.11688165712
    41 28.6694887057522
    42 28.3765484464244
    43 28.9208636891517
    44 29.4262461328771
    45 30.8506425327595
    46 31.553001683219
    47 29.2602164417546
    48 28.862875559871
    49 30.2858340779763
    50 28.2777673379879
    51 28.1983549377292
    52 28.6629818178767
    53 29.0895305854125
    54 30.0243697504862
    55 29.7302628780511
    56 29.8520292999989
    57 29.604854919812
    58 29.3129949515351
    59 29.5900786360133
    60 29.7571226585003
    61 28.4428802222854
    62 28.3416566845504
    63 29.5808651277677
    64 29.4665790531545
    65 30.0706870344691
    66 28.499896390316
    67 29.449398904042
    68 28.1914448388252
    69 27.9737172614513
    70 28.480186459367
    71 29.5290216067917
    72 29.7580584299483
    73 28.1805994986222
    74 30.2841140065965
    75 29.2935132982901
    76 27.8506369328495
    77 28.7905594090633
    78 28.4722860924354
    79 28.7570578510199
    80 29.5033946939397
    81 29.1882532713069
    82 28.8425573778575
    83 27.1167317301012
    84 28.1944085653604
    85 27.5895594503855
    86 29.0323287018385
    87 28.335052080775
    88 28.3458575716349
    89 29.2581971287247
    90 29.5704795095147
    91 28.3677016132997
    92 27.2603179813201
    93 29.0903594543014
    94 26.5909892393953
    95 28.0277206704283
    96 28.1100039083287
    97 29.3866423037381
    98 29.0108021620355
    99 29.5680434944992
    100 28.4066489835862
    101 28.2847013234861
    102 28.095388090096
    103 26.8719808099003
    104 27.4803226717905
    105 28.679730400749
    106 28.0227273344491
    107 28.314855392983
    108 27.6349849128962
    109 29.2515024418569
    110 29.4000159403212
    111 27.2725656239474
    112 27.1009783731209
    113 28.092799144366
    114 28.8256261848138
    115 27.8113190568639
    116 29.0572159951335
    117 29.5751157383339
    118 28.4312334462614
    119 29.1757719784298
    120 29.0501625859371
    121 28.398851938338
    122 27.6009204309066
    123 28.2582183635242
    124 27.3528583309383
    125 28.5766054200854
    126 29.1279800599383
    127 27.569665509304
    128 28.8885747940029
    129 27.0688233551646
    130 27.7672808647746
    131 27.6257064720903
    132 30.250541363407
    133 29.0457947363616
    134 29.6463475357465
    135 29.0112950843715
    136 29.4859981688326
    137 30.1297124171169
    138 28.6726918193169
    139 28.0796552324027
    140 28.3113919825352
    141 28.4963677057931
    142 30.8193610544728
    143 31.1222462643802
    144 30.3511584485852
    145 29.0381829763013
    146 29.4640241353802
    147 28.7142385669483
    148 29.7788999400582
    149 29.139750642537
    150 28.5968395438657
    151 30.192646746428
    152 30.8503561654763
    153 29.0408423748803
    154 30.4709094503567
    155 30.1655137250097
    156 28.1147221089692
    157 28.0235632482494
    158 29.5780632057447
    159 30.2299338961011
    160 29.4740482541091
    161 28.5501921093358
    162 28.8009460206642
    163 30.2922598617375
    164 28.269476222526
    165 28.919098189902
    166 28.8427202017611
    167 29.7357164667734
    168 29.1376255245327
    169 29.0498742278781
    170 30.0941653887923
    171 29.5209828829673
    172 28.6272338930187
    173 28.8515185225654
    174 26.2276044960992
    175 28.7147787887554
    176 28.1833907322646
    };
    \addplot [semithick, forestgreen4416044]
    table {%
    0 55.114010185253
    1 55.6523554857136
    2 56.3168276746908
    3 55.6004074748091
    4 55.9818446151342
    5 55.892854545216
    6 55.7301283725815
    7 55.5803685680285
    8 55.2971401187863
    9 54.9291693582808
    10 54.4931911292387
    11 53.9278172614244
    12 53.6188228440916
    13 53.9770249857457
    14 54.5189875174178
    15 55.2625214797939
    16 56.2668785283301
    17 57.3587388258852
    18 58.6774967945453
    19 60.6556019454401
    20 62.3255888216499
    21 62.3036927983937
    22 62.8194527544992
    23 63.0180749443218
    24 61.8104631027628
    25 61.9207723666392
    26 63.221098384598
    27 63.3597767456157
    28 64.3660497198263
    29 65.165524349273
    30 65.5852961975999
    31 65.8469157149296
    32 66.0255169431336
    33 66.2838596045694
    34 67.1190802725266
    35 67.8794596476892
    36 68.3780028994826
    37 69.358196066742
    38 69.5286871506212
    39 69.7423324407659
    40 69.5017277512322
    41 68.9850725225889
    42 69.5229930785481
    43 69.7401864908817
    44 69.2850705374926
    45 69.1391190442221
    46 69.52481943326
    47 69.672241537378
    48 69.8123680619106
    49 69.5550112868769
    50 70.0936992702314
    51 70.0794934474484
    52 69.8469511691132
    53 70.4902714262825
    54 69.7304064809343
    55 69.5074608938614
    56 69.4201042507319
    57 69.2964470422615
    58 68.8442424066633
    59 69.0665966173787
    60 69.0039634460015
    61 68.9037362786848
    62 68.5981957980851
    63 68.8593396354899
    64 68.4792831403405
    65 68.887281745858
    66 68.750187809816
    67 68.453547011638
    68 68.9626483342957
    69 68.9928208707212
    70 68.1280887435821
    71 68.3318865215174
    72 67.2770282170846
    73 67.2403910912051
    74 67.0235604539233
    75 68.2293021354909
    76 67.3273466173873
    77 67.9346864512294
    78 67.7298359497011
    79 67.9124752242239
    80 67.1915874222354
    81 63.7723643328469
    82 63.8938752173062
    83 63.5704264711656
    84 63.5292502900248
    85 62.8433059579112
    86 63.5837907861057
    87 64.1365175817444
    88 62.6961169759997
    89 60.0107728157179
    90 60.9588218853198
    91 61.4603069642356
    92 61.1652801936218
    93 58.7675909365027
    94 57.5007919710135
    95 61.4525861538997
    96 63.594027006109
    97 59.9628579402341
    98 56.1499814630887
    99 58.4133258962579
    100 58.5844791916437
    101 57.5946643246523
    102 55.6949656422669
    103 55.8625336080866
    104 57.2185557446206
    105 55.0309442779718
    106 53.5625762942105
    107 53.2368758636226
    108 55.5302150624514
    109 55.0186617779731
    110 53.306853849443
    111 53.2477114345562
    112 54.7297133362071
    113 56.8164358284049
    114 56.8568035169498
    115 51.0050627623375
    116 53.7856556684329
    117 52.4804974112406
    118 53.1341090088126
    119 50.4936074266449
    120 50.9930702711411
    121 52.4883851418599
    122 51.6482604879698
    123 50.9273545110049
    124 50.4866932691849
    125 50.225770918753
    126 50.03417861764
    127 52.1871452023356
    128 52.9942596154863
    129 53.3846143348726
    130 51.8942546376319
    131 52.3168367102667
    132 53.3914131022766
    133 51.5502113732291
    134 49.764681836174
    135 49.5182099704103
    136 52.1270423298396
    137 53.0660193876325
    138 52.2545373106339
    139 49.8649135099429
    140 49.4694990226401
    141 50.1687780722555
    142 48.8686247314671
    143 48.9879262473206
    144 49.3619130256307
    145 48.8704791497183
    146 48.8315616434798
    147 47.8235637565664
    148 47.4296221825616
    149 48.6981159385787
    150 48.9704540571522
    151 49.1196740725813
    152 49.127505833021
    153 49.1790846939726
    154 49.0164783256778
    155 49.6046976489141
    156 49.0510771931654
    157 50.2381492074272
    158 51.7787561201772
    159 49.9957939744697
    160 50.4444851629375
    161 51.3445289752277
    162 52.1747376981948
    163 51.488952453785
    164 52.1532768702366
    165 51.728383445812
    166 48.7582717558231
    167 48.7272897311765
    168 50.1342775710678
    169 51.6263351748796
    170 52.2918472617
    171 51.7761261594874
    172 51.6632002182245
    173 51.9405139596714
    174 52.0510630019629
    175 50.588309101967
    176 49.4233065833424
    };
    \addplot [semithick, crimson2143940]
    table {%
    0 35.8812629433853
    1 35.6881574502959
    2 28.0064478874408
    3 30.5401167164002
    4 36.5510913883904
    5 43.0888641478104
    6 39.3235852897966
    7 32.1890362567177
    8 29.3116277133911
    9 25.610803012489
    10 24.4808113927055
    11 26.2598948763276
    12 24.887411312138
    13 27.706391790216
    14 27.9988541398996
    15 24.8287643100271
    16 23.5303081296304
    17 22.7687137983578
    18 22.8782733488105
    19 23.3873698527747
    20 21.7296555411409
    21 22.3381046030832
    22 20.163712569025
    23 20.7744255954921
    24 19.7940094303015
    25 19.4968272639842
    26 19.9421869107846
    27 22.9671734829303
    28 23.9497948568356
    29 25.3919281578522
    30 26.4346469343405
    31 25.3841699591263
    32 27.2768283712061
    33 27.6323342226855
    34 27.0025418088526
    35 29.6645210506823
    36 28.7225256841099
    37 27.1199982560201
    38 27.6490664589362
    39 30.0958203745245
    40 30.0798255359839
    41 30.5729316778558
    42 29.1693683480997
    43 29.1597885335884
    44 27.6042372308587
    45 29.403814100414
    46 27.136078347809
    47 26.6555306694338
    48 28.1116854395872
    49 27.6024240923895
    50 28.0877656646562
    51 26.3446889998187
    52 25.591653100501
    53 27.2036577632254
    54 26.4148940995651
    55 25.3208427130657
    56 22.2745999681494
    57 27.1581058354605
    58 28.056037120029
    59 23.1180546435512
    60 24.0370971334511
    61 24.5871791824874
    62 25.5351767118006
    63 24.8575962543684
    64 24.5473491246838
    65 26.9552444184301
    66 23.9758866779264
    67 26.7859922426007
    68 24.8413511497372
    69 24.3461637678327
    70 25.7166493191681
    71 25.6116720684471
    72 24.9776027943447
    73 22.3454714809928
    74 25.0598561958136
    75 24.4065340422723
    76 22.5171322733658
    77 23.2790653798016
    78 22.9927336144836
    79 27.6648037666654
    80 26.1171394571219
    81 25.230212717184
    82 22.3091717968841
    83 22.4082748868174
    84 20.8694620200633
    85 23.7593799210788
    86 21.1865271818378
    87 24.794217366283
    88 23.3279434994287
    89 22.6284667047038
    90 21.8994711798196
    91 21.1333195257568
    92 21.8862640666222
    93 22.6769605842857
    94 20.3156380683165
    95 21.7200879264388
    96 22.9310705592551
    97 23.0063162555321
    98 23.6785384720165
    99 19.9833389740467
    100 21.9877241337482
    101 21.5842322839353
    102 20.9716261452029
    103 20.0792215831457
    104 20.8227065550528
    105 20.9797447903636
    106 19.9815601347067
    107 21.4576235084022
    108 21.5828582899931
    109 22.5602010790687
    110 20.7751380855515
    111 21.221368456748
    112 21.4872726242512
    113 20.984885365505
    114 21.6010847853889
    115 21.2269342253388
    116 21.0289933867064
    117 21.3432797625572
    118 21.8591941984286
    119 21.0815218759561
    120 21.0935764666131
    121 21.110233251719
    122 21.8481756228674
    123 21.1739368975535
    124 22.5550109402898
    125 21.5471795524568
    126 23.5557607479044
    127 19.2430525860299
    128 19.8758160230316
    129 19.7476683597325
    130 19.6092049410449
    131 20.7240105786683
    132 19.8540171444854
    133 20.9615857849365
    134 20.7419343378124
    135 19.5362653384345
    136 20.0771424964264
    137 21.2730106077289
    138 19.586540586498
    139 19.7504780433597
    140 19.9647736031636
    141 20.2338485795298
    142 19.8906370063497
    143 19.4176380016095
    144 21.2424827002309
    145 21.1135517270272
    146 21.9272253943173
    147 21.8891955642064
    148 20.7817323243755
    149 21.6665663336976
    150 20.4609934418882
    151 19.6756681328014
    152 20.6798034813538
    153 21.287605149956
    154 20.644670976352
    155 21.8662007691399
    156 21.0376046168225
    157 21.593124096006
    158 20.770477592172
    159 21.4287018259758
    160 19.7767472003484
    161 19.5254985139656
    162 19.1675882391669
    163 20.0569775778566
    164 19.9004214811242
    165 20.7887479388034
    166 21.3261393125413
    167 21.083420673091
    168 20.1276983122394
    169 18.8754869356662
    170 20.7662399124972
    171 20.1605816190571
    172 20.3091254119345
    173 20.2290490645823
    174 19.5011857769778
    175 20.3428377813767
    176 20.7746374193007
    };
    
    \nextgroupplot[
    tick align=outside,
    tick pos=left,
    title={fc\_layer\_0},
    x grid style={darkgrey176},
    xmin=-8.8, xmax=184.8,
    xtick style={color=black},
    y grid style={darkgrey176},
    ymin=18.5187988840706, ymax=79.3954918031593,
    ytick style={color=black}
    ]
    \addplot [semithick, steelblue31119180]
    table {%
    0 45.2365536854343
    1 51.9133962354964
    2 43.5486604485505
    3 40.5175870753878
    4 34.1926077725072
    5 31.6761711896931
    6 29.156788557478
    7 28.2331275971765
    8 30.8260395036825
    9 30.4806529324966
    10 33.2784755129079
    11 30.6407447105532
    12 29.8349824007999
    13 29.8239836937909
    14 31.2760268608147
    15 31.1282848228028
    16 29.8906266670195
    17 30.3660915668974
    18 31.3718168811909
    19 32.5900331921689
    20 32.4846689432984
    21 32.9792911375027
    22 32.8982950064317
    23 33.1144233333842
    24 33.5113248239214
    25 33.9871387349249
    26 34.8515218893837
    27 34.6996410165252
    28 34.9669759454543
    29 34.8064839277381
    30 34.3206570237163
    31 33.2938826699175
    32 32.7355103115923
    33 34.0124775720224
    34 34.3514025096475
    35 34.344151221671
    36 34.1221183563138
    37 34.9341585300546
    38 34.7983941864676
    39 34.5235141715835
    40 33.7040621722081
    41 33.9664412667594
    42 34.8376129096437
    43 33.9010677799721
    44 33.7686779407148
    45 33.9562137092503
    46 35.2840438171138
    47 35.0981025435717
    48 34.860246148539
    49 34.8853446480794
    50 34.6541182433455
    51 35.4664052864452
    52 35.4586352834088
    53 34.917802373285
    54 34.2112655630283
    55 35.364754398457
    56 35.1837662493511
    57 34.3031604248679
    58 33.8269587135923
    59 35.859564316119
    60 34.7639225607828
    61 34.9807208669532
    62 34.6144352354242
    63 35.6274120003736
    64 35.1156915620527
    65 34.3959460790953
    66 36.5247448574979
    67 33.7990971502275
    68 34.95479997726
    69 35.3731499564711
    70 35.234101840689
    71 35.1178287992409
    72 33.9361970208321
    73 34.6202423522876
    74 34.0659284148796
    75 34.268489422773
    76 33.7899548856806
    77 34.2534146997815
    78 33.3014470416693
    79 34.5035033993985
    80 35.0048330990849
    81 34.4549410325309
    82 35.6614250455627
    83 34.5452011243374
    84 34.8266233579582
    85 35.5862586469358
    86 33.0001020268886
    87 35.2515607283349
    88 34.9738341006071
    89 34.0634225537897
    90 33.0605681942963
    91 34.7162008851158
    92 36.1632512235553
    93 34.7660786612434
    94 35.1998250045959
    95 34.1751691451476
    96 34.8911242668161
    97 34.9951339709195
    98 35.3521666639035
    99 35.705949905055
    100 35.4803287280659
    101 35.089804178209
    102 34.9046028151292
    103 35.0587042441904
    104 35.9245388470843
    105 35.2932126794478
    106 36.384862323806
    107 36.484864114958
    108 35.9631758889215
    109 36.8977944680169
    110 36.7760895698759
    111 35.7676021581429
    112 35.5871741362347
    113 35.8240573259255
    114 33.926071579197
    115 36.0608156471435
    116 36.8168045145157
    117 34.4246451524972
    118 36.4402602441818
    119 34.7704803367873
    120 34.977706562924
    121 35.0741234978253
    122 36.5471000190943
    123 34.8174371644031
    124 36.0898195353311
    125 34.6068714796134
    126 33.6477941915029
    127 34.0177826391531
    128 34.0984228655744
    129 34.2421640090451
    130 33.026158808957
    131 35.7675787870456
    132 33.374431085015
    133 34.3566676285198
    134 31.4828595604835
    135 31.6729518188797
    136 32.4826468190718
    137 32.9474250221422
    138 32.4899335711396
    139 34.0650626612476
    140 33.2370701738347
    141 33.2469321343467
    142 34.2511029295043
    143 31.607668652214
    144 31.6195715830912
    145 31.9393703864436
    146 32.3046354974668
    147 33.9547157221969
    148 32.3568639022573
    149 32.7289986200581
    150 33.2860803185046
    151 33.1336032539531
    152 33.8478293623504
    153 33.2749650838959
    154 33.3684895207217
    155 33.3174541119929
    156 35.1348395403755
    157 32.6539774156902
    158 32.5698397827527
    159 34.2687380917204
    160 33.5867392585631
    161 31.5150572665791
    162 32.6140427696917
    163 33.7921715784597
    164 34.5868909134174
    165 33.666994737729
    166 34.6540101365595
    167 32.927778309217
    168 32.9269992609774
    169 32.5243892283104
    170 32.1048097005073
    171 32.2241116082869
    172 33.3786026590058
    173 34.2802600377243
    174 35.9225423118575
    175 34.9646340128773
    176 34.4335123009377
    };
    \addplot [semithick, darkorange25512714]
    table {%
    0 66.4592098437449
    1 68.9441656901155
    2 50.3662534362102
    3 32.3192343307915
    4 39.8150194754088
    5 45.0606344316261
    6 44.4875343353561
    7 41.9633073989442
    8 33.8198847743153
    9 33.2974224377537
    10 30.51715754137
    11 31.1163652545689
    12 29.0136257992858
    13 28.2935359311678
    14 27.026864498609
    15 22.2539045464678
    16 21.4220079225266
    17 23.4137511421553
    18 23.0104751872043
    19 22.258881768416
    20 21.6549270186571
    21 23.1885986171435
    22 21.812955974175
    23 21.9464406860356
    24 22.6592290289725
    25 22.8302372478636
    26 23.2148985844116
    27 22.2120161458657
    28 21.8112464228877
    29 21.3450343691296
    30 21.2859212894838
    31 22.043424839105
    32 21.5344274905653
    33 22.7161785517543
    34 22.5709694551246
    35 22.8799247320135
    36 22.8065048719212
    37 23.1915559516896
    38 23.0085880290025
    39 22.4348762343848
    40 22.5010751025425
    41 22.4763615220182
    42 23.0426479396257
    43 23.1960476434263
    44 23.5368768253224
    45 23.1240990517304
    46 23.1601431639431
    47 23.2312330354961
    48 22.7044137049018
    49 23.1650225950603
    50 22.4742711147545
    51 22.3786836571157
    52 22.8813739913652
    53 22.696767882684
    54 23.0128601361183
    55 22.8126185231242
    56 23.4248868349836
    57 23.3017381489744
    58 22.9603721626974
    59 23.5335926936227
    60 23.193723859185
    61 23.6366146562312
    62 22.9454848892202
    63 23.3658260944222
    64 22.4829620682735
    65 22.977341273413
    66 23.3054760749923
    67 23.0042190455611
    68 22.6896065615411
    69 23.1856756226027
    70 22.067198532218
    71 22.5597560000419
    72 22.8380519306216
    73 23.4901912899225
    74 23.3812606623413
    75 21.8801158700197
    76 21.907133694584
    77 22.7393993974711
    78 21.2893265073081
    79 21.4737310453575
    80 22.2360878909604
    81 22.296177317129
    82 23.8445253316096
    83 23.0150701548472
    84 23.4227219526533
    85 24.1002690008272
    86 24.3129648216119
    87 22.5020566857209
    88 23.47521010452
    89 23.6059999004009
    90 22.5047424467426
    91 22.7024493861746
    92 23.3412641746505
    93 22.9751365718338
    94 23.6820927340717
    95 23.1431357022202
    96 24.2802880003475
    97 23.6119433845648
    98 23.6828409357127
    99 23.7666007431208
    100 23.4429201844556
    101 22.6330285576574
    102 23.9208970001722
    103 24.8214110910099
    104 23.897471781905
    105 23.7681429513301
    106 25.16128705239
    107 23.9414563175093
    108 23.6141856814166
    109 24.2253739719459
    110 23.6871936671425
    111 26.0261312916759
    112 22.9127269955248
    113 24.1505677291391
    114 22.8049982441257
    115 23.8496104250178
    116 23.8763030913365
    117 25.9541223799451
    118 25.2581035248029
    119 23.5331479342256
    120 23.9583493275026
    121 24.5611227683542
    122 25.0902355150619
    123 24.0297515727672
    124 24.8138930810033
    125 23.338160969403
    126 23.3150557026053
    127 23.629509774755
    128 25.5845445538552
    129 26.2543754582618
    130 25.0958563473817
    131 26.0472778778232
    132 26.5340473260625
    133 26.6529956448882
    134 26.0496342894294
    135 24.6638595812125
    136 24.9506413502726
    137 25.9246105489807
    138 25.345912125886
    139 26.462487903854
    140 25.2033517029214
    141 25.994981384735
    142 26.3031945676671
    143 24.0372648156617
    144 25.2149458593807
    145 25.3874121040092
    146 25.7780010746005
    147 24.5364220283594
    148 24.83400142133
    149 26.4224392433225
    150 25.6094756739519
    151 24.1750214126496
    152 25.200151607838
    153 24.8535907566692
    154 23.9612926792169
    155 25.2756656463133
    156 25.9440465880055
    157 27.7172093611505
    158 24.5000394725045
    159 27.2805093398334
    160 26.0257187821577
    161 23.3749433797661
    162 24.473929009647
    163 27.0969899775973
    164 23.9409850369281
    165 25.9789724045699
    166 28.0953808384532
    167 25.8006002340609
    168 26.2959568552342
    169 24.4991335827662
    170 27.1647789332641
    171 25.8751980390769
    172 24.9942984308839
    173 26.8150922676166
    174 25.34517817114
    175 25.6259838351495
    176 29.1876440282777
    };
    \addplot [semithick, forestgreen4416044]
    table {%
    0 71.7415684011828
    1 71.7384343283262
    2 72.3835540733037
    3 73.1522338902108
    4 73.8113320026431
    5 73.5786965695383
    6 73.439304278092
    7 73.0149686857073
    8 72.4176640367569
    9 72.3204423192876
    10 72.1421182175582
    11 72.0386506969831
    12 71.8480393987924
    13 71.7541562675639
    14 71.7882383622539
    15 71.9973117390994
    16 72.1245722160507
    17 72.3547111565853
    18 72.586489169002
    19 72.7748155613046
    20 72.6374846139112
    21 72.8131941430857
    22 72.4149932982996
    23 70.4494482389466
    24 67.3111792237394
    25 67.5873094030201
    26 68.1283592221123
    27 70.3750224203553
    28 73.1183931443802
    29 75.4137454703955
    30 76.4197369757882
    31 76.6283693977462
    32 76.627733163035
    33 76.3362492785862
    34 76.1382783816901
    35 75.8565820779249
    36 75.64400888967
    37 75.7504596986696
    38 75.5143824028032
    39 75.0621710616958
    40 74.7236113625476
    41 74.6837380273436
    42 74.4377367702333
    43 74.1380643070695
    44 73.8918985427602
    45 74.2929231950107
    46 74.4930216246537
    47 74.967229947733
    48 74.587523164802
    49 74.5451364735057
    50 74.5284525746755
    51 74.5100503595759
    52 74.5958071154959
    53 74.4622422993809
    54 74.2778632176013
    55 74.2085161634129
    56 74.1856022590888
    57 74.4030738132617
    58 74.2739463766182
    59 74.3490628076153
    60 73.9757919129365
    61 74.1312139289695
    62 74.2034888846293
    63 74.4626676553085
    64 74.2398038678931
    65 74.235430302673
    66 74.1670058741195
    67 74.0611186256308
    68 74.0068991094645
    69 74.1065549000542
    70 74.4535577572012
    71 74.1822604312634
    72 74.2936131955652
    73 74.3206278083638
    74 74.6480254337554
    75 74.8910856874531
    76 74.6100362491035
    77 74.1612161176911
    78 73.7323281165835
    79 74.1570840257083
    80 74.6160294092702
    81 74.3384759589376
    82 74.393071273023
    83 74.0008791409348
    84 74.4350832770006
    85 74.3096386438377
    86 73.6825324485206
    87 74.0505096064627
    88 74.0666608780404
    89 74.6485690475047
    90 74.0988281378598
    91 73.6893573968355
    92 73.560700078542
    93 73.5475964303124
    94 73.7455349857741
    95 73.7160250791486
    96 73.4782694415996
    97 73.9046273376176
    98 73.8881447964272
    99 73.8106154476006
    100 74.1809719582025
    101 73.3774069386953
    102 73.3550127921067
    103 73.3394141134017
    104 73.1817909682214
    105 73.7312235018236
    106 73.747818750292
    107 73.7801194390832
    108 74.0335986564909
    109 74.0454890396477
    110 73.4086324852417
    111 74.025149750599
    112 73.5603030652372
    113 73.828647662378
    114 73.6452169786693
    115 73.6559936621321
    116 73.5820859684544
    117 73.1575326676987
    118 73.2641404468131
    119 73.6604742812398
    120 73.7183732269581
    121 73.0442063231358
    122 73.4816766268042
    123 73.4009619911481
    124 73.4087215716449
    125 73.4098226761808
    126 73.2643365848435
    127 73.3352611431361
    128 73.3979204352243
    129 73.1803638854808
    130 72.9445686245865
    131 73.8111773123206
    132 72.7982208292513
    133 73.2342466492614
    134 72.7528019296688
    135 73.0089570614338
    136 73.5648019084026
    137 73.1350091617507
    138 73.0604700990874
    139 72.6749044504323
    140 72.7590953822579
    141 72.6001090281827
    142 73.6613479722023
    143 73.1142103598509
    144 73.1783748709868
    145 72.8230422308766
    146 72.2113560166305
    147 72.555568264365
    148 72.6883514032864
    149 72.5692208556639
    150 72.2580344908009
    151 73.0499579805944
    152 72.6156316747073
    153 72.7300346446352
    154 72.8685475472116
    155 72.9589318665737
    156 72.5316773492053
    157 72.2306363687632
    158 72.8366749815178
    159 72.7246092755468
    160 72.3699842510772
    161 72.6151270975713
    162 72.6563117635164
    163 73.2704987823432
    164 73.3684217930531
    165 71.9287804848849
    166 71.8399114857323
    167 71.9133059853254
    168 71.4659302723845
    169 71.5216391200539
    170 71.5887869184471
    171 71.6588480145315
    172 71.6901951898347
    173 71.5837603820243
    174 71.4750734719203
    175 71.532280765738
    176 70.7576909031676
    };
    \addplot [semithick, crimson2143940]
    table {%
    0 45.2420991832074
    1 51.7853110464516
    2 44.0438953059916
    3 39.3669424559327
    4 34.4521521549269
    5 31.9167173361672
    6 30.159430558862
    7 27.7588204031051
    8 32.100407788082
    9 28.4811675977514
    10 26.6287455581518
    11 27.0910738578375
    12 26.4126217213712
    13 26.4036686279607
    14 29.9490297856138
    15 29.2683343355184
    16 27.4750675786645
    17 27.4712033836801
    18 26.835063617558
    19 28.6319451526771
    20 28.2762246384593
    21 29.674628405209
    22 28.8806989341418
    23 30.2486093177455
    24 30.2849266497635
    25 31.3731354263621
    26 30.9137713319091
    27 31.0734763286848
    28 30.5516610019788
    29 31.0967656819805
    30 30.6394983355185
    31 32.1971513440011
    32 32.4651106821579
    33 33.6672288311073
    34 33.3414892503595
    35 33.0516713964411
    36 32.9384822836408
    37 34.1861839241854
    38 35.0579075608332
    39 35.1946940076431
    40 33.171758853752
    41 34.3964599341282
    42 33.6553128864285
    43 33.2040202618686
    44 32.5166074562011
    45 33.9504537837437
    46 34.9426680068654
    47 35.1567541791285
    48 36.0176988305298
    49 37.2358617536063
    50 35.4548321517681
    51 34.7858374719246
    52 34.5445627344526
    53 35.1527565649132
    54 35.1638587534398
    55 36.7500130498916
    56 36.7441450696497
    57 35.9304580387889
    58 35.8268520980895
    59 37.3366069478465
    60 35.2801888471353
    61 35.1055856441483
    62 35.5357518166644
    63 36.7869719620563
    64 37.5283004678457
    65 36.8553571279787
    66 38.5298453585491
    67 36.2055822672606
    68 37.6433260165519
    69 36.4246418023288
    70 36.2410893313802
    71 36.1590030322782
    72 35.3176823887563
    73 35.6678916057192
    74 34.7497013477251
    75 36.3940264904897
    76 35.1452584873244
    77 36.921325524003
    78 34.940980566965
    79 35.3453311835347
    80 35.1004604641944
    81 35.1969929644274
    82 35.5837233391051
    83 35.3273043205176
    84 35.6226277556756
    85 36.6693662683782
    86 33.8598326688441
    87 36.7281410281457
    88 35.6115855136439
    89 35.6884208815359
    90 34.2157965200467
    91 36.7092287820595
    92 36.1586325973508
    93 34.6167437566529
    94 35.747357828389
    95 34.855812551079
    96 35.6550863609515
    97 37.5876311832256
    98 38.6362220666531
    99 36.6853809079929
    100 35.8994092611401
    101 36.5468476785437
    102 34.6004970801344
    103 36.3244741364828
    104 37.6849533570852
    105 36.4004494319692
    106 38.5083493751237
    107 36.711616836282
    108 36.8648191000903
    109 38.4126020180121
    110 36.9266634295726
    111 36.9225762108684
    112 38.2804714521218
    113 36.3402218847866
    114 37.0278278270969
    115 36.1131309923188
    116 37.9554608092776
    117 35.4770277835255
    118 35.5325317365528
    119 35.8832439870728
    120 35.5702694988501
    121 34.2576374225725
    122 38.800002041936
    123 36.9683490979509
    124 36.5592220696275
    125 35.6325180975621
    126 34.9077897259136
    127 35.472590799536
    128 37.974313325139
    129 37.360689116133
    130 34.3365474271807
    131 37.604105237694
    132 34.7295049519792
    133 34.931826605672
    134 34.8664535524795
    135 35.5767386940519
    136 33.5212081651408
    137 31.92135501999
    138 35.0025052840139
    139 35.0900477480155
    140 35.0982688498038
    141 35.3917579648856
    142 36.2065535631977
    143 33.7641802609131
    144 34.9758478277025
    145 36.6759649593275
    146 34.2869412091601
    147 34.3828919034422
    148 34.411721382638
    149 34.7578069843043
    150 36.1835023428822
    151 35.8087848486538
    152 35.1111673693505
    153 34.1772483322529
    154 33.2467577191334
    155 34.5458395038904
    156 38.4215329748615
    157 35.1371063057833
    158 32.67461165209
    159 36.8397591111112
    160 34.1153662195902
    161 32.8195663418126
    162 35.6160661016898
    163 35.286337705048
    164 36.3734561444694
    165 36.1690151101432
    166 36.0847115341108
    167 32.8865668179275
    168 34.3461124704414
    169 32.5043692510028
    170 33.553190779673
    171 34.0763281402088
    172 35.0375808497456
    173 33.7529767758965
    174 34.4144043664391
    175 33.7431349165111
    176 34.9846164893903
    };
    
    \nextgroupplot[
    tick align=outside,
    tick pos=left,
    title={fc\_layer\_1},
    x grid style={darkgrey176},
    xmin=-8.8, xmax=184.8,
    xtick style={color=black},
    y grid style={darkgrey176},
    ymin=12.5532651227255, ymax=75.3078313848282,
    ytick style={color=black}
    ]
    \addplot [semithick, steelblue31119180]
    table {%
    0 46.9797158430389
    1 48.6689654728294
    2 41.2247789779406
    3 39.5796531978298
    4 31.637415197373
    5 35.6837315533001
    6 39.1573644919341
    7 35.2279864901638
    8 35.0462646699562
    9 34.5025267100001
    10 31.5749034386448
    11 31.5781122029033
    12 32.6838967574994
    13 34.4787893262385
    14 29.8526056107064
    15 27.1252119568636
    16 29.1949331756075
    17 37.1598401114001
    18 37.6164105841827
    19 33.0544890497955
    20 32.4994181573677
    21 34.4216245494071
    22 33.8220261129297
    23 35.8944875296913
    24 37.4590418096214
    25 37.3853767421463
    26 36.0840505212124
    27 31.8361074865109
    28 34.7993217180214
    29 35.1535039234873
    30 35.838192617776
    31 35.6394229230561
    32 34.4379390856763
    33 31.0608626990069
    34 31.6339448652816
    35 34.1752603398945
    36 34.4850145372101
    37 32.2417133315806
    38 37.5403243349801
    39 32.9815308430004
    40 30.4344693198681
    41 30.2792377330535
    42 28.8556230946938
    43 30.2303408793389
    44 27.882872387446
    45 27.9275011928014
    46 29.3109301205212
    47 29.1030707606281
    48 28.6910927015866
    49 28.916054237954
    50 28.7585769063207
    51 26.9349710544526
    52 26.5091454810146
    53 26.6584841099133
    54 27.4370256299357
    55 29.4576081810494
    56 28.3310656127249
    57 28.4179318403772
    58 26.411623660087
    59 26.8466435715957
    60 24.0148695000325
    61 24.2444845038807
    62 24.8135106143811
    63 25.7220714341054
    64 29.4484958953148
    65 28.2095841468781
    66 33.3220298635288
    67 29.8762188043851
    68 29.0351574641868
    69 25.2522202640207
    70 27.1079192866869
    71 26.5902186519039
    72 25.9289377831106
    73 28.0078587776358
    74 25.9074668443098
    75 28.3802775882417
    76 28.9046244453057
    77 29.5390879932408
    78 26.2635070028418
    79 29.8426216577345
    80 31.5334240564374
    81 28.6355868051203
    82 31.3679003117811
    83 28.50524233634
    84 26.9090624048935
    85 28.6994357144976
    86 25.345537173224
    87 31.249572859749
    88 29.3437215568636
    89 26.3800976155164
    90 26.495078895285
    91 27.698501984859
    92 28.1040598245999
    93 26.2766935214636
    94 27.6440838339816
    95 27.6069493632424
    96 29.4993793394277
    97 29.1394560801653
    98 27.1287995837268
    99 26.5832061171408
    100 26.2478974515149
    101 26.2991096112729
    102 25.9333111995614
    103 28.7520742664437
    104 28.0549769978144
    105 26.4377459802887
    106 29.2025006493925
    107 29.1844995896288
    108 26.6757115440552
    109 27.5314337802409
    110 28.3631083116246
    111 29.8348725692432
    112 28.7713794850763
    113 26.4734450809106
    114 25.0384975368696
    115 28.4297557717482
    116 30.1673214685669
    117 27.110752146505
    118 28.5016999446615
    119 26.711905784267
    120 25.9518437432865
    121 26.8059533757365
    122 29.1719607817856
    123 29.4025133156711
    124 27.2397546691934
    125 25.9952151382854
    126 25.9276959200725
    127 27.571406849472
    128 26.4264604561991
    129 27.5096083952333
    130 24.4326377888226
    131 27.5226183343023
    132 25.5074181957733
    133 27.2111644839469
    134 25.0793931623255
    135 27.5712887957143
    136 25.4833861503187
    137 25.6627085389462
    138 28.9713788178398
    139 28.3173465040939
    140 26.9833324940541
    141 27.4229698068551
    142 29.3860716090705
    143 23.4003318359743
    144 23.1904285641228
    145 25.711942468865
    146 24.8955332012556
    147 27.8579243280619
    148 25.7799563679599
    149 24.9562106524695
    150 26.665889213607
    151 25.0466058316262
    152 28.3530926927316
    153 28.4063905494376
    154 27.5637840476233
    155 24.8268123268769
    156 28.6938312753244
    157 25.6577637397681
    158 26.1842815632278
    159 26.0545020235347
    160 26.8663891693304
    161 25.0107413126417
    162 27.5199428895633
    163 28.8872740691304
    164 28.0416715124439
    165 27.1386015310588
    166 28.4762544334817
    167 24.9356275801767
    168 26.1751630492943
    169 25.7596035573152
    170 28.6581897768605
    171 26.0337810735476
    172 26.61222184335
    173 25.5796331248598
    174 29.8401577967082
    175 25.7377887613195
    176 25.9784032699494
    };
    \addplot [semithick, darkorange25512714]
    table {%
    0 52.5378711088941
    1 61.4671292637896
    2 43.1739625385373
    3 36.9560526949339
    4 49.3850456940743
    5 54.7899334364484
    6 52.1254329163273
    7 43.3515030215341
    8 37.9747462400857
    9 35.1021412150121
    10 28.6013271019792
    11 30.1633456625268
    12 29.2430936695316
    13 31.4683524453875
    14 32.4104254378914
    15 21.7663848834047
    16 23.2775790456183
    17 21.4270565819159
    18 22.0016117360508
    19 19.415254650218
    20 20.5122215556928
    21 20.6570464580396
    22 20.0420155114587
    23 21.1632314294041
    24 21.5183818590397
    25 20.5591560066964
    26 22.1764772585138
    27 22.1175848621351
    28 22.1830448574897
    29 23.1943308389781
    30 19.3327116191371
    31 18.0395400107631
    32 18.0343891967101
    33 20.6708660793645
    34 18.713165664206
    35 18.5762944177476
    36 19.9849877625252
    37 18.5670514112728
    38 18.0362643899914
    39 16.6879793548158
    40 17.5870601435823
    41 17.3120102244311
    42 17.7834975826259
    43 19.2937418276715
    44 19.9289357104766
    45 19.0076284295495
    46 18.8902390811851
    47 17.2825948961664
    48 17.0626534410856
    49 16.1732393798562
    50 16.5669253147869
    51 16.9243146094075
    52 16.1644462142231
    53 17.1069307997722
    54 16.8996143345868
    55 16.7145745670726
    56 16.9785698465918
    57 16.7478616212332
    58 17.4802086302518
    59 16.8289240239279
    60 16.3794946839094
    61 17.0634448778549
    62 17.0519539102433
    63 16.9517209839979
    64 16.5475952754279
    65 16.629391261437
    66 16.6427037386318
    67 17.6633611178857
    68 15.9124570415395
    69 16.8862752402339
    70 15.4057454073665
    71 16.0122750210769
    72 16.5685900517461
    73 18.4905457696074
    74 17.6628208604423
    75 16.7026486485928
    76 16.888520684699
    77 17.3594824506336
    78 17.0357000506744
    79 17.6191759841152
    80 17.2087913805023
    81 16.6128794267517
    82 19.9686740043492
    83 17.5968793388884
    84 17.7823905713778
    85 17.3615883898102
    86 16.4687295364038
    87 17.7106392253466
    88 17.809074456124
    89 17.5620757925119
    90 18.0864055515893
    91 16.2960413662442
    92 17.7415410665227
    93 18.4415635127023
    94 18.5353426923089
    95 17.3388907152324
    96 21.0564899631038
    97 18.6885398636196
    98 20.1620877037856
    99 19.2100117061371
    100 20.6963337438725
    101 18.28618540699
    102 19.4631041020731
    103 19.8162544719515
    104 19.3863135223494
    105 17.8811108907265
    106 19.6105481285935
    107 20.3825965636468
    108 20.1089130919233
    109 20.3668530844269
    110 20.6753062245812
    111 22.7413695278492
    112 20.663870274832
    113 17.2047392718646
    114 19.5303412261928
    115 20.9805746120899
    116 21.4506580513291
    117 22.7606559018603
    118 20.7000344059327
    119 19.8441503539129
    120 18.7784240865536
    121 21.7457278588987
    122 22.2914780977061
    123 20.0828122468901
    124 24.3191104574043
    125 20.4613255997963
    126 20.2474610219968
    127 21.6562226264071
    128 21.8479829113369
    129 22.5567915614099
    130 21.5387068833
    131 20.1644060708558
    132 20.9363883097957
    133 22.7207767726765
    134 21.093566977396
    135 22.2477717331415
    136 20.7847255450622
    137 21.1311408162292
    138 22.3947702533412
    139 22.8539375031605
    140 19.9629231572473
    141 23.0115235451883
    142 20.0681572744902
    143 23.4640283322275
    144 23.8277254387462
    145 22.2098569664991
    146 21.5353392551746
    147 20.0915028850268
    148 20.6012522750875
    149 22.1184465369403
    150 22.4900607404506
    151 20.7956651013904
    152 21.5858939626316
    153 22.4705097040521
    154 20.2143771152773
    155 18.1238465752066
    156 22.5456604926259
    157 24.8758773734155
    158 22.0598525683661
    159 24.7856570815729
    160 22.3489564689905
    161 21.6914349053808
    162 21.0812750203389
    163 19.9255188758329
    164 21.0305921899394
    165 24.3867239712259
    166 20.6069973278997
    167 21.9096598433351
    168 22.1572523459601
    169 24.3013413511463
    170 22.0076183007947
    171 21.8923465078091
    172 22.5124232952489
    173 20.3944381863897
    174 20.9285213029515
    175 21.9445765345877
    176 24.0672287307585
    };
    \addplot [semithick, forestgreen4416044]
    table {%
    0 55.3116364205542
    1 56.5301369728011
    2 56.3920394310223
    3 55.5126411644192
    4 53.2795741520133
    5 51.7017262412048
    6 51.1581281579196
    7 50.9120665882408
    8 50.1156411536951
    9 48.8959505606008
    10 47.5310956477441
    11 46.210169896527
    12 45.9396638112104
    13 46.5529629950335
    14 48.4379736467287
    15 49.743778211759
    16 52.2274919251705
    17 53.6959996249057
    18 55.7184862482858
    19 56.2604228812865
    20 59.0214205646276
    21 60.2849177160152
    22 60.5795410557552
    23 60.0033400133837
    24 67.7030634458967
    25 61.3083368135954
    26 60.4006613679506
    27 61.9364745816194
    28 69.2538884731266
    29 69.8094243871043
    30 68.8626441231946
    31 68.1855493290525
    32 67.5274587501549
    33 67.4816406403381
    34 66.8899920153618
    35 67.2055370787777
    36 68.0779603992231
    37 68.5274316057916
    38 69.2296982739913
    39 69.8501196760501
    40 71.6026995627656
    41 72.0852542663987
    42 71.2126446681865
    43 72.4553511001872
    44 71.7676708920889
    45 70.9441844577698
    46 71.3375441495361
    47 71.1124999304321
    48 70.3063585802382
    49 70.3223920824744
    50 70.986264376765
    51 70.8703996024645
    52 69.2020501972961
    53 68.9859670054229
    54 69.8301089840466
    55 68.5711654501819
    56 67.8688367710842
    57 66.7662195422655
    58 66.8853803201105
    59 67.110936273593
    60 67.0095327192478
    61 65.0778735867373
    62 65.5839085158466
    63 66.033784129019
    64 66.1887351846497
    65 65.5733860161802
    66 65.8713105161128
    67 66.1625000556303
    68 65.7144845385608
    69 65.2808507006675
    70 65.3608685920334
    71 65.4751473780512
    72 66.0322929353247
    73 66.0466248727001
    74 65.6588796302993
    75 66.1714192880348
    76 65.9409720529482
    77 65.4644768809045
    78 65.8599641538032
    79 66.3785009547082
    80 66.9629838786222
    81 65.6521793872517
    82 66.2673285989946
    83 65.7164421457096
    84 65.5719438158722
    85 65.1242646371821
    86 64.9675816646315
    87 65.9273614568405
    88 66.6227823841372
    89 66.1023865630803
    90 66.3546415274754
    91 64.3650156145381
    92 64.6739790671624
    93 64.6993018688678
    94 64.2626156160759
    95 64.6457562061038
    96 65.0758701878631
    97 65.0391913115624
    98 65.6384591817265
    99 64.1204721103985
    100 64.8878158891261
    101 64.4588895569287
    102 64.2900475091392
    103 64.4973690324629
    104 64.4579470939265
    105 65.6828645325922
    106 65.3941997253131
    107 65.1760229148951
    108 64.9871645276783
    109 65.4492475623514
    110 65.0621015885318
    111 65.4053362982347
    112 65.383208664911
    113 64.8414995189531
    114 65.9445811059754
    115 66.6489851463547
    116 65.7386799436754
    117 65.0406585256852
    118 64.5218361406519
    119 64.1852915532549
    120 64.1096707909535
    121 63.7376502929743
    122 64.4341597504895
    123 64.7649623834832
    124 66.2630700752953
    125 64.877552959894
    126 64.4765262533795
    127 64.1220245646519
    128 63.8539987930003
    129 63.6145195962218
    130 64.8870276241222
    131 64.5945989367389
    132 63.8669767189171
    133 65.0585725723923
    134 61.6512639593893
    135 64.3574356902416
    136 64.1666896941794
    137 63.7978725651019
    138 63.6260281651633
    139 63.7319264491931
    140 63.8769312543269
    141 62.9248574311365
    142 63.9129042967985
    143 64.1412786158725
    144 63.4374405090203
    145 61.8696052191777
    146 60.7466870380929
    147 60.9816961248307
    148 61.7188673026372
    149 62.2943969894428
    150 61.379970771486
    151 62.0869460213077
    152 60.8697978157091
    153 61.6785187821865
    154 60.8416018437482
    155 61.246531079292
    156 61.3462954999904
    157 60.7542550440556
    158 61.4378582853349
    159 60.1045774671555
    160 60.0248640392525
    161 61.5835690152091
    162 61.0549502310977
    163 61.4649679448966
    164 61.2398677720525
    165 59.8144622314349
    166 59.2798883225622
    167 59.8050864923732
    168 58.7920905247394
    169 58.8188630939943
    170 59.4300531523838
    171 59.5582098222782
    172 59.3722853455211
    173 59.0357990823205
    174 59.2333982668673
    175 59.9301138959976
    176 59.6022203012444
    };
    \addplot [semithick, crimson2143940]
    table {%
    0 47.4147301685843
    1 48.8990187929578
    2 40.6694243493912
    3 40.434073066418
    4 32.8474250228064
    5 35.0110300251826
    6 33.6524165155257
    7 32.880905773946
    8 34.2775799726674
    9 31.7263100161076
    10 32.0234768314725
    11 30.6820636099476
    12 29.3768277334245
    13 33.7261992733875
    14 37.0664074627266
    15 31.4486555479649
    16 26.8298129177294
    17 30.0845539439595
    18 31.782687427515
    19 34.6844425597055
    20 34.1979247258532
    21 35.4133278888295
    22 29.5889997113904
    23 31.0778695015292
    24 31.8945993070218
    25 36.5425519241483
    26 35.9203070097917
    27 32.9596550004511
    28 33.5657872819432
    29 32.139646124123
    30 32.8572584145374
    31 34.3264107733497
    32 32.2586666523293
    33 30.3692530220442
    34 30.6265225947715
    35 33.7081484718079
    36 30.8564825706898
    37 30.9715286796987
    38 34.0646724561025
    39 32.0439705542907
    40 31.5344231122153
    41 29.7259241469507
    42 30.082966527636
    43 31.2511922647796
    44 30.3650917285169
    45 29.7741000012563
    46 31.6534407112406
    47 32.4962272379536
    48 32.1053687369389
    49 32.9263081561059
    50 32.6465080144643
    51 28.7118145625354
    52 30.9518067237806
    53 33.0351556654092
    54 32.4857880582081
    55 34.7121650164769
    56 36.4617463045187
    57 31.5463377659252
    58 30.7046171263227
    59 31.5405474219929
    60 26.8387551828284
    61 27.023392141904
    62 27.9834353367219
    63 31.2701981795781
    64 35.3986151264553
    65 33.6353049516873
    66 37.1620506711933
    67 32.3714736277328
    68 33.0791313121777
    69 30.529476086797
    70 31.3657220630521
    71 33.6115357359901
    72 29.7990685070611
    73 33.4004834566493
    74 30.3785735289661
    75 33.5077246194472
    76 30.8167279738734
    77 34.3759367699999
    78 30.3801064370952
    79 31.1639265276748
    80 33.9991781282522
    81 33.3969905778061
    82 34.5838105294359
    83 32.0896995721832
    84 32.2045469188945
    85 32.0942249218765
    86 29.2137750384612
    87 34.3595480147145
    88 33.0414947636305
    89 31.0893989221191
    90 29.6281017151256
    91 32.7595323490679
    92 31.4426853985127
    93 31.1282319734715
    94 32.3681630303636
    95 32.3093639911847
    96 31.8296715297663
    97 32.4605423963833
    98 31.0137159949932
    99 29.0757597790455
    100 28.8570171124441
    101 29.9910393593111
    102 29.6969569801717
    103 32.8314732530257
    104 31.6110568685968
    105 30.4696365968547
    106 33.1859183622526
    107 30.7728222896884
    108 30.0666792017774
    109 31.4108142954966
    110 31.9233765515054
    111 30.7499665869407
    112 31.8685091514196
    113 28.2779980153932
    114 30.518004921449
    115 30.4387433729019
    116 31.3555310407729
    117 30.7081282686999
    118 27.6778567523824
    119 28.5369484637999
    120 27.528537458074
    121 27.8596783037512
    122 33.5714015430212
    123 31.6506162704557
    124 31.3634190077659
    125 32.7821006934102
    126 35.4503809778168
    127 29.5788726746812
    128 29.9448566751558
    129 30.5907264457027
    130 27.8993932812558
    131 32.7979792205125
    132 28.7011708961916
    133 31.2786185341864
    134 29.7122499361915
    135 31.381012861018
    136 26.5625171577327
    137 28.0785088618031
    138 30.7156977280785
    139 31.7727266688263
    140 31.0412779518647
    141 33.7343605571947
    142 32.8150734785086
    143 30.5908807916906
    144 29.5612119899531
    145 30.4247731623875
    146 31.1057372532333
    147 32.1614851817928
    148 30.471044141037
    149 26.767571845685
    150 30.7309251627975
    151 29.6821327300648
    152 27.9725523400444
    153 28.8061132360827
    154 30.8318837426493
    155 29.982956068098
    156 34.4132985642516
    157 29.4318482335803
    158 27.7762006362049
    159 31.8237331000747
    160 30.6049837049361
    161 27.7400728304608
    162 28.6499497392243
    163 28.8570524928044
    164 30.8030060074757
    165 31.7461250034261
    166 31.5937540034838
    167 27.8676576108901
    168 30.5754897059025
    169 29.4626910547442
    170 29.7077020802717
    171 29.1208159784091
    172 32.2149263653227
    173 30.072724845327
    174 31.6857755299853
    175 31.0410792666231
    176 30.5392026813909
    };
    \end{groupplot}
    \node at (7,7.5) [anchor=north] {
    \begin{tikzpicture}[scale=0.75] % Nested TikZ environment
 % AdaBelief
 \draw[red,  ultra thick] (0,0) -- ++(0.8,0);
 \node[anchor=west] at (1,0) {{AdaBelief}}; % Increased spacing
 
 % Adam
 \draw[steelblue31119180,  ultra thick] (3.75,0) -- ++(0.8,0);
 \node[anchor=west] at (4.75,0) {{Adam}}; % Increased spacing
 
 % AdaHessian
 \draw[darkorange25512714,  ultra thick] (6.7,0) -- ++(0.8,0);
 \node[anchor=west] at (7.7,0) {{AdaHessian}}; % Increased spacing
 
 % Apollo
 \draw[forestgreen4416044,  ultra thick] (11,0) -- ++(0.8,0);
 \node[anchor=west] at (11.9,0) {{Apollo}}; % Increased spacing
    \end{tikzpicture}
};
    \end{tikzpicture}
     \\ % Replace with the correct path to your .tex file
    \end{tabular}
    \caption{The cosine similarity (in degrees), y-axis, between the calculated batch Hessian diagonal and the corresponding optimizer approximations on a \emph{big} batch (1028 samples).
    Optimizer updates are denoted on the x-axis.
    Note that these results represent only the Hessian diagonals for the network's \emph{weights}. For the corresponding analysis on biases, please refer to Figure \ref{fig:cosine-bias-big-batch}.}
    \label{fig:cosine-big-batch}
\end{figure}


\begin{figure}[h!]
    \centering
    \begin{tabular}{cc}
        % This file was created with tikzplotlib v0.10.1.
\begin{tikzpicture}[scale=0.75]

\definecolor{crimson2143940}{RGB}{214,39,40}
\definecolor{darkgrey176}{RGB}{176,176,176}
\definecolor{darkorange25512714}{RGB}{255,127,14}
\definecolor{forestgreen4416044}{RGB}{44,160,44}
\definecolor{lightgrey204}{RGB}{204,204,204}
\definecolor{steelblue31119180}{RGB}{31,119,180}

\begin{groupplot}[group style={group size=2 by 2,
    horizontal sep=1cm,  % Adjust horizontal spacing
    vertical sep=1.5cm, }]
\nextgroupplot[
tick align=outside,
tick pos=left,
title={conv\_layer\_0},
x grid style={darkgrey176},
xmin=-93.3, xmax=1959.3,
xtick style={color=black},
y grid style={darkgrey176},
ymin=12.0853491510403, ymax=86.5325725606409,
ytick style={color=black}
]
\addplot [semithick, steelblue31119180]
table {%
0 22.4548421491706
1 20.3344079514906
2 19.0408400044025
3 15.5490111941372
4 15.4693138514767
5 16.4693020231618
6 16.9655672886026
7 16.7190710007294
8 17.3545370870222
9 18.6364830946902
10 19.7142870454391
11 21.4646948223494
12 23.1339764645527
13 24.5450390411707
14 26.1323743910934
15 27.2714514410847
16 28.3535430348293
17 29.4189599418338
18 30.2313573884806
19 31.0185748040875
20 32.003035607517
21 32.1111793141284
22 32.8234632403097
23 33.5455959915504
24 33.6452478728912
25 33.8044848888576
26 34.1609505229329
27 35.0367919824479
28 35.9850824896608
29 36.4816268393823
30 36.7546447755834
31 37.6278345185191
32 38.0002820878829
33 38.5099162263112
34 39.3273860057875
35 39.6461701707008
36 39.960394721824
37 39.9929481279723
38 40.0693522202708
39 40.1132417667412
40 40.5983503159622
41 40.8234846053887
42 40.7174330591843
43 40.8469402800987
44 40.7881652554394
45 41.0981976425387
46 40.9938347417053
47 41.2407910332792
48 41.1170683741368
49 41.1232479826659
50 40.8725141958686
51 40.6826358572132
52 40.7678541684536
53 40.3860128269538
54 40.2658092028393
55 39.9170639620994
56 39.8495131286572
57 39.1019645528452
58 38.6679805882228
59 38.3312004186767
60 37.9186190397788
61 37.6519101483925
62 37.1789100123518
63 37.3770660817122
64 37.1420145968553
65 37.2809410795673
66 36.8255108089189
67 37.4189454061238
68 37.5098582681495
69 37.2000418915913
70 37.3930041118319
71 37.2483072342182
72 37.3003457455305
73 37.2455227457876
74 37.1963421385623
75 36.9135899949812
76 36.6016773889328
77 36.0506316380595
78 35.6925783978715
79 35.5878012747546
80 35.3261915040218
81 35.1383187257592
82 34.830701484442
83 34.5026036539743
84 34.156938806631
85 34.0117152976631
86 34.3683440146724
87 34.4087727838367
88 34.7760180601058
89 34.9249213369376
90 35.0914028988786
91 35.2166122384223
92 35.3403465792178
93 35.3891435701711
94 35.6487634246155
95 35.9390179174444
96 35.7099119470652
97 35.6633113015552
98 34.9713846179456
99 34.8950304460231
100 34.6331414557607
101 34.141255947746
102 34.1206806843338
103 33.7889776630831
104 33.5024161993458
105 33.2146930418181
106 33.2345955113123
107 33.2799025395321
108 33.4099435351576
109 33.2686055431023
110 32.7914803467139
111 32.8762860553322
112 32.6828419108457
113 32.7291380224942
114 32.5166612943541
115 32.3511215750974
116 32.0802396307758
117 31.9031714337398
118 32.0975203086145
119 31.6849624398024
120 32.1808652458855
121 32.1182909353999
122 32.434078334146
123 32.104296736649
124 32.1534946724334
125 31.8883161645196
126 31.9500177085531
127 31.8991646816452
128 31.4361707546604
129 31.774542349098
130 31.3830046219256
131 31.5144947744922
132 31.4994112159766
133 31.8821272215846
134 31.9683864855055
135 32.077095233502
136 32.1314935498365
137 31.8841709671666
138 31.8856675886424
139 31.7337604020157
140 31.7107954774185
141 31.3794326859637
142 30.8285001962565
143 30.5406899469794
144 30.4577603589124
145 30.2615931490701
146 30.0779658039681
147 29.9498169827133
148 30.0524596813811
149 29.818954289909
150 29.6844841777794
151 29.7303370913743
152 29.844431839694
153 29.8769406623565
154 29.5960513454804
155 29.7573225901828
156 29.9541645873049
157 30.0628352759388
158 29.9452597303729
159 30.0607477684783
160 29.7120083529683
161 29.3980159427948
162 29.3233462048522
163 29.4911691966874
164 29.488704344302
165 29.5205004494164
166 29.3363813800988
167 29.3934606082312
168 29.6133487872695
169 29.7603451768031
170 30.2281799506814
171 30.3135539260787
172 30.2435597151681
173 30.0088493655435
174 30.3746867446667
175 30.4023420073349
176 30.2578352022351
177 30.1077325970566
178 29.7082217874389
179 29.6847892390976
180 29.6513440584284
181 29.8523592939089
182 29.8588331655275
183 29.5598095672501
184 29.2669686235629
185 29.3725691905369
186 29.5064888426432
187 29.5491394077693
188 29.9133337075789
189 30.0260214377909
190 29.8644842192529
191 30.0114874611345
192 29.8437441173275
193 30.159744658223
194 29.9559039770162
195 30.0310763890031
196 29.9286846230848
197 29.5498076565019
198 29.4060388054342
199 28.7431450573528
200 28.8230005109963
201 28.7139690880661
202 29.0117059262478
203 28.8843392542572
204 29.1667430468525
205 28.6428343927505
206 28.8714671516034
207 29.3367461145812
208 29.1182757775647
209 29.4142006047823
210 29.1771647031339
211 28.6544432677674
212 28.1671309726076
213 28.1498583177458
214 27.7492193288404
215 27.6238222798851
216 27.7867564197896
217 27.654122457848
218 27.9454247583909
219 27.8206965131604
220 27.6345715284565
221 27.7409407561881
222 27.6319366967866
223 26.9929066937185
224 26.8053280978956
225 27.0049654136913
226 26.6460992036033
227 26.6051993847118
228 26.2171644527444
229 26.6017125983113
230 26.992191745993
231 26.7130822063969
232 27.0999036461109
233 27.6050372804787
234 27.7596055198718
235 27.6472917225486
236 27.1866806023086
237 27.1389383244906
238 27.1527882711949
239 26.6637166213715
240 26.4994499293543
241 26.9940523169914
242 27.1944724027261
243 26.9227397827499
244 26.7399082811536
245 26.5727458646416
246 26.837746282259
247 26.6167819876437
248 26.9159496271646
249 27.1266208639631
250 27.334717060586
251 27.3703169352822
252 26.9591255876329
253 27.4020960260412
254 27.9261282951877
255 27.9991182008647
256 27.6966183987457
257 27.3325262489757
258 27.1719972581469
259 26.6523394048539
260 26.2782991247682
261 26.0072518845953
262 25.7499032211479
263 25.7051852542998
264 25.7092434324715
265 25.8231622836909
266 26.616699030887
267 26.597452466036
268 26.4584169662297
269 26.8105398100701
270 26.9026622306363
271 26.9438455114494
272 27.5081471053374
273 27.3315714390214
274 26.9670376028773
275 26.396929047902
276 25.9673357020743
277 26.2478004770834
278 26.7778028467011
279 26.9562299547636
280 27.2626273585741
281 27.9658921389293
282 27.1826676376474
283 26.9505412324168
284 27.0807841611461
285 27.5462988383089
286 27.1641116656433
287 27.2069883460012
288 26.8317899600959
289 27.2121720944973
290 27.0726713336937
291 26.7324802905615
292 27.7584850046859
293 27.6209916940206
294 27.2626491919302
295 26.5692970487455
296 26.3790566509138
297 26.1443221540409
298 25.96320107218
299 25.4868706120425
300 25.4949416218182
301 25.370263882846
302 24.9369688908516
303 25.3890740144018
304 25.5723102067268
305 26.5081917624364
306 26.7158193167176
307 26.8233284224313
308 26.9267876593036
309 26.6119938171901
310 27.1231125590033
311 26.6544932545356
312 26.3366337905938
313 25.9782472239136
314 25.4416649848522
315 25.1163205502404
316 25.0169010743703
317 25.3887380141155
318 25.6494854223723
319 25.7138558572008
320 25.0337835003722
321 25.3451288845067
322 24.908818904906
323 24.8800592620853
324 24.9599829316665
325 24.9234027094208
326 24.5862946022191
327 24.0022245161403
328 23.3399807338405
329 23.8376232690227
330 23.4138968686167
331 22.9779331697902
332 23.4294650305506
333 23.705709656346
334 24.0423756013836
335 23.9716001762252
336 24.4254627224422
337 24.888240423885
338 25.5003287739444
339 24.9516037881369
340 25.4459869647633
341 25.5572505667687
342 25.4375579631669
343 25.0228019012282
344 25.2815586794568
345 24.997838794523
346 25.7254025173406
347 25.1812617829954
348 25.3852578912467
349 25.7618052823648
350 25.164176066417
351 25.6976455913296
352 26.0022559264197
353 26.5198777902414
354 26.4838926871537
355 27.0566898062042
356 26.7598456168148
357 27.1527031023826
358 26.7491268972168
359 26.4816611672785
360 27.1070196323514
361 26.7342528892565
362 26.601289263076
363 25.9820042904627
364 25.428803450892
365 25.2335885919875
366 24.5790730394592
367 24.6095156589255
368 24.5728678246616
369 24.1292686788846
370 24.2724265791688
371 24.6885834368989
372 24.399127296086
373 24.623687762983
374 25.2909329053858
375 25.4504231043247
376 25.2881504945084
377 25.0564412980594
378 24.7018458512922
379 25.0245333735754
380 24.770759527738
381 24.56272949447
382 24.7668651955507
383 25.1664444687318
384 25.1076009549327
385 24.869525692812
386 25.6068468131116
387 25.4510645168848
388 25.1529655281528
389 25.4423480326972
390 25.1888085615538
391 24.9852246244164
392 25.1816549470038
393 24.8056383960244
394 24.4387256637316
395 24.9955394477155
396 24.952376009284
397 25.2311368563507
398 25.9758252192102
399 26.1895117729496
400 26.3451370195527
401 26.8537615033286
402 26.6587868998715
403 26.4060858937925
404 26.2541904899663
405 25.3881036202149
406 24.8793486651553
407 24.5615003122797
408 24.5625345536486
409 24.4829824633348
410 24.1860012507039
411 23.8772278322433
412 24.2331909852186
413 24.6263265662099
414 25.0294638852726
415 25.4853946100401
416 25.7001689718223
417 26.5743663774574
418 26.6805985223749
419 26.5389905523386
420 27.3398262371843
421 26.5394921627198
422 26.66752235783
423 26.7674840770532
424 26.3309599029489
425 26.4544878437314
426 26.5509724649388
427 26.0991271416932
428 26.2873683268468
429 26.3291543972989
430 26.0841005399945
431 26.2810778601417
432 26.0995807285535
433 26.0635883053342
434 25.9389387209849
435 25.2886591774098
436 25.9258225486482
437 25.5039325363816
438 25.0052107739283
439 24.7469465086873
440 23.9440837272101
441 24.5406203719716
442 24.7292565988071
443 24.8971433879117
444 25.6008169395362
445 26.0686117154581
446 25.8383730177796
447 26.2965895740675
448 25.9769392054708
449 26.2087867315249
450 26.7304874493737
451 25.9405741004447
452 25.4013529555361
453 24.7754133410544
454 24.6197945074363
455 24.3531154746156
456 23.7440377018251
457 23.2111908381805
458 23.6116960652069
459 23.8768477633812
460 23.4287256365291
461 24.0985851355774
462 23.8868088738379
463 24.2352124487262
464 24.0003208786518
465 24.8127336874509
466 24.8880818574578
467 25.3616440150175
468 25.2423243682316
469 24.5731590174978
470 24.9696995854982
471 24.8485600828506
472 25.1992092054152
473 24.6695533472742
474 25.0354982213069
475 24.5963310376654
476 25.3967273099226
477 25.5575871772035
478 25.7273193495015
479 25.7110861528491
480 25.2976767336514
481 25.7313142765128
482 25.5924949620862
483 26.0001981607971
484 25.6733159356053
485 25.4443678471789
486 24.6835225280555
487 24.4132933113853
488 24.2683306356772
489 24.9604947901379
490 25.0541932653905
491 24.6029546095417
492 24.241024070619
493 24.0334013933842
494 23.9561624388236
495 23.6663507100018
496 23.294108814212
497 23.6238802283508
498 23.0178950694406
499 22.7388235714062
500 22.9026718269573
501 23.080362809816
502 23.4412874934812
503 23.6418379778776
504 23.3408122752101
505 23.6676080203121
506 23.4471711647273
507 22.9186469245234
508 22.9053765130444
509 22.4907192811211
510 22.8506922652438
511 22.3780829152973
512 22.0727348810658
513 21.8627711630661
514 22.4310788575557
515 22.4037915174161
516 22.5311874176001
517 23.3037116437687
518 23.4389583577832
519 23.7109417645674
520 23.5069103425977
521 23.8967605043865
522 23.8681710106083
523 25.4606634731923
524 25.3869783669956
525 25.2140352862281
526 25.8099115932012
527 25.3888260016207
528 25.5870121513343
529 25.5315093491393
530 25.7318882895473
531 25.034060909448
532 25.7327404753
533 24.4244370067708
534 24.4081272818443
535 24.6740923402189
536 24.7499349927748
537 24.7907319417419
538 24.9602830491128
539 24.8002569523544
540 24.5330174129042
541 25.4559070879058
542 25.235610623984
543 24.8021165034306
544 24.50970087768
545 24.6926330507599
546 24.7121022822612
547 24.7907264716845
548 24.9543322961862
549 25.1641128451565
550 25.4764168184649
551 25.5387197854116
552 25.5689857731587
553 25.9423416709376
554 26.4502910347472
555 26.2959436389054
556 25.6674173879702
557 25.1866990968222
558 26.0496954819229
559 26.016734931586
560 25.1458532881172
561 24.817407274487
562 23.953268271221
563 23.9576827005891
564 24.1096950208846
565 23.8938193622524
566 23.7319937792836
567 24.021544924608
568 23.6453257353758
569 22.6695327483336
570 22.7143299134236
571 22.9579688260281
572 23.311451093948
573 23.4190043080898
574 23.243463538447
575 23.077378398668
576 23.5781978312166
577 24.0464071033913
578 23.5001369183057
579 24.1336670924618
580 24.2102332865201
581 23.7512006177959
582 23.8688866467247
583 23.5447785446736
584 23.1749240115116
585 24.4881256285702
586 24.6444853454374
587 24.1242755404661
588 23.5704112242615
589 23.5742945245327
590 23.8761879476655
591 23.9403598193481
592 23.9701013573215
593 24.398180167824
594 25.0911168864512
595 23.9910903369336
596 24.105799382916
597 24.3051485116404
598 24.218546610625
599 24.4388189545171
600 24.7791908978645
601 25.3598318544089
602 25.6968711643361
603 26.4056654369291
604 25.50462815257
605 25.4273561201416
606 25.2249387544432
607 25.663521456423
608 26.087060823257
609 26.8383600173647
610 27.2946114865471
611 26.9005373711369
612 27.0817333553652
613 26.3413561717108
614 27.0684544243602
615 27.7403123934084
616 27.6035744401596
617 27.1045101025562
618 27.3846437560594
619 27.300782600808
620 27.1840845627951
621 27.2971279855588
622 27.1690893025969
623 26.9679034375825
624 27.0134925339379
625 26.202239035464
626 26.7553688165783
627 27.3910876854069
628 27.1683799100717
629 27.1027673572026
630 26.7385991907047
631 26.2149778901675
632 25.9101511022488
633 26.0317840453436
634 25.2187937318733
635 25.2169231570475
636 25.2048993572997
637 24.6416682463101
638 24.7917741308947
639 24.5032976737157
640 24.1797430098818
641 24.6709947059962
642 24.7048382316923
643 24.6533617393901
644 24.9238764268465
645 25.2894252439405
646 24.3829703555777
647 24.1469838661684
648 24.2613696062103
649 24.3866375779411
650 24.4081321898368
651 23.7031813893049
652 23.983895407103
653 23.7057717405768
654 23.558060299808
655 23.3980881523569
656 24.3889141115064
657 24.7940063323034
658 24.7870321529251
659 24.3125553821446
660 24.8526300025711
661 25.5667785702316
662 26.5973437879961
663 26.8657107278838
664 27.3894666665603
665 27.2039623520125
666 26.8187961641083
667 26.8698998755474
668 26.8616350984707
669 26.5751274235962
670 26.4396457846512
671 26.589521881387
672 25.2048623708977
673 25.6023953646189
674 25.6872325903016
675 25.802542195319
676 26.0979247714785
677 26.382398534278
678 26.1362335744262
679 26.3905169968627
680 26.8190208165253
681 26.7966504516083
682 26.6565003982405
683 25.2382195287572
684 24.654088293032
685 25.1907733783729
686 24.3497209810607
687 23.491250194073
688 23.5176723648082
689 23.7077398455473
690 23.0975209578873
691 22.2217491337592
692 22.3338653078738
693 22.9927742307763
694 23.7615021291924
695 22.6307598027871
696 22.991724814204
697 22.9393353924959
698 23.1348809954697
699 22.8084195488156
700 22.585708419979
701 23.527668219413
702 23.6689226063729
703 24.126191890493
704 23.3432029718381
705 24.0743085999022
706 23.9956454548001
707 24.1944318838206
708 23.9677931028173
709 23.8028305269248
710 24.8852678246629
711 24.4174277573511
712 24.5881812560455
713 24.6833303431549
714 24.7280879911011
715 24.5382636161311
716 25.6232886757258
717 25.5088821257315
718 25.6032434052589
719 25.9557553291998
720 24.9917641087669
721 24.4977215363422
722 23.8846694616953
723 23.682029754901
724 24.1381508363562
725 23.7898558473141
726 23.2689218756221
727 22.9768441943536
728 23.0639287772173
729 23.2276561218765
730 23.1916866709998
731 23.984358757036
732 24.3174632979362
733 24.2857250933784
734 23.9220287271305
735 23.6690038553422
736 22.8039747991686
737 23.7493034163935
738 23.5680591515492
739 23.2947682667021
740 23.6582838204257
741 23.1346953275254
742 22.9401004075118
743 22.8734683608722
744 22.575439065177
745 23.818753291311
746 24.1469892602502
747 23.392500542277
748 23.2545748842716
749 23.1280287395367
750 23.0360759781289
751 22.7444719677331
752 23.1578593554859
753 22.787889777145
754 22.9541183458533
755 22.2689377716247
756 22.5734644008441
757 22.2691474358635
758 22.0295553490973
759 22.1431718926963
760 23.0695646863767
761 24.2131734875471
762 23.8416956877144
763 24.4988672613584
764 24.7317382172637
765 24.8428291492038
766 24.1737317936851
767 24.0791472088001
768 24.3582125385906
769 24.5925819145436
770 23.1750838097666
771 22.5625690960168
772 22.8462174254812
773 22.335829681481
774 22.06085150061
775 21.999548690192
776 22.4926330102126
777 24.1837572342121
778 23.8772068738542
779 24.2757774032974
780 24.4874116591158
781 24.0233276961306
782 24.1034066637383
783 24.557386858537
784 24.7339659149319
785 24.5105256477729
786 24.166153888208
787 22.3629225208851
788 22.2593657883731
789 21.7951419916912
790 21.6260324878344
791 22.2980839867193
792 22.183970921972
793 22.2721179896751
794 22.7966546765359
795 22.6769835332775
796 22.6440663762743
797 23.2520020908383
798 23.8679009603383
799 23.6507825432364
800 23.3429678725423
801 22.8827690598911
802 22.4170418294757
803 22.4420988977362
804 21.6800077856625
805 21.5670938066632
806 21.8253309556904
807 21.9519743481814
808 21.8926857680646
809 21.6821109602253
810 21.5686654758754
811 21.7262937657729
812 21.9547929210662
813 21.1923685964726
814 21.0637041953951
815 21.0054668984206
816 20.8218652503967
817 21.2017399869577
818 21.5804945287142
819 22.1495506046405
820 23.0627628995274
821 23.164527109146
822 23.2019528578809
823 23.9661458669328
824 25.0310691245277
825 25.5021386500446
826 25.0445785320312
827 24.2824345381816
828 24.5937657534298
829 24.7920977420921
830 24.2090808148363
831 24.0204623102837
832 24.1623421325857
833 23.2848001358949
834 22.4603342166446
835 22.4944341127299
836 23.2328590147828
837 23.4356331123033
838 22.7626937209605
839 21.5997671109272
840 21.8049310119717
841 22.391294849131
842 23.1881377235498
843 23.2184297267327
844 24.1029991026818
845 23.9034466020428
846 23.4863152677606
847 22.8548822672587
848 22.162365146603
849 22.7694699694684
850 22.7302241443499
851 21.5654122612481
852 20.2385022901787
853 20.3945907252823
854 18.8618496646343
855 18.6734318638992
856 18.8694403171364
857 18.9245416325198
858 19.2491479546623
859 19.0239455917739
860 18.8309112776103
861 19.5026478505397
862 20.5828522146185
863 21.3960037859499
864 21.8261277337956
865 21.9652342259807
866 21.8404862394793
867 22.3858664278304
868 22.24673643442
869 22.1242993581043
870 22.621580775
871 23.1636425121305
872 22.8961994951644
873 22.5472548906949
874 22.3994044902848
875 22.1605142004056
876 22.1294110828251
877 22.8632237676061
878 23.1638375968717
879 23.694077997085
880 24.2323483658051
881 23.4139356299597
882 23.1800874014868
883 23.190629669875
884 23.9060826705963
885 24.5427951607568
886 25.1204658064634
887 23.9239010946704
888 24.2384129042547
889 24.6925819743704
890 24.2751313185063
891 24.6727677478767
892 24.0932574390283
893 23.8401638349547
894 23.8339885399226
895 23.3571995551589
896 22.5003851486539
897 22.7991213762414
898 22.3452414312301
899 22.0832697780721
900 22.716254908876
901 22.5836227907462
902 23.0534136770831
903 22.6918692138177
904 22.0308742350757
905 21.6434773232149
906 21.5432434326698
907 21.2670475164209
908 20.7251317057298
909 19.8339340422789
910 18.7751899928243
911 19.4062749443907
912 19.1829402463272
913 19.6208832823826
914 20.1076063354549
915 20.7358785987116
916 21.2191361083585
917 21.5199149046786
918 22.355802859313
919 22.6290979599852
920 22.3830203671992
921 21.2770237265373
922 21.3514228016972
923 21.0687558669264
924 21.0926252579691
925 21.3361559754287
926 21.2115398135548
927 21.4020818742125
928 21.9740298979163
929 23.4887422148522
930 23.1705627004994
931 23.6461509395106
932 23.6138384776515
933 23.7081583732206
934 23.2119599554656
935 23.7903852849963
936 23.9272832465658
937 24.9751579240143
938 24.081994866975
939 22.6844739889745
940 23.8421514351186
941 24.1417194670637
942 24.989489956667
943 25.6022086323549
944 26.1046952025414
945 24.9886319621659
946 24.8546673457344
947 24.1960279914382
948 24.3258421409606
949 24.7663415554412
950 24.1931579548483
951 24.2181959523081
952 23.5117155844944
953 23.0723472814865
954 22.3607031388136
955 23.1073209799604
956 23.53117594634
957 23.0029445687455
958 23.3726703331938
959 22.8990527688747
960 22.8574589568554
961 22.4242489821606
962 22.5903521281868
963 22.6149332841356
964 23.4628164882974
965 23.1133608712948
966 22.8070875535905
967 22.6700645475249
968 21.9481867444471
969 22.3687475232602
970 22.4028928184634
971 22.8578124858508
972 23.0352490715372
973 23.2911455946257
974 23.3530079949399
975 23.3094431945976
976 23.6542414618315
977 24.0630228690893
978 24.7478647671046
979 24.8853224264189
980 24.76911812894
981 24.347502057037
982 23.7736496199387
983 23.947499930378
984 24.8244781630706
985 25.0920397119002
986 24.8636179287609
987 24.6773510192944
988 24.8551467992157
989 24.3110472754192
990 24.7854762088211
991 24.6751500242258
992 25.4113342510162
993 25.0864764476092
994 24.3856989407487
995 24.3294876305865
996 23.9970752359364
997 23.6716136798459
998 23.4331917407687
999 23.9752953997689
1000 23.5232841966078
1001 23.7123943595609
1002 23.3288001324437
1003 23.2568167556427
1004 22.6508815815444
1005 22.2654903795586
1006 23.5672102482864
1007 23.6395913098592
1008 23.6099079587812
1009 23.2284939317633
1010 23.7381748215409
1011 23.3213806668629
1012 22.8024196452746
1013 21.9265439553269
1014 21.2663281822331
1015 21.1247013743641
1016 19.7869952366601
1017 21.3717118443907
1018 21.5942892648283
1019 21.3929719324694
1020 21.0869446602493
1021 22.3948097403558
1022 23.2474383802
1023 25.4112301530678
1024 26.2743929502438
1025 27.0692873062118
1026 27.4131628821425
1027 25.9964719560377
1028 25.4372927958097
1029 26.3556776844852
1030 26.60128347398
1031 25.6574853428009
1032 25.3109058375199
1033 24.1461493968901
1034 24.0283235181361
1035 23.2366211453644
1036 23.3079432212501
1037 24.3311647792529
1038 24.0327919316774
1039 23.9263357246657
1040 24.7405212811113
1041 24.9187037513934
1042 24.6558631276125
1043 24.200110440156
1044 24.5689693918007
1045 24.5876266615199
1046 24.4832359521516
1047 23.2572586800166
1048 23.7200437750823
1049 23.3981637042107
1050 21.8106862056363
1051 21.7756218606352
1052 21.8281850524136
1053 21.7011261815874
1054 20.7930544148545
1055 21.0815139156824
1056 21.5175941610467
1057 22.0717078560294
1058 22.1349807682797
1059 21.5818608084474
1060 21.6268307539563
1061 21.5188212798716
1062 22.509583018349
1063 22.6709641696039
1064 22.7096940961636
1065 23.2376495688576
1066 22.5485438816615
1067 22.5328421119035
1068 22.3560839159356
1069 22.5879996084891
1070 23.102198574468
1071 23.1169838810063
1072 21.981861692174
1073 22.0572020664667
1074 22.6059354959656
1075 21.6902256311972
1076 21.8731637823151
1077 21.7236861597532
1078 21.9108814332984
1079 21.5812887477615
1080 20.9285109847421
1081 20.7719361977321
1082 21.9698668465803
1083 21.9024846178026
1084 22.1457042691906
1085 22.7556982870689
1086 23.5977444352898
1087 23.7919417402312
1088 24.2847067964993
1089 25.2897439036401
1090 26.1381308666306
1091 26.2202602896469
1092 24.9980481226789
1093 25.1474824066004
1094 24.7607091024137
1095 24.3720895580812
1096 23.4610807600189
1097 22.9716966883452
1098 22.733795241179
1099 21.9574363091013
1100 21.333876587991
1101 21.9051310568326
1102 22.9522503954444
1103 23.1527960086653
1104 23.7430473675797
1105 24.4420055833354
1106 24.5502947694944
1107 25.5943402464891
1108 24.8851791170139
1109 24.8565304642196
1110 24.6556910422663
1111 23.5785154861209
1112 22.4194800136305
1113 22.584180517074
1114 22.0926405595238
1115 21.7376502205043
1116 21.5337068033538
1117 21.2039060447554
1118 21.545682094112
1119 21.2596781492965
1120 22.0431813150398
1121 22.0191427393783
1122 22.0941155254615
1123 21.8229057570109
1124 21.4496638733576
1125 21.5368832883073
1126 22.1574697307599
1127 21.5361999649183
1128 22.0345129890037
1129 22.8462556730747
1130 22.5613212059274
1131 23.0550295927343
1132 24.8356962159821
1133 24.8847896394568
1134 24.8871170619724
1135 25.2424416644855
1136 25.2172773365686
1137 25.3957155694892
1138 24.3997860886356
1139 23.3736773843154
1140 24.1394287570074
1141 24.8137147827424
1142 24.3680303428265
1143 24.8883504149841
1144 25.474543943276
1145 25.3380043322913
1146 24.808564925174
1147 24.9004017209309
1148 25.2525697329259
1149 25.6691414232092
1150 24.638746351739
1151 23.615817502657
1152 22.3250969887218
1153 21.7187130740795
1154 21.433646218404
1155 20.9019041125333
1156 20.6639239107533
1157 19.967069409867
1158 20.3096000945124
1159 20.6098689403145
1160 20.9733119143925
1161 21.461035603366
1162 21.5544429596618
1163 23.2709102908154
1164 23.4505975512924
1165 23.2920829493562
1166 23.6509505470474
1167 25.5204430336964
1168 25.2097475517035
1169 25.6251303555157
1170 25.9211347931673
1171 26.9789128792755
1172 27.5035210382991
1173 25.2950903237185
1174 24.813421480429
1175 25.1021908523443
1176 24.9143937815276
1177 23.3753791523729
1178 23.4503532906941
1179 22.8564257089245
1180 22.3424918690549
1181 21.9729680368756
1182 21.7485605293782
1183 22.8889080912011
1184 22.9447747580495
1185 23.7512199585835
1186 24.4753517074142
1187 24.3179108950489
1188 24.6079050510931
1189 24.9438227004518
1190 24.8685771038331
1191 24.0716225287193
1192 25.1903763859117
1193 24.7563856951929
1194 25.065790213455
1195 24.0053078894201
1196 23.6143884492567
1197 24.1273850601076
1198 25.3598924414206
1199 25.5827032140056
1200 25.9539938414518
1201 26.8860565681118
1202 25.7664087370944
1203 25.3313498276612
1204 24.8534996433764
1205 24.9408858839798
1206 24.6016108600803
1207 24.6279176428817
1208 23.5987325340621
1209 23.2517039174266
1210 23.664334142287
1211 23.3313669796205
1212 23.2780865712479
1213 23.2873156497184
1214 25.3681382910066
1215 25.2543202690753
1216 25.2487036515744
1217 25.5129088602177
1218 25.8813502493636
1219 25.3396854772935
1220 24.6427937697753
1221 23.7416789590492
1222 23.4008297596736
1223 23.5698280064647
1224 22.0834078053894
1225 22.0606788777988
1226 22.5964631457693
1227 22.2808160214862
1228 21.2066746739353
1229 22.0628673102213
1230 21.911747518834
1231 22.2418790153612
1232 23.5757842632491
1233 23.2251528466723
1234 23.3533099187358
1235 23.585259453876
1236 23.0123958928766
1237 22.4220029657905
1238 22.3799604742262
1239 22.0101672933384
1240 22.7094590567985
1241 22.5833908012561
1242 21.7704440802158
1243 21.840343298573
1244 21.7000942364295
1245 22.0787109887615
1246 22.8150246593525
1247 23.3942026896208
1248 23.3140787852857
1249 23.1836500633395
1250 22.4098807010796
1251 23.0769952137201
1252 23.3867664854326
1253 23.7699657825358
1254 23.2472600612485
1255 22.3540401653579
1256 21.5729369400262
1257 21.876146693015
1258 21.9935333212337
1259 23.0991182174811
1260 23.269799725904
1261 22.5960945507417
1262 21.9070698474795
1263 22.6377600536676
1264 23.2868132328614
1265 24.0859343243623
1266 24.2631498153344
1267 23.8947447596333
1268 23.7434501188247
1269 22.645680501099
1270 22.3404047693257
1271 22.2974984761044
1272 23.2377536496691
1273 22.7679420700933
1274 23.1091037422071
1275 23.2999859726638
1276 23.5879149671635
1277 24.3711945025007
1278 24.5904171067241
1279 24.8625535425216
1280 25.3987012499282
1281 26.2445324637423
1282 26.5176067426819
1283 27.0397143055767
1284 26.4709306698279
1285 26.1688156955646
1286 26.3151214185623
1287 24.9309163141179
1288 25.9584091156538
1289 25.7258152750834
1290 26.5684408460963
1291 25.817618373786
1292 25.5183171069673
1293 24.3108146186378
1294 24.5035204670241
1295 24.7645520153429
1296 24.9100490322559
1297 25.1600316395257
1298 24.2423500287939
1299 24.520762923703
1300 23.5459762947438
1301 23.2782837360784
1302 22.4656530791221
1303 22.910786979683
1304 22.4814274865895
1305 22.5420185079803
1306 22.3067719626437
1307 23.5385382853642
1308 23.8886914670982
1309 23.9759094219313
1310 24.9320548169982
1311 25.7677323301757
1312 25.7365640795566
1313 25.8151657186059
1314 26.577111297239
1315 26.4473425490902
1316 26.4420358791683
1317 26.6305167379933
1318 26.3406815747583
1319 26.1011507136343
1320 25.3737802729053
1321 25.1313986843873
1322 24.846376562421
1323 24.6100376678936
1324 24.581790523411
1325 24.3534737880113
1326 24.7617183006152
1327 24.0112595855768
1328 23.9824150222654
1329 24.1234911288932
1330 23.783112650055
1331 23.8116829973491
1332 24.6040088507382
1333 24.9441628136559
1334 25.0148971318651
1335 25.5604554558298
1336 25.0178864776278
1337 25.3365147867416
1338 25.2028698079943
1339 24.8453254391498
1340 25.0428606366906
1341 24.3774293547483
1342 24.0630110674412
1343 23.8763972159879
1344 23.7349537224072
1345 23.5658614085459
1346 23.5224133270643
1347 23.0139195479173
1348 23.665296486326
1349 23.7946767780083
1350 23.9163134640875
1351 25.402542590263
1352 25.9780331628588
1353 25.6028028203995
1354 24.9089694376202
1355 25.0067043878598
1356 24.768349052211
1357 24.7079766004206
1358 24.3658616795976
1359 24.2771487500479
1360 24.3161819004543
1361 24.2131366035269
1362 23.609826247341
1363 23.2854757117893
1364 23.9709285153836
1365 23.6722735383562
1366 24.9686050058521
1367 24.5012704952391
1368 24.8632597859781
1369 24.8793580470833
1370 25.5920738033569
1371 24.6192041893109
1372 24.8004252181595
1373 26.2323778997378
1374 25.8018215555089
1375 25.3443119192117
1376 24.7442049017448
1377 25.3856747050342
1378 25.2739880502376
1379 25.416442196994
1380 24.0512536872832
1381 24.8181566802263
1382 24.922234569976
1383 24.3607489061093
1384 24.0275009194043
1385 24.7064606537432
1386 24.0098151647901
1387 24.0595867845262
1388 24.0869288499854
1389 24.0729737159815
1390 24.3470287648753
1391 23.7376167904431
1392 23.4771736694338
1393 22.8875920133805
1394 22.5731978998628
1395 22.1019256665222
1396 22.5688353429226
1397 21.8151054004591
1398 21.40018553601
1399 21.0597491186519
1400 20.8069909828393
1401 20.4034613947468
1402 20.4124735812045
1403 20.6479446206988
1404 21.3432641438777
1405 20.7793116242628
1406 20.4989902318042
1407 20.3144302535576
1408 21.1083957812798
1409 21.0312417864437
1410 22.5713104258885
1411 22.8708901968287
1412 23.0814718201448
1413 23.3098563860364
1414 22.6054944393836
1415 23.2667980314133
1416 23.3851022630513
1417 24.2098510978096
1418 23.5202861065836
1419 24.5901378007826
1420 23.5672845765648
1421 23.737698658426
1422 23.6087117287311
1423 23.6667656360976
1424 24.469491515072
1425 24.7279596164099
1426 24.486380609867
1427 25.4627828817938
1428 25.6937400548805
1429 25.0391827729424
1430 24.7268645321017
1431 24.1343693435298
1432 23.9772137178813
1433 23.9960173094631
1434 23.8935669456226
1435 23.9332907974359
1436 24.3367666081615
1437 22.8515802845911
1438 23.0721077676566
1439 23.7331694629469
1440 24.4726996426751
1441 25.0376997768571
1442 24.6579964812884
1443 24.4827813277282
1444 24.4780304267013
1445 23.453448058354
1446 22.532339615367
1447 23.2037624782179
1448 22.4963703983785
1449 21.5483576659574
1450 21.4829566957414
1451 21.3103194132537
1452 21.2159236475067
1453 21.3534323100664
1454 21.3161786189116
1455 22.4241628131636
1456 22.8479161578102
1457 23.1329049992458
1458 24.2316020657503
1459 24.9192039649457
1460 24.2325931812906
1461 23.950362281256
1462 24.6440983949679
1463 25.429671077485
1464 25.2274254315077
1465 24.8416746270122
1466 24.9633815425004
1467 24.9336452864322
1468 23.7992404934661
1469 23.5673178660489
1470 23.8908391574904
1471 24.0539484041864
1472 24.0642133475618
1473 22.5802298826058
1474 22.8627143569151
1475 22.4328075458698
1476 22.8122792612587
1477 21.8698296069825
1478 22.4400981213054
1479 23.0749317627046
1480 23.3098646509367
1481 23.1309850960301
1482 23.002968277366
1483 23.5465901018258
1484 23.5667706402524
1485 23.7561705154552
1486 24.3783536779345
1487 24.5264160378319
1488 24.7979256103423
1489 24.5777710016725
1490 24.2227149657566
1491 24.3191756512519
1492 24.2760329613767
1493 24.3009315954028
1494 23.7041508181919
1495 24.0600259873984
1496 23.9613781530927
1497 24.2638792943845
1498 24.1714518757744
1499 24.1677462573051
1500 25.3854781324302
1501 25.9907646804977
1502 25.5221010359349
1503 25.5104313824625
1504 25.9235504151022
1505 26.4508774823625
1506 25.4324267373945
1507 25.6431401512576
1508 25.5857116586194
1509 25.3283901722487
1510 24.1798579176337
1511 24.0014829565427
1512 24.3871751945303
1513 24.4537578821094
1514 24.372468683534
1515 23.1776842924068
1516 23.1066764991399
1517 22.5452668843291
1518 23.1963967906285
1519 22.8612679485066
1520 23.1674471722293
1521 22.6221817846102
1522 22.3400895959365
1523 22.7243000198091
1524 23.3541683938513
1525 23.8445689981401
1526 24.0718494539368
1527 24.6103784337812
1528 22.6061774120749
1529 22.8513447083877
1530 22.8251105899805
1531 23.0247789252766
1532 23.651126360635
1533 23.4752434440438
1534 23.3377501614403
1535 23.0266689028144
1536 23.4429553529839
1537 22.7749673916048
1538 23.21589921494
1539 22.8861089844243
1540 22.6822782279153
1541 22.8870149355415
1542 22.0884339640827
1543 22.2353460709378
1544 21.9087832835038
1545 22.9298516821764
1546 22.457187176117
1547 22.4147109766176
1548 22.8545718487726
1549 22.628385115908
1550 22.8567392580432
1551 22.4490463250473
1552 22.7737565562383
1553 22.4410654709509
1554 22.273462831933
1555 20.908194159237
1556 20.7201282330715
1557 20.9769424362444
1558 22.0265478002037
1559 22.4103063590928
1560 22.4777430779539
1561 23.4886239288817
1562 23.6081533568887
1563 23.6762014621441
1564 23.9890990184717
1565 25.2384745019516
1566 26.052581983511
1567 26.175412761136
1568 25.466050205407
1569 25.9327556302304
1570 25.4019118325504
1571 24.9318927889318
1572 24.7889625064745
1573 24.6071406954063
1574 24.9109647579855
1575 24.8693485554442
1576 26.0345525381005
1577 26.4611776983925
1578 25.6921686124825
1579 24.9725818902227
1580 24.8944804301728
1581 24.6611680578274
1582 24.8729596807442
1583 25.5139044489808
1584 25.2630564737835
1585 24.6408132529012
1586 24.0978733083902
1587 23.249976072274
1588 23.2833305476601
1589 23.8223770407669
1590 25.5232967829308
1591 25.9480577762079
1592 26.2876411164315
1593 25.3251593646076
1594 24.9454408802922
1595 25.1560652899278
1596 25.0133573610675
1597 25.6873509870685
1598 26.3532717320215
1599 25.9018055505079
1600 24.5016541894507
1601 24.7896734926451
1602 25.1538900282535
1603 24.994954503882
1604 25.2876031388608
1605 24.7680297232002
1606 23.9318036753984
1607 23.732010743494
1608 24.4284050782194
1609 25.0140264238778
1610 24.4982836226661
1611 24.3003878426861
1612 24.0216861317731
1613 24.3620075215946
1614 24.4227490051757
1615 24.7890296977816
1616 24.2606744357219
1617 24.4208533458304
1618 23.978699810374
1619 23.3149483137425
1620 24.8857242153513
1621 24.9009854666351
1622 24.3854253522436
1623 25.1115243001476
1624 24.6700521475524
1625 24.7470045779694
1626 25.4112999281498
1627 25.6788127564503
1628 25.1238628809071
1629 26.1998050207301
1630 25.0654200845906
1631 24.2886906444736
1632 24.288078631184
1633 24.3174681378058
1634 24.4655593517625
1635 24.7480413490637
1636 24.2669830347268
1637 23.7909837594022
1638 23.7576729783982
1639 22.4898707392736
1640 23.217886271952
1641 23.7482052360277
1642 23.6502807038638
1643 23.0044850705489
1644 23.4265179870491
1645 23.0062271025927
1646 23.3027017586273
1647 23.2550099515473
1648 23.5759951758533
1649 24.6383544580145
1650 23.7709946830143
1651 24.4617555532744
1652 24.0958567276961
1653 24.1424883792174
1654 24.2764854922544
1655 23.94629066575
1656 24.1399609900817
1657 24.4866660255392
1658 25.3781034469273
1659 25.0996745224752
1660 25.9605190000863
1661 24.7999053960893
1662 25.6603563684374
1663 26.1410730884241
1664 26.2753508969306
1665 27.2941545234093
1666 27.6793781253615
1667 27.5679941754939
1668 27.207488046446
1669 27.102929945704
1670 27.1038878381916
1671 27.7326002697707
1672 27.9588221742997
1673 27.6662886844666
1674 27.1178030167521
1675 26.0478487663659
1676 25.7698706930354
1677 25.1857293819978
1678 25.0512126940412
1679 24.9161789528121
1680 24.4110382679579
1681 24.4347765031321
1682 24.0899471219139
1683 24.5494239803898
1684 24.1706938855169
1685 24.8541568802172
1686 25.0912626039291
1687 25.7832966600874
1688 25.2705120361956
1689 26.7621072811032
1690 27.2162090816819
1691 28.7753194408197
1692 29.016940702288
1693 28.6123339719787
1694 29.1188354849104
1695 28.9628105948157
1696 28.1971978796174
1697 28.5982945890854
1698 28.8345860149052
1699 28.3060640335675
1700 28.4204497797222
1701 26.8835434657246
1702 26.6057461828159
1703 26.4738676837325
1704 26.5281556854784
1705 26.0903308362802
1706 26.1059579844988
1707 25.1096796509888
1708 24.0862718127288
1709 22.5832584943695
1710 21.7178237124172
1711 22.2293153283973
1712 22.0379030998719
1713 21.4788433091772
1714 20.9881195221916
1715 21.4432700206338
1716 21.7613841343992
1717 22.6406064794536
1718 23.1812391798032
1719 23.1886763005883
1720 23.6264335252373
1721 23.8477244311735
1722 23.5225315294835
1723 23.7079354826215
1724 23.9924018157058
1725 23.0645983936267
1726 23.0343194789774
1727 23.3449248523594
1728 23.1908785730551
1729 24.0469678551034
1730 24.5601259099904
1731 23.6964069043392
1732 24.8450551920554
1733 24.7656440746177
1734 24.5613897917935
1735 25.098803739156
1736 24.6264125772173
1737 23.5966037772262
1738 24.2538896225112
1739 23.5223193754708
1740 22.9465655145813
1741 22.9475847085467
1742 21.9411899092414
1743 22.7129711709627
1744 23.1076337998705
1745 23.0757873117973
1746 23.115070849721
1747 23.2109311987681
1748 23.1420252416158
1749 23.1546318502951
1750 23.34575739125
1751 23.8646041754749
1752 23.9981957035981
1753 23.5107386983772
1754 23.8888960403455
1755 23.6059973077305
1756 24.4739002526703
1757 25.3787190072356
1758 24.6646634194297
1759 25.1258610728834
1760 24.5389821038652
1761 24.5950156819803
1762 24.8601082697929
1763 25.2665243596316
1764 24.4721952432847
1765 24.8571122604884
1766 24.1606423055594
1767 23.3175536733508
1768 23.1566417380221
1769 22.8811252680208
1770 23.1503612253372
1771 22.1909471681975
1772 21.5186588370209
1773 21.3350716027118
1774 21.4378659100567
1775 21.5085147795411
1776 21.128910110122
1777 21.3226795600496
1778 21.8158497997058
1779 21.6640942594012
1780 22.3182836483617
1781 22.3082359145854
1782 22.6369694413782
1783 21.9035881407769
1784 21.5614270967368
1785 21.494482722741
1786 21.976668545381
1787 21.4972143653493
1788 21.7014872168447
1789 21.906211358467
1790 21.228752705841
1791 21.8487157120572
1792 22.2767055554517
1793 22.6215877337443
1794 25.0698500164483
1795 24.692213781468
1796 24.6795963635265
1797 25.5932488195387
1798 26.0731725271291
1799 25.5759420906411
1800 25.273913545536
1801 24.4153598516575
1802 24.2299882874259
1803 24.4339334225079
1804 23.9578706244654
1805 24.25866184064
1806 24.1697429249997
1807 24.0159112622103
1808 23.3999602663784
1809 24.5264358606654
1810 24.1535778219348
1811 24.1735156213238
1812 24.3036992191063
1813 24.2896758047868
1814 22.1860019584638
1815 22.0555431376782
1816 22.1510969745405
1817 23.403465319772
1818 24.3162973728506
1819 24.2919161279637
1820 25.1617544111605
1821 25.1518605988167
1822 24.6709878401902
1823 24.7524649480046
1824 25.9726636056784
1825 26.5146848344038
1826 26.217061130089
1827 24.2620332316641
1828 23.692293289974
1829 23.3763658309547
1830 23.1468207684196
1831 23.5603973462862
1832 23.5696691078117
1833 23.2680774841851
1834 22.445517257619
1835 22.1703403009212
1836 22.7989532289747
1837 23.3119437155375
1838 23.4224441227881
1839 23.487663933005
1840 24.084338105285
1841 24.0749166804509
1842 24.435092537813
1843 24.606147045393
1844 26.0722075361874
1845 26.0166169772675
1846 25.2869954796726
1847 25.3050413796334
1848 25.4547924948328
1849 24.9389608210253
1850 24.9449152298508
1851 24.859076556224
1852 25.8506015532829
1853 25.8595048840206
1854 24.3038143567134
1855 24.724268942214
1856 25.371485458482
1857 25.0493566494421
1858 24.1250959711308
1859 24.8672410925194
1860 23.741628218961
1861 23.3546584921049
1862 22.7339696515096
1863 23.035100026114
1864 23.4179402683496
1865 22.8253730678359
1866 23.1164958032424
};
\addplot [semithick, darkorange25512714]
table {%
0 27.8786013176605
1 30.0695590587269
2 37.2233003076325
3 44.7620995329456
4 52.0452326040839
5 59.0072727151973
6 64.4701909794769
7 68.7354883415253
8 72.9056114960367
9 76.6054783041362
10 80.2619715105668
11 83.1486078602045
12 83.0698303261524
13 82.9549156395685
14 82.7586668044181
15 82.5511419137003
16 82.3291311380316
17 82.3005213205718
18 82.0271826244223
19 81.7184745359382
20 81.3206495805577
21 80.6926862823662
22 79.8913044455399
23 78.9162860466331
24 77.9970450824691
25 77.1251549362345
26 76.2103193135855
27 75.2195517361077
28 74.249723767351
29 73.174494216347
30 72.1762932650794
31 71.3976870507206
32 70.7757790560556
33 70.2857310866243
34 69.7713499378068
35 69.2286046683268
36 68.7410087507604
37 68.3134513623233
38 67.9808095351829
39 67.5644581720763
40 67.1432406820602
41 66.7245714991991
42 66.3134158549255
43 65.9186662114362
44 65.5496366456518
45 65.2162635224696
46 64.8412098056473
47 64.4747077639826
48 64.0730636885532
49 63.8214374436036
50 63.5523954935657
51 63.2946083932948
52 63.0697999004467
53 62.8299943298965
54 62.5734578725718
55 62.2158400782559
56 61.849935054401
57 61.5328363519228
58 61.2206320652394
59 60.9246012961082
60 60.6823868110435
61 60.2893186344108
62 59.8609251541721
63 59.3532568784465
64 58.7185984518342
65 58.1436828594617
66 57.5528311831297
67 56.9201917356596
68 56.2848472793474
69 55.6755166185778
70 55.0783349751736
71 54.5782379032821
72 54.1125772022323
73 53.729074182853
74 53.3875127555502
75 53.1206004692566
76 52.9423458829856
77 52.7215946947227
78 52.5359293130706
79 52.3134727392362
80 52.084880943165
81 51.8334464355504
82 51.5704844480891
83 51.3958213904938
84 51.2637612660098
85 51.0018693989176
86 50.7406525864387
87 50.5957192972195
88 50.3634951404566
89 50.209563219093
90 50.0422816972736
91 49.9545182917395
92 49.7811136628556
93 49.6175171085497
94 49.4525481873681
95 49.3122294909182
96 49.1376245765621
97 48.8739676368356
98 48.7691360201457
99 48.5839042232287
100 48.3453064119053
101 48.0984110426242
102 47.9081511652633
103 47.6932146587742
104 47.5459497180983
105 47.4292809174687
106 47.3021284108471
107 47.1889321053107
108 46.9910113784854
109 46.8938563363976
110 46.8880617296752
111 46.7917153131671
112 46.6419315518255
113 46.5277411639344
114 46.3793059144553
115 46.3553948325826
116 46.2371681469725
117 46.1501879808753
118 46.0886571052486
119 45.8629467488501
120 45.6739928096304
121 45.5176181818962
122 45.4381611732844
123 45.3253300892345
124 45.1399506880712
125 44.863911177447
126 44.7767236653346
127 44.6083987639202
128 44.4403262464699
129 44.389317585551
130 44.2883553720869
131 44.2114873339598
132 44.1059427694439
133 43.9378043847863
134 43.8248715971577
135 43.807452778819
136 43.6505184720129
137 43.613566333479
138 43.4443011861174
139 43.2860082751612
140 43.050883279752
141 42.8133187613472
142 42.6190065918218
143 42.3436335445841
144 42.1583183032388
145 41.9661280889734
146 41.8512159492529
147 41.6841831873561
148 41.6370219915089
149 41.52961550127
150 41.5230702966569
151 41.364227456044
152 41.3441622925951
153 41.3132213612009
154 41.1750280571494
155 41.1835258498728
156 40.9600504832717
157 40.6938486387056
158 40.553451425345
159 40.3667802724146
160 40.1963456557597
161 40.3388070862561
162 40.224084118075
163 40.4066407698544
164 40.5188124670638
165 40.3908276643457
166 40.3484343472976
167 40.2518064177742
168 40.1243253245161
169 40.1159827356006
170 39.9889585480758
171 39.8058567408852
172 39.6388134251448
173 39.3156548097749
174 38.9409437855295
175 38.922375098685
176 38.9057259918113
177 38.830929738475
178 38.9914386062934
179 39.0892410688674
180 39.159227689071
181 39.1707683862496
182 39.3243901086022
183 39.331736359049
184 39.4682149060937
185 39.0840983764726
186 38.9428512580395
187 38.8493331599214
188 38.5526877671398
189 38.2844008553776
190 38.0743212188832
191 37.7401414183268
192 37.2921120060718
193 37.0349478941648
194 36.8426721551317
195 36.738014988249
196 36.6810796946352
197 36.5649744600305
198 36.3303731138272
199 36.0407232425267
200 35.9467384292897
201 35.8400108221871
202 35.8113950720136
203 35.6691629642521
204 35.5571481429636
205 35.4897178669795
206 35.3004403658257
207 35.1359969051771
208 35.2562125668716
209 35.3066837502762
210 35.411624011613
211 35.6302152614479
212 35.8114815917085
213 35.9626465454812
214 35.9445712767123
215 35.8644307961181
216 35.6792051942474
217 35.8281303165183
218 35.5300695152195
219 35.3119806898448
220 34.7110054914802
221 34.2229854349228
222 33.7807598953846
223 33.5450388078794
224 33.4142192796601
225 33.3114036301188
226 33.3508430236276
227 33.1337540779664
228 33.0056764947875
229 33.0954251787962
230 33.4125627483339
231 33.6961591589309
232 33.9723566162366
233 33.7935555212794
234 33.5149285237598
235 33.4755392321693
236 33.2609139026572
237 33.1301726791654
238 32.9405454887071
239 32.713898272931
240 32.6125058669164
241 32.1440678361718
242 31.9554266280429
243 32.0858800462356
244 32.2195836923765
245 31.9600120849709
246 32.1063949980832
247 32.1906576524703
248 32.6744782567205
249 32.8702270747133
250 32.5590074991053
251 32.6601798617239
252 32.5406593758752
253 32.5389438301039
254 32.556442420503
255 32.8331108783761
256 32.9055969278644
257 33.0326111398485
258 32.4766278443855
259 32.1442972959716
260 32.0427621583755
261 32.1913311091099
262 32.0744229051233
263 31.9873224491765
264 31.8921184375056
265 31.6198634979339
266 31.0770324940491
267 30.9496966607786
268 31.1389244551227
269 31.4142678441017
270 31.7996933697082
271 31.4308422593705
272 31.7895171271999
273 31.8309412771445
274 31.89547404994
275 32.0555009722547
276 32.7610069628373
277 32.7011932124973
278 32.7186098607754
279 32.5833764123269
280 32.0921957351558
281 32.3451954511648
282 32.1989364543564
283 31.942659692379
284 31.6365871956029
285 31.2377816027972
286 30.6546194124357
287 30.2063401340277
288 29.9887458371406
289 29.7539613653997
290 29.75686465128
291 29.5299865621342
292 28.742359335719
293 28.4502101815242
294 28.2452004873251
295 28.5701382784115
296 28.7549565082456
297 29.1269156516696
298 29.2094425331545
299 28.8382740532623
300 28.8610730588362
301 28.7023770976521
302 29.088135411216
303 29.2919745738583
304 29.4007484289811
305 29.3057488279936
306 29.0828166257554
307 29.110434027278
308 29.4969788686857
309 30.0546956999951
310 30.1640722476951
311 30.2485576008743
312 30.6106058177241
313 30.7414456563871
314 30.7713801925657
315 30.7152875943146
316 30.9376154953308
317 30.5690937697975
318 30.0903628918751
319 29.9414636321399
320 29.966715754278
321 29.8069201168769
322 29.444286551315
323 29.2609930892156
324 29.6838070727977
325 29.8798183305199
326 30.2400496488857
327 30.8132723654014
328 30.791883999039
329 31.0828581147384
330 30.8596093649299
331 30.9651064985657
332 31.1275431720555
333 31.4522616371783
334 30.765871481538
335 30.375449100013
336 29.765406898517
337 29.4528721718804
338 29.2153108916165
339 28.8157408875794
340 29.2423175830281
341 29.2309869341435
342 29.0693049225539
343 28.4969088596425
344 28.9902690134421
345 29.3540670484701
346 29.4769141612079
347 29.4824468520011
348 29.9898958685165
349 30.3088772076603
350 30.2183101211506
351 30.6179488118504
352 30.7378484619759
353 30.8872891861616
354 30.6868897918358
355 30.0052383341845
356 29.8675969571468
357 29.8371637976668
358 29.5088021839711
359 29.3190811839605
360 29.3480367808645
361 29.1078308544258
362 28.5150781736238
363 28.6449432174686
364 28.6772379262278
365 29.3829519659494
366 29.2519944757342
367 28.883324848872
368 28.4633484959235
369 28.2546685026024
370 28.2939571287304
371 28.0619490785847
372 28.212340695476
373 27.9821913589463
374 27.7381236143433
375 27.4716744012049
376 27.5876421335913
377 27.8113174416328
378 28.1383376719021
379 28.7313904331046
380 28.6835888791602
381 28.5645019747899
382 28.7106486069091
383 28.6659620463515
384 28.4997467979961
385 28.4654526775088
386 28.5348327492303
387 28.2418249681202
388 28.043943778878
389 27.1129187978562
390 26.5443706058346
391 26.3796295195818
392 26.1644182053671
393 26.3673325418801
394 27.1127805046102
395 26.7427719307391
396 26.5373347491196
397 26.7522176861564
398 26.915142913656
399 27.132779284433
400 27.4337375896226
401 27.6904678965953
402 27.9024799534783
403 28.0842625198203
404 27.3657519889164
405 27.393875824533
406 27.2950836459492
407 26.9526254978619
408 26.8043565927242
409 26.4069664335757
410 25.8801882769788
411 25.4511243064759
412 25.5248250092161
413 25.4838325940126
414 25.5296727865921
415 26.1670003850124
416 26.2370933142852
417 26.1865822445771
418 26.2994182179416
419 27.1413150366144
420 27.3225920323906
421 27.7381344672176
422 27.5592304966633
423 27.3878457668636
424 27.9279596401613
425 27.4065952873066
426 27.569929645254
427 28.0644457759331
428 28.0627608213823
429 27.4998460449495
430 27.4715719859275
431 27.5189164148889
432 27.7780823601248
433 27.6264896578454
434 27.2271082372564
435 27.0998354449298
436 26.6489881485153
437 26.3256994246618
438 26.2778101696664
439 26.3378015889917
440 26.2899946064286
441 26.2435995637837
442 25.668787056695
443 25.379868639427
444 25.2586348162232
445 25.4491130632758
446 25.522659946477
447 25.91261402547
448 26.0767163305267
449 26.4639837065643
450 26.9803111944676
451 27.1739823380239
452 27.8934751119504
453 28.0721699321847
454 27.7779527543164
455 27.6581966372545
456 27.418081985391
457 26.997381214586
458 26.587560195033
459 26.1967883884923
460 25.7499093517206
461 25.8013196502849
462 25.3149225380361
463 25.1580983506249
464 25.8346331210026
465 25.5856634171114
466 25.6294269330371
467 25.3176927545875
468 26.0035295906061
469 25.6311022850124
470 25.4039310244021
471 25.0181771124957
472 25.0187768226544
473 25.3304868146693
474 25.0614906560787
475 25.9482147085498
476 26.709006138076
477 27.3397180173116
478 27.0283570172279
479 27.3603121862607
480 27.5177322813434
481 27.5370412193035
482 27.257712501029
483 27.2203235337244
484 27.0191846918691
485 26.080367162167
486 25.5137339328709
487 25.3864265483443
488 24.9037046735166
489 24.5795993705504
490 24.6347333123836
491 24.5405503152381
492 24.8978211276328
493 24.5461787967397
494 24.2782167514678
495 24.8317021834829
496 24.7230214731852
497 24.4306617765279
498 24.7844353108032
499 25.2792552943397
500 25.2159998292232
501 25.2081093847304
502 25.2391204443791
503 25.4746160291456
504 26.2733469972823
505 26.3203468097047
506 26.3645860070999
507 27.1042252908354
508 26.9719489021814
509 26.8436126279491
510 27.2588543590695
511 26.9888750702875
512 26.8123661715645
513 26.8409615311281
514 25.761402733739
515 25.5089582984944
516 25.4690268239828
517 24.4385958082363
518 24.2311164661431
519 24.1837208551042
520 23.6386825243923
521 23.7419004064096
522 23.8955682439666
523 23.2051530230002
524 23.7006188620852
525 23.4002701819312
526 23.9549860473912
527 24.1779709134306
528 24.4541719437952
529 24.275932650432
530 24.9483220619055
531 25.1717636605478
532 25.3974474205251
533 25.8053791713123
534 25.9196536871183
535 25.7736619931637
536 25.4526100645079
537 25.72379284038
538 25.4264993979102
539 26.1248651565144
540 25.7260754892707
541 25.5314457641254
542 24.9686572456423
543 24.9182789878912
544 25.0840415549037
545 25.1112756183629
546 25.1142368206102
547 24.6385232958128
548 24.7666994111457
549 24.1907250021211
550 24.2254050073827
551 23.9945409307973
552 24.3252239966759
553 24.6821676103761
554 23.784127620061
555 23.9951932027723
556 23.9293276183061
557 23.9473659087861
558 23.7110531673766
559 24.0398056492338
560 23.5497939931902
561 23.7775583992706
562 23.4273670779306
563 22.7467356201458
564 23.2500395037541
565 23.3655760414433
566 23.475527962347
567 23.6011702756619
568 23.7297891737843
569 23.0119163313391
570 23.3253369614431
571 23.4039731649102
572 23.5171249740044
573 23.6353573596739
574 24.0928640922967
575 23.6569599897654
576 23.455395784233
577 23.237483249039
578 23.2799385231948
579 23.3652685715002
580 23.5288038117629
581 23.4156800131492
582 23.494489345701
583 23.6623116312914
584 23.1151633529406
585 23.227876596694
586 23.013498063114
587 23.3676353172154
588 23.4533153281579
589 23.810659392403
590 23.8787589478511
591 24.0434042497379
592 23.8785140943979
593 23.7312334261483
594 23.6810385843423
595 23.8081583224795
596 23.9258375661605
597 24.2247847941044
598 24.1490433928443
599 23.615278116957
600 23.3292372446365
601 23.1422810576299
602 22.991788284765
603 22.8665886171011
604 22.8978754025294
605 22.6364953744505
606 22.5500780481199
607 21.8492112003454
608 21.3645567388406
609 21.4071112662799
610 21.0687219231053
611 21.4740433185069
612 21.3170841677124
613 21.9803350648672
614 21.8973092548766
615 22.2097403222129
616 21.9506222851882
617 22.237197589522
618 22.7613394805679
619 22.9748525497098
620 23.3645301700596
621 23.4267963496903
622 23.383060859896
623 22.9025947662871
624 23.1203904045806
625 22.8191086115188
626 22.880529814644
627 23.1706360103754
628 22.8758236740421
629 22.8640636894515
630 22.8318798602002
631 22.2321289291483
632 22.5945550686527
633 22.9338864551571
634 23.1433646433108
635 23.2801379036244
636 22.9967353498324
637 22.6876324717647
638 22.8940929850266
639 22.8066866861842
640 22.9244101193793
641 23.252261688427
642 23.3320657750963
643 22.9465899758281
644 22.4547408677432
645 22.4458504554331
646 22.9034294862434
647 23.3950908265442
648 23.3173081645417
649 23.5680392970274
650 23.2521177880767
651 22.988378463191
652 23.2341487428732
653 23.2682267167656
654 23.7084643517025
655 23.8142779763985
656 23.9143039410227
657 23.6677070840621
658 23.6801158554902
659 23.5460216014171
660 23.2763101141573
661 23.1451975128908
662 22.7215778143725
663 23.009618233197
664 22.8781292085735
665 22.7374763756292
666 22.3764873797832
667 22.7368314655894
668 22.4274136890685
669 22.2483035176936
670 22.5875277800537
671 22.9133019953871
672 23.3312961897754
673 23.1996994093562
674 23.3197318137214
675 23.5652873668569
676 23.906597070989
677 23.2802850676706
678 23.6325167926302
679 23.5800208170985
680 23.7564313244221
681 23.722615016364
682 23.270863255729
683 23.3158642533616
684 23.4850242327552
685 23.3523970791715
686 23.0944738614671
687 22.9162980908888
688 23.77246279717
689 24.218723711066
690 23.9138810494236
691 23.9032110266517
692 24.1012632277071
693 24.148832209569
694 23.6636573694843
695 23.4921065097943
696 23.8505306860539
697 24.2857763946124
698 23.5643916027645
699 23.1069408628627
700 23.2103985038973
701 22.7096255817842
702 22.6252802636445
703 22.8780857392639
704 23.1107258293379
705 23.1868610673928
706 23.3626583409192
707 22.8103418235082
708 22.8905963311981
709 23.1376613485529
710 23.1510216977881
711 24.0866376137412
712 24.4638429848717
713 23.9475690035638
714 23.6756627966148
715 23.45642674891
716 22.8421284764384
717 23.0396359688517
718 22.5987622336567
719 22.5408480886266
720 22.5759466433902
721 22.3385249556133
722 21.9201799848948
723 22.1568507793648
724 22.3909506135162
725 22.7630481472899
726 22.6261350364201
727 22.7386805851167
728 23.0582260080622
729 23.0918966563539
730 22.9812189286735
731 22.3688119267385
732 22.4575370123805
733 22.33793867663
734 22.2691822220283
735 22.2238018451808
736 23.5314613041005
737 23.3745990946258
738 23.0437222305974
739 23.461798314904
740 23.0343156328929
741 23.2379591774815
742 23.4511399035552
743 23.2186625709121
744 23.0146978000814
745 22.9439431590378
746 22.0617177757707
747 21.936566048069
748 21.8059097056102
749 21.5372025702842
750 21.9266318672259
751 21.8840364445859
752 21.3007972041233
753 21.1463165036282
754 21.3220087920905
755 21.0589394055956
756 20.8960093671236
757 21.0570048911434
758 21.3047871668164
759 21.5824628368915
760 21.6546141104822
761 21.5151615357386
762 21.9222566814938
763 22.5070967642736
764 22.4875565118726
765 23.4582490093785
766 23.4143168615537
767 23.4906533830962
768 23.458785848066
769 23.0520891722084
770 23.4185055828386
771 23.917488125742
772 24.1345497793742
773 23.7550393928902
774 23.451691340068
775 22.8281849378834
776 22.8544077652718
777 22.5026023106241
778 22.5614885595958
779 22.947683387531
780 22.724251139324
781 22.3198500335609
782 21.8022063032681
783 22.5178763528429
784 22.7828671110382
785 22.4372928947037
786 22.4355732963552
787 22.498572815849
788 22.5662547422284
789 22.4746106065974
790 22.1292000944165
791 22.3748833023322
792 22.2085533578165
793 21.6082118032455
794 21.3832156118585
795 21.8027124232004
796 21.9838453111298
797 22.0071226091016
798 22.0360713335432
799 21.8627082811211
800 22.1122452074126
801 21.7143277233356
802 22.1612793750355
803 21.754921426874
804 22.0394676023415
805 21.6262813590875
806 21.2761945670279
807 21.5897115294592
808 21.6236632611878
809 21.2387868115485
810 20.8391599812863
811 21.1618496314699
812 21.0423586482081
813 21.2836150233185
814 21.4062600290161
815 22.0498199527409
816 22.2245628337609
817 22.1854278095742
818 21.9379122251076
819 22.0239863921434
820 22.2268229838819
821 22.4129423606436
822 22.2904707301318
823 22.361590302121
824 22.3787240403429
825 21.7058792698784
826 21.6460773961112
827 21.4196790112591
828 21.5385031876839
829 21.3868550508869
830 21.6499317775027
831 21.3455428256149
832 21.1857433582911
833 21.3458010174323
834 21.0502259960084
835 21.8418871023967
836 22.0845338845192
837 22.198232796202
838 21.8678294975694
839 22.3640286075865
840 22.7990247532343
841 23.0440089020289
842 23.2910302202173
843 23.2219797664915
844 23.3929921444937
845 23.0397471950999
846 22.9151740214405
847 23.2609477288799
848 23.6461876450667
849 23.8176474496397
850 22.9890074376412
851 22.8329091621862
852 22.3398358477043
853 22.7006826205532
854 22.7742938832682
855 22.8719293609566
856 23.0798301834033
857 23.2795663156514
858 22.8905428686773
859 22.4508541277034
860 22.4512263515321
861 22.2012965225796
862 22.5863140618318
863 22.8629109811152
864 23.6758807871482
865 23.9151908801931
866 23.6383772452557
867 23.0980002876464
868 23.2189139489341
869 23.4410083884671
870 23.8277804809064
871 23.8968805636859
872 23.9932423829287
873 23.8669398130087
874 22.8365214030557
875 22.1495902391855
876 22.4323157762505
877 22.1597014774997
878 22.2439825138186
879 21.8948022129085
880 21.8335409490847
881 22.4028556663925
882 22.1431725858593
883 21.4392446656949
884 21.5959768578786
885 22.0519693627822
886 22.3711670301241
887 22.3703372409095
888 22.8107171470603
889 22.9745860128538
890 23.4608511238309
891 22.9616679022068
892 23.0145728555179
893 23.3646219711407
894 23.2063957920637
895 23.1442359465189
896 22.4217779152308
897 22.7566209890691
898 22.3013526339644
899 22.3478608005096
900 22.1533373691397
901 22.6250871878896
902 22.8520641424996
903 23.3407647729268
904 22.9720224895236
905 22.6924131487441
906 23.069355977882
907 22.4656409015587
908 22.4595329302336
909 22.0995711796412
910 21.4244484554057
911 20.9433744535954
912 20.9604763888223
913 19.9976127229741
914 19.8814791958623
915 20.034491186607
916 19.6114840488846
917 20.2650578206623
918 20.028442186448
919 20.1525737590219
920 20.4913907963388
921 20.1411843293989
922 19.592983640968
923 19.5558808964173
924 19.7611151539823
925 19.516186228663
926 19.8254655663816
927 19.3256515926092
928 19.3666700674579
929 19.4565518555301
930 19.3511508816357
931 19.7576993445374
932 20.1694516330148
933 20.6399405699345
934 20.8757569574358
935 20.7106924389586
936 20.6029623947606
937 20.7093880936212
938 20.6654955397544
939 20.5253077850395
940 20.3722950289883
941 20.3431082393846
942 20.4801891992347
943 19.9838629553219
944 20.5285277222279
945 20.6730281741772
946 20.5123757730583
947 20.7822039529488
948 21.1194472973935
949 21.2300446332142
950 21.4115171464227
951 21.3829904010177
952 20.8392779907533
953 21.3841843643115
954 20.3872013690231
955 20.4085521789195
956 20.5434058563536
957 20.4482842469418
958 20.5268772695174
959 20.5001789700789
960 20.4368758490094
961 20.3053432743448
962 20.5247365991844
963 20.2363331599146
964 20.7087079088153
965 20.9271967937611
966 20.8377921515622
967 21.0636055760614
968 21.6923695473752
969 22.5984650233797
970 22.7450628326825
971 22.6350850133052
972 23.3767424001075
973 23.5828456212745
974 23.213903804708
975 22.9393249912659
976 23.5443218045084
977 24.0574214220193
978 23.2766674327657
979 22.6928435236626
980 22.7472057950904
981 23.5820644365448
982 23.079261395433
983 23.0557485506135
984 24.101580799482
985 24.0780384973711
986 23.7397359809981
987 23.2755517544273
988 23.052613542778
989 22.6261585356484
990 22.7869668299671
991 22.3192653238683
992 22.5309564416999
993 22.6038505357135
994 21.8445040604623
995 21.8921735836754
996 22.7318020652102
997 22.8479332762933
998 23.0138499306527
999 22.9307503361836
1000 22.4131911769379
1001 22.5510195270795
1002 22.0198279850723
1003 22.2924988923877
1004 22.6091030767553
1005 23.1892996944005
1006 22.2903138524863
1007 21.7215472568209
1008 22.3287772471455
1009 22.8420976076835
1010 23.0057730337107
1011 22.6690229317227
1012 22.7992374969344
1013 22.4816400469347
1014 22.1756200700255
1015 21.7829949041257
1016 21.8295290772285
1017 21.8762282014056
1018 21.6052519015555
1019 21.4004949379859
1020 21.5025032909392
1021 21.2446466617243
1022 21.7047358491342
1023 21.3951234033315
1024 22.0494434155188
1025 21.8497937385514
1026 21.5853265110196
1027 21.4598393663216
1028 20.9106687805918
1029 20.8700463828701
1030 20.4451054922328
1031 20.8278758437575
1032 20.284864812583
1033 20.3785157461272
1034 19.8851468768215
1035 19.6814572071015
1036 20.0590175910852
1037 20.3136679359688
1038 20.9157917098907
1039 20.735208272439
1040 21.2324780849142
1041 21.0587338885469
1042 21.131306510914
1043 21.2001984822337
1044 20.9393465416965
1045 21.6662931095139
1046 21.1909695106551
1047 20.9650463519886
1048 20.8531493609614
1049 20.7095129600871
1050 20.1642657371946
1051 20.2914093538742
1052 20.6757176819194
1053 21.0415272403613
1054 20.7661877012978
1055 20.0690129756695
1056 19.9751503258373
1057 20.3768647019113
1058 19.8797419410223
1059 19.9213693658442
1060 20.2955718570576
1061 20.1438590777144
1062 20.0276483273672
1063 19.853703827697
1064 20.4023168298485
1065 20.5444667458264
1066 20.7190225010453
1067 20.4666673011286
1068 20.6982336810554
1069 20.547867154884
1070 20.2875599958046
1071 20.7874669226232
1072 20.5662224053286
1073 20.3307012266881
1074 19.9059634871746
1075 20.4498058625462
1076 20.3740291235821
1077 20.3062547054323
1078 20.0712133513297
1079 20.7478004555642
1080 21.1369110139233
1081 20.6625003406673
1082 20.5794832246952
1083 20.719025523163
1084 20.8207359138187
1085 20.1912876275064
1086 20.2543466858133
1087 20.9761857592857
1088 21.2204651009929
1089 20.6409380230964
1090 20.4524268985909
1091 20.5296805148388
1092 21.0538809899819
1093 21.1591625801085
1094 21.2312440123423
1095 22.1101196617265
1096 21.9187890393515
1097 21.5734973843372
1098 21.4575596085041
1099 22.1002352126746
1100 21.835857673179
1101 22.1701457756962
1102 21.8300317652229
1103 21.6194370531154
1104 21.4663819007788
1105 21.1855406492635
1106 21.4825711662099
1107 21.1793341106782
1108 21.4241310754657
1109 20.8070442915058
1110 21.195467159712
1111 20.5068692221287
1112 20.5908931535979
1113 20.7051822306821
1114 21.2631210946033
1115 20.8054632885953
1116 20.6500754686169
1117 20.7795767265559
1118 20.3965522654918
1119 20.1862594094322
1120 19.7659297923322
1121 20.0445965581869
1122 20.8928321385164
1123 20.8770676232898
1124 20.9645921410456
1125 20.9176504267371
1126 21.6128197708594
1127 21.5777943022115
1128 21.5181434623034
1129 22.8028130015194
1130 23.2098641859668
1131 23.117495247162
1132 22.1326917457286
1133 22.0881621945485
1134 22.0609904024789
1135 22.3981160992519
1136 21.6602734041603
1137 21.924635022973
1138 22.1044181188847
1139 20.8572327681841
1140 20.537666151918
1141 20.5737997125215
1142 20.4934765071374
1143 20.439532952949
1144 20.2919293994395
1145 20.1473951149058
1146 20.3518789937321
1147 19.6499897679343
1148 19.6821022244339
1149 19.9586599111059
1150 20.0404388770076
1151 20.6151512407642
1152 20.3096859826399
1153 20.071735878146
1154 19.8773284653962
1155 20.0968363430675
1156 20.0430565476051
1157 20.3331304379304
1158 20.4813664313674
1159 20.4226978465804
1160 20.4266539533635
1161 19.9760047956808
1162 20.2147730028226
1163 20.5891306659028
1164 20.6735198250922
1165 20.2963394904121
1166 20.141396641543
1167 20.0056412263303
1168 20.5229880187878
1169 20.6574044863476
1170 20.8318800478126
1171 20.7723225418165
1172 20.9804771962806
1173 20.6626247526065
1174 20.4595056235179
1175 20.4994268164698
1176 20.5437579390816
1177 20.8823889242947
1178 20.154752866323
1179 20.0755054261957
1180 19.8661995260894
1181 19.6159896324488
1182 19.6597974904014
1183 19.4981487884563
1184 19.3564300972879
1185 19.4242428825618
1186 19.2648866366418
1187 18.8332893306508
1188 18.9326246613571
1189 19.4187982772294
1190 19.5442363646677
1191 19.6593078493188
1192 19.4757138992431
1193 19.7386917183229
1194 19.9524474989676
1195 19.7728797229671
1196 19.8772020672726
1197 19.9837851664016
1198 20.3453922747716
1199 19.8853643655975
1200 19.8007922993069
1201 19.821085248975
1202 19.666888203024
1203 19.6608664493928
1204 19.4824052705074
1205 19.1659744678301
1206 19.0014363021575
1207 19.051778662099
1208 18.6209062931432
1209 18.4296790569692
1210 18.5885622883614
1211 18.560628602866
1212 18.4075458002902
1213 18.7960438466257
1214 19.0237500923098
1215 19.5043559767978
1216 19.7827322073475
1217 19.6308824256259
1218 20.5940772492412
1219 20.3935242285776
1220 20.2428517109326
1221 20.3926745621857
1222 20.6991137952546
1223 20.3316201256629
1224 20.2588115494851
1225 19.9023774341515
1226 20.7421420602764
1227 20.8467914558374
1228 20.0404288982513
1229 20.1068092508263
1230 20.0066200505536
1231 20.1421595033678
1232 20.1978646720933
1233 20.0641780798428
1234 20.0823431665887
1235 20.48663850757
1236 19.2026445841628
1237 19.5454168878731
1238 19.2753591079259
1239 19.4870483592725
1240 20.1427349494708
1241 19.9206764008437
1242 19.6912037556552
1243 19.3834633031601
1244 19.1347616779547
1245 19.1193239018539
1246 20.0559581776155
1247 20.2465098523213
1248 20.44779129345
1249 20.5454215117265
1250 19.8955838576111
1251 19.6718682221906
1252 19.7757909029628
1253 20.4476894920451
1254 20.3647025499182
1255 19.7000359881038
1256 20.3340045005926
1257 19.6079616222159
1258 19.921448152881
1259 19.8817829519836
1260 20.1603697621888
1261 19.9763539221807
1262 19.9418806844888
1263 19.4777118851019
1264 19.4925058905387
1265 19.7309654451021
1266 18.3359139402359
1267 19.1183523032716
1268 18.6511611434456
1269 19.1488822400956
1270 18.8923610100268
1271 19.3662910258085
1272 19.4275776440506
1273 19.2967079391663
1274 19.0526409333408
1275 18.8571354986957
1276 18.8083994063094
1277 18.2083732918397
1278 17.9987026980358
1279 17.93533610709
1280 18.0687384184294
1281 17.9880484868963
1282 18.2123851083108
1283 18.3605242536865
1284 18.6035603051257
1285 19.7028922398835
1286 20.0222216262982
1287 20.1585158991489
1288 20.3395092522573
1289 19.8823247033786
1290 19.7384220072304
1291 20.0059501498655
1292 20.0906647332482
1293 20.3238632905773
1294 21.5538995683595
1295 20.8934004540589
1296 21.3242314178038
1297 21.3430680447086
1298 21.3596503329373
1299 21.3741526554191
1300 21.8142165893092
1301 21.4741048083584
1302 21.3939206246292
1303 21.0155731844579
1304 20.1237538792992
1305 19.8557924278498
1306 19.4920169854195
1307 19.2248228902067
1308 19.2333823693848
1309 19.6746118302748
1310 19.3451075744735
1311 19.56518644634
1312 19.2027767954362
1313 19.4022007341164
1314 19.3579293368628
1315 19.6968058665684
1316 19.6002770106225
1317 19.7403271949837
1318 19.6207509848666
1319 19.2885898603527
1320 19.1069791884631
1321 19.2103499829561
1322 19.09089980505
1323 19.3563205919813
1324 19.0283549073438
1325 19.1344335125936
1326 19.2046667074165
1327 19.5132835953882
1328 19.7406645460726
1329 19.7759461226703
1330 20.4213767234327
1331 20.0982980725107
1332 20.052062025119
1333 19.8955392879826
1334 20.3551135197974
1335 20.4022568065246
1336 20.6500508710012
1337 20.4122602056973
1338 20.9923458850335
1339 20.7919915209064
1340 20.0025848403056
1341 20.1880040183152
1342 20.4194662144276
1343 20.3363510824813
1344 20.1372987965922
1345 19.9298135763055
1346 19.7494557822342
1347 19.8855076652964
1348 19.4729722384258
1349 19.7062551877693
1350 19.8082241600554
1351 19.4949390143821
1352 19.9020614284999
1353 19.4341477722544
1354 19.18113862273
1355 19.0847017833199
1356 19.0263004340977
1357 18.6571223804713
1358 18.2179239986864
1359 17.7783908316782
1360 17.8497461042141
1361 17.8636612877055
1362 17.6895995385207
1363 18.0815453475814
1364 18.260087836713
1365 18.6958172108917
1366 18.5988985102356
1367 19.3594196465902
1368 19.3346068839849
1369 19.5872030454409
1370 19.6918209121376
1371 19.907331172261
1372 20.0787535385509
1373 20.1991320363903
1374 20.5243987555123
1375 19.886296519915
1376 19.6786340004754
1377 19.8461505762435
1378 20.133683826833
1379 19.8246053197843
1380 19.6787939389998
1381 19.4941236568117
1382 18.9647598307917
1383 18.8452567914477
1384 19.2909765016539
1385 20.1486453953837
1386 20.2381468615764
1387 19.7622589946349
1388 19.6797188722291
1389 20.3609363706862
1390 20.2596477909997
1391 21.1730794368148
1392 21.383987747017
1393 21.3652612737602
1394 20.6515992885709
1395 20.3686988932954
1396 20.8787872822766
1397 20.8213644951366
1398 20.7048270761958
1399 20.3814887015233
1400 20.8111395363853
1401 20.3346099262282
1402 20.2173103309303
1403 21.5050814481138
1404 21.4628044222782
1405 20.7925341058971
1406 20.6787156946702
1407 20.4738588048148
1408 21.05481144707
1409 21.1420037673254
1410 21.1470508850252
1411 20.7429464516256
1412 20.7303774701327
1413 19.3799197606116
1414 19.5783720751275
1415 20.3070267636251
1416 19.8341993423894
1417 20.6604661727243
1418 20.3801573762055
1419 20.2483497897237
1420 20.8758984299558
1421 21.1789790389484
1422 21.0496731539746
1423 21.2665583187713
1424 20.9281155062189
1425 20.5564625366891
1426 20.5390923836033
1427 19.3928245410105
1428 19.1885247492922
1429 19.2666238257267
1430 18.3505786563855
1431 18.3968956135892
1432 18.3369541896775
1433 18.4054785839395
1434 18.9271254695641
1435 19.0285908605314
1436 19.0513698014227
1437 19.5205292691399
1438 19.3629365872171
1439 19.2819646880244
1440 19.6110853099332
1441 19.3870130447417
1442 19.7393283987436
1443 19.6407607706663
1444 18.6938395437054
1445 18.2154809276823
1446 18.4974365304329
1447 18.2527921641332
1448 18.7123064562806
1449 18.6045894527377
1450 18.3702546884248
1451 18.6555271415768
1452 19.2329465693239
1453 18.8702923824415
1454 19.5702237536833
1455 19.6390507157735
1456 19.3776013104286
1457 19.7251418341536
1458 19.4759461673847
1459 19.6943372942778
1460 19.6455106043785
1461 20.1249298981196
1462 19.5283558473898
1463 20.0454517085811
1464 20.1396496460917
1465 20.6269885599572
1466 20.4923546904608
1467 20.3368214227735
1468 20.1369172687115
1469 20.5263018211075
1470 21.1143031402292
1471 20.3561511237101
1472 20.0654324484209
1473 19.7612413078927
1474 19.6824556108536
1475 19.1666555572881
1476 19.5002686634759
1477 19.7093085010009
1478 20.0790410307073
1479 19.1817332573981
1480 18.6865181085145
1481 18.728588464711
1482 19.0037338173009
1483 19.0896267521439
1484 18.807697831636
1485 18.9004627414084
1486 20.0038683583087
1487 20.0455843231486
1488 19.5442454896432
1489 19.6844428366739
1490 19.7623064576742
1491 19.6643466017271
1492 19.471364581584
1493 19.2791605830453
1494 19.359063763622
1495 19.5810960594534
1496 18.5514204970668
1497 18.1022196404242
1498 18.4834665898122
1499 19.1877814433169
1500 19.0110828367873
1501 18.9666167285633
1502 18.9364388907084
1503 20.0036164303977
1504 20.5344233764364
1505 20.6774645243506
1506 20.4575367713407
1507 20.9485730716896
1508 22.2947360164584
1509 22.3370272803861
1510 22.7270438441963
1511 22.5433349344673
1512 22.9152482715967
1513 22.2756417360788
1514 22.1811179003458
1515 22.1289987045272
1516 22.5809819713914
1517 22.0523861700627
1518 20.3101017545947
1519 20.1221177507205
1520 19.5596687403793
1521 20.2740071587134
1522 20.0096175874761
1523 19.9097632195643
1524 19.4120553174811
1525 19.2141063502909
1526 18.5244136881057
1527 18.7759408638499
1528 19.6987435593936
1529 19.2081255174873
1530 20.3383872081398
1531 19.8505464174149
1532 19.6847055475647
1533 19.7092535248355
1534 20.2405671157393
1535 20.6625690990597
1536 21.2840065099144
1537 21.0288868578235
1538 20.5475432671035
1539 20.6924531395248
1540 19.8161464359202
1541 19.5863962246955
1542 19.5150085601686
1543 19.8465857123025
1544 19.3743856783228
1545 18.9285809629811
1546 18.7907148414844
1547 18.732799846383
1548 18.4158075197906
1549 18.1469691422908
1550 17.9302181823972
1551 18.1047045089201
1552 18.3958475681078
1553 17.8951009181419
1554 18.3511619763326
1555 18.2908526739465
1556 18.1270866219594
1557 18.3735382016449
1558 18.3489717421762
1559 18.8943546504135
1560 19.0291974247456
1561 19.2578811255203
1562 19.0662379405447
1563 18.9939334158934
1564 18.4134363500578
1565 18.3986578616103
1566 19.0879880937678
1567 19.5293479404693
1568 19.6379642832436
1569 19.6710617516671
1570 19.4654045298469
1571 19.3047876171725
1572 19.4188587055787
1573 19.5193312419724
1574 19.6022516589271
1575 19.6383716568315
1576 20.0800381248779
1577 19.9541043154007
1578 19.8557329743802
1579 19.7945998672098
1580 20.7693852931245
1581 21.8366698613174
1582 21.9465991261802
1583 21.9652941025181
1584 22.0117920209264
1585 22.4932595587474
1586 21.0421235869545
1587 20.5798309257899
1588 21.2315842536437
1589 21.2297451970951
1590 20.313008245513
1591 19.0293735764967
1592 18.8963773233213
1593 18.9192962709787
1594 19.957929752303
1595 19.9239453201175
1596 20.3814020564589
1597 20.5038792631453
1598 20.1851357803698
1599 19.8800946041602
1600 20.5350767251107
1601 21.2578834843755
1602 21.4567192176703
1603 21.2512527277981
1604 21.2440915482
1605 20.756646824884
1606 20.4600004371308
1607 20.294688827472
1608 19.9699754131124
1609 20.0233483198032
1610 19.7476592928331
1611 19.0107932792228
1612 19.2682465546878
1613 19.851947168233
1614 19.0279620710569
1615 19.3545229052951
1616 19.552235492517
1617 19.7060062853434
1618 19.713204635238
1619 19.8901915525085
1620 19.9629984162528
1621 20.2250535690465
1622 21.1681067452053
1623 21.1196707067894
1624 21.2785895318326
1625 21.3544145928972
1626 21.5485192388462
1627 21.4562391718729
1628 21.5579296341836
1629 21.1690570583903
1630 20.9375394436613
1631 22.1005033715202
1632 20.7385171411568
1633 21.8602919605291
1634 21.2373021559347
1635 21.064364373094
1636 20.6621329910622
1637 20.9393710982541
1638 20.9205571963506
1639 21.2956447777936
1640 21.649492667015
1641 20.5196389998068
1642 20.9299838919454
1643 19.3369565656648
1644 19.8262127552789
1645 19.9445012111141
1646 20.2134355543704
1647 20.1830153964553
1648 20.4212139425149
1649 20.2430300744775
1650 19.853851376557
1651 19.857267092234
1652 19.5279001607201
1653 19.5900182786892
1654 19.6107219890344
1655 19.3678759508297
1656 19.2657887036356
1657 19.0732843664659
1658 19.1654433677289
1659 19.171489585953
1660 19.6037302203919
1661 19.9512440217649
1662 20.2696854430944
1663 20.4527911646149
1664 20.6704159800508
1665 20.4223972544501
1666 20.72785432569
1667 20.6470208963934
1668 21.2154032524345
1669 20.9607463038371
1670 20.8933739181572
1671 20.8293944945075
1672 20.7537932910091
1673 20.5399285164242
1674 19.9914518582751
1675 20.3999789438164
1676 20.1226354978731
1677 20.4252688859304
1678 19.9402977415072
1679 20.2077537849052
1680 19.674101886998
1681 19.4880349685171
1682 19.1830845368724
1683 19.2317356972364
1684 19.1097244348291
1685 19.9140277423592
1686 19.8628846438356
1687 19.9876578093257
1688 19.4821239842913
1689 20.1163936642326
1690 20.2787894140557
1691 20.5005535415677
1692 20.123277899909
1693 19.9223429773311
1694 20.4250422670632
1695 19.5895672319143
1696 20.4581706450091
1697 20.0434111156202
1698 20.3496699760645
1699 19.9327117586103
1700 19.7101009129902
1701 19.207954359988
1702 19.2316868386669
1703 19.1947551614115
1704 19.1926769111436
1705 19.0548568691453
1706 18.3570219729998
1707 18.3868320179451
1708 18.2774487966213
1709 18.3830078866014
1710 19.0155834976782
1711 19.4976306061358
1712 19.9973076486137
1713 20.485142706165
1714 20.3806682367945
1715 20.3358556561902
1716 20.2325671211112
1717 20.238485128242
1718 20.1346117113835
1719 19.9274730531296
1720 19.833420039676
1721 19.4697396404528
1722 19.0075128978644
1723 19.8007427726434
1724 19.4926774906181
1725 20.1560976320227
1726 20.2147049919296
1727 19.9789802120844
1728 19.6564273963261
1729 20.024905738658
1730 20.1506311513458
1731 19.8133433841278
1732 19.8792444442848
1733 18.9590977768443
1734 19.4961076530197
1735 18.8280753515225
1736 19.3296323575601
1737 19.945728230378
1738 20.0928293214999
1739 20.2109461143251
1740 19.8774375926806
1741 20.916137198254
1742 22.6263295483558
1743 22.3096321399757
1744 21.9282930458992
1745 22.1311748216621
1746 21.5592063334399
1747 20.5954106093432
1748 21.0139532733124
1749 20.2162098036996
1750 20.7430085196648
1751 19.9912125036379
1752 18.8182979955298
1753 18.7695742037132
1754 18.7150474853549
1755 18.6019504790753
1756 18.3674515341005
1757 18.8202173523161
1758 18.7952313695882
1759 19.1509603063541
1760 18.7530344122382
1761 19.2201884498279
1762 18.9042216986387
1763 18.637362132717
1764 18.6244115253552
1765 18.959393439025
1766 18.8260950125077
1767 19.0046572978025
1768 18.5655893534399
1769 18.4372536045443
1770 18.0459917379047
1771 17.4659321713248
1772 17.4340880856564
1773 17.6755621975191
1774 17.5751433530557
1775 17.4532368116814
1776 17.7248514677292
1777 17.953607640995
1778 18.3602156389874
1779 18.2954072607022
1780 18.667141489436
1781 18.8931882797484
1782 19.6385454934906
1783 20.5416464112794
1784 20.822270039129
1785 20.6155991588429
1786 21.2012429667679
1787 21.0862301739853
1788 21.1034519162816
1789 21.5022005175578
1790 21.7120658270821
1791 21.8298904284789
1792 21.2601805553607
1793 20.4872448377117
1794 21.8016452224535
1795 21.9494288434006
1796 21.1243699092079
1797 21.6886796986209
1798 22.6650695101471
1799 22.2505456303269
1800 21.4177282823135
1801 20.9238334997132
1802 20.9368071465303
1803 21.1744528753605
1804 19.6816666514555
1805 19.890838025756
1806 20.5714871697846
1807 20.2864984986476
1808 19.6911509945864
1809 19.8778782078494
1810 20.4250639078819
1811 20.6228395550449
1812 20.2772539699166
1813 20.0166832282074
1814 20.0700275803091
1815 20.0270865674047
1816 19.5478225253653
1817 18.7808412361493
1818 18.1658129987783
1819 18.7229371372533
1820 19.2795007024464
1821 19.2077656509937
1822 19.5100551915371
1823 19.7676432370769
1824 19.8918052902003
1825 20.2615891544956
1826 20.3768402855058
1827 20.6510608435534
1828 20.8451109849507
1829 20.1992198361698
1830 19.6745377611815
1831 19.8113039631234
1832 20.0178329238466
1833 19.9767770935088
1834 19.6900724542527
1835 19.3266311495732
1836 19.8705321742384
1837 20.6359703217726
1838 21.4503736698491
1839 21.0953309311545
1840 21.3604646424328
1841 21.3057091486447
1842 21.1661719168686
1843 21.2815018702927
1844 21.5115562492485
1845 20.8150850827806
1846 20.6583105653213
1847 20.3065617703841
1848 19.9775325068433
1849 20.5146314823284
1850 20.3800517866837
1851 20.3767958247953
1852 20.0010002772851
1853 20.6399700292998
1854 20.9289123556982
1855 21.2945791641911
1856 20.6421769372462
1857 20.5713269981099
1858 19.7135902849069
1859 19.3440841492478
1860 19.0832449013696
1861 19.4938553304108
1862 20.678340367642
1863 19.5734336323143
1864 19.0387953535006
1865 19.2612158904197
1866 20.2601978322588
};
\addplot [semithick, forestgreen4416044]
table {%
0 28.1803190552454
1 29.1185345259133
2 29.9203588745836
3 30.6560513730003
4 31.1061917655013
5 31.4263982179341
6 31.8420441510069
7 32.383002991117
8 32.7493885224247
9 33.4090838943031
10 34.1826490885932
11 34.9433300942735
12 35.9075951030958
13 36.9593416954186
14 37.8877857205294
15 38.9060387021493
16 39.7098019822276
17 40.4228104949681
18 41.1346407015868
19 41.8018365083109
20 42.4942766090954
21 43.0868705139753
22 43.6008915754133
23 44.1454257844596
24 44.8218267603284
25 45.45696325611
26 45.999455400756
27 46.5314210914715
28 47.0433338334628
29 47.2784496425112
30 47.3387947754449
31 47.3777405019038
32 47.2012832477109
33 46.9552282349763
34 46.6914239848103
35 46.342000227486
36 45.8866381694801
37 45.5420739936639
38 45.0687962290484
39 44.6120425904924
40 44.1402223500312
41 43.719979373772
42 43.3304082060343
43 42.5498474010073
44 41.669408242168
45 40.8253476866688
46 40.0716603914159
47 39.2012855334703
48 38.5502606664841
49 37.8785555118359
50 37.2749687858166
51 36.3471790840815
52 35.6048945134249
53 35.1751489914716
54 34.9070691198789
55 34.5107141478193
56 34.2604189078372
57 33.6379076411497
58 33.2246232717432
59 32.7347976620604
60 32.1740429789611
61 32.1368907222493
62 32.0918225336524
63 31.8375965045791
64 31.5880949122266
65 31.7297165563942
66 31.8078780635231
67 32.4934681037839
68 33.0295557260218
69 33.5266301061961
70 34.370907719929
71 34.7340749596382
72 34.9073474356884
73 35.1879956482999
74 35.7089607081966
75 35.8175458769783
76 35.812012086766
77 35.9523678375911
78 35.5828432139715
79 35.6210731763738
80 35.5999739014871
81 35.9727436449267
82 36.3679792644988
83 37.2205667971637
84 38.0990509814009
85 38.8817529787548
86 39.9855913588681
87 41.2310628751564
88 42.3096981156566
89 43.4679251921941
90 44.0791429259413
91 44.4340988373569
92 45.0636926646695
93 45.3949931532375
94 45.4617579831474
95 45.6404996205116
96 45.4106458927312
97 45.2076024106917
98 45.1739040164556
99 45.1523511578453
100 45.337421800695
101 45.7982074244383
102 45.9105329937452
103 45.9143891180279
104 45.6344647340866
105 45.7433942327281
106 46.0124941692704
107 45.8106781497949
108 45.4573533046363
109 45.2007233792109
110 44.9699182562536
111 44.3338141728636
112 43.8556530116338
113 43.9466950697547
114 44.4001724512469
115 43.9945187736684
116 43.8708893041281
117 43.9845207245846
118 44.2848130226214
119 44.1503915935389
120 44.3470554717587
121 44.4535396248349
122 45.0518622447206
123 44.9162915349624
124 44.6596753450916
125 45.3135885856698
126 45.224462633696
127 45.1700154438755
128 45.0460137573465
129 44.9848988116275
130 45.0730611340282
131 45.440228161243
132 45.4096185115847
133 45.5592190839944
134 46.0241313622428
135 45.7350803329036
136 46.288455425374
137 45.9597859395279
138 45.6828597557723
139 45.8208908527144
140 45.6561913956746
141 45.465064674707
142 44.7444726789047
143 44.4776822404045
144 43.7919350421334
145 43.5830317466254
146 42.6612785115292
147 42.3967499973974
148 42.504589441649
149 42.5701671950274
150 42.1560795514184
151 41.9500510920854
152 42.0326227583461
153 41.8271553153419
154 41.6827190050153
155 41.7919657975667
156 42.0941570585906
157 42.2744301210256
158 41.7715898434844
159 40.8571274923202
160 39.8183479994611
161 39.7477338876846
162 39.6872750997462
163 39.3732675021251
164 39.0367068056999
165 38.2763444861052
166 37.7862023394676
167 37.2132114435869
168 37.296710116398
169 37.7513766967842
170 38.3511008066381
171 37.9193407345475
172 37.5221429576105
173 37.6056552941214
174 37.5167970451911
175 37.3754340547466
176 37.2215774186005
177 36.8653242502521
178 36.7945938346708
179 36.5150762911632
180 36.4727223375443
181 36.2656727580278
182 35.7818713633329
183 35.3345971698702
184 35.2352702448449
185 35.1762720689162
186 35.5041092791829
187 35.7685513078514
188 35.8511566586795
189 35.4860391972437
190 35.0820259054099
191 34.7026835558307
192 34.7357474746373
193 34.2601730879009
194 33.9392117862353
195 33.8225854592918
196 33.2738704848318
197 32.8432826298935
198 32.7831371229126
199 33.1472891638414
200 33.6002490380647
201 34.0081451139866
202 34.4306144238982
203 35.0773036127054
204 35.4673062718754
205 35.4745990566536
206 35.8512739235187
207 36.9758125631093
208 37.3384576605876
209 37.7659942992242
210 38.1605566035696
211 37.9441152283207
212 38.3869934327774
213 39.0126295445769
214 39.3994797130844
215 39.7330652470681
216 39.8382977584049
217 39.4562142411893
218 39.2698904418168
219 38.9828797207291
220 38.6262840386569
221 38.7269577843098
222 38.276127377811
223 37.8430041388002
224 37.3858465947405
225 37.4138374006741
226 37.534579074696
227 37.3843678887384
228 37.099326328843
229 36.5763515934611
230 36.4280616317758
231 36.0050025243466
232 35.9785505284265
233 35.5928213106915
234 35.4617177841002
235 35.3456052066155
236 34.6152316800423
237 34.4582113859163
238 34.1230748778989
239 34.6271132818734
240 35.069423849621
241 35.9177003292743
242 36.3734940076357
243 36.2630556015657
244 36.6406224671723
245 36.8956022475664
246 36.9078569760306
247 36.6323059211753
248 36.9569741186288
249 36.4506644736738
250 35.6644619630134
251 35.2699708560449
252 34.5593073221752
253 34.7351331437184
254 34.547824944782
255 34.2225956551979
256 34.3740934948386
257 34.2402663064789
258 34.0729606058381
259 34.1144503843132
260 34.7717509514975
261 34.5805447417129
262 34.5619650501925
263 34.8553059357089
264 34.5465537941803
265 34.5264140889156
266 34.5878161565061
267 34.7143903973517
268 34.4645994020765
269 34.2676089059161
270 33.9164083449611
271 34.1605566437165
272 34.5635818772633
273 34.1806194154859
274 34.0635102229006
275 33.83639375608
276 34.0773280029951
277 34.1642023024271
278 34.5060696965885
279 34.7225846990134
280 34.9376295411509
281 35.360476316959
282 35.3747754651632
283 35.8092910867525
284 36.4294561443421
285 36.5646077194318
286 36.5063719192713
287 36.5058225843215
288 36.6910106954959
289 37.3112675697694
290 37.205522384521
291 36.9003859388623
292 36.5533657219983
293 36.304847727218
294 36.2994163498679
295 36.6266036564699
296 36.1740884283694
297 36.3697797025843
298 36.2094935965381
299 36.466504249672
300 37.1873413961747
301 38.0729257332141
302 38.9735879765467
303 39.6150525422723
304 39.9355130396699
305 40.9343657380471
306 42.3628440625896
307 43.3989576948677
308 44.1602996871543
309 44.3992705920354
310 45.1198591665082
311 44.8541739521035
312 44.9018741459439
313 44.5919336074531
314 44.0723678024031
315 43.69464446496
316 43.2989195485584
317 43.167104120122
318 43.9480809437324
319 44.598255675473
320 44.5148801878676
321 45.0655077487787
322 45.1260696377259
323 46.2664849443684
324 47.6727144148875
325 48.2336441752916
326 49.0320571191985
327 49.5034069269783
328 49.3756235920853
329 49.2739172837393
330 48.9777179037754
331 49.1953832819086
332 49.8968656406294
333 49.9837983661444
334 49.6919758150293
335 49.9178736214596
336 49.5799186188627
337 49.205968600811
338 49.1651523440344
339 49.0342084069205
340 49.101572960912
341 48.7612840670086
342 48.2830270608812
343 47.9916383011017
344 47.6947423536488
345 47.2856446378994
346 47.2719820395777
347 47.287153859789
348 47.1957023184369
349 46.5113009182671
350 46.0290687322881
351 46.0206139259715
352 45.7937884409486
353 45.5498048120945
354 45.7672595846999
355 45.9460231774142
356 45.9145869601669
357 45.5131585718244
358 45.2537991931799
359 45.4280416096816
360 45.4045164531956
361 45.2502836875331
362 45.4382529762246
363 45.1119077085149
364 44.768728550082
365 44.3547907548212
366 43.7467692477908
367 44.3374437685636
368 43.7679867393467
369 43.5573994276427
370 43.4592696553656
371 43.6225555162617
372 43.3075341125613
373 43.5275261431337
374 44.0035243814259
375 44.4994356187639
376 44.677696132839
377 44.9388859318451
378 45.6974931917544
379 46.6200186504872
380 47.2841329723452
381 47.8066714617706
382 48.5558320030433
383 49.3415914350305
384 49.6186574069719
385 49.7195381220219
386 50.3294161177129
387 50.3583066580561
388 50.2550620674608
389 49.7538365068857
390 49.4704923250832
391 49.0676694646761
392 48.4886539854375
393 47.8993542823812
394 47.2821796906326
395 46.6747418853765
396 46.5410781417473
397 45.8992966079785
398 45.5015598652766
399 45.1670232451757
400 45.4380058754983
401 45.8595457736779
402 45.8424326563292
403 45.5387245045743
404 45.7693237343443
405 46.0572138300122
406 45.6543947387265
407 45.2064994748881
408 45.1777950331536
409 45.0469897047068
410 43.7811033677533
411 42.8574855595354
412 43.0535633226862
413 42.5124202631241
414 42.3322858656171
415 41.6990452921875
416 41.7095551342478
417 42.044158843533
418 42.0439738770425
419 41.712496210595
420 42.7014355153782
421 42.4862654278101
422 42.3982931700686
423 43.0312790156402
424 42.1810215881902
425 42.2708257175955
426 42.1732859522744
427 41.8425091243329
428 42.084194598085
429 42.395483406009
430 42.3766036532784
431 42.4358528158946
432 42.1893651836729
433 42.1638203968626
434 42.9119208353851
435 42.8083649622742
436 43.1993855187036
437 43.4954671164115
438 43.3748019177078
439 43.8447025534127
440 43.8364500752545
441 44.1742019003151
442 44.7799735179269
443 45.0796959669485
444 45.211701222856
445 45.7950415171641
446 45.7812329779249
447 46.1278732773405
448 44.9690434959299
449 44.6907351032771
450 44.2757313159183
451 43.7049494473218
452 42.6252184311873
453 42.1266168790768
454 41.8096858359195
455 41.4242832700776
456 41.1748898978569
457 40.0761533394683
458 41.4150226244877
459 41.8532850930355
460 42.223551643228
461 43.2057475891872
462 43.9382219656231
463 44.8025335999794
464 45.3931374195749
465 45.7709576758344
466 46.1221735950277
467 47.4716255941911
468 47.2974137851974
469 47.1112600055422
470 47.4142510456772
471 47.2427811844312
472 47.0245091353465
473 46.5183397702634
474 46.8456382610305
475 46.826403548245
476 46.5537609466601
477 46.2282177618442
478 46.6088984450707
479 46.6513446833998
480 46.640857152756
481 46.4139865109697
482 46.8374759587523
483 47.1250603420386
484 46.9087415516662
485 47.0392326651147
486 47.8018639867567
487 48.0904042026978
488 48.330930496243
489 48.5077590874226
490 48.5768892489007
491 48.9062131264523
492 48.9634334450207
493 49.2738062848749
494 49.0400945637419
495 49.3636264736951
496 49.4415243375988
497 49.8892152204428
498 50.103373835742
499 50.6779557627775
500 51.0766162800003
501 51.4335390090307
502 51.5899487636416
503 51.8207289928215
504 51.8222979557919
505 51.2445493344823
506 51.0212220889477
507 50.4877139141429
508 50.3015782714306
509 49.8475894293108
510 49.6587864673087
511 49.7502992869787
512 49.6076994225368
513 49.0054541547161
514 49.6979662923271
515 49.9575623293676
516 50.0503655787713
517 50.6146888791518
518 50.7046394748901
519 51.5318822832455
520 51.4260360091536
521 51.2309033332326
522 51.0511268148227
523 51.0606310862209
524 50.7227424722548
525 50.9458512592594
526 51.0647549917155
527 50.4470576359911
528 49.9803236046058
529 48.9794985030747
530 48.8663160768458
531 47.7822469626996
532 48.1772570024409
533 48.4489881131045
534 47.8850831545193
535 47.8410750278457
536 47.085642348423
537 46.9536319319555
538 47.3884926623323
539 47.74936155708
540 48.2395225654887
541 49.7611680600369
542 50.3090961996571
543 50.5316343737691
544 50.9742482810973
545 51.3892212475929
546 52.0348707997853
547 51.7111069852042
548 51.703471846123
549 51.377721262924
550 50.7868344869649
551 50.3650616328029
552 49.9998825347088
553 49.6079135328318
554 49.8116234000656
555 49.0622622903173
556 48.7390405977644
557 48.6638993968968
558 49.0419769340164
559 48.9407642370586
560 48.8478181141241
561 48.4785484656815
562 46.9617468853629
563 47.008090703726
564 46.9571027910318
565 46.8327384950142
566 45.8364483245932
567 46.0090713655246
568 45.197279780924
569 44.2791006491414
570 44.0750893654089
571 44.1178016231157
572 44.4163512383847
573 43.769838690253
574 42.9628702308616
575 42.9643276932781
576 43.4077287301944
577 43.6036552499078
578 43.5664761822191
579 44.6023777320833
580 45.3953792133213
581 45.0349153022085
582 45.6335594934783
583 46.2940296376042
584 47.1311476593914
585 47.3333945498517
586 48.1557105109107
587 48.7385669108902
588 48.9802183000161
589 48.6771442434125
590 47.9490534668899
591 48.4001342278222
592 48.4696762243865
593 48.2810924111453
594 48.0768061808576
595 48.2843090831619
596 48.0138264800977
597 46.4845198648258
598 45.9866931508341
599 46.1003178106196
600 46.5693274205211
601 46.3736156465658
602 46.4711256075686
603 46.4719923019313
604 46.0516187289392
605 45.9681450992793
606 45.6239511177079
607 47.0261916499123
608 47.6544555513384
609 48.2271648462794
610 48.262750205814
611 48.683017869321
612 48.4214199352499
613 47.7631678689072
614 48.1341823516703
615 48.0611605133146
616 48.6224241746974
617 48.2694277534423
618 48.2786933440032
619 47.9499786732085
620 47.9177897185176
621 47.5543093080631
622 47.6694389185329
623 48.657685779819
624 48.3567251027995
625 48.2414441363439
626 48.1487008547236
627 48.4583861917049
628 47.8164001382455
629 47.6229629981115
630 47.3970638034928
631 47.0325271652152
632 47.4281289914363
633 47.1416689707401
634 47.2116666905739
635 46.7357700546916
636 46.6412911394397
637 46.279089608626
638 46.5542170338224
639 46.5283461455492
640 46.624557464546
641 47.0459926085476
642 46.8993436889418
643 46.8382484192801
644 46.6359879369963
645 47.3171937237148
646 47.4346044423107
647 47.6703567157574
648 48.3381397963977
649 48.9887760204633
650 49.1788398917229
651 49.6218418132339
652 48.8827682782023
653 48.9004665731869
654 49.0455822391699
655 48.8525042318768
656 48.5622143889307
657 48.4061297705904
658 47.7253632519002
659 46.2364531903277
660 46.5516751932349
661 46.5881892753845
662 48.0502209355172
663 47.2983632055584
664 47.3888297751973
665 47.6421772998323
666 47.7284910390011
667 47.3574066190296
668 47.3282596917439
669 48.1522707490719
670 46.9905613214057
671 46.4131957466707
672 45.0030717094475
673 45.2951047818176
674 45.0920039901738
675 44.5655871091398
676 44.0815844365314
677 44.4202451730926
678 44.4621615903843
679 44.3809623058062
680 44.791882836662
681 44.6914744299712
682 44.2697243844193
683 44.0786004777719
684 43.2216262735494
685 43.4913700076846
686 43.6480769867796
687 43.2364477463488
688 42.1413099929503
689 40.792582931895
690 40.0182106242978
691 39.1072452033486
692 38.7980767246112
693 38.1376561124359
694 38.6312136506114
695 38.2830389452607
696 37.8019271937408
697 37.6795059192815
698 38.5705913217729
699 39.7828233661443
700 40.4687353714213
701 41.5978458408457
702 42.43996031166
703 43.7712491236589
704 43.8180286555943
705 44.3335905948688
706 44.7422252986779
707 44.5734495635519
708 44.9391429299847
709 44.7084365901049
710 45.0567586146574
711 44.48757319685
712 44.4587594686499
713 44.3253710339866
714 44.4714202639725
715 43.9917449460366
716 44.663634600114
717 44.8593780931397
718 44.3888935720278
719 44.8302555694878
720 44.5063553337108
721 44.8590672486714
722 45.621613391021
723 45.1155762068168
724 45.5147743177541
725 45.6167162403986
726 45.603331347525
727 45.5645365592269
728 46.3958669471537
729 46.2423152775751
730 45.709877671018
731 45.5659719153885
732 44.9845080851853
733 45.4237153030304
734 45.9836065551751
735 46.7197888603225
736 45.6708936133627
737 47.0100449989073
738 46.9954133027793
739 47.1644567856328
740 47.931519715076
741 48.3113582234296
742 48.5052887051388
743 48.8259156871359
744 48.3846742735801
745 47.9670673636774
746 48.6909375030061
747 48.0813083104202
748 48.2546844833522
749 48.5279276266135
750 48.684591979209
751 48.4450018670894
752 48.9908588692264
753 48.6531301233655
754 48.5170532998259
755 48.8317889763554
756 49.7080515178839
757 49.4823542677236
758 48.9054633349029
759 48.0875925223999
760 48.1807936613383
761 47.9450303444844
762 46.9981625487406
763 46.1156213070869
764 45.7880222387371
765 45.3923652961848
766 44.3783295722771
767 43.8253353308668
768 43.504355986662
769 43.7887109248286
770 42.9765210346286
771 42.9247194778881
772 43.3652705638698
773 43.5586466671617
774 43.4681384766194
775 43.1458315837271
776 42.2457895822149
777 42.6850976317651
778 42.0589323120598
779 40.9990850708417
780 40.5878004766966
781 40.9134130157888
782 40.9879433253101
783 41.2939497526661
784 41.8061760216037
785 41.0816432277035
786 42.3598524472927
787 42.5395234174506
788 43.3148206346329
789 45.1285879837888
790 46.2641928748979
791 45.9596342802693
792 46.2392686471074
793 46.8539415668732
794 46.8232795295416
795 47.591372115452
796 47.2492660461887
797 47.5078266308575
798 47.5910341552194
799 46.7487969482209
800 46.5002762814591
801 46.603888210686
802 46.1836758649247
803 45.1348845308106
804 44.487514047441
805 45.0139968556167
806 44.9198007985448
807 44.5193256088697
808 44.7654936914169
809 44.8392345164406
810 44.4684007158075
811 44.9846054584573
812 45.2670934925341
813 46.2406954631501
814 46.4708864106928
815 46.3250728967002
816 46.6389176845705
817 47.0797212037383
818 46.5937880134783
819 47.4128400199089
820 47.4770381670824
821 46.9808973990829
822 46.4562655158769
823 46.5933309980309
824 46.5272862054344
825 46.4981102477236
826 46.2386517163158
827 45.5816549675798
828 45.9535434459819
829 45.6670754773794
830 45.7490882712503
831 46.0117654536613
832 45.9768053569417
833 45.4486499987537
834 45.9858539111414
835 44.5895823674027
836 44.7752665317945
837 44.9618237623366
838 44.7847021818834
839 44.4400506750786
840 43.9950026122424
841 43.4520589997732
842 43.8093723892507
843 44.0009701618919
844 43.826713278751
845 45.0548777194582
846 44.215261430892
847 43.9038687936148
848 43.8834804571479
849 43.6248594436262
850 43.9804326029216
851 44.1430156259727
852 43.5682826983858
853 43.9246149261368
854 42.8544610065839
855 42.3277895006383
856 42.2773190283204
857 42.1400868360189
858 42.4604161144835
859 43.208540182971
860 43.635412306129
861 43.7746493894138
862 43.8208236430491
863 43.9810742258943
864 44.7015814741585
865 44.7479146185169
866 45.1585375698756
867 45.9387207582966
868 45.595909909388
869 45.1366211266199
870 44.8033030593738
871 44.7698540762167
872 45.3100085576073
873 44.3647235435109
874 44.3739149496239
875 44.7594010106205
876 45.186643054877
877 44.1462950317803
878 43.6944660802486
879 43.7075818586917
880 44.1658749829423
881 44.106979565064
882 43.8890893201014
883 44.2233726524588
884 44.3573723921703
885 43.1622674867468
886 43.2681009806764
887 44.3137236889679
888 44.5424370843084
889 44.0746527096718
890 43.4162919468372
891 43.462005140937
892 43.0832620337707
893 42.8766270971838
894 42.7309625525199
895 43.2814248074713
896 43.0248709890813
897 41.6504409415195
898 41.3226871013667
899 41.7304734727557
900 41.3245990420479
901 40.630962753756
902 40.222162908863
903 40.0321845028047
904 39.5942660710866
905 39.9618674919411
906 39.8439755813256
907 40.0485218461367
908 39.9672871302212
909 39.467998984745
910 40.2779166253369
911 40.8342042766318
912 40.8483651975795
913 41.1105472092562
914 41.4891997912303
915 41.5133854935652
916 41.2737176166901
917 41.2043426767665
918 41.9313813441439
919 42.137917615871
920 41.6411891764067
921 40.943348606124
922 41.437225876684
923 40.7410767267032
924 41.0143830387526
925 40.8461830565817
926 40.7431515500847
927 40.9960990180834
928 40.098489900373
929 39.8862121402204
930 39.7245472331559
931 40.5704286004063
932 40.6463569165324
933 40.9932124656499
934 40.7087351708609
935 40.8572420277253
936 40.7576531246514
937 41.1926252115306
938 41.5170800760381
939 41.939391192461
940 42.1824909310364
941 41.9831611857608
942 42.0109487512826
943 42.6202286437091
944 42.0399415729662
945 42.0202325507558
946 42.000262342928
947 42.5651900361307
948 42.6086070866804
949 42.2365528650141
950 42.2344509788244
951 42.1498386844783
952 42.250224139387
953 41.2704971123182
954 41.7350299404535
955 42.406975164203
956 42.6354713646784
957 42.5162102554956
958 42.4877290585933
959 42.8171187414926
960 42.7443481410616
961 43.0378783460386
962 42.9757993656303
963 43.711100518021
964 43.8734950475479
965 42.5666521598521
966 42.3768676751231
967 42.277620913873
968 42.3089767577302
969 42.0946014545111
970 42.3855856901487
971 42.1995248500407
972 42.0907622321876
973 41.6785535564894
974 41.623391381394
975 42.2089669589486
976 42.470862225097
977 42.6063471754555
978 42.7203033817342
979 42.0759824815397
980 41.9555159800949
981 41.869662328913
982 41.8028155769165
983 42.2560204095154
984 42.6425412697232
985 42.4999090643466
986 42.2726810816187
987 41.6752500769013
988 41.7141546957355
989 41.8865769225266
990 41.5711461155289
991 41.9076551030193
992 42.2428080542866
993 41.367415198458
994 40.5992929854259
995 41.2602656622277
996 41.3560547291754
997 41.1997545408781
998 41.177066956561
999 42.2852316305599
1000 42.5975336204267
1001 42.2051220946409
1002 42.2656798566099
1003 42.4586578340049
1004 42.9486096310591
1005 42.5371611187273
1006 43.3037772710424
1007 43.5526967558533
1008 43.5331258482485
1009 42.2674956129261
1010 42.149321615727
1011 42.2560113478666
1012 41.8815367168178
1013 42.0615335322454
1014 41.7470222036943
1015 41.7585800645703
1016 40.9894694968965
1017 40.6600117273593
1018 40.8096359532881
1019 41.3774581494445
1020 41.4436538067159
1021 41.5174670087069
1022 41.3522778751162
1023 41.1145508963843
1024 41.7211654823725
1025 41.5513763575568
1026 41.854260651234
1027 41.8589156159064
1028 41.8736199138556
1029 42.2432315116115
1030 42.854252000527
1031 43.5959740999462
1032 44.1387489137507
1033 45.5902633240672
1034 45.5150260216789
1035 45.7811529384973
1036 46.0213176833109
1037 46.9717383154999
1038 47.4352053922191
1039 47.4788163193166
1040 46.4731975600111
1041 46.5185148801152
1042 46.5219101173009
1043 46.7079103467863
1044 46.8761625450632
1045 47.5058283746977
1046 47.1042096060019
1047 46.8164000326066
1048 46.788760418853
1049 47.1364736003436
1050 48.1312194985279
1051 48.0010138002386
1052 48.5969878835989
1053 48.1160555043203
1054 47.9376850414586
1055 47.3547121817792
1056 47.6125361427413
1057 47.724422149961
1058 47.8988755109504
1059 47.1801600269966
1060 47.2690091142087
1061 46.6642705033129
1062 45.9476310893934
1063 45.1901457396973
1064 45.3950093747169
1065 45.6111138932276
1066 45.0505666136057
1067 44.8099550875726
1068 44.4932959574372
1069 44.7281208614998
1070 44.1351841361363
1071 44.0339107433156
1072 43.797578499781
1073 43.7753106014284
1074 43.5280759743831
1075 43.8269781666007
1076 43.5363405627151
1077 43.2382041428331
1078 42.2064774940239
1079 41.5990088100921
1080 40.7316733878632
1081 40.5559035210554
1082 41.2261425334258
1083 41.8934208540573
1084 42.1231370912046
1085 41.1423913814537
1086 41.5069985691586
1087 41.3919199416202
1088 41.7945695937839
1089 42.4990832893712
1090 43.2755048835505
1091 43.4442286586618
1092 43.6192657045477
1093 43.3968345667795
1094 42.4645358840979
1095 42.3279167494421
1096 42.5786104420937
1097 42.836007409457
1098 42.6425691376882
1099 41.6371941308602
1100 41.8225058038145
1101 41.3400244766588
1102 40.7595008004573
1103 40.0127560170088
1104 40.5658395640157
1105 40.2312202768948
1106 39.4570504340847
1107 38.9125711645284
1108 39.2683645363681
1109 40.0116741802386
1110 39.1243670498296
1111 39.1107234851716
1112 38.7475565416152
1113 39.0671981153966
1114 38.6824106107921
1115 39.0806209423888
1116 39.2089880745611
1117 39.1995138832751
1118 38.5583821000108
1119 38.399332982348
1120 38.6054305546265
1121 38.7070925002264
1122 38.3764899485044
1123 38.6945161570872
1124 38.7027771977753
1125 38.608189890962
1126 39.3680641868158
1127 39.4019005161164
1128 39.8494067294727
1129 40.4317811504555
1130 40.9276424060904
1131 41.2837162980347
1132 41.458907024895
1133 40.8384091560551
1134 40.6612069713821
1135 41.254239605549
1136 41.1867293129643
1137 41.0330888424516
1138 41.4608887644178
1139 41.469133052665
1140 41.9431876010838
1141 42.5435493050444
1142 42.8474713680343
1143 43.3509164204961
1144 43.579813188171
1145 42.7529587710599
1146 42.7382344636058
1147 42.6772812189346
1148 42.6407089381086
1149 42.5899503785838
1150 42.446894605814
1151 42.150036618518
1152 42.8841341488115
1153 43.2697742733654
1154 43.9247956250754
1155 45.2053580986562
1156 45.1070548869539
1157 45.7616258052094
1158 46.3351502151166
1159 46.5209231672804
1160 46.4893785802631
1161 46.0701452742388
1162 45.1663694985898
1163 45.5040299278132
1164 45.6625248961194
1165 45.2816064351625
1166 45.6281084008517
1167 46.1932827702727
1168 45.2495068255503
1169 44.6275185224184
1170 44.8333830795896
1171 44.8376277687258
1172 44.9547989574057
1173 44.0404779261159
1174 43.3324705917707
1175 43.4426865327107
1176 43.1706154839201
1177 42.7957811160132
1178 43.2031201696599
1179 43.3851619517176
1180 43.0040063687225
1181 43.3253879867867
1182 43.2757407662299
1183 43.7171400121746
1184 43.7886800022327
1185 44.4771027741246
1186 44.5586122243553
1187 43.5160969136238
1188 42.7292243905395
1189 42.3368041409681
1190 42.9139107182987
1191 42.799607630463
1192 42.9618600343738
1193 43.1061390796085
1194 43.4421152912236
1195 42.404525396938
1196 41.8353492423858
1197 43.0522368300187
1198 43.6974669962752
1199 43.504105536666
1200 41.8311515420243
1201 41.2604287483475
1202 40.8347936100663
1203 41.0889763600327
1204 40.8364497891544
1205 41.6413288755065
1206 42.4275569825886
1207 42.01600570505
1208 42.354584708132
1209 42.7714702533839
1210 43.3755372829946
1211 43.8905245169932
1212 45.0228120971039
1213 44.2231815149454
1214 43.9613300909491
1215 43.1585567922502
1216 42.848498514011
1217 42.4778223119755
1218 42.1120020453709
1219 42.5290116063683
1220 43.033155278597
1221 43.1304531193175
1222 41.6443307031493
1223 41.6527745069234
1224 42.0444448799841
1225 42.0146588639987
1226 41.4620088344707
1227 42.1632184782971
1228 42.5858470294928
1229 42.3745054697074
1230 42.1401940883854
1231 42.7191197546144
1232 44.0639786742981
1233 44.3522190959255
1234 44.0361089917536
1235 43.8626060735289
1236 44.57222305337
1237 43.9936785906536
1238 43.4190105504605
1239 43.1572849044714
1240 43.0673234766916
1241 43.3497903006295
1242 43.239891541456
1243 43.1857200765408
1244 43.2980963578089
1245 43.6981920062761
1246 44.2692099280659
1247 44.1376007504639
1248 44.6525058205558
1249 45.0404859714408
1250 45.3801491194394
1251 44.8774365688509
1252 44.6653655461434
1253 44.3476300540136
1254 44.4799357532498
1255 43.9341904423031
1256 42.6546306895027
1257 42.7808535869902
1258 42.6004973026121
1259 42.066477420205
1260 41.6205206080285
1261 41.4674953813404
1262 41.3735939003265
1263 42.0359598038767
1264 41.8716845118922
1265 42.6350926476894
1266 43.458051756161
1267 43.601085740027
1268 43.5514972432221
1269 44.0924924531717
1270 44.1530025387478
1271 43.3075473167015
1272 43.5032787742113
1273 43.3475647729549
1274 43.302354513152
1275 42.8642064545172
1276 41.6495134965797
1277 41.8423136227206
1278 41.1476571226935
1279 40.9910605926128
1280 41.8155168119941
1281 42.778519750821
1282 42.5262480178247
1283 42.7753822027625
1284 42.4258130625463
1285 42.7562166791547
1286 43.9942012336253
1287 44.3204153941491
1288 44.3392071967522
1289 44.3828705622781
1290 43.6819399248181
1291 43.4758076704157
1292 43.0152499094992
1293 42.2302412933427
1294 43.0066650161987
1295 42.4811378675882
1296 42.1039153282604
1297 41.5120641357744
1298 41.7835256477067
1299 41.5310884926724
1300 41.106377274949
1301 41.6270781457295
1302 42.2833326920928
1303 42.3109365243633
1304 42.1754531346169
1305 42.6984593407417
1306 42.5441551398865
1307 42.9956248025693
1308 43.8743649288014
1309 43.1309030290656
1310 44.0905650355651
1311 43.4654774035205
1312 43.3541583480544
1313 43.8112634158037
1314 44.4064157929987
1315 43.8313543357536
1316 44.2544012971916
1317 44.3930161222787
1318 43.7072449602117
1319 43.6930407843751
1320 43.4247561075619
1321 43.3983245772573
1322 43.9598000573362
1323 43.8935150309091
1324 43.0364903293879
1325 42.7380585903392
1326 42.5226888516991
1327 41.7206414825397
1328 41.2472076257887
1329 41.7490331298505
1330 41.0918905763104
1331 40.7079631653489
1332 40.0384515639569
1333 40.1760581676918
1334 40.3128291626931
1335 40.495055491799
1336 39.9825982755037
1337 40.0251613506721
1338 39.465172602658
1339 39.378032930047
1340 39.5118724921432
1341 39.7083094856476
1342 40.4023521909609
1343 40.1301075067554
1344 40.1782526704983
1345 40.5353671091842
1346 40.8622630208285
1347 40.5255840610131
1348 41.1052603925178
1349 41.6666622631302
1350 41.5197412827793
1351 41.4043808931247
1352 40.3155631569624
1353 40.0497411348996
1354 39.2695897982258
1355 38.6970542579282
1356 38.2551455871533
1357 39.1748228570159
1358 39.1253600999006
1359 39.4177127218968
1360 39.9862986577318
1361 40.5156505404226
1362 41.0788489787845
1363 40.8732916778331
1364 41.1952244350567
1365 41.034292693285
1366 41.5728518055289
1367 41.332361883039
1368 42.1408720402862
1369 41.4077343648425
1370 41.6658726662859
1371 41.1323173615628
1372 40.928274113601
1373 41.450285492203
1374 42.0895131493777
1375 42.3152997114668
1376 42.3454321684601
1377 41.961114991871
1378 41.7629101802066
1379 41.3371709491976
1380 40.1409723523234
1381 39.4585367422907
1382 39.3226042023291
1383 39.4655350437278
1384 38.6765356861666
1385 38.5978797523772
1386 38.5862882564401
1387 39.6683182602959
1388 39.555380508132
1389 40.4286982949223
1390 41.7800828480519
1391 43.2364393746252
1392 43.1627941599835
1393 43.2073691747932
1394 43.5422703721269
1395 44.4335916588686
1396 44.163734401786
1397 43.4772174134232
1398 43.8684835658025
1399 43.7945943297825
1400 43.0013284333908
1401 42.633678336451
1402 43.108476461022
1403 42.9563937538248
1404 43.7916848403805
1405 43.9469492861715
1406 43.5651468780999
1407 44.0793146046883
1408 43.6846990854676
1409 43.5353523771875
1410 44.2932956450176
1411 44.6430263458432
1412 44.7119866528732
1413 45.0484026238416
1414 44.3056368915743
1415 44.220568797494
1416 45.1152267673629
1417 44.367252883226
1418 44.7843089837908
1419 44.5441078678991
1420 44.79721577049
1421 44.4737377675285
1422 44.755523839449
1423 44.6169863850345
1424 45.129892534112
1425 44.8618813791519
1426 45.1132709262116
1427 45.0361289567659
1428 44.3656905330263
1429 44.6505480587789
1430 43.9727247651313
1431 43.9794990501803
1432 43.7818041597101
1433 44.1007086357139
1434 43.5186557587861
1435 43.829983668279
1436 44.293554174985
1437 44.2598017074891
1438 44.6598578848693
1439 44.2274648191841
1440 44.6552759150553
1441 44.3763750938902
1442 43.8622176845982
1443 43.3607444524667
1444 43.1410872433648
1445 42.2311795103669
1446 41.5563714763883
1447 41.2457706751017
1448 41.0472882404119
1449 41.2860719881856
1450 41.075357363703
1451 40.8730932594739
1452 40.4158106786827
1453 40.1122901128199
1454 40.2085359917111
1455 41.2552846439039
1456 41.529455154042
1457 43.0821627058787
1458 43.4428998905861
1459 43.1991256697749
1460 43.0041539166116
1461 44.1664352560403
1462 44.4224713232435
1463 44.8141618699454
1464 45.6415469116083
1465 45.0152310262991
1466 44.0325741634132
1467 43.2849519896975
1468 42.8904992734747
1469 43.2846876047587
1470 42.7782721684931
1471 42.1114373970086
1472 42.0582511029306
1473 42.2505703566183
1474 41.8039344411201
1475 42.6414173379858
1476 42.7685839702559
1477 42.8192598503619
1478 43.1772266229798
1479 43.9874385928044
1480 43.9309432747317
1481 43.3470896769767
1482 43.6795371737436
1483 43.0847246355611
1484 43.1618904668927
1485 42.5305627822354
1486 42.7404386012955
1487 42.0802067367702
1488 42.2747193514365
1489 42.0183353073
1490 42.1790297508407
1491 42.2354160100891
1492 41.878301761875
1493 41.82417375763
1494 41.5066614824568
1495 42.3157019545619
1496 43.0291102223558
1497 43.6770625298586
1498 43.351013155524
1499 42.4832176812451
1500 42.7681663321015
1501 43.074158078894
1502 43.2858750234579
1503 43.6144229057328
1504 44.2902609339975
1505 43.953149475969
1506 44.1591568656025
1507 44.0150917026619
1508 43.7958595865033
1509 43.5902955159887
1510 42.860878134619
1511 42.9098432713152
1512 43.2459640937074
1513 42.817141727861
1514 41.3724738805658
1515 41.3588236086874
1516 40.2755062128492
1517 40.3562096547631
1518 40.5137756052995
1519 40.6514800786308
1520 40.822591142399
1521 40.7406011819145
1522 41.0040273118133
1523 41.0388497398705
1524 41.1818336881182
1525 40.7194740098576
1526 41.3613100072557
1527 41.3883389441601
1528 41.6881295364437
1529 41.9450317653282
1530 41.4989472717365
1531 41.4739829953935
1532 40.2322308726076
1533 40.1503192904295
1534 40.6975851003668
1535 40.1722277940085
1536 39.869696442314
1537 40.1360961446045
1538 39.5248174527714
1539 39.6291391685811
1540 40.0033821570686
1541 40.1926057248082
1542 40.3155528578533
1543 40.6916487568951
1544 40.0605018430056
1545 40.8427539096687
1546 40.0323334044974
1547 39.4627432521085
1548 39.1285522841096
1549 38.9542449873351
1550 39.9065574059238
1551 39.9022724331713
1552 40.3690025526687
1553 40.0994057696101
1554 40.6406517592401
1555 40.2407547630531
1556 40.9300014391233
1557 40.8937039570573
1558 41.5675705543381
1559 41.6870841718278
1560 41.6565681727414
1561 41.4041151840368
1562 41.3226934976154
1563 41.9974160769937
1564 41.7376622246851
1565 41.5287462683514
1566 41.9700894442754
1567 42.2168941641485
1568 42.0306730087691
1569 41.6796636598097
1570 41.1112798128982
1571 41.6094468514629
1572 41.5264543049059
1573 41.0526324261316
1574 41.1755925865559
1575 41.5917269738474
1576 40.8068462516244
1577 41.1884684949936
1578 41.9726303696623
1579 42.7440597898568
1580 42.2937023330989
1581 41.1195889353508
1582 41.6619895559893
1583 41.1008892513795
1584 41.9056558749939
1585 41.2968473253115
1586 41.6892954235491
1587 41.7717202733795
1588 40.9152182646322
1589 40.4075513126512
1590 41.5600665514902
1591 41.7865893872455
1592 41.2302947438254
1593 42.174363019266
1594 41.0918482227013
1595 42.2743992323729
1596 42.050739397385
1597 41.8386013857269
1598 41.4271858375935
1599 41.4176132897825
1600 40.3821497937066
1601 40.6079652885302
1602 40.4648514327082
1603 40.8965262814838
1604 40.67471127015
1605 39.3849254007053
1606 39.2365332357076
1607 38.8426842155283
1608 38.8782592862771
1609 39.2459117327214
1610 40.2556605764399
1611 40.4961041343592
1612 41.3982935190323
1613 40.8386281360285
1614 41.56487591159
1615 42.3083099136011
1616 42.8993256664915
1617 42.4525327878023
1618 42.7846391200498
1619 42.7081643494373
1620 42.568011557775
1621 41.6373432853578
1622 41.3825579379611
1623 40.7170579276475
1624 40.2964662915119
1625 39.9873074733947
1626 39.7299100217147
1627 40.7039819615829
1628 41.0766270501975
1629 40.8147196667924
1630 39.5533947468296
1631 40.8588276277183
1632 40.8321326251895
1633 40.8272833521511
1634 41.0194398631651
1635 41.0690991958876
1636 40.2169593865242
1637 39.6997383979923
1638 39.377232568793
1639 39.4357972084301
1640 40.429307614928
1641 40.2971849967738
1642 40.1317224674878
1643 40.3127405297619
1644 40.4655049846186
1645 40.6785557139325
1646 41.0255233185386
1647 41.5306270223267
1648 41.0925533664396
1649 40.9522341800935
1650 41.1996445161881
1651 41.2123670666012
1652 40.6773304386129
1653 40.5410801031645
1654 40.3459696601285
1655 40.1969495272873
1656 40.7570111232309
1657 40.4769260273989
1658 41.6671140602455
1659 42.6313893036885
1660 42.5096402636515
1661 41.3947629657174
1662 42.060741629856
1663 42.9060739162816
1664 42.9129706195239
1665 42.5257788036298
1666 42.6626925986124
1667 42.7701834314873
1668 42.5727550593255
1669 41.5397898811085
1670 41.4954167793902
1671 42.4980534276211
1672 41.5038255650025
1673 40.7288243121959
1674 41.180572751559
1675 40.9892411736544
1676 40.8126313874251
1677 40.682819470683
1678 41.3050976550583
1679 41.0658801190131
1680 40.7475960471871
1681 40.4588580582746
1682 40.8561022288194
1683 41.3574975039827
1684 40.939305785709
1685 41.561371662524
1686 41.7388735875006
1687 42.3544411054414
1688 41.3184125978684
1689 41.1712147081849
1690 41.1386552293322
1691 41.1496199981013
1692 41.2794414390177
1693 41.5874077984407
1694 41.882227856368
1695 41.6203757389585
1696 41.6366336791783
1697 41.2752615716471
1698 42.1152365892607
1699 42.8439362886229
1700 43.5973237973855
1701 43.7016030349222
1702 43.9931969863619
1703 44.0091369412328
1704 43.3649219093663
1705 43.8198953407803
1706 43.4477211313688
1707 44.1613152388322
1708 43.2866878659418
1709 43.4021224940464
1710 42.5022985329042
1711 43.2480092979311
1712 43.9262155449855
1713 43.2889633200165
1714 44.119883809068
1715 43.8925774872951
1716 44.3444143943601
1717 43.5895527377658
1718 43.8843972520746
1719 43.7428686923175
1720 44.1487147910433
1721 43.7301338273105
1722 42.1448100466694
1723 42.0111602987841
1724 41.5177985167714
1725 41.6837081088442
1726 41.7332646092948
1727 42.3495398584832
1728 41.5264469274503
1729 41.9369901304463
1730 41.9434125492666
1731 42.0901327743162
1732 42.3944776129737
1733 42.653127633151
1734 43.2101721233217
1735 42.0651779520785
1736 41.6589348192783
1737 40.5031750166621
1738 41.7109359687505
1739 41.0655653607637
1740 40.8750599078668
1741 39.8553876510669
1742 40.4886503810059
1743 40.6604131729828
1744 40.2747317834006
1745 40.9306343105363
1746 41.3327487041987
1747 42.0742095611982
1748 41.7834039919866
1749 42.4005408590753
1750 42.4496289515812
1751 42.9143901600543
1752 42.6927626932214
1753 42.6246363497047
1754 42.8996278347718
1755 42.7606747276249
1756 42.1774333673524
1757 41.4939152285987
1758 41.3646759429745
1759 40.9470934328228
1760 40.9821805489123
1761 41.6727678269232
1762 41.7080244682408
1763 41.8355441413213
1764 41.9180740369432
1765 42.0030042218127
1766 42.1162377389397
1767 42.0477703360847
1768 41.3570952417666
1769 40.912436567544
1770 40.8655123338711
1771 40.2715569799222
1772 39.8685287517725
1773 39.1786953808343
1774 38.8981557516293
1775 39.7518752384263
1776 40.2972012739376
1777 40.1265876868311
1778 40.3170701970765
1779 40.5534780156477
1780 40.318854621525
1781 40.0844571297116
1782 40.5029063403015
1783 40.5274323534287
1784 40.6779970834598
1785 39.6563312977102
1786 39.0379707802524
1787 39.6721946010108
1788 40.1547871623295
1789 40.2582793744929
1790 40.7115720927056
1791 41.310064809998
1792 41.1348818543637
1793 41.4027881818186
1794 41.1143678937019
1795 41.4961306066105
1796 41.7849665434385
1797 41.2731379862629
1798 41.3955614587181
1799 41.3860762271106
1800 40.7882685804446
1801 40.3245200052371
1802 40.7977014875215
1803 40.4083042156517
1804 40.5267258666813
1805 40.4942478963496
1806 40.4956833082298
1807 40.2418433166376
1808 39.8798764245627
1809 40.9799845642659
1810 41.7744356313365
1811 41.5298599844864
1812 41.1179086303889
1813 41.3926977530967
1814 41.6490064238992
1815 41.1213852430514
1816 41.1004105876313
1817 41.2069154327928
1818 41.7841575783267
1819 41.5416769546236
1820 41.4206533745182
1821 41.5573760031554
1822 41.8738441760751
1823 41.9824374874844
1824 42.415776743276
1825 43.4064880720936
1826 43.6197626415975
1827 43.7099006278355
1828 43.7308640168339
1829 43.2093953633823
1830 42.678488703039
1831 42.4975741570103
1832 42.4287300327967
1833 42.8668176591978
1834 42.348599932945
1835 41.8150257309454
1836 41.7882863363109
1837 42.1715190717376
1838 41.5280894474203
1839 41.1727111672909
1840 41.3498744419528
1841 41.5449471263944
1842 41.5324743021106
1843 41.7763948299168
1844 42.3760547048416
1845 42.9886138979988
1846 43.4322246129447
1847 43.3983607864427
1848 43.1772440149944
1849 43.5449508723911
1850 43.256183679754
1851 43.8225811148471
1852 43.2586187422037
1853 42.6935139095172
1854 42.5072899715483
1855 41.5389077304955
1856 40.4166768397671
1857 40.4555059753704
1858 41.4241833443088
1859 41.6322571465758
1860 41.6680510324505
1861 41.3988246225335
1862 42.2678993191028
1863 42.5782151140163
1864 42.3967479551842
1865 42.7247409702401
1866 43.3409084393117
};
\addplot [semithick, crimson2143940]
table {%
0 24.2350763123709
1 22.1079784547144
2 20.9723316971481
3 17.4708881478724
4 17.1724134094192
5 18.0833529105139
6 18.4209045252299
7 17.890212545858
8 18.2406077096833
9 19.4210126178337
10 20.2992388312068
11 21.891164724369
12 23.2571121819894
13 24.7188413837996
14 25.9197361845998
15 26.468508387435
16 26.8247986824767
17 27.3920538790458
18 28.0198604841048
19 28.3663109483974
20 29.1264514350439
21 29.1394180649562
22 29.6327414181217
23 30.0099762036159
24 30.184255175069
25 30.5296536079502
26 31.2048018932673
27 31.7919845243403
28 32.15861102834
29 32.6020723658115
30 32.5771670859473
31 33.022544454863
32 33.1020112281363
33 33.3104565284894
34 33.6797930959823
35 33.934971945321
36 33.7232365124507
37 33.6088474270817
38 33.5680552629904
39 33.2694421555834
40 33.6298446563176
41 33.7680588374993
42 33.8039537817649
43 33.84020966607
44 33.7410000917392
45 33.8212442361108
46 34.0273683477554
47 34.4030691275048
48 34.5346038166191
49 34.742479368511
50 34.7197632599503
51 34.6940066715168
52 34.75167284094
53 34.8209384610603
54 34.9305217206212
55 34.6716675177666
56 34.4826948535387
57 34.1317669973246
58 33.9066495674633
59 33.6805314919664
60 33.2871450712655
61 33.3631972590292
62 33.1676497272273
63 33.2161499998806
64 33.0145074215704
65 33.4554974229896
66 33.1822349140623
67 33.782653965219
68 33.8726550652097
69 33.7880839774532
70 34.3502436690013
71 34.3364884342403
72 34.595191572265
73 34.6742948807312
74 34.8312161953726
75 34.61416545649
76 34.5882408848182
77 34.198727258075
78 33.8839300281517
79 33.9761735580691
80 33.7201695332919
81 33.3446029555458
82 33.1563496411162
83 32.9849462188149
84 32.7566993813411
85 32.5879239653968
86 32.899052707779
87 32.8252099322056
88 33.2954600255537
89 33.4795178907253
90 33.5819210820843
91 33.9316079153213
92 33.9800493845569
93 33.9257558507752
94 34.0686737909076
95 34.4604209850358
96 34.2453977525448
97 34.2724951097127
98 33.7855020325029
99 33.5520948126922
100 33.3943071132914
101 32.8773282974064
102 32.7985260218835
103 32.5778748284072
104 32.3895472597439
105 31.8473403449105
106 31.8328350097862
107 31.8489930346677
108 31.9949197363141
109 31.8429120881887
110 31.2213435010924
111 31.2883801106347
112 31.1331708191794
113 31.0354392571801
114 30.9260293647825
115 30.8335351549039
116 30.5734536671144
117 30.3613212385687
118 30.3216307851973
119 29.9518724985771
120 30.4721765362415
121 30.190748537909
122 30.5383346542683
123 30.16718450757
124 30.0022112918332
125 29.8351391149798
126 29.7186825075327
127 29.5821325765241
128 29.1020988907015
129 29.5460034176778
130 29.2409340363197
131 29.5442308035284
132 29.4097374453167
133 29.8860430680862
134 30.1164391204367
135 30.304992281687
136 30.5262026963512
137 30.1824689176272
138 30.1057762744809
139 29.7501571768656
140 29.7139658971367
141 29.2655039491537
142 28.7902419533674
143 28.39502411494
144 28.2994410917481
145 27.9595543971189
146 27.7934012304599
147 27.8009799405904
148 28.0988508286696
149 28.2388482734326
150 28.076270452761
151 28.2120213084082
152 28.4730762020809
153 28.5803543951753
154 28.3502829677724
155 28.6235083882511
156 28.706669905315
157 28.9608129542265
158 28.767616275519
159 28.8040565415204
160 28.722869962864
161 28.5880061632764
162 28.5468083757189
163 28.9195635632746
164 29.1029538083409
165 29.2330389715404
166 29.2642836230262
167 29.0750634472409
168 29.5035703058098
169 29.6861394421315
170 30.0130405976954
171 30.068048001389
172 29.9344854675874
173 29.5239307911037
174 29.8152037544925
175 29.8215526038697
176 29.6843564136477
177 29.9299318753544
178 29.5681830505095
179 29.2962186896763
180 28.9597919825792
181 29.1567804033465
182 29.1828438582524
183 29.001372675433
184 28.5549257225312
185 28.3830271286497
186 28.6613737609306
187 28.5434880073751
188 28.8241612366507
189 29.1234295937502
190 29.539849278449
191 29.7751836923308
192 29.7404592382367
193 30.1634563999215
194 30.1272381396962
195 30.552679424025
196 30.2719046266098
197 30.0061473698159
198 29.9076542218809
199 29.4658579465472
200 29.2715797902468
201 29.1370535948084
202 29.2382823468692
203 29.1453085125605
204 29.2180011747135
205 28.5243536700653
206 28.8468752268372
207 29.0065340041539
208 28.6101317349525
209 28.9672620599953
210 28.9939645387484
211 28.4765692421114
212 28.0159189163584
213 27.7967026328933
214 27.7016309736601
215 27.8558154905364
216 28.0008373535006
217 28.2884196649077
218 28.741480154755
219 28.3791959893558
220 27.9909949981645
221 28.1868634264287
222 28.030348604605
223 27.7850608909771
224 27.4909691246964
225 27.3864910585718
226 27.0347080381604
227 26.6624382659643
228 26.1829041156793
229 26.4347352252592
230 26.8037807555407
231 26.5930007319567
232 27.0870513747997
233 27.3376864461789
234 27.427653566247
235 27.2977627178583
236 26.8174835365041
237 27.0937236770598
238 27.4478142184337
239 26.8929806773839
240 27.0145227771301
241 27.4673428591471
242 27.7409713233831
243 27.3724966876958
244 27.4114464647977
245 27.6309983930452
246 28.001624673146
247 27.5952478634159
248 27.6775629860655
249 27.7542179007884
250 27.7760764800089
251 27.8064131798287
252 27.5066698874611
253 27.8110197628277
254 28.0747675417716
255 27.8926994515733
256 27.4688475504799
257 27.2115521159451
258 27.0242913764223
259 26.9097887365097
260 26.4580320773351
261 26.1557187257135
262 25.8860394781468
263 26.1169904866006
264 26.2079679143193
265 26.489439992531
266 27.2565571085852
267 27.4059331625365
268 27.2713529731805
269 27.7872739400326
270 27.984816045639
271 28.0545532341531
272 28.5194831628029
273 28.3445743063756
274 28.436083793553
275 27.921352720481
276 27.3740292570377
277 27.9156144618294
278 28.311429728201
279 28.1876524437205
280 28.4396251199876
281 28.9255749276026
282 28.2806185942514
283 27.7551604237439
284 27.3465196659703
285 27.5491986827666
286 27.6145269877517
287 27.0136510454634
288 26.7772753287526
289 26.9642741511289
290 26.749090032261
291 26.5653347085746
292 27.2893475121293
293 27.189017739293
294 27.0000176671794
295 26.6880805734315
296 26.3358116284121
297 26.4701685024844
298 26.209264595702
299 25.9839659614851
300 25.9785333012274
301 25.8469891526532
302 25.3741763327109
303 25.9416860345697
304 25.7551733399827
305 26.3173981401688
306 26.7286736834186
307 26.4999986285061
308 26.818267001829
309 26.4422870088453
310 26.8006364616649
311 26.3413203201503
312 26.0993763927778
313 25.7903716163874
314 25.7225497100237
315 25.3612556399842
316 24.6526004590918
317 25.2071970854016
318 25.4556418745818
319 25.420838618026
320 25.0140311665165
321 25.1931690405002
322 24.9297495449482
323 25.0349037250421
324 24.9398979374842
325 24.8081930831249
326 24.8044200672825
327 24.0044608069888
328 23.3116576407856
329 23.8739363136395
330 23.5956964529409
331 23.2165013765074
332 23.6105331167116
333 23.6739262633711
334 24.2312998712911
335 24.5539548215964
336 25.0212965969041
337 25.3512350625363
338 25.7290061934725
339 25.1831102308306
340 25.4567630705329
341 25.6059903484601
342 25.5981874861115
343 25.3870971118229
344 25.5297581389473
345 25.0083698552633
346 25.5791110985022
347 25.4630163001745
348 25.2849610991778
349 25.9533696771164
350 25.6894733718024
351 26.0955545382044
352 26.1571021070236
353 26.4506633591358
354 26.197512321105
355 26.6988760291739
356 26.4435036377284
357 26.8701812794181
358 26.9956033213714
359 26.6673151061668
360 26.8089102368606
361 26.6866688879358
362 26.6515070154486
363 26.2180238806178
364 25.720314544404
365 25.6716870235714
366 25.2569022922973
367 25.0428310135897
368 24.8792330863366
369 24.1495788480737
370 24.0729120952319
371 24.444322829598
372 24.1447064682857
373 24.1776903258607
374 24.5118612278264
375 24.5433054035043
376 24.3872026917804
377 24.4400592116917
378 24.4668866088333
379 24.9781372208046
380 25.1966483506818
381 24.9747329756367
382 25.4072241431155
383 26.093034859661
384 26.6250227025307
385 26.6215318943284
386 26.9843409651936
387 26.4231316799637
388 26.0867719818843
389 26.2226513176597
390 25.7570404671951
391 25.3908184516969
392 25.3868003167527
393 24.9676384433746
394 24.608011449481
395 24.4846956568927
396 24.4557449313507
397 24.9664611828534
398 25.3509425475004
399 25.4163219933924
400 25.6846478736973
401 26.2133837519966
402 25.6937008920868
403 25.6114283711483
404 25.1062898208443
405 24.6107334656777
406 24.475822098703
407 24.053178763898
408 23.969218283984
409 24.0126494119559
410 23.9662334060051
411 23.6609466634328
412 24.4272815195254
413 24.8303303183397
414 25.5025696442437
415 26.2229419282296
416 26.1968173183075
417 27.0587248228783
418 27.804564185354
419 27.8007596285726
420 28.4702742861514
421 27.9877180355973
422 28.1052513538718
423 27.9546131529932
424 27.553241544822
425 27.6590115635081
426 28.0594131180872
427 28.152556768334
428 27.8117084861484
429 27.8983646580214
430 27.6662218828074
431 27.6823837866072
432 27.9720565362245
433 28.2578876557523
434 28.1514627625078
435 27.600082387365
436 27.7907741781226
437 27.0708251263247
438 26.5802114437135
439 26.4636537605665
440 26.1473518589104
441 26.1955517726944
442 25.8210195534625
443 25.6195778184615
444 25.9570585783893
445 26.3926216419636
446 26.3718242904929
447 26.9131602541122
448 26.6868309568912
449 26.5984147647489
450 26.6923256257519
451 26.7955924369117
452 26.6337570017151
453 26.0083845629829
454 25.8300959327026
455 25.2622976762353
456 24.2373720062408
457 23.2692966422575
458 24.001040935004
459 24.5694611477026
460 24.3009488119063
461 24.4466200482771
462 23.969007004698
463 24.433546317732
464 24.7188162938776
465 25.5175037094513
466 26.3361007457409
467 26.549401117027
468 25.9247346181585
469 25.0275388414464
470 25.2683723673193
471 25.1502770784988
472 25.307133031215
473 24.5641988506026
474 24.788658723976
475 24.2974394325921
476 24.1789263348083
477 24.5699995481167
478 24.6127786570835
479 24.6693876662023
480 23.8957618487453
481 24.6949423724013
482 24.508535009226
483 24.7283961302387
484 24.4012184852539
485 24.6954016416806
486 24.4854225558923
487 24.3113028105048
488 24.4628024060669
489 24.8335541799575
490 25.2198506896471
491 24.532191067001
492 24.3433633908776
493 24.6139472594657
494 24.3890937107934
495 23.622222870628
496 23.4235838967402
497 23.6505366672246
498 23.0941330317037
499 23.3977293854218
500 22.7175966973011
501 22.9148845529327
502 23.2115201491439
503 23.2033878357016
504 22.6293370485257
505 22.8267245827601
506 22.9168150268939
507 22.877196179037
508 23.0118913902707
509 22.245095784312
510 23.2200127746703
511 22.7427629923971
512 22.3916573432893
513 22.4378677082754
514 23.3469096188852
515 23.4889604104621
516 23.4376744235146
517 23.6708873344401
518 23.6842992914196
519 24.0664965517158
520 24.012580720605
521 24.4336614911746
522 24.4455402454664
523 24.1449581241088
524 23.7958959416082
525 23.8092383547446
526 23.7813879462027
527 23.3065875353464
528 23.5545076337591
529 23.3963369180632
530 23.4934546338147
531 22.7878647541266
532 22.9078809087038
533 23.5876188414109
534 23.8755778160624
535 24.2388472228227
536 24.5231317122396
537 24.8611313150023
538 25.2638808973524
539 25.580068028873
540 25.3258954418036
541 26.0359340562421
542 26.4545032549167
543 25.6200347136518
544 25.4560274225926
545 25.7125460501353
546 26.187576556602
547 26.5561179120425
548 26.4911516117997
549 26.626706026033
550 27.0069735664478
551 27.552111464063
552 27.3971514747998
553 27.750534662192
554 27.8763415058357
555 27.624609550697
556 26.7382959603933
557 25.7336031008158
558 26.2982998241169
559 26.0178152229834
560 25.2822938473746
561 24.7477636128673
562 25.0305790896001
563 25.3312290644524
564 25.9437386878447
565 25.46680167692
566 25.1505281682363
567 25.8886717953605
568 25.7653579097741
569 24.7545682055796
570 24.3641238471778
571 25.0507913996286
572 24.193115410255
573 24.5991550839959
574 23.9281696944294
575 24.0646927710162
576 24.7586655893943
577 25.0386900079907
578 24.1965062249981
579 24.9968140855972
580 25.3011223180386
581 24.5404732642999
582 25.4381722646588
583 24.6968985190183
584 24.4495195911919
585 24.8117040501244
586 24.4192930024878
587 24.1024913768882
588 23.8079568384361
589 23.3192593499019
590 23.6857192879678
591 23.4076531157189
592 22.9380485363942
593 23.2881719701156
594 23.5339048597256
595 23.1554397661927
596 23.489124103183
597 23.3085608810076
598 23.6837718765852
599 24.6886941322892
600 24.9927418317107
601 25.3808514287036
602 25.7185475789431
603 25.9586431670365
604 25.5895150717169
605 25.6470507814636
606 25.5309507303209
607 25.8316229464749
608 25.9173225336254
609 25.9634272837815
610 26.3725207507975
611 26.1435119396808
612 25.7672892123173
613 25.2459016128258
614 25.0103877288399
615 25.4484864599129
616 25.7558515483936
617 25.3895546456022
618 24.9916371333729
619 25.3491001269718
620 24.7890699033133
621 24.8692042558985
622 25.1883215158711
623 25.5963414566331
624 26.3131486530223
625 25.4092482288636
626 25.3598612175732
627 26.2163210126651
628 26.0125090369995
629 25.5854468936391
630 25.6329258720327
631 25.0801541252114
632 24.364278444627
633 24.1112864334785
634 24.1135444516834
635 24.6581235966841
636 24.0299704512942
637 23.491881235641
638 24.0105529389344
639 23.5275937656211
640 23.4498019458019
641 24.4275106682752
642 25.0321727501319
643 24.9125910956609
644 24.3569937280697
645 24.32947841006
646 24.4625947368556
647 24.0690178581489
648 24.0850676300025
649 23.8935969637201
650 24.1971799055521
651 23.6701372140272
652 22.9633050287103
653 22.9541498283775
654 23.1729108779424
655 22.926555562002
656 23.5532064099061
657 24.1454765899463
658 24.149815333063
659 24.2005363921032
660 24.2483036535058
661 24.3569388736824
662 25.9261238302363
663 25.5770357623877
664 26.1415329635963
665 26.4586103604275
666 26.5686980685415
667 26.0976795690344
668 25.9423868037261
669 26.0277967855942
670 25.7668647733098
671 26.275276764951
672 25.3262453282468
673 26.0493259407906
674 25.5557297560802
675 25.5564784640214
676 25.3214582059737
677 25.8221444385516
678 25.9718816668079
679 25.927333989612
680 26.2571319718012
681 25.6824944248181
682 25.251083479134
683 24.2887119232651
684 24.207079482334
685 24.3254764364031
686 23.9399648915369
687 23.0122895305819
688 23.2360720368469
689 23.2332805472578
690 22.3347339658014
691 21.6994376793593
692 21.5036813423263
693 22.0881330575498
694 22.5484950289912
695 21.8282383518828
696 21.8609198600172
697 22.083263260223
698 22.1210838993658
699 21.9221985861116
700 21.9962289034356
701 22.7197341214233
702 23.7379695564508
703 23.75377696326
704 22.7897929223012
705 23.172543800844
706 23.4511656688625
707 22.9060442552503
708 22.4438831111064
709 22.603503262498
710 23.3673714459564
711 22.9215107202752
712 22.7696723735117
713 23.2815163669102
714 23.7849131280256
715 23.6166506400833
716 24.4833754486761
717 24.7137543024771
718 25.0841393161533
719 25.1675150520778
720 25.0388274728558
721 24.9258756054607
722 24.807212478216
723 23.7380528500037
724 24.2619920645654
725 24.7233035827237
726 24.1311016687415
727 24.6206507446733
728 24.4346378727642
729 24.6517417690023
730 24.2874536095994
731 24.9396407161665
732 24.7681250254297
733 25.321858097127
734 25.1263744180044
735 24.6905144806048
736 24.1291020315172
737 24.2088653945721
738 24.1495895177475
739 24.3261975656697
740 24.2447075814123
741 23.6575050633644
742 23.4066557545715
743 22.9828001621254
744 22.1047414254148
745 22.3226811724911
746 22.4925893107923
747 21.8880796334987
748 22.375720958071
749 22.3745287092065
750 22.5853378732529
751 22.7268214685281
752 23.3664677413862
753 23.3342451903269
754 23.5795005462896
755 23.7640617422576
756 23.7056197098676
757 24.3928164043891
758 23.879790724352
759 23.1883761700419
760 24.0610156078301
761 24.2878625377732
762 23.5655474219084
763 24.2366349823592
764 24.917269721186
765 24.7927756646584
766 24.1195127099027
767 23.4367136777067
768 23.6554881124867
769 24.3037653134395
770 23.0484869901027
771 23.141469532251
772 23.4656026283619
773 22.9330872012528
774 23.4142930521038
775 23.4405373585335
776 24.1738540382988
777 25.6530620212469
778 25.0935244102034
779 25.0331403698713
780 25.4710986095979
781 25.4660237417129
782 25.7689492858701
783 26.1120416663192
784 25.1230396068197
785 24.8052275825799
786 24.4225988182329
787 23.4830387310152
788 23.7081729676082
789 23.7369677513588
790 24.2942961457471
791 24.4094571362953
792 24.0751441744597
793 23.919690484188
794 24.603360273966
795 24.6583755207524
796 24.6578127249189
797 24.7617291410483
798 24.4806344899727
799 23.7226054084218
800 22.6074405531671
801 22.1670093895061
802 22.0820400634322
803 22.3807429045786
804 21.3224361354197
805 20.9090434079002
806 20.6370241725276
807 21.0216538213989
808 20.8711601782497
809 21.0918931259268
810 20.8669529342482
811 21.0762454046054
812 21.0734375931688
813 21.2992886329397
814 21.6165497112716
815 22.4936546693616
816 23.2342229944663
817 23.4638318199823
818 23.8204663363274
819 24.9583268850203
820 25.4182464136152
821 24.8599785243596
822 25.3364447892563
823 24.8638123694744
824 25.4106941798785
825 25.8094895860678
826 25.416886996516
827 25.2423830975711
828 25.9205741031669
829 25.2987853994259
830 25.9176212908943
831 26.902681207728
832 26.7892106735936
833 26.8687946757317
834 26.213801804194
835 25.6465683863886
836 25.9571964161294
837 24.8173993346992
838 24.1256887759358
839 23.5764495971355
840 23.3396439331117
841 22.9268632385979
842 22.2510302338044
843 22.4182641314089
844 22.6777885804981
845 22.089289952127
846 21.6107369486667
847 21.5055053116301
848 21.0310994543676
849 20.8197720437816
850 20.5105728702611
851 19.9898347584636
852 19.6982125995587
853 19.4661908026534
854 19.4962271336614
855 19.7780965728328
856 19.8365826013643
857 20.1418228494902
858 20.5909023882913
859 21.0108028697909
860 20.799831284463
861 21.46742367782
862 21.9076135083531
863 21.4343072772132
864 21.3063746588293
865 21.4604972753521
866 21.5158468676796
867 21.9261545083127
868 21.7921043264402
869 21.8424467437931
870 21.6274454226234
871 21.3976686700334
872 21.3516338432228
873 21.6639336061068
874 21.7207412463961
875 21.5539787627603
876 21.973160382854
877 21.8208544528044
878 22.354608240286
879 22.0571783716388
880 22.8879720844582
881 22.8554086944439
882 23.4988034470218
883 23.8115504600397
884 23.5165335006987
885 23.3391409418314
886 23.3532755531101
887 23.0584713082957
888 22.6238384002645
889 22.654423116388
890 22.4040111315435
891 21.9189542143002
892 21.1585164278778
893 20.4749659057209
894 20.3612300626264
895 19.9392509976634
896 19.1760899164845
897 19.1580894354457
898 19.1638225622474
899 19.7566146450248
900 20.9882554120743
901 21.3137578285098
902 21.4329015623253
903 21.3576334799673
904 21.7935510254768
905 22.080627800546
906 22.4877871246098
907 22.8170129172924
908 22.4873849760611
909 21.6230890952432
910 20.6097991120496
911 20.8569334809785
912 21.08980834831
913 21.4908306748115
914 21.601556969552
915 22.2042429352075
916 22.2377536788572
917 22.3969932442905
918 22.9320578303897
919 23.0806361640714
920 22.1158313535562
921 22.2143405570386
922 21.9968923616643
923 21.8793597301477
924 21.436852021369
925 21.7198072920448
926 21.804271779564
927 21.4390574656553
928 22.4153499249711
929 23.7870713837454
930 24.3978728512781
931 24.1318681151279
932 24.2026166897732
933 24.1619374289047
934 24.6339964445406
935 24.4096732976931
936 24.534884055189
937 25.7191098410785
938 24.9361595383849
939 23.3643432213312
940 24.0494759607905
941 24.190474214161
942 24.7442721967672
943 25.3971201830225
944 25.1245375219649
945 24.3289920718973
946 24.1265336943825
947 23.4553609587591
948 22.9920911139186
949 23.5612115271296
950 23.1165739839005
951 23.8623359252089
952 23.6534379627859
953 23.1179576235745
954 23.1148039194524
955 24.0207076665297
956 24.4156064835345
957 23.4986291875129
958 24.2993079577958
959 23.8959828219573
960 23.9272031606574
961 22.5720429142588
962 22.4117642475542
963 22.5158485054049
964 22.5558160057048
965 22.1658077580901
966 22.0889987527661
967 22.4249906195669
968 21.6565482665839
969 22.2289376146546
970 21.6571525512245
971 22.1852001990266
972 23.0070684304667
973 23.4921484339131
974 24.5764229791526
975 25.0545218331793
976 24.748673003377
977 25.2044415797148
978 25.7985113422249
979 26.2775510047265
980 26.4979608779027
981 26.6644059419888
982 25.8963681874485
983 25.9993188869068
984 26.5665712051974
985 26.0572494773918
986 26.5716396024966
987 26.0202814217026
988 25.5855734895968
989 24.9894243147257
990 25.7705847213347
991 25.6781350305597
992 26.1356329252699
993 25.4889818613038
994 24.3285766138532
995 24.2101675118877
996 24.238040013277
997 24.9796154016197
998 25.6122171511154
999 26.0817942813796
1000 25.4631463875505
1001 25.6541798729182
1002 25.4982778426414
1003 25.787960986611
1004 25.6303893053212
1005 25.3369664322737
1006 25.3519844830166
1007 24.787364581627
1008 24.2296384454443
1009 23.4344516767961
1010 23.4701647903499
1011 22.924991126142
1012 23.3582810561271
1013 22.6479469284248
1014 22.3812784222354
1015 23.0513542784967
1016 22.2608361393304
1017 22.7515810642151
1018 23.2276240119173
1019 23.3786292060765
1020 23.3986588440287
1021 23.5386946561687
1022 23.1911591525358
1023 23.5518660345926
1024 23.3882551725379
1025 23.2767210941564
1026 23.2628417192063
1027 22.719656472974
1028 22.0109414253264
1029 22.2164527863215
1030 22.6773673894639
1031 23.2976197373702
1032 23.3650547097905
1033 23.5761052063885
1034 23.7746999982802
1035 23.1528214235834
1036 23.3038089094436
1037 24.7619208499622
1038 24.7774131877166
1039 24.8079324078568
1040 24.2228787865024
1041 24.0430441297629
1042 23.3600639946759
1043 23.1574086086646
1044 23.9744214232639
1045 24.0765639938558
1046 24.7186181353811
1047 23.2791926845686
1048 23.1489831670315
1049 22.9265544748306
1050 22.5146418571081
1051 21.7902844544924
1052 22.0180559179305
1053 21.5896158422307
1054 19.9904514499942
1055 20.6741816066008
1056 20.2925987175119
1057 20.2997053048555
1058 21.7397002733546
1059 22.1626948189252
1060 22.4694371681472
1061 22.9923031186289
1062 23.1210691200218
1063 23.7026646277561
1064 23.9247886064356
1065 23.155846682509
1066 23.3966556113951
1067 23.4572558678057
1068 23.7069131614463
1069 23.3820321568549
1070 23.3698677192047
1071 23.1347486029165
1072 22.7351945892909
1073 22.3202320393838
1074 23.0809029655392
1075 22.8152481373564
1076 22.3627256539643
1077 22.2331716900577
1078 20.9116145909175
1079 20.9383309157417
1080 20.8175964443017
1081 20.4492310491441
1082 20.1625183223739
1083 20.5626146377891
1084 20.1077988937416
1085 20.5357472069177
1086 20.6269801136376
1087 20.6241105143851
1088 20.6069490007334
1089 20.3343296622437
1090 20.1970043216238
1091 20.437833891281
1092 20.7870361093942
1093 20.6556042910964
1094 20.6109809236103
1095 20.9684625113375
1096 20.9652304480134
1097 20.9423843148617
1098 20.6986986078747
1099 20.9353762377261
1100 21.1259409859322
1101 21.192408309254
1102 21.2847903315134
1103 21.0528716343416
1104 20.8840893003677
1105 20.4789073023141
1106 20.540031748771
1107 20.4289241414691
1108 21.1663587109164
1109 20.8902043670214
1110 20.8880337007859
1111 20.8585847501049
1112 20.6654758752913
1113 21.0278764547295
1114 21.4332698607252
1115 22.1571974983373
1116 22.0133366246351
1117 22.5913347422028
1118 21.6007066528143
1119 21.9765383070225
1120 22.3518945744098
1121 22.7448732714976
1122 22.6818494778115
1123 22.553713853627
1124 22.644612275177
1125 22.3245597048731
1126 23.188277925577
1127 22.908938191139
1128 23.8342167499009
1129 23.9998757104696
1130 24.1830434585592
1131 23.9675659060111
1132 25.4584106668182
1133 25.5526968310605
1134 26.1577080157202
1135 25.9821870130173
1136 25.4956552805502
1137 25.7426215271386
1138 25.2087405576229
1139 24.4479241994841
1140 23.546432677375
1141 23.4033033847461
1142 22.1424397351173
1143 21.9248943506712
1144 21.8027036835951
1145 22.1281150838749
1146 21.411457680654
1147 20.9673022976135
1148 20.7862670771725
1149 21.042865115583
1150 21.8877329884467
1151 22.564239165347
1152 22.5077647797394
1153 22.6043792857766
1154 21.6160748106765
1155 22.1729638564204
1156 22.5491084858148
1157 22.3567736318382
1158 23.7754649067704
1159 23.6554260594929
1160 23.2874987039271
1161 23.5169341948743
1162 23.3148891215439
1163 24.9476634683973
1164 26.3483512156781
1165 25.8479963246375
1166 26.2152786595068
1167 28.9744669511555
1168 27.9441891451622
1169 28.0026754593271
1170 28.3801640474645
1171 27.7416907676496
1172 27.8671958733147
1173 26.9294942987405
1174 25.8593979060577
1175 25.2118466979811
1176 24.8588916093476
1177 22.5831409471838
1178 22.4073624541558
1179 22.8787172969357
1180 22.4336094225896
1181 22.7780797019777
1182 23.0277210752974
1183 23.1055768996289
1184 23.4134582586052
1185 24.0703454761004
1186 24.4500386564991
1187 24.490397872255
1188 25.3749452168903
1189 25.7368637174132
1190 25.9141531746011
1191 26.0383350904687
1192 27.1936735554193
1193 26.9893144736751
1194 27.0970862069525
1195 26.6887943922864
1196 26.4954180499477
1197 26.839079537056
1198 25.6919313387075
1199 25.4114307268052
1200 25.6769307844287
1201 25.4175117072343
1202 23.9736673244017
1203 23.4406716868021
1204 23.4967776654195
1205 23.8672867941618
1206 24.5097558532786
1207 24.060183536401
1208 24.6686643289603
1209 24.2519382486495
1210 23.9344084149975
1211 24.0265067089264
1212 23.8901659004565
1213 23.6553375370333
1214 23.3268370007179
1215 22.9715204997065
1216 22.2744210587
1217 23.1899987044182
1218 22.7661559444064
1219 22.5281633525338
1220 22.285917173386
1221 21.9397875337001
1222 22.078605632673
1223 22.1590222378235
1224 22.2563075608679
1225 22.2167439280742
1226 22.6581268572288
1227 22.5833540344258
1228 23.507998225176
1229 23.8519839819174
1230 23.9881695156494
1231 23.7484835356726
1232 25.0185698805557
1233 25.3650748229958
1234 25.3299812735332
1235 24.998340780517
1236 23.7651986103444
1237 23.2966311184514
1238 22.7943806223399
1239 22.4700497755259
1240 23.3186967513097
1241 23.1030268957884
1242 22.1093699019314
1243 21.9055215894998
1244 21.4245258589559
1245 22.4413875923304
1246 23.5826985807629
1247 23.442157269954
1248 22.6423909899843
1249 22.9051030358969
1250 22.3520480516886
1251 23.2346274945026
1252 23.4939900411461
1253 23.6527442555631
1254 23.8690087195866
1255 23.6874354093213
1256 23.4279076002151
1257 22.9730726994943
1258 23.6098832188677
1259 23.9469286444941
1260 23.947753313963
1261 23.631140090541
1262 23.1674272018415
1263 23.2360446114725
1264 23.1265874664302
1265 22.8579937939991
1266 23.1143723656621
1267 23.4096119581189
1268 23.6274226951442
1269 23.2978501228267
1270 23.3303953483341
1271 22.6279511969212
1272 22.6356076769067
1273 22.6921138148662
1274 23.6973715160187
1275 23.5787186816324
1276 23.6681256066209
1277 24.3699623466766
1278 24.2974232020625
1279 24.6934327522596
1280 25.4535091203176
1281 26.4546489689609
1282 26.8138665167334
1283 27.7004473449682
1284 27.4567686652638
1285 27.4813997844869
1286 27.5009331953823
1287 26.4753794673668
1288 27.2477750604387
1289 27.1982060466653
1290 27.9621969093915
1291 28.0841208449539
1292 28.6939334624438
1293 28.0157532351281
1294 27.639906195559
1295 28.7010837861856
1296 28.4576529375743
1297 28.5763611566429
1298 27.1094163660509
1299 27.8668822541221
1300 27.5333608215419
1301 27.2543407882456
1302 26.2883110618564
1303 25.6638963956892
1304 25.9230667757944
1305 25.6488002126922
1306 26.0205480306065
1307 26.4397119805042
1308 27.3658037602518
1309 26.4733776635679
1310 25.2439395445806
1311 26.0253222670034
1312 26.0552995326688
1313 26.1663206173099
1314 26.4777486282124
1315 25.9409961746224
1316 25.2289317301897
1317 25.5706226977276
1318 24.8284496708775
1319 24.5065612813525
1320 24.2106937873553
1321 22.8904639843762
1322 23.1502425126677
1323 23.1194378914256
1324 22.0124815409827
1325 22.0626728046689
1326 21.6745921480432
1327 21.3505232198297
1328 21.6218571575416
1329 21.5608838855169
1330 21.9767377706667
1331 22.013188342821
1332 22.1911053824686
1333 22.6485061337913
1334 23.4356974781339
1335 22.9211908166242
1336 23.5574151809829
1337 23.7777186605377
1338 23.7947165296859
1339 24.3812959260853
1340 23.7879642436616
1341 23.6690105024363
1342 23.5757227893987
1343 23.005041357934
1344 23.7938594290879
1345 24.1853542883316
1346 23.3927042517936
1347 22.7742826678898
1348 22.6573964924639
1349 23.0710059795892
1350 23.8533747815239
1351 23.9498312206553
1352 24.2543667118457
1353 24.2822416023841
1354 22.5558790035602
1355 21.9808932280835
1356 22.4956599937787
1357 21.92281328202
1358 21.8007370555695
1359 20.6845201870095
1360 20.1082184994675
1361 20.8315305751788
1362 20.3410170199207
1363 20.455343386682
1364 20.8923155661192
1365 21.4917646495684
1366 22.2212806885184
1367 23.0790985454696
1368 23.9233498741818
1369 24.2201887953132
1370 25.2034358815832
1371 24.6787713213347
1372 25.1014598379634
1373 25.5684163833816
1374 26.3685991460545
1375 26.3012870459147
1376 26.2953980017053
1377 26.1507878494506
1378 25.5604680438053
1379 26.0857442501582
1380 25.0425880227112
1381 26.1550368593037
1382 26.1633152489183
1383 25.2680944509855
1384 24.4632992469929
1385 24.3752685931866
1386 23.9600924630268
1387 24.0140572146572
1388 23.4481346803158
1389 23.4239924695146
1390 23.4245554390614
1391 21.9645691992492
1392 20.9756436393221
1393 21.5012056489466
1394 21.2396180840042
1395 21.3895742575096
1396 20.9955257248844
1397 20.9592130577696
1398 21.5604601481764
1399 21.6338538294309
1400 21.4745677111099
1401 21.5831934724788
1402 21.8911638748398
1403 22.0370259018122
1404 22.4464198738983
1405 21.9693941651764
1406 21.9698728039681
1407 21.9591639632519
1408 22.3621576989754
1409 21.4956050032491
1410 21.0626805557574
1411 20.5580243703539
1412 20.6120177842766
1413 20.9601582241048
1414 21.3754960191601
1415 21.3003309386795
1416 21.4804282386993
1417 21.3512434538401
1418 20.6638085105832
1419 21.7532882507337
1420 22.5243643634203
1421 22.9730170066123
1422 23.4211049077006
1423 23.1560628534674
1424 22.8770832399089
1425 23.8851110995555
1426 23.5846949806642
1427 23.7378602398328
1428 23.9667328197474
1429 23.7435826304169
1430 23.2497304304901
1431 23.2845303096943
1432 23.4206689915252
1433 23.1936889321211
1434 22.7202434659133
1435 21.5987668005884
1436 22.4768551268717
1437 22.6415389732411
1438 22.7496390469558
1439 22.3052594659992
1440 23.3207169882008
1441 24.2703019953358
1442 24.2389643191417
1443 24.4668348545676
1444 25.4939803974343
1445 26.151936898179
1446 25.4493175831142
1447 25.4071559307951
1448 24.8507253909941
1449 24.7736704260762
1450 24.9867204583485
1451 24.2331968307464
1452 24.1537083028309
1453 23.8718402780502
1454 23.4932800977728
1455 23.1127821492952
1456 22.910853922701
1457 23.567830980449
1458 24.8378396394845
1459 25.0007517643609
1460 24.5256982607994
1461 24.9192781309643
1462 24.5470544292446
1463 24.8195696088978
1464 24.3383291882877
1465 24.2479215824643
1466 25.2850003166511
1467 24.7074395415695
1468 23.7913834405941
1469 23.9462675062261
1470 24.1948318133768
1471 24.1494411430244
1472 24.638306265056
1473 24.7356978833004
1474 24.964545722353
1475 26.0077393833654
1476 25.2047098262533
1477 25.0506655764013
1478 25.8553143081474
1479 25.7278354698059
1480 25.4892148644305
1481 25.3562840903876
1482 25.420662686764
1483 24.745022843857
1484 25.0670088203326
1485 24.4237330318019
1486 25.0563204284712
1487 25.0314654978078
1488 24.1579528238548
1489 24.9701819743057
1490 24.5388702524096
1491 24.8847626538191
1492 24.1033664608278
1493 24.718340127272
1494 24.1784462546764
1495 24.4392977225606
1496 24.3040488775096
1497 24.8783293486023
1498 24.7967066190358
1499 23.6156734805964
1500 23.5600074622591
1501 23.3820190173153
1502 23.6417514226744
1503 23.5077665036751
1504 23.7046421508281
1505 24.2878509289648
1506 23.9574252681419
1507 24.5215700735802
1508 24.7960057858042
1509 25.4566580208607
1510 26.0111901504799
1511 26.4317234959414
1512 26.3669505998581
1513 25.8431387789377
1514 25.8942285602722
1515 24.5832225327491
1516 24.8400991303246
1517 23.5833701746642
1518 23.7505722554968
1519 23.2924798296851
1520 23.2741728583395
1521 22.6812754653363
1522 23.0316727334192
1523 24.0974909109042
1524 24.1585404167706
1525 24.5841674566088
1526 23.8215394831297
1527 24.2215830252702
1528 23.8921645650585
1529 23.9983398168542
1530 23.6019739534269
1531 23.4747223440924
1532 23.3688452302866
1533 23.2131990119844
1534 23.3072815943265
1535 22.6706543656563
1536 23.0470611887235
1537 22.2469263790731
1538 22.0319990555456
1539 21.9206593128036
1540 21.4934065691642
1541 21.6414653249831
1542 21.6999561264098
1543 21.679788494842
1544 21.7234791422445
1545 22.3594218303955
1546 22.0714822484681
1547 22.2047940776252
1548 22.448285495141
1549 22.0236360093769
1550 23.0220940733196
1551 23.1308047859357
1552 23.0883893913276
1553 22.2124782236046
1554 22.1774715082184
1555 21.5491768330905
1556 21.6020265229935
1557 21.5968875779988
1558 22.7279138012469
1559 23.5259818499033
1560 22.6889179352792
1561 22.7185865048088
1562 22.3210319153738
1563 23.090261269983
1564 23.1030557993514
1565 23.6889643481783
1566 24.1060235630543
1567 24.4643684461899
1568 24.0435993450878
1569 24.6992504730401
1570 24.8959605117071
1571 24.7620239042786
1572 25.0892040749736
1573 24.6250709867743
1574 24.7986052561601
1575 25.3895531298208
1576 25.6225274853022
1577 26.0606285893556
1578 25.8439828074223
1579 24.7852597355604
1580 24.9437147401329
1581 25.1335594195181
1582 24.8972562333856
1583 25.2686593771937
1584 25.5189220204758
1585 24.8854425215776
1586 24.9806037907445
1587 24.7293011164761
1588 24.2978519492738
1589 24.0226365299334
1590 24.21392185868
1591 23.6500375391721
1592 23.4454819727386
1593 22.8691829426612
1594 22.0018405929206
1595 22.118988296521
1596 22.0397626264772
1597 21.4884209684774
1598 22.0510775002036
1599 22.4980101045781
1600 22.7934345322176
1601 23.6788646489217
1602 24.1313069226452
1603 25.2070050494792
1604 25.709602098165
1605 25.6993550624685
1606 25.2574415808877
1607 25.8460125372248
1608 25.0262366730253
1609 25.2735328051744
1610 24.7075720436959
1611 23.772420495892
1612 23.6073530009555
1613 23.4132521877238
1614 23.319898852535
1615 23.5012772171718
1616 23.9325541934701
1617 23.6942508069759
1618 24.6377030453153
1619 24.1684428860394
1620 24.9741150315158
1621 25.2569672904104
1622 25.404816815778
1623 25.1157114978574
1624 24.7448869272904
1625 25.6175284279582
1626 25.9387252498923
1627 26.2271471501624
1628 26.2240523821479
1629 27.4726551297639
1630 28.2015911867792
1631 29.1966730169338
1632 28.9943413156364
1633 28.7405799495653
1634 28.5934223274332
1635 27.8555457002092
1636 27.9276892954783
1637 27.4606881244534
1638 27.4589594508577
1639 26.6751887106436
1640 24.8982603142354
1641 23.8480429565023
1642 23.7181437977503
1643 24.4528541030267
1644 24.4308008622009
1645 23.8147659846598
1646 23.1307393131004
1647 23.7518494204708
1648 23.3680447041425
1649 24.4698398068218
1650 24.5926403398127
1651 24.8441981877564
1652 25.3669320807873
1653 24.9021894267952
1654 25.9831980186311
1655 26.1190689219072
1656 26.3102815461346
1657 26.3721044832027
1658 25.8477652296981
1659 25.1718269948907
1660 26.6799247304487
1661 26.1251789372271
1662 25.9629964719909
1663 26.1720554771449
1664 24.9999413260919
1665 25.8810616102099
1666 25.9723402163159
1667 25.8333982485051
1668 26.5218413809303
1669 26.1004080558622
1670 25.0803260730085
1671 25.0704817773922
1672 25.9701390131912
1673 25.3115414184117
1674 25.3141938886454
1675 24.9616239469755
1676 25.86156597174
1677 25.9508988474104
1678 27.0287725369885
1679 27.3191719769127
1680 27.9366062563202
1681 29.4326325781418
1682 28.5613628983127
1683 29.5528553429623
1684 30.2042054033243
1685 30.863586678002
1686 29.9675126127297
1687 31.0727110893275
1688 29.5475092052478
1689 29.7172178562942
1690 29.4285226335299
1691 30.1640284670222
1692 30.3728779415765
1693 29.8105051309936
1694 29.4658436367795
1695 29.2774336590346
1696 29.9399105934229
1697 28.8926928002228
1698 29.0586308809469
1699 28.5043498855855
1700 28.6321033317627
1701 27.5744730372777
1702 27.11260155985
1703 26.6700005710078
1704 26.7201031431098
1705 27.0246857550259
1706 27.5173111740563
1707 27.2320575798314
1708 27.5934540440753
1709 27.4638539914096
1710 26.4371911059296
1711 26.113039532387
1712 26.7029066473772
1713 26.8416043460325
1714 27.2675466190259
1715 26.4458092025148
1716 24.8933818698135
1717 25.7563422077859
1718 24.8860365171445
1719 24.6123063817708
1720 24.6727681551574
1721 26.1183327270728
1722 25.5972856186075
1723 26.1413018301681
1724 26.1309405042359
1725 25.1935504173999
1726 25.2185658305657
1727 24.6046418703293
1728 24.665395759329
1729 24.9905900540592
1730 26.1511428107977
1731 24.7962684287023
1732 25.6079893172452
1733 25.0891366758315
1734 24.9311698561419
1735 25.5183079357331
1736 26.343660909552
1737 25.8001023563098
1738 26.7839422862542
1739 26.844838861284
1740 26.902526140499
1741 26.7127171088368
1742 26.0703757449265
1743 25.8425864996219
1744 25.8655350798524
1745 26.374968180916
1746 25.6343410277603
1747 26.2201880378398
1748 25.573985662089
1749 25.2635935049956
1750 25.0218460238356
1751 25.7418140288429
1752 26.2083584347191
1753 26.4481036749918
1754 27.4912396960032
1755 26.8203983701437
1756 27.4952356836941
1757 28.4715661424824
1758 28.3495273342547
1759 28.9090238680511
1760 27.7865004744971
1761 27.4577251221601
1762 27.1490867321081
1763 27.2803236009171
1764 26.1490942442068
1765 27.3422153153609
1766 26.8872057397931
1767 24.8986164282242
1768 24.8381896443504
1769 24.8242269759358
1770 25.9222686143359
1771 25.3015811264887
1772 25.1880996311778
1773 26.0146602929467
1774 26.6509118716187
1775 26.675576096874
1776 26.4787594956888
1777 26.9040961935
1778 27.5677269586543
1779 27.3289563679303
1780 27.127283714113
1781 26.9088240303249
1782 26.8738227905988
1783 26.6269988290838
1784 25.5580468353537
1785 24.2134031980166
1786 25.0068804561584
1787 25.127549063346
1788 26.2328647709084
1789 26.15259668812
1790 25.6025505668414
1791 25.9988835082338
1792 25.3022840732441
1793 24.9176592068143
1794 25.6977959564108
1795 26.0957278404658
1796 25.9930212429606
1797 26.1194979211328
1798 24.2863286462613
1799 24.4043148301491
1800 24.4348097424041
1801 24.1026369137098
1802 25.2110111878299
1803 25.009440003805
1804 25.3548944492751
1805 25.330711917901
1806 24.6466150302819
1807 24.9499311142308
1808 25.1575555523187
1809 26.2600774110745
1810 26.4638427712469
1811 26.1109035702759
1812 25.4518766002832
1813 25.4723970119146
1814 25.0700621155673
1815 24.4406803522723
1816 24.7220300574294
1817 24.3977946231271
1818 24.3599017650019
1819 23.6578756936208
1820 23.625387838939
1821 24.1531363394318
1822 24.2550056956668
1823 24.6807554620667
1824 24.8370701734875
1825 25.6551854154804
1826 24.6806579724644
1827 24.5332528750518
1828 24.2012508159421
1829 24.2514997759684
1830 24.0980364438741
1831 24.5295024000544
1832 24.6680785281183
1833 24.4675058645046
1834 23.4088034118661
1835 23.8753870203108
1836 24.4860493845634
1837 25.1247796903064
1838 25.4952498678285
1839 24.0707918835082
1840 24.4111102937224
1841 23.560828780762
1842 23.1455004770504
1843 23.9711772908893
1844 24.7575672456323
1845 23.8364955193585
1846 24.4769814108056
1847 24.8639424273497
1848 24.9480157464311
1849 25.6788991409156
1850 25.6654970262577
1851 26.012584905981
1852 27.9356761913661
1853 27.2590356265675
1854 26.7938445080422
1855 27.5288343111645
1856 26.6134493319985
1857 25.3100720228715
1858 25.49101582387
1859 25.17282831521
1860 24.7203269523498
1861 24.555168763484
1862 23.3450387042273
1863 23.8267186219841
1864 23.6372804054478
1865 23.606077434829
1866 24.26121666523
};

\nextgroupplot[
tick align=outside,
tick pos=left,
title={conv\_layer\_1},
x grid style={darkgrey176},
xmin=-93.3, xmax=1959.3,
xtick style={color=black},
y grid style={darkgrey176},
ymin=21.1224279126551, ymax=68.8632601806864,
ytick style={color=black}
]
\addplot [semithick, steelblue31119180]
table {%
0 33.3584233667678
1 31.8917398332453
2 31.9937133645679
3 31.1990278149163
4 31.4917294934757
5 31.2046061796534
6 30.19414458597
7 28.7698789700266
8 27.9042927923624
9 26.4061477122761
10 25.2351821234699
11 25.1564008133451
12 24.9808487817882
13 24.8431941621413
14 24.8059775986055
15 25.1896035008641
16 25.7191417704526
17 26.2510604860465
18 26.4902025706978
19 26.5921395854335
20 26.6206379870071
21 26.433050534037
22 26.1771002639071
23 26.0649903127417
24 25.7579175688122
25 25.4811148789952
26 25.6279657087011
27 25.8796626680328
28 25.9077053585205
29 26.3913336547921
30 26.7292325410634
31 27.3771750543217
32 27.8700288916625
33 28.078194372627
34 28.4359684315799
35 28.7106365611781
36 28.3334599137649
37 28.428573941644
38 28.5033496591637
39 28.3965037419707
40 28.9091949685821
41 28.5812154386706
42 28.5638304903366
43 28.8773754321888
44 29.1414596964262
45 29.3217637862437
46 29.2830276183619
47 28.9708960101563
48 28.8889697490367
49 28.7273882852277
50 28.1307547064797
51 28.6461466171547
52 29.1474445516841
53 29.1203438036351
54 29.4176259592655
55 29.3830424664639
56 29.3820873375785
57 29.0872602602671
58 28.8059138586964
59 29.218292380809
60 29.1128755514995
61 28.8431008867962
62 28.612264837322
63 28.8105918531601
64 28.1161505652644
65 27.5411344997413
66 27.8061899243539
67 28.3899551027133
68 28.73176045406
69 28.1160958273049
70 28.2221859226491
71 27.9167821295679
72 27.6540759796674
73 27.1208998489919
74 27.2398899484567
75 27.4236605118903
76 27.217185554096
77 27.2197186688593
78 27.0079928797476
79 26.8261027975727
80 26.8263921372713
81 26.7020870595292
82 26.6246819491825
83 26.8448935772508
84 26.6954015605143
85 26.9064250036465
86 26.9970144318319
87 26.6610863842702
88 26.891220151823
89 27.4481467702121
90 27.4309866996068
91 27.7400549072505
92 27.7319053564717
93 27.3854893417751
94 27.4160665946568
95 27.1845449032162
96 26.7569802705586
97 26.6390786996028
98 26.3945469601767
99 26.1842097473096
100 26.0837654157299
101 26.3067242622947
102 26.5090352349752
103 26.4226642937213
104 26.187321379068
105 26.0223672438284
106 26.2153210629192
107 26.0759208192534
108 26.1753522655367
109 25.910737378351
110 25.8265750166818
111 25.5975615521287
112 25.2676878389223
113 25.6547825486848
114 25.8545120291169
115 26.4392375603668
116 26.285017574466
117 26.5683365392533
118 26.2738670338369
119 26.2616023061346
120 26.5994923871346
121 26.2821409078603
122 26.9780915988747
123 27.4394505462021
124 27.6122664488944
125 27.3207964837935
126 27.529800245617
127 27.6988076842233
128 28.067531654066
129 28.1705200080871
130 27.905447206964
131 27.8424721716915
132 27.4037060617981
133 27.4498525214248
134 27.6529648740201
135 27.7353861446408
136 27.8816861028395
137 27.1875952395558
138 27.0325694646549
139 26.9607378497901
140 27.246275438194
141 27.6446014843114
142 27.5890148665678
143 27.3358102600616
144 26.9391340496036
145 26.8096741821504
146 26.5985991043551
147 27.1660374955191
148 27.3176804831964
149 27.4013708046688
150 27.0891649684155
151 27.0496984690659
152 27.2603757828929
153 26.9988987234882
154 27.0641214990221
155 27.496252811957
156 27.8711412208305
157 27.9926484998894
158 28.1334439917382
159 28.2851815055137
160 28.393132495878
161 28.6123890119627
162 28.3510410061056
163 28.4712402600928
164 28.9603364868557
165 28.6103266797365
166 28.7403528310947
167 28.6630988474634
168 28.3986652515473
169 28.4031156883712
170 28.5026967893891
171 28.4698059749593
172 29.0497847774409
173 28.8188697725166
174 28.3681171874832
175 28.2075481234175
176 27.705873539504
177 27.3573156322885
178 27.3566423035675
179 27.1612114680074
180 27.1120658252123
181 26.5269582062773
182 26.3876462656353
183 26.5553772319128
184 26.7845968098606
185 27.081388570632
186 27.2558038543566
187 27.6819360754744
188 28.1727398706157
189 28.2057502104124
190 28.1042263325217
191 28.3690329357833
192 27.7148886839734
193 27.7412388214677
194 27.1129343860246
195 27.1431661983577
196 26.9993111499703
197 27.0744678596305
198 26.6793451344129
199 26.739642985737
200 26.8971406455305
201 26.8953857956228
202 27.3642983858214
203 27.1214138817837
204 27.3275967383026
205 27.1310947958006
206 27.2820107348349
207 26.9542013883557
208 26.8799402431918
209 27.1788371398835
210 27.4349886603342
211 27.3609974942606
212 27.1758268773346
213 27.2989026128435
214 27.9170088991957
215 28.015798240269
216 28.4488725806388
217 28.2150972672194
218 28.0396782894299
219 27.5660671958625
220 27.1922888518679
221 27.1336325603807
222 27.1644672304313
223 27.0835028020917
224 26.9024308563417
225 26.6700555540171
226 26.1614502506344
227 26.2862822151844
228 26.5448176650152
229 26.613673598601
230 26.7078374657684
231 26.7202455424265
232 26.7504709374646
233 26.4623724570171
234 26.2836340562342
235 26.2612230421492
236 26.0477553557497
237 26.0197199713308
238 26.2618928467956
239 26.1847644739787
240 25.7730452729492
241 26.1149839537811
242 25.8588985359198
243 26.2405615071401
244 25.8303418083906
245 25.9524474253705
246 26.0309399848537
247 26.2368121047153
248 27.0914994347935
249 27.3111213490152
250 27.5371740800537
251 27.0392206874862
252 27.7216884481178
253 27.4703301728634
254 27.4890830334604
255 27.3432530245279
256 27.680510427738
257 27.2466879047339
258 26.7224500962759
259 26.3873592748665
260 26.1272278272556
261 26.310822770348
262 25.7016603517102
263 25.947069157229
264 26.2011048244573
265 26.619521877391
266 26.917131336616
267 26.8289376915663
268 26.3800814173319
269 27.0800572358767
270 27.2386142634753
271 27.2964586671331
272 27.5888467710617
273 27.4014927445099
274 27.5714522350103
275 27.2563279658183
276 26.8952431312689
277 27.2601080444177
278 27.3532476468667
279 26.7917963915777
280 27.5457839658892
281 27.6732618786102
282 27.1933497394988
283 27.0487585691348
284 26.8708619036763
285 26.8245074766389
286 26.9037504298699
287 26.470078676836
288 26.0793363451466
289 26.099211350175
290 25.5160409071615
291 24.8548196402286
292 24.7914425119993
293 25.1376148001373
294 24.9385674660795
295 25.2798075653699
296 25.0510783994048
297 25.7738568583041
298 25.9821384907674
299 26.2222541534529
300 26.226360564874
301 27.1467207919623
302 27.8537983461687
303 27.4706754613246
304 27.5504164767952
305 27.6184679194374
306 27.6405447581493
307 27.6519183316166
308 28.1675506518888
309 28.1754771558028
310 28.5105551989635
311 28.3122127138173
312 27.9002195771996
313 28.4037768455997
314 28.6415973100611
315 28.3956824701528
316 28.4047773291148
317 28.1706661900017
318 27.8374321432958
319 27.9445264844935
320 28.0634275570745
321 27.8399136468674
322 28.7339823693276
323 28.3895647014327
324 28.974724487176
325 29.3986301800502
326 29.6865308344704
327 29.571669089597
328 29.8147613972626
329 29.5862392514986
330 28.938661499051
331 29.2818849873295
332 28.535687222005
333 29.5021727554133
334 28.9216784364568
335 28.2790973341372
336 28.3418535476968
337 28.3540948110628
338 27.8078514977156
339 27.5420451205148
340 27.4721312967949
341 27.6228262806661
342 27.8937020038161
343 27.0703536505083
344 27.4393846197274
345 27.4166079532718
346 26.9804266624105
347 27.4119958264571
348 27.422661493855
349 28.1190743396553
350 29.0542799503118
351 28.5314238902647
352 27.974612500917
353 28.3457163484407
354 27.975973255164
355 28.0578921955178
356 28.2643400089534
357 27.9100468591965
358 27.6550000744373
359 27.2059505078712
360 26.4690361955123
361 26.5960252787547
362 26.6669123451988
363 26.4816939137313
364 26.7905804212656
365 27.4230583631065
366 27.1902970318818
367 27.7485868047483
368 27.8419004717087
369 27.7212810057788
370 27.927381687
371 28.9943907385191
372 29.0137255295196
373 28.9633606595798
374 28.4798667833678
375 27.7487523615831
376 27.69852309373
377 27.3316263310191
378 27.982348148489
379 27.9413904333183
380 27.754526146969
381 26.3590139686016
382 26.4594667138849
383 25.9583295087357
384 25.9918607964262
385 26.2192329273795
386 26.8726667026813
387 26.7121121486034
388 26.8011436347569
389 27.1088817180384
390 27.1235922439044
391 28.1439796553926
392 27.8941322852487
393 28.0167776319006
394 28.4741448304515
395 28.5544538660577
396 28.2587558289415
397 27.9060124861052
398 27.2045761946384
399 27.1930146215245
400 26.9333070162291
401 26.2497461474271
402 26.8336121162831
403 27.4049576318171
404 27.520944350924
405 27.7010518673936
406 27.4276619917891
407 27.9938824783595
408 28.3597177889431
409 28.7440807422392
410 29.2116883360027
411 29.2084035044704
412 28.8848315976782
413 29.2300815159306
414 28.9334723602574
415 28.8001114752306
416 29.3575150359635
417 29.4783730879456
418 29.4312854823552
419 29.133855994218
420 29.0144967697423
421 29.0727976947705
422 28.8050108637917
423 28.4315014745369
424 28.1335176658952
425 27.8906427547846
426 28.1718601934898
427 28.7372936698301
428 28.3910399307805
429 28.0586425270208
430 27.800265881869
431 27.923441822368
432 28.3151813747686
433 27.9931349094405
434 28.4511024234284
435 28.4869494061923
436 27.5897491488594
437 27.5341910973928
438 28.0944649510696
439 28.2237468497569
440 29.0049298473129
441 28.5419318354631
442 28.2979594980399
443 28.2531891544074
444 28.1980607704601
445 28.3050227358849
446 28.6279433952972
447 27.8432424928993
448 28.1146053793791
449 28.4951010083859
450 28.3188028839185
451 28.8289339638929
452 28.9858522171415
453 29.0533327865596
454 28.9383264394906
455 28.9501050352478
456 28.6391497414838
457 28.8352655396234
458 28.3392070613895
459 28.238388644276
460 27.9324331296965
461 27.7808687268573
462 27.8783177218282
463 27.9658306395228
464 28.1422539663054
465 28.5889927862049
466 28.5866819193591
467 28.1390351405568
468 28.229329119732
469 27.7309418239562
470 28.0943626406892
471 28.3925161955153
472 29.1647142526426
473 29.0255917759515
474 29.0867261735569
475 28.4779175920791
476 28.9818734713637
477 29.2411234879757
478 29.0581745714469
479 29.1975404041498
480 28.7353804036549
481 28.4320152542655
482 27.6341905060135
483 27.6881188085258
484 27.2028322501811
485 27.4601679309831
486 27.4041888386967
487 27.7601156178481
488 28.1020002119806
489 28.3293357586625
490 28.2152494015364
491 28.720501669318
492 28.5612827607871
493 28.6036229132293
494 28.7378724015547
495 28.0682460088361
496 27.9010719489797
497 27.3913454819375
498 27.1709898798832
499 26.9054098781364
500 26.8736915540221
501 26.3653093435003
502 26.3922395170141
503 26.275577344096
504 26.2931152748366
505 26.7259485738638
506 27.0998820573576
507 28.1551213614395
508 27.8658907926004
509 28.9444414447325
510 29.182593488496
511 29.55345711696
512 30.05482579471
513 30.8020506013833
514 31.1239544953843
515 32.2293100903342
516 32.7137722601966
517 32.888190590689
518 33.0968460615702
519 32.3678541049133
520 32.5644381425019
521 32.385459248529
522 32.2451065121285
523 32.1445002837295
524 32.0874288840568
525 31.3396549417459
526 32.3215395203906
527 31.1111626582967
528 31.1827626427876
529 31.1273496420509
530 31.6328524677917
531 31.6457019922434
532 31.2517191490495
533 30.593752150415
534 30.6244620692512
535 30.3741734848983
536 28.4167076877284
537 28.4549081912354
538 28.8730265985961
539 29.002343608483
540 28.114125029573
541 27.5797759578508
542 27.5451402535894
543 27.8292762805047
544 27.3848504468296
545 27.7313024235395
546 27.8550080555922
547 27.5343952816007
548 27.547415203031
549 27.1677393064545
550 27.2098667850753
551 27.4779402020451
552 27.4983747553037
553 27.9203465817218
554 28.6131097182435
555 28.6934523885206
556 28.469789209311
557 28.8818244999671
558 29.3119572788963
559 29.4763302741894
560 29.6204373454947
561 29.4720519239446
562 29.7759145691915
563 29.2637607802679
564 29.4795289115509
565 29.3653965704476
566 29.4627443918189
567 29.9373810848564
568 29.6254901209782
569 29.5495671502328
570 29.8118033494112
571 30.1632953589796
572 30.0571378156875
573 30.0640967706736
574 29.1837108832199
575 29.0549460135296
576 29.1465685744626
577 28.7771814039234
578 28.0165407166901
579 28.6086643569049
580 28.3581524781424
581 29.2181176967681
582 30.1158214444699
583 29.9990484461606
584 29.7909316311493
585 29.4177351501411
586 29.4148088466725
587 30.2274931443784
588 30.8392581566368
589 30.0924530679199
590 30.1957078968388
591 29.7754168409907
592 28.9564218871111
593 28.7260318166963
594 29.6588232204774
595 29.8937939134834
596 29.5631604802062
597 29.2263821511501
598 29.0052745557686
599 28.7754975538408
600 28.857430756753
601 28.871757991566
602 28.6635510770495
603 29.4467621247826
604 28.5547529419964
605 28.6642989010741
606 28.7319478865777
607 28.6519553630416
608 28.1411032574211
609 29.1231735143318
610 29.5426775239612
611 29.7366110887819
612 29.8055384210352
613 29.3790569697221
614 30.1863016014812
615 30.1127541383055
616 30.606195436617
617 30.3619213053725
618 30.3723800536196
619 29.8734426836976
620 29.1700369656839
621 28.663330544805
622 29.0775373993163
623 29.2147440374745
624 28.6220116801833
625 28.5038186552932
626 28.5220754667707
627 28.212523135097
628 28.7383913929775
629 28.7552443151469
630 28.9078812562856
631 28.3248425213385
632 27.5917044176317
633 27.744026966319
634 27.8238615164177
635 27.5332006847379
636 27.0316949291934
637 27.782103142299
638 28.4359864918364
639 28.7140636066842
640 28.3313395623181
641 28.7731330017338
642 29.1443322166853
643 28.7646076247607
644 28.4394222388176
645 28.8285432154073
646 29.1812034230134
647 28.9625339804107
648 28.6894392637611
649 28.3234756447467
650 28.4867572529819
651 28.1462715420931
652 28.5634816236916
653 28.6254923903295
654 28.9443882385706
655 28.7518361101941
656 28.9875604786013
657 28.9614235331555
658 28.4751145300442
659 28.9278128588043
660 29.4362460376045
661 30.1362328932988
662 30.3914552487236
663 30.2974755789165
664 30.1277612660338
665 30.9993917919342
666 30.6176542349564
667 29.8972845293082
668 30.0544977647341
669 29.8427889565979
670 29.3598121226712
671 29.8960096159895
672 29.1553717522261
673 28.6993931554018
674 28.7555955531814
675 28.1809690514394
676 28.1405219443694
677 28.494832025438
678 27.9937212999842
679 28.1462849827566
680 28.4810906708228
681 27.4086239666031
682 27.0184139080668
683 27.881401219133
684 27.7662398158538
685 27.4014721947338
686 26.9045794166535
687 27.0184439883954
688 27.7144937967902
689 27.563932949894
690 27.5011957489321
691 27.2164692889887
692 27.0883404933116
693 26.450123641416
694 26.6782356332518
695 27.3818341662266
696 27.7480833219486
697 27.4621477062997
698 26.8210356102225
699 26.5632135044563
700 26.3646769261272
701 26.8350089296617
702 27.5023060902363
703 28.0903833823503
704 28.4417364691615
705 27.9323801151959
706 28.3187454007568
707 28.4665997541819
708 29.4515337827021
709 29.4378245395776
710 29.6043259163262
711 29.5536114520131
712 30.2222979981263
713 30.3601010965589
714 29.5666019915742
715 30.4518033098502
716 30.4002583089226
717 31.2120127508773
718 30.3289480716881
719 30.2949015877245
720 30.7159292157708
721 30.9703238878225
722 30.5495911391646
723 30.6107975880324
724 31.4538101935998
725 30.9751733389839
726 30.8493919109786
727 30.6691087389854
728 30.804907099584
729 31.246082462834
730 31.3964882092735
731 31.2366031274614
732 30.8845913571651
733 30.5899847242907
734 30.1652015628221
735 29.8913686027717
736 30.1231256742225
737 30.351695221634
738 30.3474987877221
739 30.1765403560318
740 30.220453052575
741 30.359881352113
742 30.4295128857101
743 30.7646841237712
744 31.1221291926201
745 31.7916764698182
746 32.2283107750764
747 31.1620403371996
748 30.7380803220562
749 30.4572696682236
750 30.5852429873197
751 30.3399413503453
752 30.7768992387546
753 30.1343716368805
754 29.9122332463352
755 29.6871667665325
756 29.2293995948482
757 29.2296363969274
758 29.5450174131151
759 29.7529898663685
760 29.1303183143256
761 28.6582227115588
762 27.9900807049149
763 28.8936159892416
764 28.9499242121037
765 28.3783914421638
766 28.2007220788264
767 28.3736613277039
768 28.6553862549757
769 29.2415292975099
770 29.6686831619094
771 30.4344379518762
772 30.3264899833343
773 29.4746184572482
774 28.9201015143266
775 28.6439529654224
776 28.9358719466787
777 29.8340243519103
778 30.5715122060104
779 29.4788773425206
780 28.259510548008
781 27.5838953794819
782 27.3264155201678
783 27.3815256908282
784 27.7630731383331
785 27.8585288302566
786 27.7300856983075
787 26.9519078842475
788 26.1279314119792
789 26.4342008016199
790 27.0319152480482
791 27.6242984625238
792 28.3590190871431
793 28.0662624615695
794 27.9258644475878
795 29.180610163145
796 29.4006033758638
797 28.8657219262355
798 28.8376841531512
799 28.7550005239069
800 28.8184650884383
801 28.3549766546026
802 28.3434325769436
803 29.606031721003
804 29.8954778159711
805 28.9157820469417
806 28.1747599065677
807 28.6786913058656
808 28.4795235231773
809 28.6016814711412
810 28.2572248105225
811 28.067393750717
812 27.8343978785647
813 27.0358545761575
814 26.8787638806924
815 26.9443541975232
816 26.8928756760222
817 26.6380394865937
818 26.4358590953848
819 27.361099637087
820 27.640581054872
821 28.1178325904943
822 27.8921342432492
823 27.7629052034674
824 28.0841245988622
825 27.7840975543663
826 28.479813634957
827 28.7527557024989
828 29.9183868328248
829 29.2477440764332
830 28.8114443381252
831 29.0376712977854
832 29.7625180862714
833 29.6163178037807
834 29.5448529112789
835 30.6854618067527
836 30.1885473416111
837 30.5248631486271
838 29.187657549501
839 29.1656551213607
840 30.0348876587581
841 30.0708254904341
842 30.2403815971389
843 30.2348071784172
844 30.2709413599833
845 29.3434558169847
846 29.3643142083999
847 28.9700306185775
848 29.1948656677181
849 29.0892159435787
850 29.0785135004407
851 28.6772811925544
852 27.9415733500396
853 27.9871918563607
854 27.6041941916931
855 28.003416429429
856 27.9489686370498
857 27.9939715687012
858 28.0830644752289
859 27.9600359708145
860 27.4093948268207
861 27.8596984859203
862 28.4517505904507
863 28.6274683508334
864 28.7828546658274
865 29.1218758755107
866 28.9003559172499
867 29.1478435300804
868 29.4790607503792
869 29.8016817304789
870 29.8640018768418
871 29.2113495607859
872 29.0608872708228
873 29.3318506614049
874 29.1588403573935
875 29.0707306009792
876 29.9534412031395
877 30.5659641235442
878 30.3529221431893
879 31.0260367641156
880 31.323308616675
881 31.2718130532294
882 30.9012174938278
883 30.2703274454481
884 30.9522531191315
885 30.1346274963633
886 30.2199887551195
887 29.9977118944758
888 30.371556905661
889 29.6371339218861
890 29.1808499930613
891 29.9205401329986
892 30.0844940274835
893 30.4218830330673
894 30.1585345446542
895 30.0188021453369
896 29.161515657132
897 28.603936362284
898 27.9766666589225
899 27.7410979901695
900 28.2864797591617
901 27.3884963252493
902 28.1168043689265
903 27.9307040664993
904 28.130210139904
905 28.7889497115536
906 29.1254880225871
907 29.2428755426984
908 29.0040217511842
909 29.2405049571573
910 29.0364585337514
911 29.7065948988164
912 28.476474720805
913 28.7859710018776
914 27.9754480236071
915 27.4991941109684
916 27.2663332940846
917 26.6234516425324
918 27.3951092675052
919 26.8620584477152
920 26.3343708609497
921 25.7561616240196
922 26.4304206739321
923 26.6724722050715
924 27.1658681717303
925 28.2081522462546
926 28.2915490044126
927 28.9784869680027
928 28.7950956761801
929 30.2429745595908
930 30.4333621255192
931 30.7654480112417
932 30.917823260152
933 30.1504202180966
934 30.6020801314232
935 29.5077238361083
936 30.320973831654
937 30.3841354159517
938 30.402509213187
939 29.8086119506419
940 30.1368879299519
941 30.150382384259
942 29.9600849340677
943 30.1386909747404
944 30.1599522344348
945 30.5718619796538
946 30.2534969322923
947 29.8185075238461
948 29.5971456590986
949 29.0014718037706
950 29.0343891092902
951 29.5873474242171
952 29.5123651517986
953 29.7590365636845
954 29.1377914569225
955 29.0294062313894
956 29.3498657763257
957 29.2152380025272
958 29.7799839734903
959 29.6639206072759
960 29.5731746481099
961 29.1035203249179
962 29.5794639020612
963 29.641496381042
964 29.4634979783256
965 30.6071472726683
966 29.4682969627497
967 29.3841497460918
968 28.7989039887055
969 29.5590815817493
970 29.7253012697056
971 30.2676938180147
972 29.9855413278414
973 31.1574450032216
974 31.4227013300517
975 31.1111971550096
976 31.8415385578739
977 33.8069517065968
978 34.0414800130573
979 34.6214281809207
980 34.8111856499971
981 34.8420531162851
982 34.7648991324364
983 34.5502921933881
984 35.4316021447354
985 35.3098962825092
986 35.5747501381993
987 34.500494258941
988 34.795292992956
989 33.3196307912982
990 34.1431708931714
991 33.7735550133333
992 33.895653454842
993 33.5276139975925
994 32.5427893807388
995 32.0205663737696
996 32.3990392134042
997 31.876991430688
998 31.7087286953686
999 32.4924853143613
1000 31.0686546023875
1001 31.058627698902
1002 30.4297397274805
1003 29.7589565158521
1004 29.5190996212204
1005 29.5145035550837
1006 28.4719121078224
1007 28.7796633679895
1008 29.259530053104
1009 29.0555626191427
1010 29.7267611217378
1011 29.9087586433131
1012 30.8972524802037
1013 30.8832865484564
1014 30.7783837193678
1015 30.2264393339511
1016 30.3593611253517
1017 29.6944057410928
1018 28.9931972347095
1019 28.6008390157392
1020 28.3107666413006
1021 27.8043666432472
1022 27.0366905810925
1023 27.9803342717477
1024 28.4045341711312
1025 28.4635298142407
1026 28.1246194378176
1027 28.5520947232512
1028 28.7756891887164
1029 29.0592089544004
1030 29.3068177696022
1031 29.6740768380455
1032 30.2326680545768
1033 29.4880170615028
1034 29.2696182046917
1035 29.9509504980968
1036 30.8328688438076
1037 31.1430248750075
1038 30.8513500833833
1039 30.8271437138631
1040 30.810470418156
1041 30.2862477742229
1042 29.6109763371186
1043 29.2434165318535
1044 29.3357944603902
1045 29.4448214422807
1046 29.4265655326896
1047 28.8700391405684
1048 28.7198541851576
1049 28.6690323028195
1050 28.2233451061117
1051 28.7318007505603
1052 29.2819844160789
1053 29.4627261674712
1054 29.3566513528375
1055 28.8981663016841
1056 28.8610864929607
1057 28.6289278600623
1058 28.5766140352491
1059 28.3752398350339
1060 27.9948307217093
1061 27.7595047825388
1062 26.8955313692608
1063 26.5639506280849
1064 26.8769809240295
1065 26.6713752209689
1066 26.2903179540752
1067 26.9452317494945
1068 27.3741355270717
1069 27.3502885693934
1070 28.2963596134065
1071 28.4241221207353
1072 29.019396440052
1073 29.5430160060465
1074 29.8730811176905
1075 29.5181972753195
1076 29.3945723777652
1077 29.0553284848832
1078 29.3448475263106
1079 29.6827303394029
1080 28.864212921526
1081 28.59870021524
1082 28.5363329545444
1083 28.6371315840765
1084 29.1701667639235
1085 29.2900638533022
1086 29.1006042208079
1087 29.1388309060914
1088 28.7121564108327
1089 28.5954742530642
1090 28.372070790053
1091 28.0973351459618
1092 28.4202502360251
1093 27.9351763232186
1094 26.6718760950181
1095 26.6749060366974
1096 27.0593711107356
1097 27.8753347714314
1098 28.4466434383786
1099 28.8835616924236
1100 29.2307778550406
1101 30.3129172339799
1102 30.0006957931959
1103 29.9096518983113
1104 30.9071777828164
1105 32.157487063044
1106 33.3992778006907
1107 32.9105639547276
1108 32.797296404932
1109 32.2479530604129
1110 33.2770705598494
1111 32.314919628008
1112 32.3163854018815
1113 32.9749729742954
1114 32.056196835521
1115 31.4427184902586
1116 29.769581412092
1117 29.6502277662669
1118 29.4796920862245
1119 30.3363330185278
1120 30.3391511301103
1121 31.5981441946572
1122 31.6897478589217
1123 30.9644963824892
1124 31.8905331642302
1125 32.0123426267791
1126 32.3745049814156
1127 32.1214314945904
1128 32.1338776187828
1129 31.6789254979902
1130 31.9137606883787
1131 31.0713413576857
1132 31.2768241422294
1133 31.9190455174724
1134 30.9640348485248
1135 30.1556082358026
1136 30.8262794866689
1137 31.7925192407666
1138 31.335199260805
1139 31.1943187750328
1140 32.4485703829755
1141 32.2604246920905
1142 32.3898783751535
1143 32.5310462346887
1144 33.0005923281387
1145 34.3775552783342
1146 33.6657805166956
1147 33.3812383231256
1148 33.61952318385
1149 33.5685061791592
1150 30.8878432246907
1151 30.9601045965488
1152 30.4122121426019
1153 29.9138270810082
1154 29.0939942800518
1155 27.9932051657831
1156 28.4477206823408
1157 27.7686162171468
1158 28.4016188966651
1159 28.420549903996
1160 29.2862177429418
1161 29.9285811658774
1162 30.846480017635
1163 32.6645892005883
1164 34.4840751438094
1165 34.5005602748559
1166 34.5656897386448
1167 35.0631564276868
1168 34.4718056352289
1169 34.3952099368545
1170 34.1009889428053
1171 35.3782463256937
1172 34.546306532041
1173 32.8302050480938
1174 31.6676785875568
1175 32.1407204381464
1176 31.2943307273642
1177 31.1312378558759
1178 31.9603745305981
1179 32.1331622456855
1180 31.965537303211
1181 30.2134187597257
1182 30.6078013209903
1183 31.0376575365927
1184 31.0235495595947
1185 31.0810091688099
1186 32.0071582192818
1187 31.5784175715532
1188 30.2819226725415
1189 29.6408450657833
1190 29.5579154572281
1191 29.7167643534412
1192 29.3118614429699
1193 28.8638725836511
1194 28.6766395875185
1195 28.4371874706759
1196 27.9718213467754
1197 27.9466966408565
1198 28.8302396263789
1199 29.4535016148285
1200 30.1566245901973
1201 30.3652890589512
1202 30.3155136744402
1203 30.70743984225
1204 31.3043448682943
1205 31.7581625874332
1206 31.6347369595175
1207 31.9299952567721
1208 32.2508121428705
1209 32.1326981854954
1210 31.1218533189915
1211 30.9873168214405
1212 31.4388786273421
1213 31.3663125180034
1214 31.4085065779544
1215 30.6053147658188
1216 30.6656674193902
1217 30.2419044469001
1218 30.330348580356
1219 30.0052422003595
1220 30.070206056221
1221 29.3012095536874
1222 28.9952381988299
1223 28.4620077857538
1224 27.8843819452
1225 28.2341033447241
1226 29.0570807627509
1227 29.529522739227
1228 29.4333905668898
1229 29.9417490445965
1230 30.0200236846111
1231 31.1200176546978
1232 31.4971981063401
1233 31.7840520646625
1234 32.5227158409143
1235 32.5146942919736
1236 31.5983101201908
1237 31.7189121314916
1238 30.9961724821655
1239 30.5232635893622
1240 30.8642173214984
1241 29.9366509988655
1242 29.8746914712493
1243 29.5849555535587
1244 29.4092618522336
1245 29.021877835298
1246 29.6892670798931
1247 29.5026251244185
1248 29.5559661169659
1249 30.0782847550936
1250 30.2848584160727
1251 31.3634949308353
1252 31.6454135257557
1253 32.8981404756981
1254 32.0717759974591
1255 32.3326718734421
1256 31.835137079583
1257 32.1188227668396
1258 32.1716755725315
1259 31.9499256900722
1260 31.495058013007
1261 31.4389223985153
1262 30.3468855834434
1263 29.1280657580812
1264 29.2128826675517
1265 29.2759066795727
1266 29.1208952471711
1267 28.7839603640873
1268 28.8491492280821
1269 29.1829489350794
1270 29.1431144018697
1271 28.763847059483
1272 30.5510950497289
1273 31.1903003404657
1274 32.630678884696
1275 32.672312144522
1276 32.1730960878776
1277 31.6831034066044
1278 32.0260380393567
1279 31.8365889907834
1280 31.5976754573269
1281 31.1418751581464
1282 30.0784297514194
1283 30.3093349640388
1284 29.0199886654436
1285 28.8611741793614
1286 28.9489090742335
1287 29.8234234857207
1288 29.8601335580514
1289 29.4560306367455
1290 29.6595848619429
1291 29.7428673124151
1292 29.4450634376639
1293 28.8909332823359
1294 29.4285890844822
1295 28.9542075739809
1296 29.616870841975
1297 28.9584249404009
1298 28.8149092957567
1299 29.2645748437214
1300 30.6819223447967
1301 30.4018778364125
1302 30.3224643447916
1303 30.4785989683874
1304 29.8588294442567
1305 30.4975428774786
1306 30.8019267349548
1307 31.4676773782838
1308 30.8358438307006
1309 30.6739158428944
1310 30.1911263927245
1311 31.055056713169
1312 30.9782696982404
1313 30.7161919758173
1314 31.2041867214376
1315 30.5330358661946
1316 30.0151107434201
1317 29.6148740683323
1318 29.5128258670391
1319 30.2887877563006
1320 29.71657973797
1321 29.0416459836335
1322 29.463285555814
1323 29.0414495297094
1324 28.8891769463362
1325 29.6198151907341
1326 29.404225527055
1327 29.9507965153928
1328 31.5440997242391
1329 30.6759759837983
1330 30.5157675081357
1331 30.3839023504783
1332 29.9385739410412
1333 30.8683995184213
1334 30.995171583993
1335 31.0064167265775
1336 31.3807400947969
1337 30.9646405010765
1338 29.3188804034661
1339 29.3851269145213
1340 30.1113326060172
1341 30.189167863094
1342 30.9683753134888
1343 31.5018385189449
1344 31.7897093308823
1345 31.5485531739499
1346 31.0977471703595
1347 30.8792293077156
1348 32.789933871964
1349 33.0720978054279
1350 32.6944374335748
1351 33.5575711723512
1352 33.2309656250667
1353 32.5521755207029
1354 32.8191124560397
1355 33.2613405525561
1356 33.8303143628389
1357 34.0834913895829
1358 32.488671407285
1359 31.8044212462511
1360 31.4815300444944
1361 30.8300961721611
1362 30.4270285668596
1363 30.308225552854
1364 30.874484066653
1365 30.5406842631352
1366 31.1697838615155
1367 31.5695117567026
1368 31.8604214140547
1369 32.2747440733722
1370 32.0280512802172
1371 32.5826895302504
1372 33.4759818625345
1373 34.148409986274
1374 32.7934747863919
1375 33.4542440589622
1376 32.8859526163795
1377 32.5578727678823
1378 32.4564046962253
1379 32.5992898345305
1380 33.7079928652654
1381 33.4633417775622
1382 32.9364892702217
1383 31.9328407694975
1384 31.5593728812038
1385 31.5187284972958
1386 32.5691548033678
1387 32.64457630361
1388 32.8223503237685
1389 32.1595245129813
1390 30.9337167892919
1391 30.1551427213813
1392 29.5928740327818
1393 29.5549031653499
1394 29.5542676753333
1395 28.5509243606493
1396 27.2890304596967
1397 27.3203763670562
1398 26.7532394890329
1399 27.5648306646554
1400 27.7785575162672
1401 28.0743430726692
1402 28.2146896955555
1403 28.7975711352074
1404 29.0798047802951
1405 29.9929350510793
1406 29.8667527203606
1407 29.9472878221136
1408 30.1869814284991
1409 29.6123015640494
1410 30.0009174663585
1411 29.9037514686008
1412 30.6330475915221
1413 29.9052205468508
1414 29.9682923505599
1415 29.19858161342
1416 28.4804603479809
1417 27.4561010720385
1418 27.1221261690754
1419 27.8161927245498
1420 27.869603106655
1421 29.1614772646278
1422 29.1563828007694
1423 29.8399795502837
1424 30.1518577276943
1425 30.259148768503
1426 31.496937176315
1427 32.6478490584281
1428 33.3965710693803
1429 33.4689891495671
1430 34.2469629026479
1431 33.0614292697937
1432 32.6949754161422
1433 32.6948446917555
1434 33.145956936076
1435 34.7120901927414
1436 33.7236095807916
1437 33.5661910420329
1438 33.0381205612106
1439 32.5059088404557
1440 31.3649386480404
1441 32.2302771109676
1442 32.1879533785362
1443 31.9367895280183
1444 31.1963916957404
1445 29.9317631147232
1446 30.2025996445019
1447 29.6952345490632
1448 30.4868525902677
1449 30.4611508041461
1450 30.7413774057468
1451 29.7535210097615
1452 30.1669989950532
1453 30.4298870252057
1454 30.9092210640308
1455 31.212189186826
1456 31.5948467955946
1457 32.3553448661175
1458 31.6841090331184
1459 31.7733565503023
1460 31.7492731057727
1461 32.2441443014406
1462 31.476951403224
1463 31.6354108805601
1464 31.1044774616648
1465 31.3807281497674
1466 31.602549878724
1467 31.0990691300997
1468 30.8370715839851
1469 30.8760781699741
1470 31.5411446620431
1471 31.1613595591352
1472 33.2066265693581
1473 32.4219244588813
1474 33.2246897854592
1475 32.1901561779557
1476 31.5929506418662
1477 31.6378480979883
1478 31.5365346970105
1479 31.4944048557539
1480 30.7580205333062
1481 31.0207027444392
1482 29.1392987733934
1483 29.7191776593403
1484 28.9020113724205
1485 29.3014776120803
1486 29.897324082085
1487 29.5346225426474
1488 29.6140623623162
1489 31.0220071911189
1490 30.8427740766602
1491 30.7323296603068
1492 30.9538214340587
1493 30.5949581274461
1494 30.5296804011096
1495 30.465667148595
1496 30.2240458649823
1497 30.802682809411
1498 31.0297378107786
1499 30.1724155754795
1500 30.3051197301708
1501 30.0222604743597
1502 30.1999683943511
1503 29.9546233496497
1504 31.5441610419874
1505 31.639472970038
1506 30.9905742520355
1507 30.4784238973592
1508 31.1880518968176
1509 30.946372136584
1510 31.8139637512171
1511 31.9950555269041
1512 32.1018032455619
1513 32.0783423711333
1514 30.9098060445455
1515 30.793960983357
1516 31.9955149293351
1517 32.346012819529
1518 31.2732153481949
1519 31.3175827692944
1520 30.8618055766303
1521 30.6452123765206
1522 31.3555767813245
1523 31.4861645793011
1524 32.1634847474277
1525 32.0466762397744
1526 31.1198652222667
1527 31.2287490054184
1528 31.2809540109472
1529 31.0380898690421
1530 32.0207448238152
1531 32.9291604367867
1532 31.7988282606281
1533 32.3599624580026
1534 31.5672335026153
1535 31.2338517115648
1536 31.6219876995367
1537 31.2677568191471
1538 31.2798817804818
1539 30.8835770172043
1540 29.2128678895223
1541 28.479177398078
1542 27.9975669196244
1543 27.3937126015659
1544 27.2116708804011
1545 27.8626597065703
1546 27.7583954889533
1547 27.9160606133038
1548 28.3055424109954
1549 28.3112118713592
1550 28.5052487028167
1551 28.8305538390453
1552 29.0846975279534
1553 29.1792979740543
1554 29.6365804295724
1555 30.0156882509572
1556 30.1589110382484
1557 30.0971149912175
1558 29.790528199174
1559 30.0617311598377
1560 30.3114800416927
1561 30.1198476394255
1562 30.5088850997566
1563 30.178507356911
1564 31.5325200418615
1565 31.1263425368236
1566 31.7275978588498
1567 31.2409884142817
1568 31.4439762840103
1569 31.2308746635395
1570 30.6269475792768
1571 30.7618640085463
1572 30.5790152188404
1573 31.4461949683767
1574 30.1218642501757
1575 30.6422793046802
1576 31.9334108946848
1577 32.7237939852333
1578 33.7342120219021
1579 34.2285662114334
1580 34.7228928910732
1581 34.7436985418443
1582 34.8346553840726
1583 37.153805321913
1584 36.9359845993981
1585 36.3915258939496
1586 35.3759570094022
1587 35.1350238635981
1588 34.3803055793749
1589 34.1582222751137
1590 33.9035274958596
1591 34.0384249580335
1592 33.9228431964481
1593 32.6624928181768
1594 32.630655158604
1595 32.5132095999611
1596 31.9436666973432
1597 31.7654038293115
1598 32.0270659801655
1599 31.803550513855
1600 32.2996287169214
1601 33.5196675955247
1602 34.6832305401193
1603 33.7518876058106
1604 34.2587934360461
1605 34.1167988236463
1606 32.9026130818011
1607 33.1664490052284
1608 33.0181790564178
1609 32.8001517766552
1610 32.2292253011429
1611 30.6888853001943
1612 29.6626493331035
1613 29.2315745326109
1614 28.3527426683315
1615 28.8208273951634
1616 30.5663326989085
1617 30.362967878789
1618 30.3270297783951
1619 30.4398936800536
1620 31.3166696168905
1621 31.7451484226736
1622 32.5942998022417
1623 32.5386821100497
1624 32.3836289212425
1625 32.5195724668235
1626 31.9422154484924
1627 31.9380964437667
1628 31.6699162421271
1629 32.7003932726626
1630 32.9328058268429
1631 32.8460006487454
1632 33.4762763778356
1633 33.7860621660095
1634 33.6884650300158
1635 33.2206051131224
1636 32.4475112462262
1637 32.539563784974
1638 32.2132078902897
1639 31.1325566886018
1640 30.4008803194912
1641 30.4555650794867
1642 29.5954074794156
1643 29.5943955004507
1644 30.9252739874098
1645 30.9755821542715
1646 31.6431581073615
1647 31.6799716808332
1648 32.137683320787
1649 32.8339460782537
1650 32.5710291141415
1651 32.5616268616669
1652 31.9146732310406
1653 31.396986752306
1654 30.8579491668521
1655 30.6115923525329
1656 29.659859758622
1657 29.6776132881952
1658 29.5993584121165
1659 28.9420390585825
1660 30.3071532701997
1661 29.9810879754367
1662 29.6994728674167
1663 30.3655591882273
1664 30.4646822344719
1665 31.4789799401764
1666 33.1435603987391
1667 33.3609329721779
1668 34.8177522156378
1669 35.3016254204005
1670 34.2748851054702
1671 35.0691543655401
1672 36.1790655920171
1673 35.8107576615669
1674 34.8654827210448
1675 35.8041866163804
1676 34.8555695353198
1677 34.8930144145183
1678 33.5133899759975
1679 33.6298779476361
1680 34.8233980584252
1681 34.6314724758762
1682 34.7432217915893
1683 35.1320426023685
1684 35.6267561765397
1685 34.6432150162405
1686 35.4670806709451
1687 36.5123322370782
1688 36.7783487295578
1689 37.7646431742402
1690 36.370807964005
1691 37.1829484019625
1692 35.9711997817388
1693 35.7578979162222
1694 35.278461142702
1695 34.8997480843031
1696 34.4034909879576
1697 33.4556655243591
1698 33.4923961123131
1699 32.932939234153
1700 33.4023361949024
1701 33.6774509667261
1702 34.2126506991634
1703 34.5033934446539
1704 35.0371831890273
1705 35.2696690087207
1706 35.1073972761309
1707 34.7406463940741
1708 34.0819626040589
1709 33.3990434958365
1710 33.9281450636269
1711 32.4413741846781
1712 33.7425478784796
1713 32.8493988333051
1714 32.9988364522181
1715 32.6896375199418
1716 32.4878319181085
1717 32.6845229026354
1718 33.5073025132287
1719 34.3340499266445
1720 34.3400211534915
1721 34.9234269222899
1722 33.3883702472871
1723 33.7020517608744
1724 34.151996774873
1725 34.6749155383408
1726 34.8768130673015
1727 34.7532741273117
1728 34.4627970161087
1729 33.9048940138603
1730 33.9319412043545
1731 32.9946822224297
1732 33.3223162067621
1733 32.5703671016969
1734 31.5956442789345
1735 31.3982288966581
1736 30.9540697052012
1737 31.6972096801282
1738 31.4699745303099
1739 31.4255025084592
1740 31.0641562979097
1741 31.9062944224469
1742 31.3316839739972
1743 32.2940949038296
1744 32.4329725063408
1745 31.721899834815
1746 31.8943789528668
1747 30.7921223871189
1748 31.1929537526173
1749 31.0137285175498
1750 31.0603656827781
1751 29.8378609137747
1752 30.6930484385725
1753 29.8754404089189
1754 30.128750107599
1755 30.1103744582628
1756 29.882945284782
1757 30.804894954922
1758 29.7985071740095
1759 29.6630141272754
1760 28.7210598957817
1761 28.9409082822661
1762 27.5329400233455
1763 28.3026099054531
1764 27.8952842620398
1765 28.1789378859317
1766 28.3787994271472
1767 27.3182496335484
1768 27.4030058961069
1769 27.3021713939765
1770 27.6757398002525
1771 27.5267038278553
1772 27.7758043210064
1773 27.8990256542184
1774 28.5586959103594
1775 28.3499362470941
1776 28.4424008529991
1777 28.2856385314259
1778 29.6321894871289
1779 29.5777131606153
1780 29.8348806949762
1781 30.3144879707269
1782 30.0219642681554
1783 29.524460946867
1784 29.2767381663828
1785 28.9781727332559
1786 28.687582372947
1787 29.0205724490604
1788 29.3464676390484
1789 29.3193788029696
1790 29.6764546167004
1791 29.638159730302
1792 30.64109126647
1793 30.8138396850156
1794 32.030226816246
1795 32.1139628044925
1796 32.8674986740399
1797 33.001772971815
1798 32.0368144238442
1799 32.071643331892
1800 31.2118027860255
1801 30.9523614890441
1802 30.8056195394594
1803 31.4954794017775
1804 30.5476543425935
1805 31.6254483837422
1806 31.039525614048
1807 30.740190768153
1808 30.5051188800228
1809 32.0282661723787
1810 32.079970250663
1811 32.0040329038953
1812 32.3214330876857
1813 31.5720606887737
1814 31.0720996961476
1815 30.4778196793997
1816 30.5422532665593
1817 30.7350198190631
1818 30.6903360212936
1819 29.9993656455078
1820 29.783814641407
1821 29.500356679823
1822 28.6202923966956
1823 28.4737171710218
1824 28.8054888535233
1825 29.4801859335027
1826 29.7824769657016
1827 29.8174207498632
1828 29.547104697471
1829 29.1485366317569
1830 29.3521572520987
1831 30.0616896104832
1832 30.3437321990518
1833 30.2415678690415
1834 29.5070228083808
1835 28.7948854796418
1836 28.2029030677143
1837 28.859723727607
1838 29.2328220050382
1839 28.6821813986612
1840 28.6884102736794
1841 27.9977178587486
1842 28.3076164634035
1843 29.1237904608838
1844 29.7195177653225
1845 29.7289690800645
1846 30.0056887665495
1847 30.4956257116273
1848 30.4119856467767
1849 31.9583046967788
1850 31.4009113618521
1851 31.4117844005184
1852 31.3896301033497
1853 30.5130189858844
1854 30.4776828110888
1855 30.4390583427821
1856 30.5306015888228
1857 29.2435410565424
1858 29.0667858016033
1859 28.5010900618256
1860 28.7667903331528
1861 29.3420606689404
1862 29.1479914036502
1863 29.8019480194725
1864 30.5415201528759
1865 30.2682650089382
1866 31.0275683252023
};
\addplot [semithick, darkorange25512714]
table {%
0 56.1312842403538
1 58.9883425037856
2 61.1976707732338
3 64.0732263376307
4 65.117479022753
5 65.6823110859843
6 65.4633597393894
7 63.5199632238782
8 60.622621794092
9 57.9559222464011
10 55.509226247828
11 53.5728757504563
12 52.045130220376
13 50.6153583004393
14 49.0767827618167
15 48.1395806689615
16 47.2749163348492
17 47.2791435423364
18 47.7262186325758
19 47.6579139745653
20 47.3724466293155
21 46.1057972590075
22 44.5476121621572
23 42.7420479743729
24 41.3950509389498
25 39.8480795890745
26 38.3856272472972
27 37.0146567963217
28 35.6227333618562
29 34.0332358924971
30 32.5301097794832
31 31.7183506682557
32 31.3253923799843
33 30.9158846386705
34 30.4923466130461
35 30.0563050848074
36 29.7487584468586
37 29.3249679143555
38 29.2147895301784
39 29.138660470361
40 28.9481302855867
41 28.781143071185
42 28.3959747393209
43 28.4660209505276
44 28.5290091906497
45 28.7865151082437
46 28.6687786712571
47 28.8131151945996
48 28.5178361350485
49 28.5459402657752
50 28.6022246943747
51 28.9035841399721
52 29.1654877697916
53 29.154016448986
54 28.9797688431103
55 28.8409255921613
56 28.9901911737146
57 28.9128987295298
58 29.3966817296046
59 29.5018562143511
60 29.6526788150937
61 29.5302122917418
62 29.3197673493339
63 29.2916004410158
64 29.1260460133808
65 28.9961500154315
66 28.7634623412477
67 28.8123993616814
68 28.503815574901
69 28.4414475968092
70 28.4475084954907
71 28.3648654718947
72 28.4797131316577
73 28.3415130079359
74 28.2943452213497
75 28.2705283459202
76 28.2713424483948
77 27.7991753012672
78 27.5094746449004
79 27.340736302448
80 27.1512218027137
81 27.0870673753927
82 27.1178461912672
83 27.1564535659503
84 26.9974761175787
85 26.7532943737273
86 26.5132003276643
87 26.6665070161637
88 26.7144517667146
89 26.7907990094191
90 26.9252093245462
91 26.6599718831035
92 26.6295741069866
93 26.6515168293978
94 26.8100782696753
95 27.1655773059861
96 27.1308358185689
97 27.1317989988352
98 27.4251737705388
99 27.1513432953672
100 26.9615790714434
101 27.025960266488
102 26.6098012663471
103 26.3059579469814
104 26.0010588603209
105 25.552237642815
106 25.8102789658892
107 25.7145446307259
108 25.4425614561862
109 25.3842967446431
110 25.2285957473563
111 25.0904902985193
112 25.4224691145503
113 25.6323399893474
114 25.5270091739223
115 25.4425984806706
116 25.3285804384826
117 25.54552306758
118 25.5755459983343
119 25.7260140649047
120 25.9303113025764
121 26.0755940882061
122 25.8932922683221
123 25.9508403420339
124 26.3006223380363
125 26.631027072485
126 26.5764881105261
127 26.4301688147457
128 26.19600237711
129 26.0519960786071
130 25.7325690687995
131 25.7322869585092
132 25.51727043505
133 25.1911923957782
134 24.9706408954717
135 24.7733832258382
136 24.7794672141432
137 24.822906010343
138 25.0287659014978
139 25.0730567928069
140 25.2823042523402
141 25.0598466572241
142 25.1634556557387
143 25.0820509318477
144 24.9808393486236
145 24.750624213292
146 24.4939765872844
147 24.0654908341918
148 23.7349914810044
149 23.7818713175059
150 23.4807525950379
151 23.7774183434796
152 24.0812586183766
153 24.4612163542725
154 24.6359587205412
155 24.9964911153638
156 25.5194171193112
157 25.8220441738745
158 26.0072337933391
159 25.9551266172708
160 26.1444364464605
161 26.0954607855743
162 26.0252995861698
163 25.9861930115761
164 26.038227069935
165 26.1276451520623
166 25.9974597702369
167 25.6705741008414
168 25.8246679747317
169 25.8261194652092
170 26.0919713799422
171 26.2372465984893
172 26.0869278570592
173 25.8542057498828
174 25.7919909120656
175 25.8752716801841
176 25.9211371134973
177 26.0486208480845
178 26.2551280308921
179 26.5751727889527
180 26.5236267813078
181 26.4762621495491
182 26.7712915220996
183 27.0486084501422
184 27.2226572167665
185 26.9529728991741
186 26.8661647774663
187 26.932614818407
188 26.5981860556069
189 26.4604326967183
190 26.2966468918285
191 26.183443244331
192 25.867553691886
193 25.62588980324
194 25.4713452388507
195 25.6005443351914
196 25.7144801465871
197 25.8690790023956
198 26.1598339922922
199 26.3492477708145
200 26.4026794197768
201 26.334911328017
202 26.2415855840978
203 26.2879458452726
204 26.2053329107757
205 26.11575714867
206 26.1931125252277
207 26.2356038095021
208 26.3891924134371
209 26.0506890743838
210 26.41952390974
211 26.810784169938
212 27.2399925891218
213 27.3256842859549
214 27.3999650531442
215 27.0814604908699
216 26.5679558362127
217 26.6872872445139
218 26.4531792996483
219 26.3818606534584
220 25.9814153213079
221 25.6983544813242
222 25.2254928999636
223 25.1719386660453
224 25.2738495946105
225 25.5721230468146
226 26.0670327135281
227 25.9960091902681
228 25.7337211872571
229 25.9714755806927
230 26.2164230655049
231 26.2584587194543
232 26.4296596530998
233 26.4491828952115
234 26.2657679732079
235 26.1442974331938
236 25.5167333582866
237 25.5559375945307
238 25.6775498639886
239 25.4322440444007
240 25.6549817877534
241 25.313696635773
242 25.2548913613771
243 25.2294583199304
244 25.4240649316151
245 25.3863405899108
246 25.696424931214
247 25.3644802294057
248 25.7820722280463
249 25.7329938124757
250 25.1364785789508
251 25.4383925889273
252 25.9807315981233
253 26.1741115426684
254 26.3150700869784
255 26.5522152689923
256 27.0619625715699
257 27.6707330559545
258 26.9653135205371
259 27.2024288033935
260 27.2058348066517
261 27.6345543915034
262 27.0164201480235
263 26.7124954321425
264 26.2683015425565
265 25.8328174806423
266 24.9116990350055
267 24.7035619871565
268 24.7617753915024
269 25.285992437244
270 25.5330036198108
271 24.8433102159418
272 25.8601973961482
273 26.1068203803649
274 26.1332868651055
275 26.8444404034797
276 27.5785089850467
277 27.4340552828154
278 27.5482831180864
279 27.1933559828193
280 26.8954039537169
281 27.2042739856219
282 26.7905423122857
283 26.5284195829318
284 26.7300409625498
285 26.3010566590182
286 25.9411554845333
287 25.6061857756909
288 25.613772090155
289 25.0579860531904
290 25.0023912989184
291 25.0181692758828
292 24.0689414260777
293 23.8883615545216
294 23.4443710109456
295 23.2924657430201
296 23.5385645869252
297 23.6840053291333
298 23.4597574895746
299 23.4547093421967
300 23.5248177696839
301 23.4275442534143
302 23.7749880107591
303 24.044586105667
304 24.3457468622817
305 24.5952636804441
306 24.297230156202
307 24.5441876019192
308 25.9612403051541
309 26.3648045051018
310 26.578489155926
311 26.6201763831239
312 27.243762963566
313 27.6094604124162
314 27.761051490789
315 27.6928880280662
316 28.2794041590011
317 28.0306059144163
318 27.1944255403076
319 27.2064230712842
320 27.5585491566135
321 27.3968942800011
322 27.1106659910228
323 26.5931822508236
324 27.0325478594132
325 27.3309600910596
326 27.5921057700719
327 27.6734869189273
328 27.5451981227168
329 27.5841707326875
330 27.1217832988046
331 27.6586353038874
332 27.8534493748
333 28.5448177145105
334 28.1687555539213
335 29.0861420823676
336 28.4686527776256
337 28.683948391684
338 28.6424489911857
339 28.8084280874652
340 29.6241620186477
341 29.2014416335805
342 29.1786298527668
343 28.6755865106226
344 29.1372534665565
345 28.4083195694705
346 28.3014859294302
347 28.8895219500149
348 29.8707417841076
349 30.4807904719782
350 30.1363757507513
351 31.0248140634524
352 30.9059108229501
353 31.2827875629965
354 31.1737203676992
355 30.9093335588725
356 31.1321406106782
357 30.9087032343132
358 30.4207827935413
359 30.0738978589302
360 30.4615409714713
361 29.8124310714678
362 29.5275525163577
363 29.7249341974659
364 29.4939867564432
365 30.1852619443644
366 29.9850179266812
367 29.5419089295512
368 28.9511163384221
369 28.9155830656679
370 28.8060022873582
371 28.8014600825074
372 28.8387599028381
373 28.2386984057233
374 28.1676361081768
375 27.7020067770639
376 28.4749022737871
377 28.4758913122693
378 28.892097501829
379 29.504653175822
380 29.4834756759502
381 29.2056714244168
382 29.6477263878948
383 29.5232593402941
384 29.1201832114688
385 28.8666494772172
386 28.4047179303101
387 28.1237449749447
388 27.8776370757232
389 26.766113714905
390 26.0990478121854
391 26.0009359854084
392 25.4301396150744
393 25.5099220811969
394 26.0830181449041
395 26.0361255640783
396 26.2346093121528
397 26.4923285546949
398 26.3141707809256
399 26.3026279183106
400 26.7311188682045
401 26.9950439316415
402 27.6184423253134
403 28.3521980438259
404 28.1102818343148
405 28.2409315232078
406 27.7087067825376
407 27.4738771771222
408 28.3153020361915
409 28.5750289475214
410 27.8195777095684
411 28.2236040780154
412 27.8180178173182
413 28.0091288417888
414 28.1087183674937
415 28.5854424189088
416 28.9167319541898
417 29.2545260373355
418 28.8630572452744
419 29.6580200271403
420 30.1277628305131
421 29.7113498882707
422 29.4636249390618
423 28.660779890627
424 29.0902677302954
425 28.4466432918455
426 28.9724976572711
427 29.1403996992153
428 28.9935359269773
429 28.2481820952862
430 28.2824353246182
431 28.8483962178245
432 29.1475808542084
433 29.4528230690805
434 29.1554794513178
435 29.5506650713005
436 28.5165259473162
437 28.2328730679163
438 28.1053501391495
439 28.0654683956865
440 28.2962953512789
441 27.6627934321102
442 27.4040030356893
443 27.0674188428731
444 26.8059997812055
445 26.6213328382734
446 26.8082903036665
447 26.8865968906892
448 27.3834946582088
449 27.4918271922176
450 27.7401039332447
451 28.2748190653616
452 29.4486961204559
453 29.4746320792284
454 29.420174755191
455 29.5110602383902
456 29.4395648337692
457 29.3940348008143
458 28.6379366560878
459 29.3714034425611
460 28.7884752454997
461 28.5881829410908
462 28.0225434038478
463 27.7715336577562
464 28.7016639650481
465 28.5585564334242
466 28.8904570824118
467 28.7720751125034
468 29.257763032311
469 27.8573134860238
470 27.6786162796394
471 27.8338163781529
472 27.5309402521482
473 28.233287806217
474 27.5278635518592
475 28.3289804440129
476 28.6310181155846
477 29.1965061759656
478 28.7773877163997
479 29.748137828707
480 30.1384192725317
481 30.126987250786
482 29.7985998991211
483 29.6593666245755
484 29.6353517901322
485 28.5858419146458
486 28.2484319956882
487 28.1654753961214
488 28.0693732370265
489 27.6131268019659
490 27.6871371097153
491 27.4935778648949
492 27.824425174253
493 27.4946445672512
494 27.278113200027
495 28.0115261184458
496 27.9180050190586
497 27.7491138761522
498 28.1005036541749
499 27.977467537558
500 27.542989963939
501 27.1334252549674
502 26.9583535137023
503 27.0101117368449
504 27.6476962592019
505 27.2144683635213
506 27.2108873520073
507 27.6632274965042
508 27.8344908974147
509 27.9185004017814
510 28.9879506805929
511 28.9135946682917
512 29.3910784099116
513 29.4766282078278
514 28.4978121501116
515 28.7520543761944
516 29.2173932491395
517 28.2987163902547
518 27.7608136386979
519 27.811130182175
520 26.8366049018931
521 27.0701272332784
522 26.9961848070778
523 26.9006993868355
524 27.4208018102007
525 26.753978441113
526 27.138276016811
527 27.4454381961351
528 28.0373257246233
529 27.7434700568394
530 28.5653876136142
531 29.1789930304601
532 29.1455255555129
533 29.2539056683192
534 29.6967344445598
535 30.185343336357
536 29.9075574172308
537 29.9385999017879
538 29.7435234575821
539 30.6468157951203
540 30.2948694509279
541 29.6030297228272
542 29.1332064503589
543 28.8968864949502
544 28.9119723135505
545 28.5680743309531
546 28.1321998241819
547 28.1175201855787
548 27.6412394608392
549 26.821272290673
550 26.6878029027697
551 26.6582199732246
552 26.9497214574887
553 27.220506488576
554 26.7340280655283
555 26.9038721033336
556 27.1152306553037
557 26.8611378204677
558 26.668727162839
559 26.7373688578098
560 26.1549222633579
561 26.128106803809
562 26.0804701492981
563 25.7734233867666
564 25.7107568449621
565 25.9308884833177
566 25.3485595780193
567 25.4595001851659
568 25.7046657351445
569 25.3248588734132
570 25.4442940801783
571 25.7589780694468
572 25.4552192878047
573 25.266312220049
574 25.8968344948819
575 25.5436779360575
576 25.9393229103672
577 25.7764827102744
578 25.7534436725532
579 26.0130888164566
580 26.1292933711298
581 25.9674636857487
582 26.3859298744332
583 26.6779677192258
584 25.7780152360636
585 25.7983405551176
586 25.3102310697216
587 25.5875100223512
588 26.1355891581483
589 26.1766390162652
590 26.4747564608586
591 26.5911323220052
592 25.7650341782641
593 26.1157685568396
594 26.6981009696232
595 26.6987034932843
596 26.7664175979887
597 27.83957055341
598 27.7941943063415
599 28.1781993216667
600 27.8195139645303
601 28.1190895567209
602 28.5074573692259
603 27.9026148715452
604 27.5848997488012
605 27.3822490872686
606 28.0001463497039
607 26.6137391397877
608 26.8536277484642
609 26.8329362629812
610 27.3861863697222
611 26.811089547739
612 26.7578488778511
613 27.500842862438
614 27.2806353591249
615 27.6894434560239
616 27.2447787277184
617 27.3958408462103
618 27.183293142715
619 27.0450358816671
620 26.9537819824371
621 27.6890245398858
622 27.9431643167594
623 27.3657495245297
624 28.3031306812706
625 27.9674152195323
626 28.1810062902093
627 28.8974447921436
628 28.357348703239
629 28.3302864970221
630 27.6042810153814
631 27.0026488213362
632 26.5829848057433
633 26.6481326844946
634 26.0821103023086
635 26.1678409371099
636 26.1412005292159
637 25.2406959887333
638 25.087312311382
639 24.4726845316201
640 24.9321930117146
641 24.815757968653
642 24.9865164634955
643 25.0498488817327
644 25.3364899634278
645 25.4663534323416
646 25.5410878495844
647 26.3159771070705
648 26.7894515210425
649 27.208676702048
650 27.075364166863
651 27.2258596500551
652 28.0284719179421
653 28.0931907379795
654 27.8942328677482
655 27.784573881558
656 27.5619951438321
657 27.1303156283459
658 27.0596737889966
659 27.0755009116256
660 26.9510905538332
661 26.9661982085175
662 26.3983644356273
663 26.4547794654031
664 27.0712232781692
665 27.3295320164321
666 27.3029865685104
667 27.4001934556835
668 27.6958730609838
669 28.0360711080858
670 28.381358962668
671 28.6024278653561
672 28.9901965939761
673 29.137137950735
674 28.1770075231285
675 27.8506698314565
676 27.9032583539048
677 27.9305724935939
678 27.8949474072824
679 27.489413102666
680 27.4968741380305
681 27.3835887249831
682 27.220431042466
683 27.2953292170021
684 27.6246096789623
685 28.0205107355012
686 28.322613861123
687 28.2720646188482
688 28.9778447745006
689 29.241921225827
690 28.794446888654
691 28.901080091786
692 28.5787955763131
693 28.2232731915521
694 27.9030946172507
695 27.3353120410646
696 27.1265835567459
697 27.2549818896268
698 26.8166221168202
699 26.7311205831597
700 27.3611371630903
701 26.9932119637889
702 27.7619575838672
703 28.164186548451
704 28.5148256605553
705 28.5815094661104
706 28.6305184401398
707 28.2706789342892
708 28.1416573643192
709 27.842973334712
710 28.0429655738799
711 28.6229239710122
712 28.1352523417577
713 27.6384332590152
714 27.1535132348574
715 27.2945295207546
716 27.4803519825467
717 27.699234875716
718 27.7897472861249
719 28.1412130244054
720 27.4952560760577
721 26.7519620043021
722 26.3764418625492
723 26.2492584074347
724 26.379849858581
725 26.3373758048747
726 25.7901092074809
727 25.9477619156618
728 25.5845910857507
729 25.6061988299076
730 25.3306709293232
731 25.5909942002711
732 25.6133386061777
733 25.5299819487533
734 25.948323134987
735 25.814547008983
736 27.077211383471
737 26.680246519365
738 26.4710330313299
739 26.652071852401
740 27.4174058696064
741 27.0524063832665
742 27.8668067603559
743 28.0696291508522
744 27.7326736935317
745 28.4574213968054
746 27.7462912773072
747 27.8529766610662
748 28.2868431963161
749 28.371592928351
750 28.6481361328745
751 28.857785907185
752 28.4713274975497
753 29.2087580826442
754 29.404094502989
755 28.771520610539
756 28.874921112385
757 29.0604402486185
758 28.3979362855308
759 29.0906529747564
760 28.4236476461661
761 28.2668521053804
762 28.4517460587949
763 27.7759243497282
764 27.7281282448033
765 27.8913871510743
766 27.6627440087567
767 27.7270501509769
768 28.4887892820816
769 27.3869747165629
770 27.4757986073248
771 28.0715306862198
772 27.6312731255025
773 27.3813278809411
774 27.2774540759969
775 27.1297484046775
776 27.5010003922253
777 27.476182828419
778 27.5769761160119
779 28.0270942680493
780 28.3015229830742
781 28.0728404057377
782 28.071774519748
783 28.8863230892429
784 28.7417454294793
785 29.6186500729484
786 28.9177431735643
787 29.0713910579827
788 28.4942495324647
789 28.3760903444639
790 28.1194152133609
791 28.3102933493487
792 28.4000289261841
793 27.9313946392812
794 28.009355711679
795 27.7539280862007
796 28.4000973776575
797 27.9897412189261
798 28.1641797600143
799 27.7222521122884
800 27.9063336845082
801 27.9781336451826
802 28.4086869900745
803 28.6137032856298
804 28.6950686444819
805 28.1327913581623
806 27.886603140921
807 28.1352806411801
808 27.9023756918876
809 28.1173710814704
810 27.8607838851894
811 27.5183409796947
812 27.1078096867026
813 27.5388823981639
814 28.2164992715832
815 29.0432650829686
816 29.0373393625234
817 29.0674693972377
818 29.3069632265073
819 29.1472494569614
820 29.0411567085858
821 29.5269397379054
822 29.2839070314638
823 29.2895900194477
824 29.2210313817826
825 28.5772272885658
826 28.6137322894804
827 28.6394715027477
828 28.4929323450053
829 28.7932621656456
830 29.2074896262388
831 29.0336167797517
832 29.765621290979
833 29.1234534273294
834 28.7852852280522
835 30.1782893395202
836 30.9741264617403
837 30.8059658276099
838 31.0886867056572
839 31.3873448655882
840 31.826276493911
841 31.6745830972284
842 31.3810337345807
843 31.37539127122
844 31.0233938363102
845 29.9534391020662
846 28.8046285158923
847 28.9292955491441
848 29.4304674816282
849 29.0765296819049
850 28.9401280392434
851 29.0177722989274
852 29.075551411627
853 29.2274360954689
854 29.9809021099819
855 30.140261125155
856 30.5152585366403
857 31.1758725614418
858 30.4980112708978
859 30.549108501017
860 30.2276584402424
861 30.4743823030079
862 30.408488100053
863 31.0512845398234
864 31.1775806924624
865 31.5087459077856
866 31.3040341145057
867 30.596372498649
868 30.7799901675527
869 31.0950475014953
870 31.4241274735421
871 31.2502925324478
872 31.5178359065477
873 30.8049233300369
874 29.9263596975491
875 29.2177472788649
876 29.0473480624719
877 29.2570090514686
878 29.2820186859922
879 29.030758222818
880 28.1687830598782
881 28.2902686681725
882 27.6738201684689
883 27.3447873595724
884 27.3038565183734
885 27.2421430522993
886 27.4197412320915
887 27.2014072475504
888 27.3471269430675
889 26.6409014358038
890 28.1957351689554
891 28.4610837482551
892 28.686721033866
893 28.6586434321478
894 28.9436184144353
895 28.706917036438
896 28.1151946184659
897 27.8087640039263
898 27.2732070023852
899 27.7763027971969
900 26.7177778948158
901 27.4339461819116
902 27.2516314300093
903 29.0029360559904
904 28.8184787150901
905 28.9043972305758
906 29.642696660688
907 30.4029002856241
908 30.773511124168
909 30.9911161298956
910 30.4743564026338
911 29.8271810217543
912 29.8326660236151
913 28.5232293678867
914 28.7275625825241
915 29.7196612007365
916 29.3516108990434
917 28.977528831874
918 28.9029828592062
919 28.3088217099035
920 28.999184286216
921 29.0411002499941
922 28.8166545422437
923 28.6476663899758
924 28.3188832206526
925 27.2754399847133
926 27.063008519981
927 27.2025053724437
928 27.3719457244553
929 27.3576962210187
930 26.9940618821222
931 26.6667651851943
932 26.5889788997417
933 26.5227834224463
934 26.4899285961105
935 26.5826023805158
936 26.8576741941306
937 26.2870068890002
938 26.500152424456
939 27.1670828560118
940 26.9813906012013
941 26.7209008896755
942 27.4464391069686
943 27.8118007569617
944 28.8862361051963
945 29.0433982277539
946 29.1545031814376
947 29.6904462731227
948 29.6902689433322
949 29.3165982003872
950 29.3664882975679
951 29.261122284087
952 29.3094756667654
953 29.6121247948339
954 28.6808723139318
955 28.6565201453493
956 29.0260213332678
957 29.0044547905829
958 28.7053090531564
959 28.9294452015283
960 29.3584267562315
961 30.0543186959636
962 29.8206332867148
963 29.1590976844454
964 28.9807864393963
965 28.6969907625883
966 28.3823797004376
967 27.9084336453174
968 28.0563437534078
969 28.1002861693624
970 28.1703452607985
971 27.2119483765911
972 27.9364018662221
973 28.3141464394908
974 28.485615281166
975 28.5072838136541
976 28.6241329777387
977 28.9681496274455
978 29.3416887811755
979 29.7003009617614
980 29.5494853217645
981 30.5195630433117
982 29.5761611386967
983 29.366234286648
984 29.6792197338002
985 29.640845730467
986 29.5110043185213
987 29.5215370419415
988 28.7000440850808
989 27.8849397916833
990 28.5577542969157
991 28.2085111165433
992 28.8482290194475
993 28.7287856331291
994 28.5774336738273
995 29.0421077596404
996 29.9595569128892
997 30.4880437749326
998 30.159028905237
999 30.597023945004
1000 29.785873002642
1001 29.6487829478595
1002 28.9361275399838
1003 29.9750835577714
1004 30.0918503646438
1005 30.2295588502168
1006 28.7935462895938
1007 27.9445089105101
1008 28.2482534610136
1009 28.4674616529838
1010 28.52421805899
1011 28.4974914131899
1012 28.4686429116231
1013 27.8773948240125
1014 27.5746082562155
1015 27.1052722695432
1016 27.9955915838432
1017 28.1843497070459
1018 27.9015338721015
1019 27.6057308354171
1020 27.501679295367
1021 27.4367376831672
1022 27.8019677900306
1023 27.2023040335492
1024 27.850020593342
1025 27.5742912295136
1026 27.3123277835593
1027 27.3188593193776
1028 28.036008442369
1029 28.5134126373438
1030 28.5953468584664
1031 28.4879772237271
1032 28.5663062935469
1033 29.2185333221192
1034 28.8943372361769
1035 29.7566409027165
1036 29.8721920109558
1037 30.2914671842766
1038 30.0272055983573
1039 29.848523099277
1040 29.9613110342969
1041 30.4996486311941
1042 30.390494081867
1043 29.6122679229325
1044 29.7306673297992
1045 29.7485029340124
1046 29.3250763520004
1047 28.8214716018278
1048 28.7924218282565
1049 28.207815623112
1050 27.808056492405
1051 27.8845688969199
1052 27.7955577397798
1053 28.159343394589
1054 27.9326902449545
1055 27.1684717650632
1056 27.6117808720116
1057 27.3427280741739
1058 27.3795934905918
1059 27.354260125311
1060 27.5322272709549
1061 27.4611447636053
1062 27.4377009422834
1063 27.6306523072882
1064 27.5620209765958
1065 27.3279168154346
1066 26.6453590711188
1067 26.6469253355844
1068 26.4851262725324
1069 26.5342061927303
1070 26.6320268671467
1071 26.2288122877613
1072 25.9777754057609
1073 25.3923278672925
1074 25.6425427812589
1075 25.7799389855195
1076 26.6129382722825
1077 26.833966452076
1078 26.7639730821089
1079 26.6673081206819
1080 27.2250869549728
1081 26.7268684265765
1082 26.7426573544732
1083 26.6602474997251
1084 25.9960071879485
1085 26.5236190407464
1086 26.1318923755227
1087 26.7510744529315
1088 27.3816736322788
1089 27.3906169238762
1090 26.4369517214087
1091 26.958731711584
1092 27.6978404814384
1093 28.2339488755703
1094 28.3703230980595
1095 28.8074759873438
1096 28.4292096959928
1097 28.2501479230925
1098 27.612610597672
1099 28.5548557332768
1100 29.2662143286037
1101 29.7010531536961
1102 29.42155824158
1103 30.0594949966817
1104 30.1905606719622
1105 30.9010549340893
1106 31.0357025111645
1107 30.4518022852382
1108 30.5227605152497
1109 29.6739408183007
1110 29.5202344456831
1111 28.6943767348584
1112 28.8205059599345
1113 28.586617981001
1114 29.1951350858972
1115 27.8287356018716
1116 28.5178912142193
1117 29.6177754324786
1118 30.0582529186043
1119 30.6684283366989
1120 30.5147209603846
1121 30.9622873562111
1122 31.0314544732263
1123 30.4586094400329
1124 30.5809069037586
1125 31.5194647511993
1126 31.3103258847715
1127 30.7215177532234
1128 30.4873781974483
1129 30.1564006079214
1130 29.9500095315853
1131 29.7541262425692
1132 29.6022848194356
1133 30.0722516379233
1134 30.2281376056278
1135 30.084246571526
1136 30.4732788682371
1137 30.5857155030114
1138 30.2783845805771
1139 29.9798624412483
1140 30.1784472557963
1141 30.1044220802082
1142 29.5486197250166
1143 29.0349330943576
1144 28.4303231233694
1145 28.2731235419846
1146 28.2204196190528
1147 27.9025173691078
1148 27.9543778053621
1149 28.2152122614214
1150 27.8340446272104
1151 28.3955656458466
1152 28.8478693577863
1153 28.8874712598627
1154 29.2910765906523
1155 28.1875019084395
1156 27.4529442376413
1157 27.77022379962
1158 27.9620513651023
1159 28.1389119777635
1160 28.3425505429374
1161 28.6404985440221
1162 28.6992959484758
1163 29.3022758469651
1164 29.2957365979008
1165 29.9833048377608
1166 30.5531082596568
1167 30.4479637239716
1168 32.2184109655587
1169 32.0372538229324
1170 31.7810932403288
1171 32.2639473420845
1172 31.9861309317692
1173 31.957373009799
1174 31.1787356715734
1175 31.3395839588706
1176 30.6690423155465
1177 30.4743358794216
1178 28.5549503344077
1179 28.2595096572415
1180 28.7184182966109
1181 27.618445579237
1182 27.319622734399
1183 27.1906500940974
1184 27.7870609082346
1185 27.8806414306453
1186 28.068555073203
1187 27.5756785338421
1188 27.7515522201497
1189 28.9431722139455
1190 28.7591293891741
1191 28.642707442096
1192 29.2785280126631
1193 28.8761643424495
1194 28.4020729815107
1195 28.7627647992016
1196 28.7161164770751
1197 29.0441466070746
1198 29.8824714718309
1199 28.6538454541004
1200 29.8821498023648
1201 29.1715990373812
1202 28.8786161267852
1203 28.3864822726009
1204 29.1306859088074
1205 28.215911586368
1206 28.4208329700053
1207 28.0602189567614
1208 27.5203551392544
1209 27.7554794395768
1210 26.202048306992
1211 26.5257802460959
1212 26.2628040443198
1213 27.6080266421513
1214 26.7659535481432
1215 26.6595877840175
1216 26.2835382191175
1217 26.5444844621215
1218 27.167985164735
1219 27.2962911677819
1220 28.2159354804951
1221 28.2260609634046
1222 28.5081749984286
1223 27.5222642300598
1224 27.280769356141
1225 27.438518785044
1226 28.509361890111
1227 28.7411669331195
1228 27.9153307600687
1229 28.1106650228041
1230 27.5385730096539
1231 27.9247868425154
1232 28.2545831165894
1233 28.1565638213266
1234 28.7325180097718
1235 28.687474086415
1236 27.65366322105
1237 27.558513301045
1238 27.0424606041399
1239 26.5396911366117
1240 26.5084419556813
1241 26.4222598230809
1242 25.9203754208691
1243 26.1807294820416
1244 26.09641249095
1245 25.5310213670983
1246 25.8692965873853
1247 25.7545905552943
1248 25.9287981418094
1249 25.9558345401218
1250 25.5738243858473
1251 25.9977143579472
1252 26.28768885186
1253 26.3830105849048
1254 25.892384394637
1255 27.1337512931392
1256 27.9701137741479
1257 28.2006306253365
1258 28.4866624906735
1259 28.9962079919447
1260 29.0915772023271
1261 28.4178281102019
1262 27.8029377621819
1263 27.8376006122314
1264 27.9557725812118
1265 27.3444581676898
1266 26.1157725272955
1267 25.8846007770945
1268 25.3751849914191
1269 24.7318014886591
1270 24.8660145932246
1271 25.1394843182716
1272 25.5843277730483
1273 25.8682397813403
1274 26.1136860193685
1275 26.134665114378
1276 26.5364982680747
1277 27.2153976692322
1278 29.1010092596555
1279 29.0956971470697
1280 29.4660942362452
1281 29.1473161097275
1282 29.2074866525012
1283 28.8566803220273
1284 28.97848860188
1285 29.1365374739545
1286 28.6484791200175
1287 27.7183578744677
1288 26.5788293856382
1289 26.2622469055157
1290 26.3408397957336
1291 26.197063353789
1292 26.3085224917884
1293 26.4219851323465
1294 27.0712353645293
1295 27.3091445246134
1296 27.7119675721725
1297 28.5847701363791
1298 27.9235429750921
1299 28.3522351684503
1300 27.8480031481985
1301 28.5067027611375
1302 29.3086043097589
1303 29.3468587194603
1304 28.5399699191107
1305 27.8822866026592
1306 27.8029719967273
1307 27.002048205288
1308 26.844813416526
1309 27.6071833439971
1310 28.2677233021025
1311 27.7721282201557
1312 26.4621912495233
1313 26.4449767394083
1314 27.1434719323712
1315 27.2938854101211
1316 27.6086336895071
1317 28.7570538984576
1318 29.392256909181
1319 29.1648927869837
1320 28.5082892173651
1321 28.8459122404783
1322 29.2502358216732
1323 28.5458616523211
1324 27.6885459683256
1325 27.9991502618986
1326 27.391215550379
1327 26.4978029403176
1328 26.0150200706505
1329 25.7932776307688
1330 25.8106914854546
1331 25.6381979005735
1332 25.3618475092804
1333 26.1447470332861
1334 26.4805979210841
1335 26.1538905701384
1336 26.3898675391564
1337 26.7772358530139
1338 27.5658479100666
1339 27.0279380508357
1340 27.0301330072112
1341 27.6701983293678
1342 27.594736884293
1343 27.4432404228603
1344 27.0906727522347
1345 26.9787964988046
1346 26.7644134978761
1347 26.7768385236417
1348 26.5250245060059
1349 26.9853325178187
1350 27.6926212294069
1351 26.9463783474923
1352 27.4422107639949
1353 27.460468208191
1354 27.5717738006174
1355 27.9290172996606
1356 27.9896142740449
1357 27.7420872602814
1358 27.3646747977094
1359 27.4616233270822
1360 27.5490272292194
1361 27.7567958155933
1362 27.0557853471811
1363 27.2269987555777
1364 27.2818957302587
1365 27.2695450769635
1366 27.0703269062741
1367 26.9846434253102
1368 27.2375862370392
1369 26.7918756028422
1370 26.2317991990707
1371 25.8137999875947
1372 26.4045455541982
1373 26.598726139208
1374 26.6382694072991
1375 26.9761559284187
1376 27.299183503778
1377 28.5863492535907
1378 28.4072313245718
1379 28.3079765619675
1380 28.2046288948992
1381 29.1627565535818
1382 28.8109737848639
1383 28.1144854307222
1384 28.5456832934366
1385 28.4443314123501
1386 28.5438304475111
1387 27.399658171735
1388 28.3910258501724
1389 28.9629318334914
1390 29.3793985012671
1391 29.0528635457936
1392 28.9956633568704
1393 28.869087933625
1394 28.5044752477934
1395 28.2097437309808
1396 28.1688204801896
1397 27.8443401585227
1398 26.7811654251848
1399 26.3885994139141
1400 27.0275636656703
1401 26.6257332670118
1402 26.8856965595586
1403 27.3984049123682
1404 27.0235221174735
1405 27.0500285068518
1406 27.2006846900126
1407 27.5229034627121
1408 27.5781980565696
1409 27.5358800063831
1410 26.5906393961631
1411 27.3044771656021
1412 27.2864444222126
1413 27.0421014450498
1414 27.4949699552469
1415 27.6305028450804
1416 27.4792236338931
1417 27.3757735255367
1418 27.4557716206661
1419 27.2410053455516
1420 27.791775031115
1421 26.7929229939086
1422 26.5320274260518
1423 26.3360861965558
1424 25.8898311700168
1425 25.4877255504955
1426 25.5774132188577
1427 25.1886100954961
1428 25.0099261646292
1429 26.1952091257715
1430 25.2448088232967
1431 25.3680650265305
1432 25.3935579866812
1433 25.9788036607352
1434 26.3734477526064
1435 26.5563121966733
1436 26.500539897409
1437 26.9083756711304
1438 27.1777302969477
1439 26.0852157603384
1440 26.7747944828537
1441 26.9936016731676
1442 27.3964334393709
1443 27.0901066465497
1444 27.0865647370329
1445 27.1570059407947
1446 26.7598738075338
1447 26.4284706794487
1448 26.1629983957827
1449 26.363635342761
1450 26.0143357440143
1451 26.7745754916794
1452 27.1786644381696
1453 27.551757416418
1454 28.3299872515984
1455 28.055976339442
1456 28.1918958988179
1457 28.1055637478088
1458 27.7813372250395
1459 27.6938548900428
1460 27.8833083198939
1461 27.4116949805331
1462 26.9634916787304
1463 26.6721228961926
1464 26.0566408558996
1465 26.2882388023306
1466 26.6752359255192
1467 27.4156185046951
1468 28.2808130604071
1469 28.9556816340432
1470 29.7623727397429
1471 29.4267276472408
1472 29.2511097330601
1473 29.3228581412906
1474 29.1295280378889
1475 29.418863690736
1476 28.8797040462303
1477 29.0240426789468
1478 28.5003493145059
1479 28.4828942077616
1480 28.1492426046512
1481 28.1173192410275
1482 28.4977914569909
1483 28.557230683176
1484 28.4603692739238
1485 28.1065029077213
1486 29.6920746692895
1487 29.2578207682929
1488 29.4942775446775
1489 29.7858224602113
1490 28.7114899352523
1491 28.5255473449864
1492 28.365950357733
1493 28.0562570781436
1494 28.3934270929564
1495 28.6269502454101
1496 27.7061989050934
1497 27.5174631704208
1498 27.7522576599912
1499 26.8156303143281
1500 27.8468988661994
1501 28.2263990973196
1502 28.0221070250573
1503 29.3249188547663
1504 29.1253076589555
1505 29.18440715918
1506 28.6537705629054
1507 28.8893386816332
1508 29.7354001421093
1509 30.593756450404
1510 31.0031712010786
1511 30.9552857709385
1512 31.5587512924083
1513 31.0353472421053
1514 30.7536272624953
1515 30.7206207912239
1516 32.0424540118048
1517 31.5617400743388
1518 30.6573416805119
1519 30.5591897284049
1520 30.1812136373146
1521 30.399587500352
1522 29.9255545130449
1523 29.1571067583306
1524 29.4550454848907
1525 28.9384178312623
1526 27.9082214766438
1527 28.2607633430357
1528 28.120523478945
1529 28.2076130918814
1530 29.1386613664071
1531 29.8084121707837
1532 29.8694679101834
1533 30.6102994073767
1534 30.2486881219327
1535 30.7879973935237
1536 30.6177679397568
1537 30.5275811339034
1538 30.4231175398033
1539 29.9854664084372
1540 28.2143772689768
1541 27.4628090709915
1542 27.3439332444354
1543 26.4478999445484
1544 26.3249633031643
1545 25.9811179858954
1546 25.8815727442882
1547 25.6619824405046
1548 25.5161412218369
1549 25.2061190417432
1550 25.5090864041053
1551 24.96659593828
1552 25.7694847905477
1553 25.687756895706
1554 26.1243491634595
1555 26.0867059527391
1556 26.1345140232596
1557 26.4356833133027
1558 26.8094601949821
1559 27.140480490792
1560 26.8393101647668
1561 27.3187472962148
1562 26.1972984216202
1563 26.2108510643336
1564 26.0398115517244
1565 26.0500266776364
1566 26.4201241666913
1567 26.5991166166521
1568 27.3130353634836
1569 27.3710997667659
1570 27.7152977224131
1571 27.2275377573448
1572 27.2106008192323
1573 27.790004798646
1574 28.5291327455193
1575 28.6794306618074
1576 29.2471201072314
1577 29.5941582811298
1578 28.4820970842869
1579 28.4768217053245
1580 29.7612482970305
1581 30.6632085135176
1582 31.152349270335
1583 31.0550593327784
1584 30.1289207849235
1585 29.5928209757144
1586 29.0467685030986
1587 28.2926817144062
1588 28.6986888529713
1589 28.5026127439839
1590 26.9831040675711
1591 27.0183367342224
1592 26.7196255965575
1593 26.1310168552868
1594 27.5820445276531
1595 28.6765249779959
1596 28.5120131045968
1597 28.6591451995599
1598 28.1851083366675
1599 28.6815549149351
1600 28.6977111402702
1601 28.4183670876189
1602 29.4000300739443
1603 29.7089798024557
1604 29.6928076896251
1605 29.1097665814404
1606 28.7667243450518
1607 28.8961527741766
1608 28.9711232465833
1609 28.6506015833244
1610 28.5167138623881
1611 27.9472009731556
1612 26.9094265589135
1613 26.7085506326332
1614 25.7145702072422
1615 25.7833582632805
1616 26.0833894920234
1617 25.8793232976395
1618 26.2068121699977
1619 25.8371198167206
1620 26.1262336873007
1621 26.9587593105893
1622 27.2540761370663
1623 27.3767973871252
1624 27.9247270321283
1625 28.6866665416598
1626 29.10544250128
1627 29.3547627540284
1628 28.7901050089533
1629 28.9638795579655
1630 28.7088854547639
1631 29.1350840830773
1632 29.1146410305025
1633 29.6384483182369
1634 29.3064560814174
1635 28.3567256399004
1636 27.7021601788859
1637 27.477432067953
1638 28.1149585098599
1639 28.7686663631561
1640 28.6532128987942
1641 27.7549938020311
1642 27.3873012665432
1643 27.3119289372063
1644 27.5942121128222
1645 28.0641840336604
1646 28.8640530036355
1647 29.1236891730004
1648 29.6898424904083
1649 29.3374016082059
1650 29.3644944327777
1651 29.3570619933343
1652 29.6258681766663
1653 29.6836208404511
1654 28.9646260034541
1655 28.304088257403
1656 27.3909998889999
1657 26.7053708380966
1658 26.2851923759246
1659 25.6798287886113
1660 25.8624978134221
1661 26.3045874583634
1662 25.7283215315728
1663 25.4570947194972
1664 25.2075573081488
1665 25.6212784689301
1666 26.2893538607504
1667 26.7268096510594
1668 26.405621587956
1669 26.7699686566883
1670 26.7410526262361
1671 26.4674353906966
1672 27.0137341600553
1673 26.6211860830574
1674 26.6237502733523
1675 26.3545501552121
1676 26.1227180355261
1677 26.0081661036564
1678 26.0767403797649
1679 25.8140092035751
1680 25.8621711715733
1681 26.1027520921561
1682 26.048161905755
1683 26.7434277680258
1684 27.0673918961355
1685 26.888578984418
1686 27.6753977095838
1687 27.7386550680377
1688 27.5764042259714
1689 28.0450600853901
1690 28.2798400409059
1691 27.8926263064661
1692 28.0763061081583
1693 27.6516852409186
1694 28.1070794322352
1695 28.3966407701312
1696 28.7085968470018
1697 29.5777095771091
1698 30.1810401933869
1699 30.5133890872136
1700 30.5710662851974
1701 30.984050846863
1702 30.5773651236739
1703 31.0391400283058
1704 30.6131308166079
1705 30.6741714210722
1706 29.7210153623757
1707 28.9508822272457
1708 28.9992202715223
1709 28.4429396077369
1710 29.1053617658194
1711 29.1143674782913
1712 29.5750748773298
1713 28.8863140747547
1714 28.9879651242982
1715 28.8155902473504
1716 28.5635951935051
1717 28.4512749695835
1718 27.9524776853775
1719 28.837103187981
1720 28.5393424379554
1721 28.6550789504487
1722 28.3319022863287
1723 28.2226416417861
1724 28.1396275599757
1725 28.5542969963051
1726 28.5701048086441
1727 29.1833714377129
1728 28.8492265497964
1729 28.1773957614976
1730 27.6358441954643
1731 26.9176544974661
1732 27.042059906273
1733 27.6231246357601
1734 27.8125970763401
1735 27.3834829682889
1736 27.6404906042571
1737 27.3146287511048
1738 27.5464387809158
1739 27.148571765754
1740 27.0094275010296
1741 27.62182549274
1742 27.7744214683848
1743 28.3794286205067
1744 28.2478114662766
1745 29.0070736155473
1746 28.3684977334212
1747 27.8812868704113
1748 28.0909031406775
1749 28.102947517746
1750 28.3708604254781
1751 28.244028326245
1752 28.819418432989
1753 27.999320813096
1754 27.4709686658183
1755 27.0323127288945
1756 27.7461558400491
1757 28.8460746055221
1758 28.4388834298812
1759 28.2586225407012
1760 28.0669689796811
1761 27.8525523811025
1762 27.3574738959156
1763 27.1145823756513
1764 27.4557624803079
1765 27.1114880260451
1766 26.4855532220541
1767 25.9186798502471
1768 25.7111511634693
1769 25.817955264871
1770 25.7077117494678
1771 25.8739204433974
1772 25.5280748771434
1773 25.6140246696785
1774 25.2966348089765
1775 25.3316275359249
1776 25.6282459564541
1777 25.7400212804765
1778 25.5857598257692
1779 25.504130772095
1780 26.1058037955954
1781 25.9982018551515
1782 26.0586475611186
1783 26.7912627457354
1784 27.3316184498372
1785 27.2669964903651
1786 27.1713215169939
1787 26.7244846492184
1788 26.9748760162865
1789 27.3452774563846
1790 27.1859866860856
1791 27.0100963070341
1792 27.0809800628859
1793 26.0605786415531
1794 26.7114450298273
1795 26.3829654413909
1796 26.0713554578429
1797 26.4092182338033
1798 27.1571667853381
1799 26.9295995007358
1800 26.2799166783531
1801 25.9809721704889
1802 25.9150282681621
1803 26.8984627829119
1804 25.7022281338354
1805 26.8431118745307
1806 27.443570988882
1807 27.8685587623877
1808 27.8121706349314
1809 28.6781407071765
1810 29.138839263461
1811 29.2833890726953
1812 29.6305879497452
1813 29.4752467077408
1814 30.0831910405887
1815 29.5910655965399
1816 29.2373703174744
1817 28.2503657618454
1818 28.283594288725
1819 27.9051355097709
1820 27.6609780572625
1821 27.9716916525713
1822 28.2682538996155
1823 28.0534536523378
1824 28.1081854986622
1825 28.321081794813
1826 28.7920918562621
1827 28.8514406037176
1828 28.6774017496893
1829 28.2488450124475
1830 28.5838560163068
1831 28.2036718493448
1832 27.1422517368884
1833 27.0867440226816
1834 26.8594076152507
1835 27.2805242306514
1836 26.6642089663185
1837 26.7891811124096
1838 26.6732798622227
1839 27.1914617455032
1840 27.217446052232
1841 28.1590913949729
1842 28.3275796528161
1843 29.0140619223097
1844 28.5512746254028
1845 28.1163483306453
1846 28.5422908999146
1847 28.050935957209
1848 28.7633653799961
1849 28.0819303098648
1850 27.9717778794095
1851 26.9536730437284
1852 27.3464544778073
1853 26.9249080986232
1854 27.8454871831619
1855 27.5713098758413
1856 27.2118675978393
1857 27.622102609261
1858 26.5794658738955
1859 26.8161461217771
1860 26.3956603589466
1861 26.3872591875217
1862 26.2621793718303
1863 25.7344241839315
1864 25.2502792418988
1865 26.1460410774649
1866 26.2760341573844
};
\addplot [semithick, forestgreen4416044]
table {%
0 56.242929938981
1 56.4261620970276
2 56.5660040942556
3 56.696787421412
4 56.7628407122986
5 56.6999564538493
6 56.7877020453998
7 56.8084770591686
8 56.7910546629132
9 56.7687242902331
10 56.635934559473
11 56.521137003621
12 56.3792414700273
13 56.2395880281617
14 56.1040609201016
15 56.0689728081475
16 55.9816419513905
17 55.945463838508
18 55.8666176276458
19 56.003131716706
20 56.2929012057766
21 56.5931677811522
22 56.9562462373533
23 57.456365030096
24 58.0163071599564
25 58.6048945488828
26 59.2885378900496
27 59.998660087785
28 60.7291222166847
29 61.2938670712802
30 61.8860486835184
31 62.3859974680761
32 62.7831090548378
33 63.1265558840189
34 63.4165372824327
35 63.5159141906175
36 63.6904460239047
37 63.8107481177752
38 63.9949594719501
39 64.1939450221808
40 64.3083490773424
41 64.5461237854722
42 64.9093440019576
43 65.1400026651875
44 65.3857573610923
45 65.7086587757669
46 66.0260192904186
47 66.2814177284209
48 66.4183765529701
49 66.5102068157315
50 66.6078830171008
51 66.5924735558223
52 66.5807666697164
53 66.5783786004325
54 66.6906248894371
55 66.6932223503214
56 66.6511532833277
57 66.5573783043641
58 66.5604178075101
59 66.5516678571569
60 66.4494499440877
61 66.4552429403104
62 66.3735480816142
63 66.3545666832166
64 66.1833785145337
65 66.1510472426033
66 66.1276223874768
67 66.1926886389036
68 66.1372165029319
69 66.0624444744232
70 65.9914056679809
71 65.8243022491643
72 65.6423980325994
73 65.4313202395001
74 65.3018873939313
75 65.0357154223632
76 64.814150836266
77 64.6614603901095
78 64.443403934393
79 64.3182191818866
80 64.3653977882486
81 64.3374484812887
82 64.3928490329931
83 64.492967572289
84 64.5089165520927
85 64.6190552360165
86 64.7496485832357
87 64.7771767744004
88 64.8825652578174
89 64.7522573368579
90 64.4919777254426
91 64.2877325083411
92 64.0950760532154
93 63.9797751779514
94 63.7066020508729
95 63.3803220976405
96 62.8959258029828
97 62.6141613025505
98 62.287287196149
99 62.2679231373007
100 61.9775792815805
101 61.9083233131817
102 61.6286301798706
103 61.16524457308
104 60.8735744637566
105 60.8099554571044
106 60.9682244143077
107 60.9886437449528
108 61.1734813859944
109 60.9605877836932
110 61.33570402939
111 61.4922367559109
112 61.6516927614922
113 61.7399551363026
114 62.0233043450753
115 62.0210636278618
116 61.8022229354895
117 61.4650305576051
118 61.0499435852308
119 60.8603925056273
120 60.3786029887985
121 59.8352951861632
122 59.6372110810456
123 59.4459643717301
124 59.2207461687805
125 59.1771227306772
126 58.9122230796504
127 58.6742490231329
128 58.3144751800603
129 58.0738312836463
130 57.8298572128028
131 57.6802603154305
132 57.0134478170477
133 56.4034748464252
134 55.9234001363095
135 55.860139026784
136 56.2747847885378
137 56.4772894051082
138 56.7419690743492
139 57.0371343756055
140 57.0585237315474
141 56.9184283551807
142 56.9102087365715
143 57.233707662686
144 57.2549379339805
145 56.8415917802651
146 56.1235023924855
147 55.5887300998771
148 55.3309419757988
149 54.83552684967
150 54.4623062035238
151 54.27229076561
152 54.0368894402012
153 53.5049343247002
154 52.9900225600657
155 52.5306545216723
156 52.0038120619782
157 51.8068178198466
158 50.9966063043782
159 50.6536152456486
160 50.0105037760778
161 49.6545331793788
162 49.1753828776845
163 48.7334964379834
164 48.4843394985772
165 47.980811929012
166 47.7787176688562
167 47.422506262839
168 47.3568578295119
169 46.9623752444233
170 46.9189606502479
171 46.7527440046898
172 46.5931313805507
173 46.8457761182046
174 46.8968750497071
175 47.0042778252122
176 46.9686019129163
177 46.7135077058107
178 46.4551233873655
179 46.4522182316927
180 46.2737570251699
181 46.0194529770663
182 45.9068146786667
183 45.6717831675488
184 45.5534328615162
185 45.2936893731339
186 45.1596040199706
187 45.1888541328745
188 45.3465203784255
189 45.1256644199275
190 45.2810731520641
191 45.4778532446764
192 45.8366976827595
193 45.9699820481738
194 46.0015596920474
195 46.0676501082128
196 46.1388738838696
197 45.9661875507229
198 45.7934120051542
199 45.8475443638779
200 45.7501909414228
201 45.7317537596474
202 45.6132662104458
203 45.4691035758783
204 45.2290688143017
205 45.2900280045761
206 45.3316264281117
207 45.4607259735213
208 45.4179875055915
209 45.4929176140005
210 45.434859350745
211 45.2096197061586
212 45.1392828579139
213 44.8769485406011
214 44.8863952387643
215 44.7793738693024
216 45.01876850822
217 44.9487135330232
218 44.9448901987566
219 45.0289243344398
220 45.1345201026249
221 44.7947137721113
222 44.7071272891695
223 44.8073699446746
224 44.65100341968
225 44.3358410251899
226 43.465175274907
227 43.3729711888603
228 43.5395530020365
229 43.1415034510849
230 42.9667070440285
231 43.2271713807659
232 43.1495603290907
233 43.5027879275075
234 43.7753237366961
235 44.5327468657912
236 44.9765415774619
237 44.9967044831883
238 44.8806141524745
239 44.9714543645735
240 44.9157431506089
241 45.2231400469453
242 45.5626650328341
243 44.8750592112978
244 44.6983336931137
245 44.3406925348769
246 44.7261137910308
247 44.513243061935
248 44.2728120563705
249 44.1456607956623
250 44.4269148666184
251 43.9728801560832
252 43.5396680235814
253 43.8280389349653
254 43.6460771415658
255 43.4860648208813
256 42.972951688053
257 43.5054766834411
258 43.5628311087724
259 43.4303324674171
260 43.2858705326471
261 43.1778993342146
262 43.0699864415171
263 43.2224389227357
264 43.367681551711
265 43.4460382511891
266 43.4624027660585
267 43.1565909678079
268 43.1291126598082
269 43.3690801251269
270 43.5466473407227
271 44.0802252802681
272 44.1842615954853
273 44.1417413913201
274 43.9824596594924
275 43.553025404783
276 43.4545785194037
277 43.2940573149262
278 43.2049740950365
279 43.018586855261
280 42.8116476149448
281 42.6879172166111
282 42.7281819415376
283 42.8116139697057
284 42.9485006768059
285 43.2457299465585
286 43.2664350263498
287 43.3628797005376
288 43.665210295274
289 43.9377907518772
290 44.1786864643861
291 43.7970192894632
292 44.1432397547725
293 43.8314867192848
294 43.8247465735082
295 44.2070757941106
296 44.2913991719612
297 44.6816702282477
298 44.9051737921114
299 44.9180913588936
300 44.5223546641225
301 44.9968477621728
302 44.4188398608036
303 44.3816728984702
304 44.5806848419003
305 44.3508419768634
306 44.447490691119
307 44.1640308652675
308 43.7261280822328
309 43.5360014712075
310 43.9462855673173
311 43.9693735913424
312 45.0231403375292
313 45.8206162418403
314 46.4587209093231
315 47.3227127264674
316 47.8679936418557
317 48.4225811376416
318 49.3568014003953
319 50.2425364500832
320 50.1045571921361
321 49.5855731097602
322 48.5652565405606
323 48.3230390040827
324 48.0156391330168
325 47.3293593378806
326 47.0128589053449
327 47.070721518304
328 46.6432434753207
329 46.4347836354769
330 46.7751604375763
331 47.6933998041497
332 48.2887123852271
333 48.84573458426
334 48.9019852635852
335 48.8914148659197
336 48.5401713764087
337 47.8188804828959
338 47.3880399892508
339 47.0552126909758
340 46.6735582363688
341 46.2363879492947
342 46.090382787094
343 45.6851516811452
344 45.3693568942451
345 45.4952319444381
346 45.6423617415794
347 45.7417528725079
348 46.4031761715145
349 46.1649073798538
350 46.3562579412138
351 46.6037324302698
352 46.7293983666158
353 46.4843040495342
354 46.4940319769649
355 46.6336193021337
356 47.0342761102776
357 47.2268198054883
358 46.8911799371568
359 47.1093263821243
360 46.8852149304516
361 46.6471396291498
362 46.9328863069795
363 46.7976646502705
364 46.7428840907608
365 46.3606070754807
366 45.5793580723773
367 45.6715562180119
368 45.6212710968388
369 45.5462946613321
370 45.4064821632723
371 45.2659733618452
372 44.8504863081909
373 44.9766625642725
374 44.962451790184
375 45.2777493950704
376 45.8818239584631
377 46.2177038457878
378 46.4916008596737
379 46.6389416515295
380 46.8976439906905
381 47.1938249493816
382 47.329566735848
383 47.2081469703074
384 47.2073227679824
385 46.9572103337868
386 46.8982070529028
387 46.4201237353794
388 46.3623140772994
389 46.4427801005147
390 46.8447541713876
391 47.1384905355465
392 47.2964699906045
393 47.2924530252435
394 47.3685434034191
395 47.6518870942651
396 47.7544091992342
397 47.7466190069164
398 47.5566988846119
399 47.7722081451437
400 48.0062149734168
401 47.7193173281389
402 48.0617267754364
403 48.3257453145527
404 48.7718385920025
405 48.3876683081058
406 48.475689393177
407 48.7935551689089
408 48.8740087334866
409 48.3068919963094
410 47.7461622868134
411 47.7724448888862
412 47.2766103579121
413 47.0821526190394
414 46.444983314362
415 46.8432606773547
416 46.6831477265574
417 46.0364407588284
418 45.9716351905491
419 46.1782535183911
420 45.9340248190133
421 46.1539683866356
422 45.9694669122996
423 46.3449280774269
424 46.3742750823535
425 45.9648189142914
426 45.9774268120551
427 46.3100657032278
428 46.8850982057202
429 47.3734926958782
430 47.7719567432663
431 47.0570883202245
432 47.3708643678192
433 47.1210536933403
434 47.5087943586425
435 48.1777625142681
436 48.1094299102385
437 48.6091878306892
438 48.2829511804177
439 47.8924084415055
440 47.9263436762188
441 48.3942797603608
442 48.1799833530981
443 48.0745202784571
444 47.5893427528201
445 46.9693507938522
446 46.7509268709822
447 46.3110507019931
448 45.5254419322473
449 45.1269677767491
450 44.962864201197
451 44.4049322164229
452 43.8365142922345
453 43.4293932051143
454 43.4387717208383
455 43.5531753599219
456 43.5282017308308
457 43.3428368325257
458 43.8155317375158
459 44.0662810406427
460 43.7265511479786
461 43.6438606839445
462 43.6701166319463
463 43.9336611418307
464 44.0634616720654
465 43.5393530654464
466 42.9087718570194
467 42.8486302059292
468 42.1474512654099
469 41.9708570923474
470 41.8103460174929
471 41.8669185403819
472 42.4667542862737
473 42.4770981916785
474 42.1949630391242
475 42.313341584776
476 42.9583012781778
477 42.5836884976201
478 42.7766190174401
479 42.5840473062624
480 42.7607265276046
481 42.4293352026453
482 41.8272532616068
483 41.493798887374
484 41.3412006202568
485 41.4013517683787
486 41.3606651397917
487 41.3797168963295
488 41.7273122428346
489 41.6953127054845
490 41.4711043228239
491 41.5046031477095
492 41.4365590516327
493 41.0109258483261
494 40.7842605215901
495 40.5683511976091
496 40.4985971927695
497 40.6449716161519
498 40.4472993163895
499 40.5331325104808
500 40.1557982363466
501 40.6058533300292
502 40.6338695378497
503 40.7594353805034
504 40.6669045672049
505 40.6309574765932
506 40.4901704741117
507 39.8701802616703
508 39.7144533710468
509 39.4010427161913
510 39.6525926115313
511 39.6645323782946
512 39.4968061605288
513 40.0374366956023
514 40.6353589143238
515 40.4163839531775
516 40.1049082212868
517 40.4462373900095
518 40.6690874593834
519 40.7242639362615
520 40.6429854344345
521 40.9668634051177
522 41.4908475920085
523 41.0413954691017
524 41.194254457881
525 41.4624612802584
526 41.6072872187309
527 41.4004290365488
528 40.9981416136294
529 41.003983521089
530 41.5359376548808
531 40.9146506453658
532 40.8493152998635
533 40.9347180138202
534 40.4558629135436
535 40.4558634323994
536 40.5442934232535
537 41.0332976609379
538 41.2980029350721
539 41.9098293477324
540 41.6540110829706
541 41.904422981079
542 41.9163863907338
543 42.1851297222074
544 42.2880647983751
545 42.3698630294378
546 42.4893846166415
547 42.5426694121513
548 43.1382627659094
549 42.7508609397797
550 43.3027093100249
551 43.7349004801138
552 44.0902741221439
553 44.2846399193203
554 44.9893342647451
555 45.7697609102135
556 46.5104101028839
557 46.8917110467895
558 47.227095144346
559 48.1305261602006
560 47.8665582687843
561 47.9524005995881
562 47.760644589515
563 48.4995305127863
564 48.8424856082974
565 48.4576343640942
566 48.400055247063
567 48.5160771483332
568 48.2265145031067
569 47.913701640586
570 48.0050643129418
571 47.846123277745
572 47.839193349796
573 47.6386045681384
574 46.7830614904894
575 46.9029993761218
576 46.6045224149526
577 46.8124232368796
578 46.5454486661167
579 46.4783996429759
580 46.9027253106501
581 46.5200693376793
582 46.2497627193614
583 45.9317867771894
584 46.221268387954
585 45.8133444189671
586 46.4946202525085
587 46.6379647857858
588 46.5424841907072
589 46.2048994237817
590 45.9414725550711
591 46.7580507186149
592 47.9154853518414
593 48.2452561716453
594 48.5944888062002
595 49.2506618183725
596 49.2378856886532
597 49.1396034466893
598 49.0902755538034
599 49.3206464501137
600 49.5042908498149
601 49.1434687569847
602 48.5652376045963
603 48.4427040591696
604 48.0912927131731
605 47.9091778867328
606 46.9997796439407
607 47.0186835336485
608 47.8083655615456
609 48.1712199770148
610 47.9063551268519
611 47.8410579013189
612 47.6346437519779
613 47.3107674995349
614 47.3682738574147
615 47.2662569856721
616 47.3859760847343
617 46.8455219522066
618 46.3409301858009
619 45.9807000697987
620 45.918940975523
621 46.2204067524036
622 46.0195585463408
623 45.9152983697022
624 45.7161973882415
625 45.4590260580718
626 45.6255789030912
627 45.7550839363395
628 45.8368332455043
629 46.3964888844748
630 46.6717489190626
631 46.1825817270897
632 46.0741857525468
633 46.518230532556
634 46.851848978397
635 47.1184981781794
636 47.5822238436223
637 48.0604661763649
638 48.248450322157
639 48.1748701940874
640 47.8136085110479
641 48.1061711297902
642 48.7737423229565
643 48.4014815106374
644 48.241044423528
645 48.2220786204111
646 48.1212503438068
647 47.920897196374
648 47.9039056554051
649 47.1454584601398
650 47.7553204620374
651 47.6350500523922
652 47.8494046535191
653 48.0161677353738
654 48.0044223897307
655 48.033546104157
656 48.074449723884
657 47.5101080309964
658 47.2308291368087
659 47.1171433641652
660 46.7685616594748
661 46.9965699723287
662 47.1637764657788
663 46.7397617957354
664 46.6588620302345
665 46.80588033905
666 46.5646961632801
667 46.4234887041139
668 46.534768259573
669 46.638530140575
670 46.046423292318
671 45.6281665721104
672 44.5943210698066
673 44.4915455283024
674 44.3903275719059
675 43.8744451910615
676 43.5377313324263
677 43.7077103583498
678 43.563724409105
679 44.3919772776783
680 45.2304579024978
681 45.127633684773
682 45.5976932478821
683 46.5200519741739
684 46.4872366507143
685 47.0677696273376
686 47.6050018934701
687 47.9328579775641
688 47.9782707165675
689 47.5671087084459
690 47.6069958435899
691 47.8168125982814
692 47.5051737151117
693 46.8389133512642
694 47.4325644754738
695 47.4273485235645
696 47.3927922685624
697 47.3026029617449
698 47.338150367636
699 47.6931191960395
700 47.6165402791116
701 47.9155140992397
702 48.0190711312296
703 48.301423739556
704 47.8605778800948
705 48.0695839959949
706 47.567278139785
707 47.6144907024185
708 48.102900521253
709 47.6117555924982
710 47.4261048645125
711 47.3601940578692
712 47.2261864752618
713 47.5026149892808
714 47.2438151266096
715 46.6206763135439
716 46.7955681242041
717 47.0103919348831
718 46.5333931595722
719 46.3356285544022
720 46.4134009480568
721 46.5214811225454
722 47.2397409901268
723 46.8267648887244
724 47.4202647422741
725 47.8425043963783
726 48.3550928680005
727 48.424766253331
728 48.9996671139983
729 49.0366193168
730 49.0612128378385
731 48.5838717542927
732 48.0761381237327
733 48.0923768167575
734 48.2708690361217
735 47.6577330477968
736 46.9765286562636
737 47.2562309625044
738 46.5896197631296
739 46.32074116678
740 46.0110884430361
741 45.9244931158114
742 45.524081424657
743 45.4295157708172
744 44.7419000079579
745 45.0970688597396
746 44.9148043824234
747 44.3885983696662
748 44.3289474052013
749 44.9607021053641
750 44.9741202387686
751 45.0350777910644
752 45.5282200688434
753 45.1875694582802
754 45.2507593203847
755 45.0553540320571
756 45.6045850578237
757 45.4675879242413
758 45.2738601333041
759 44.7426003942842
760 44.6775101866626
761 44.7174161899712
762 44.4477365443253
763 44.6145390653605
764 44.9104203979031
765 44.760725276623
766 44.7538151142711
767 44.8079540231933
768 44.7078618521164
769 44.700779512545
770 45.0151146863696
771 44.9489635226586
772 44.9977349309928
773 44.8625065007903
774 44.3163405190617
775 44.2536277806188
776 43.8522554765047
777 44.0625108137868
778 44.2261645962505
779 44.2657281603659
780 43.5980771542622
781 43.6181091510713
782 43.4051905078353
783 43.1390791365369
784 44.0641759472215
785 44.290600588087
786 44.5564435598535
787 44.5918660476205
788 44.6028754288793
789 44.8500967447137
790 44.8482583580433
791 44.7662217399022
792 45.2906763376089
793 45.383379028847
794 44.5993956102903
795 44.4678817767681
796 44.2122419690552
797 43.7956260925282
798 44.1322432078406
799 43.6493228931872
800 44.023291039617
801 44.2915932915862
802 43.6993470760692
803 43.785839765779
804 43.5660125426352
805 43.1491882131615
806 43.0897853730932
807 42.703583021233
808 42.2386914267824
809 42.2704819622217
810 42.1069248802051
811 41.811815568546
812 41.8593193927558
813 41.7507328138925
814 41.7517498627794
815 42.0059017027055
816 41.9323945981902
817 42.2294062313218
818 42.0410495114231
819 42.3121471557244
820 41.8166066096149
821 41.5823939538179
822 41.4938944830628
823 42.1441434974256
824 42.0887971629821
825 42.1048614438725
826 42.1423736608294
827 42.057787256821
828 42.282878281648
829 42.2783608016805
830 42.6392838440345
831 42.7509533502894
832 42.7234747580681
833 42.1092728307119
834 42.5146399806729
835 42.712353961947
836 42.7874435404501
837 43.2161969229612
838 43.453808038036
839 43.4247202709802
840 43.5209400981474
841 43.6102508008952
842 44.206004193291
843 44.5970831966436
844 44.7076092915609
845 44.8384269441446
846 44.8965214485454
847 44.60633827799
848 44.5350419804411
849 44.8065553101725
850 44.9611560473618
851 45.1661754953915
852 44.8531048122898
853 45.2847926177699
854 44.8287714705863
855 44.5659100974945
856 44.3393531298309
857 44.4363464519645
858 44.7312101376135
859 44.5759957301072
860 44.5155896254834
861 44.5943878841241
862 44.2959585932618
863 44.0449576910028
864 44.8623688397378
865 45.0190521175318
866 45.1209360614649
867 45.5673396306459
868 45.5387223376969
869 45.5985065209771
870 45.9497552932503
871 46.078610758978
872 46.8983786113263
873 46.9544083318636
874 46.605803741247
875 47.1980727346271
876 47.3064959993761
877 46.7482249400634
878 46.3572050158551
879 46.7878514708527
880 46.735306012403
881 46.6434361680252
882 46.4924713649268
883 46.7509474700505
884 46.7408617170711
885 45.867218737934
886 46.1849231739353
887 46.794493415822
888 46.7648172717439
889 46.4595753588877
890 46.101361524396
891 46.0867523146495
892 45.7947224639647
893 45.6072792149365
894 45.5365907057222
895 45.6172746500379
896 45.6213926322954
897 44.9297757495241
898 45.2040644629402
899 45.4546524314311
900 45.3227680150136
901 44.9506605713126
902 44.8391290425452
903 44.3829450791656
904 44.1886519369928
905 44.2816600337311
906 43.9292370022148
907 43.9458337872798
908 44.1006269471411
909 43.5879788310717
910 44.2112449501458
911 44.6415057611585
912 44.4432540870593
913 45.0907967972895
914 45.1758152949778
915 45.505304452791
916 45.9261578754114
917 46.1053059188238
918 46.5832209103002
919 46.8721538476813
920 46.8606487250324
921 46.8958763118514
922 47.3472073249556
923 46.9806242488139
924 47.3860615410356
925 47.1779222701983
926 46.6890354026563
927 46.5380022736166
928 46.1517188803952
929 46.0893450263066
930 45.8662335071747
931 45.9257230055445
932 46.3374907359356
933 46.5588669375662
934 46.3897331986508
935 46.3813161020535
936 46.4254081209504
937 46.8370184475593
938 46.4441320805899
939 46.7955901213768
940 46.749950352421
941 46.5955983482725
942 46.2623755435485
943 46.5046875635959
944 46.1410423184089
945 46.0774228810158
946 46.1539582432002
947 45.8483539712635
948 45.8564978798001
949 45.5329728471225
950 45.5408354426031
951 45.7417888508946
952 45.6920992437971
953 44.9699788991969
954 45.3759860105604
955 45.5898388184667
956 45.8049926383817
957 46.4634077685235
958 47.0119731734731
959 47.0962558414553
960 47.0313111130732
961 47.3201201600529
962 47.3065184335761
963 47.390993107345
964 47.3682290525574
965 47.4320114134297
966 47.1538037883709
967 46.9729785991487
968 46.839293478031
969 46.5455798054758
970 46.9792351775542
971 46.1647786913594
972 45.8530986533828
973 45.8299864848173
974 45.4294116273919
975 45.0546740209944
976 45.0114118595551
977 44.7682179319414
978 44.5139748435478
979 44.5120291329279
980 44.2233063951318
981 44.6010368576411
982 45.0986762147375
983 45.4885858173591
984 45.5850730326556
985 45.7379864131381
986 45.7906277948786
987 45.5569946373711
988 45.6603097397474
989 45.5604951707233
990 45.4074903011349
991 45.3030084382579
992 45.2044348382508
993 45.2131422229928
994 45.2996688147564
995 45.3451291357688
996 45.7327809665678
997 45.8439216935685
998 45.6703297910818
999 46.1215789826188
1000 46.6250712879011
1001 46.7065181003066
1002 46.6509112118994
1003 46.3737019678995
1004 46.6044475024114
1005 46.5985789045256
1006 46.7180897815919
1007 46.9988411852824
1008 47.1958654360868
1009 46.6658683296401
1010 46.1749476537128
1011 46.331076662094
1012 46.101142375051
1013 45.981680484676
1014 45.9499479292451
1015 46.0051830660916
1016 46.0157647876189
1017 45.7876607473087
1018 45.9001350289508
1019 46.1519594609895
1020 46.0677569665993
1021 46.056774954667
1022 46.4617813164173
1023 46.4022635517894
1024 46.5156075972371
1025 46.4589680361652
1026 46.0576613117544
1027 46.1043702573304
1028 45.7910432916488
1029 45.9518060103418
1030 46.1154785442638
1031 46.3564784548973
1032 46.0465424965603
1033 46.6756213097978
1034 46.7463551578851
1035 46.8317925205508
1036 47.355784778376
1037 47.5100532195166
1038 47.584883152798
1039 47.5641951287336
1040 47.8580547768315
1041 48.2274416978082
1042 48.5832965261486
1043 48.7516708736187
1044 48.800126114788
1045 49.2791657399152
1046 48.9736563565035
1047 49.2690085352414
1048 49.4630016754839
1049 49.6137760989622
1050 49.5529005857147
1051 49.3957620567462
1052 49.5707428472601
1053 49.588598373188
1054 49.7390231932738
1055 49.0310381464125
1056 48.7738042182528
1057 48.4230832700252
1058 48.5069489780914
1059 48.6588669197658
1060 48.9485387107044
1061 48.6040440043638
1062 48.348390038548
1063 47.721879146233
1064 48.1020596442156
1065 48.5109569241963
1066 48.4747014269332
1067 48.8122016514719
1068 48.7968100412148
1069 48.5350139845064
1070 48.1482927413637
1071 47.6931008713871
1072 47.4324867623548
1073 47.4124209546329
1074 46.764463059
1075 46.8710213413604
1076 46.9985024695645
1077 46.4418559393837
1078 46.1636493861784
1079 45.8877355608789
1080 45.7405967684734
1081 45.9736638395083
1082 46.4528955449206
1083 46.9070715651092
1084 47.4443333234431
1085 47.3720960633303
1086 47.4360518830657
1087 47.666566152364
1088 47.7325123246923
1089 48.5663276203287
1090 48.6003263769306
1091 48.5583496845861
1092 48.7446896112239
1093 48.9650827291865
1094 48.1503302732741
1095 48.0060665062181
1096 47.7634373211198
1097 47.5523676738076
1098 47.6921603208335
1099 47.2144259699866
1100 47.5360987545386
1101 47.6485764733873
1102 47.6539942641518
1103 47.1918717229598
1104 47.6728100124513
1105 47.5736944756832
1106 47.7440985704833
1107 47.8500159282554
1108 47.6458434738258
1109 48.1941936109206
1110 47.8658089137582
1111 47.928568646018
1112 47.558907276241
1113 48.0618555551034
1114 47.8666364563588
1115 48.2819889859033
1116 48.2510295281256
1117 48.4681167453281
1118 48.7062569507189
1119 48.4636864026556
1120 48.4614665628659
1121 48.2261760709792
1122 48.3222699004882
1123 47.6434967335365
1124 47.5533760844422
1125 47.7702222633648
1126 48.0720532643022
1127 47.8748856259841
1128 47.9544358245764
1129 47.4999553779959
1130 47.4098979964911
1131 47.4939886979567
1132 47.3246203466637
1133 47.0254792651001
1134 46.6275199010207
1135 46.2312061968354
1136 45.9062944236591
1137 45.899378524744
1138 45.7565387137086
1139 46.297806363786
1140 46.825506758809
1141 47.2623511407343
1142 47.3167555158967
1143 48.0146755379227
1144 48.3936061389245
1145 48.4180699439796
1146 48.4626346947739
1147 48.5639432062861
1148 48.7135370143114
1149 48.6422490353037
1150 48.218339005206
1151 47.5190666715874
1152 47.9532622610979
1153 47.7449848069917
1154 47.7707828924563
1155 47.516401048542
1156 47.4599051664087
1157 47.178255761361
1158 47.1558331951454
1159 47.0609968929207
1160 46.9812427049269
1161 46.8073496813192
1162 46.0616507644017
1163 45.995127054042
1164 45.870475303479
1165 45.958170316211
1166 46.0032017504929
1167 45.9171651726807
1168 45.5554200941395
1169 45.1153487640916
1170 45.3297766802094
1171 45.3473605126417
1172 44.9540045343427
1173 44.6096670050758
1174 44.4583704820012
1175 44.2563319250737
1176 43.892608386918
1177 44.1532558170657
1178 43.8328489355526
1179 43.9183339474594
1180 43.8329511991556
1181 44.4665447212152
1182 44.9596186301303
1183 45.408138500929
1184 45.6164850305519
1185 46.3328482793433
1186 46.642726131633
1187 46.286898700426
1188 45.9984960205367
1189 45.6326291900663
1190 45.8467269923362
1191 45.04804947773
1192 44.5891364594208
1193 44.4388044243993
1194 44.3272573167798
1195 43.6413872280937
1196 43.078046536802
1197 43.2929690584664
1198 44.1726954705416
1199 44.3405117060776
1200 43.7148524070007
1201 44.1516769017736
1202 44.2278827903694
1203 44.8850815908865
1204 44.6250783188473
1205 45.1854108261383
1206 45.5475993280697
1207 45.4280199349179
1208 45.2448369173264
1209 45.275487175728
1210 45.1644230493037
1211 44.9510076840888
1212 45.4767741169522
1213 44.6077264283495
1214 44.3740453459148
1215 43.6523822165301
1216 43.5509295459913
1217 43.2508780094554
1218 42.951459398292
1219 43.1185431350102
1220 43.3058235227492
1221 43.5152278113053
1222 42.9087278933146
1223 42.5327385369716
1224 42.8554091175255
1225 42.9374522167716
1226 43.4589085692374
1227 44.3556112033871
1228 45.217219782755
1229 45.2115649482101
1230 45.116722157096
1231 45.1837897204017
1232 46.4116493779367
1233 47.1630768122829
1234 47.5327363898316
1235 47.6532342926422
1236 47.858030123801
1237 47.2176357671359
1238 46.692882322709
1239 46.6468771836721
1240 47.3665403097466
1241 47.7129211790933
1242 47.2503099921002
1243 46.9293657516671
1244 47.0037171481845
1245 47.8380034141244
1246 48.3829984847857
1247 49.0124257603216
1248 49.324042125758
1249 49.8184989585528
1250 49.989997111739
1251 50.3414890431454
1252 50.6620583936287
1253 51.2093564783233
1254 51.3515158653277
1255 50.7567912922271
1256 50.0395241848604
1257 50.0472449212127
1258 49.7602326526634
1259 49.6282607346167
1260 49.1501770213149
1261 48.8757318510416
1262 48.5625455014183
1263 48.405422641419
1264 48.1240737768834
1265 48.4902383204794
1266 48.7853420116482
1267 48.341156334518
1268 48.6428164963941
1269 49.0198797694893
1270 49.0951017028991
1271 48.9611775897277
1272 49.0182521955545
1273 49.620266329073
1274 49.4884406392531
1275 49.1003501473013
1276 48.5648704795884
1277 49.0148659337633
1278 48.8231017050235
1279 47.9477913057019
1280 48.5537030513497
1281 48.8750443957912
1282 48.7147157673685
1283 48.4558919372559
1284 48.6940975395627
1285 48.5794563037785
1286 49.1581974156477
1287 49.5092466136112
1288 49.2814693822224
1289 50.3350021784165
1290 49.7714192532973
1291 49.3353819450039
1292 49.075190025268
1293 48.4798849620142
1294 49.0609363228823
1295 48.8097006502501
1296 48.2858962314543
1297 47.494326131569
1298 47.3506838545722
1299 46.6516374760186
1300 46.8742899348305
1301 47.344938198964
1302 47.7267968032478
1303 47.6034246837963
1304 47.131560989379
1305 47.5925171411314
1306 47.4327955557109
1307 47.9028117760501
1308 48.7368460598934
1309 48.5139050676963
1310 48.7386957898712
1311 48.400163208172
1312 48.2647416720963
1313 48.7818651374477
1314 48.891159189587
1315 48.8218964032216
1316 48.9476030277334
1317 49.2286981868181
1318 48.6868268177454
1319 48.7307247487691
1320 48.3007970643261
1321 47.9981149619879
1322 48.328827095618
1323 48.2156346545481
1324 47.8046605357533
1325 47.8708182302424
1326 48.4669698843136
1327 48.0126445025309
1328 47.9036414454683
1329 48.2539180082301
1330 48.504532502072
1331 48.7436208787041
1332 48.198716735883
1333 48.4040392539176
1334 48.730988410519
1335 48.5952686766253
1336 47.9482413747206
1337 48.3263759802761
1338 48.0763121175805
1339 48.1020717473031
1340 48.0056361310343
1341 48.0573021582648
1342 48.5793929908774
1343 48.4090240504525
1344 48.8989085800162
1345 49.3186232833805
1346 49.9066797170514
1347 49.1010734245497
1348 49.4984270180394
1349 49.3938137405215
1350 48.9993086294267
1351 48.95890714922
1352 48.506915781318
1353 48.5157643120826
1354 47.4748747455357
1355 47.3694579136923
1356 46.7670340030346
1357 47.2003582401664
1358 46.9380029426993
1359 47.2081737194479
1360 47.4729504469377
1361 47.6843482513849
1362 47.8675146865921
1363 47.4096215215694
1364 47.5723061285174
1365 46.849141829488
1366 46.9341547401131
1367 46.8579790404079
1368 47.2531294591562
1369 47.0060941736479
1370 47.1079601514669
1371 46.8709890158398
1372 46.4114857500851
1373 46.4643358237923
1374 47.2467435681647
1375 47.1996547207257
1376 47.2702321041946
1377 47.1512885641069
1378 47.4468847979329
1379 46.995821688587
1380 46.8523525913674
1381 46.2356520132696
1382 46.4026522091529
1383 46.8606454478608
1384 46.2580684562725
1385 46.6422622097064
1386 46.2230085487117
1387 46.5101679834386
1388 46.0754599097921
1389 46.7441380494203
1390 46.7389671208631
1391 47.3287633599159
1392 47.5039635254933
1393 47.4149332140849
1394 47.3209529278012
1395 47.7798376449476
1396 48.1531304740259
1397 48.2327460929182
1398 48.0087960252373
1399 47.4407301083206
1400 47.4063124466881
1401 47.9073495142473
1402 47.9703723754284
1403 48.1623307188422
1404 48.6438070364514
1405 48.4582405805741
1406 48.5434594129504
1407 48.4672125161565
1408 48.487006330266
1409 48.5890858562625
1410 48.8198850079417
1411 48.3734984833536
1412 48.2759317657592
1413 47.7897045997347
1414 47.3532195165203
1415 47.4490707883578
1416 47.2529796167521
1417 46.9007370266319
1418 47.1701906990527
1419 46.9033358897707
1420 46.9034579710503
1421 47.1496525353531
1422 47.6904705370442
1423 48.1951494341893
1424 48.0598265232337
1425 47.8597167556183
1426 48.1632705102949
1427 47.8876270778795
1428 47.7368449613365
1429 48.1423077195588
1430 47.8948689544977
1431 47.6722950236616
1432 47.4830474795249
1433 47.7287232731467
1434 47.9320859181232
1435 47.9439751453867
1436 48.173337988337
1437 48.3190889135782
1438 48.4452469239802
1439 48.9038399514645
1440 49.5191899662639
1441 49.674186097407
1442 49.4291713580322
1443 49.3902441684348
1444 49.2084565740541
1445 48.9605023513682
1446 48.8595804239883
1447 49.3020285582991
1448 49.5790898730113
1449 49.3388207181487
1450 48.8488372515617
1451 48.7680737472419
1452 48.5097428083707
1453 47.7195168308769
1454 47.6721794245413
1455 48.1602676049331
1456 47.920236298485
1457 48.3739693001853
1458 48.2677463676144
1459 47.8017884490946
1460 48.0061328347873
1461 48.3318507847344
1462 48.34479932024
1463 48.6519398389108
1464 49.1824977851991
1465 48.6432866619686
1466 49.4302012218499
1467 48.9735379812842
1468 48.5557096320171
1469 48.6715159936608
1470 48.2516529670685
1471 47.6588720310035
1472 47.5524588162085
1473 47.8230835648089
1474 47.4995754042836
1475 48.1550858771444
1476 47.2976754239865
1477 46.8740578337993
1478 46.9647232329799
1479 47.472578565883
1480 47.1532420893865
1481 46.908695634211
1482 47.0713126791171
1483 46.8960211763187
1484 47.1494845873335
1485 46.5542811742871
1486 46.4936987693147
1487 46.8830834705969
1488 47.0688670823946
1489 46.3838346276537
1490 46.7296150965038
1491 46.7585081155327
1492 46.9097382816337
1493 46.2467789714577
1494 45.7673441397223
1495 46.6469198530856
1496 47.3927966867842
1497 47.2885257947621
1498 47.3538682220785
1499 47.6725043334624
1500 47.6828031776907
1501 48.1708069733332
1502 48.0839522367471
1503 49.126971919185
1504 50.2242764787275
1505 50.053811207088
1506 50.1387406650117
1507 50.7042323351203
1508 50.7944680609156
1509 50.7039665680272
1510 51.1352708907848
1511 51.6255879600559
1512 51.925571483064
1513 51.6086444620875
1514 51.1494775594803
1515 51.2144334100481
1516 50.5873879622458
1517 50.7275121525469
1518 50.7800484058202
1519 50.8742197983384
1520 50.4561572841873
1521 49.9579895307203
1522 50.3299349902114
1523 50.6071009586704
1524 50.4334596873674
1525 50.3641651271455
1526 50.9190047454788
1527 50.3959277256792
1528 50.4631274298975
1529 51.0252370951982
1530 51.111371037167
1531 51.5217110079525
1532 51.0517956528046
1533 50.7770196541974
1534 51.3467556500148
1535 51.3684122140287
1536 50.9455721576596
1537 51.2760869813875
1538 51.0749458279028
1539 51.1081290719897
1540 50.9605531478946
1541 50.431216294954
1542 50.4193301583911
1543 50.7061182649145
1544 50.0781228981115
1545 49.7193456265444
1546 49.5116541561197
1547 48.9146760265198
1548 48.6428678564859
1549 47.7676297030438
1550 48.6432586436424
1551 48.5005404918368
1552 48.5879631584842
1553 48.0414246731673
1554 48.1632558326624
1555 48.2077691175199
1556 48.2481692591645
1557 48.5919379367303
1558 48.9464487550549
1559 49.3666283143604
1560 49.3432764281205
1561 50.130615101051
1562 50.0043334678227
1563 50.6444720997024
1564 50.9494290000451
1565 51.1539358839167
1566 51.8581512647491
1567 51.8374291251092
1568 51.8726945236387
1569 51.7207771762741
1570 50.8772119422276
1571 51.3182262551951
1572 51.3772372540358
1573 51.3558770870387
1574 51.340258642495
1575 51.6446293651622
1576 50.9542578111904
1577 51.225917264797
1578 52.0473482690726
1579 52.3767432923543
1580 52.4858458210242
1581 51.2853058750489
1582 51.785100761248
1583 51.3839603967081
1584 51.6636990337854
1585 51.4649327233036
1586 52.0107377211874
1587 51.8673767267563
1588 51.0830907929722
1589 51.0009728605654
1590 51.7909377419408
1591 51.7018947810935
1592 51.694077961489
1593 52.4429254146114
1594 51.4963946215527
1595 51.8340308990161
1596 51.467116120527
1597 51.6302309968057
1598 51.2089901631085
1599 51.3009117650346
1600 50.9748058646828
1601 51.3275951776498
1602 50.9356724703344
1603 50.1208080859285
1604 50.5342091615559
1605 49.7599478752816
1606 49.6641985800741
1607 49.3292393194227
1608 49.5508962759169
1609 49.2134678213615
1610 49.326609419404
1611 49.0407933286686
1612 49.2741941566774
1613 49.6290250993126
1614 49.5577251314648
1615 50.0881336024399
1616 50.4593752645992
1617 50.3519993566507
1618 50.4590470376522
1619 51.0018502616871
1620 50.7567637106002
1621 51.1019767640016
1622 50.9762976348857
1623 50.7324712506455
1624 50.722518656067
1625 50.4727838927524
1626 50.5029238463213
1627 51.0141239524604
1628 50.9940908040375
1629 50.5455763453742
1630 50.486064458189
1631 50.5140251159975
1632 50.5013782607371
1633 50.4576508909646
1634 50.2699305542078
1635 50.4792912290681
1636 50.059296774276
1637 49.5359141922234
1638 49.2540007095092
1639 49.447151237925
1640 49.7171983348038
1641 49.703621462469
1642 49.953475376998
1643 50.2532682226081
1644 50.5642851171462
1645 50.0898797245165
1646 50.2543588723056
1647 50.6755591056782
1648 50.7795750096713
1649 50.3198383222208
1650 50.1676465973663
1651 50.1804835223347
1652 49.8159114625914
1653 49.4074826912762
1654 49.1921680309343
1655 49.8355489103577
1656 50.0934532322175
1657 50.021843289828
1658 50.770409831677
1659 51.7013217778634
1660 51.5126649927432
1661 51.416137642283
1662 51.6148629743598
1663 51.9803410145937
1664 52.6392408212997
1665 51.9363696654076
1666 52.08097609443
1667 51.8626623969858
1668 51.658806554571
1669 51.2483374068931
1670 51.6309100272052
1671 52.1191506356987
1672 51.6534533743463
1673 51.1847578916836
1674 50.4188676075733
1675 50.6401834570791
1676 50.4765220239693
1677 50.5692794184652
1678 50.653929648404
1679 50.1852813854812
1680 49.7458150104653
1681 49.0793919993247
1682 48.9269449183788
1683 49.7255322891902
1684 49.9996096320841
1685 50.1875086944697
1686 49.7824404570305
1687 49.95229672478
1688 49.4261503891791
1689 49.3447140249678
1690 49.3418458537683
1691 49.4995335548849
1692 49.8295532999447
1693 49.2881330219467
1694 48.7773711135784
1695 48.6184237024198
1696 48.8381024373889
1697 48.4945618631501
1698 49.3743495257491
1699 49.6193856061466
1700 49.9162513108419
1701 49.7935922917802
1702 49.5661284306292
1703 49.5437796221073
1704 50.2442762005758
1705 50.0272567599476
1706 49.6931991868148
1707 50.0984693965891
1708 48.9502904666562
1709 48.7622819133101
1710 48.3836766974573
1711 48.8472888538777
1712 49.4577795203037
1713 49.1959233088349
1714 48.9012875464663
1715 49.0514940487688
1716 49.752074991637
1717 49.4742099630488
1718 50.1420664639488
1719 50.1560907493891
1720 50.4676299942036
1721 50.6243385335588
1722 50.0988263831826
1723 49.9200601646746
1724 49.7277587507957
1725 49.6554297609089
1726 49.4436079006495
1727 49.790612078631
1728 49.6193556208308
1729 50.2254873126677
1730 50.1723899958445
1731 49.7999205945314
1732 49.9797221130586
1733 50.1938324147982
1734 50.288297168196
1735 50.1508566909169
1736 49.8749570014253
1737 49.4338875437968
1738 49.527736053029
1739 49.0583763230944
1740 48.6197990488544
1741 48.0807954935196
1742 47.9910789455629
1743 48.3404203656877
1744 48.4297874840727
1745 48.6316181066329
1746 49.1841690253149
1747 49.4426795518728
1748 49.3211243481179
1749 49.6472071483616
1750 49.8926947155313
1751 50.0258263037098
1752 49.9956070117731
1753 50.1643765470668
1754 50.3947819259668
1755 50.1717195954213
1756 49.3605724327514
1757 48.962306559944
1758 48.7348806681591
1759 48.6601900816855
1760 48.5712811429454
1761 49.4932239481872
1762 49.409016334997
1763 49.6305588492219
1764 49.3282234841104
1765 50.03151115981
1766 50.4680862968877
1767 50.5930374472123
1768 50.6581266322535
1769 50.4922307779361
1770 51.0290258500433
1771 50.0861853836933
1772 50.408037198835
1773 50.0321197703915
1774 50.0862205107722
1775 50.0809792324066
1776 50.1077931256483
1777 49.9486808241688
1778 49.673072752398
1779 50.1411034856865
1780 49.6715389882032
1781 50.1156474944695
1782 49.9103419845376
1783 49.9750705818726
1784 49.6431008915394
1785 48.9563111337942
1786 48.9027288216252
1787 48.9618224557377
1788 49.3888550585858
1789 48.7759395937235
1790 48.8854704992388
1791 48.5280822970522
1792 48.3854001172448
1793 47.8948721759134
1794 47.9707882232583
1795 48.1474483762259
1796 48.0524205277189
1797 47.5338460247577
1798 47.1776379030878
1799 47.6318688223966
1800 47.3148819772802
1801 47.5575097579811
1802 47.6313400921947
1803 47.6841223503679
1804 47.4117084740346
1805 46.8067214889881
1806 46.6460437347618
1807 46.9153828368257
1808 46.7941259830362
1809 46.5179991593024
1810 46.5715981648016
1811 46.2811793543134
1812 46.1515016646257
1813 46.0873271732047
1814 46.7757505368934
1815 47.1313351932405
1816 47.4589997859397
1817 47.7397120124281
1818 48.5288936861238
1819 49.0421671225495
1820 49.3352314433372
1821 49.5440370275362
1822 49.7620459336329
1823 49.7810048821925
1824 50.0352246934825
1825 50.5918208230439
1826 50.4168292007798
1827 50.2506885270116
1828 49.9205426092873
1829 49.3610953022986
1830 49.2824942597212
1831 49.2082609456041
1832 49.2296247884387
1833 49.4222932170502
1834 48.5756506618645
1835 48.0210855416785
1836 47.8819705300675
1837 47.8425858836871
1838 47.7357867971409
1839 47.6820424995773
1840 47.3568158231886
1841 47.3182914881206
1842 47.4186743207157
1843 47.2257907014205
1844 47.5915970097979
1845 47.7865764015473
1846 48.2519473524192
1847 48.3782815050106
1848 47.8391834471115
1849 48.1435285952034
1850 48.0778807091875
1851 48.4670571078217
1852 47.9520412917978
1853 47.9497707777628
1854 47.6690028368528
1855 47.3300897466268
1856 47.0014503202336
1857 47.222958166365
1858 47.520842584933
1859 47.3026038209799
1860 47.2143196372452
1861 47.0198899614393
1862 47.75268123923
1863 48.1786829195147
1864 47.9857355187267
1865 48.1566980643239
1866 48.598957440569
};
\addplot [semithick, crimson2143940]
table {%
0 33.2491165629085
1 31.7659772922578
2 31.7804376672036
3 31.0769176489101
4 31.0045656870388
5 30.7651428342797
6 29.5782774418659
7 27.8430404045114
8 26.80923396174
9 25.5811828165842
10 24.947939198669
11 25.0353287582078
12 24.9794009004942
13 25.2895157666291
14 25.261406493017
15 25.4845636082863
16 25.8280373447703
17 26.4400153448597
18 26.8568333581557
19 27.0871079979381
20 27.1599368913792
21 27.0095267081457
22 26.921217771769
23 26.6501185792923
24 26.5153818239349
25 26.2424594183571
26 26.2521643586899
27 26.4233413347914
28 26.3427448928011
29 26.6155106509354
30 26.8721294258293
31 27.2479265216757
32 27.3738564039981
33 27.6946185574427
34 27.8800869858497
35 28.3783430692946
36 28.1746367562784
37 28.3854730565904
38 28.5547352341468
39 28.5288917676041
40 29.0180789074699
41 28.8796889838468
42 29.1416675236235
43 29.3234324199451
44 29.8416983556896
45 30.0203584504329
46 30.0811764111362
47 29.5935789533081
48 29.5216537960254
49 29.2416803688592
50 28.4357717061277
51 28.5238760776285
52 28.5169990602828
53 28.5332520078244
54 28.3995742875197
55 28.4190333764915
56 28.4462857989396
57 28.5252721439836
58 28.3478836371616
59 28.4762504336038
60 28.3421868004733
61 28.3088513121072
62 28.4137478239962
63 28.439703819537
64 28.0503023635531
65 27.5506465954007
66 27.6871702974217
67 28.1279640365444
68 28.2831431098553
69 28.0805335478062
70 28.3915322623526
71 28.232627269138
72 28.0440846902835
73 27.7528759581558
74 27.7636948612265
75 27.9288873997517
76 27.8613360661875
77 27.6851385254138
78 27.4927141340468
79 27.3961958976101
80 27.4712834687033
81 27.4065365663635
82 27.4381930916563
83 27.7307629669882
84 27.7952809179823
85 27.8962843521571
86 27.9952930132265
87 27.8235833506456
88 27.9861507552451
89 28.3983040096475
90 28.2933803596658
91 28.3379535450458
92 28.1760661261446
93 27.6863034078811
94 27.585839622924
95 27.3778155357788
96 26.9093476929159
97 26.8046766193009
98 26.6859796025161
99 26.3572717830802
100 26.3081537675386
101 26.7702321749682
102 26.8519075666817
103 26.8734403559531
104 26.6736267099334
105 26.5015188488147
106 26.5960856536071
107 26.2196873913586
108 26.3218583659982
109 26.314100209093
110 26.4282760525133
111 26.1499734037053
112 26.147576213295
113 26.3686105803925
114 26.6813491153459
115 27.1823293303193
116 27.1647838503693
117 27.5542837357678
118 27.3408044010349
119 27.1607175962468
120 27.0438754194234
121 26.8346467859168
122 27.1583453777824
123 27.5596145504614
124 27.543989011863
125 27.3404283950988
126 27.548659759772
127 27.7130459217854
128 28.0219372720555
129 28.0695297317324
130 28.0336653283471
131 27.9314662756117
132 27.6746915201087
133 27.5618472671785
134 27.542000651107
135 27.5609607881137
136 27.7338279144323
137 27.4234064011149
138 27.2750401245199
139 27.3819653561206
140 27.583591572893
141 27.8136399585369
142 27.7717594733315
143 28.2641688766208
144 28.2336096590126
145 28.2820940258503
146 28.1136200265024
147 28.5391195099286
148 28.6621295918485
149 28.8861481331552
150 28.7887344284418
151 29.2670591241137
152 29.8580777053827
153 29.2558410966289
154 29.5101684824042
155 29.8545108178578
156 30.2830013351681
157 30.4215580942319
158 30.6555531737904
159 30.695149372446
160 30.6136050973336
161 30.6080416953212
162 30.1363275921843
163 30.2596134232995
164 30.8032888901584
165 30.5080191009486
166 30.5919888411766
167 30.4022884476319
168 30.0449023573131
169 29.9330397328613
170 30.044736591293
171 29.8128488200508
172 30.5449299249122
173 30.172536872256
174 29.453639214168
175 29.2983987285803
176 28.6645820922406
177 28.1208657266306
178 27.9141787611342
179 27.7162854170876
180 27.7383390060188
181 27.1305443079024
182 26.6939990785046
183 26.8832372782396
184 26.9576634743277
185 26.915357071134
186 27.5146399425594
187 28.2262615165032
188 28.5913214992839
189 28.6947279399817
190 28.4762607214076
191 28.4823842205875
192 27.6119839063249
193 27.6056126779163
194 27.0865931632628
195 27.0619829888633
196 26.4263756440489
197 26.0448042909261
198 25.799863728821
199 25.7080990898545
200 25.4630833672208
201 25.5985290021877
202 26.130206146797
203 25.8518267338396
204 26.0591351071386
205 25.9124820442909
206 25.9525058374784
207 25.5898100801477
208 25.8508336459339
209 26.2187337727105
210 26.628998056279
211 26.3883919157053
212 26.3753363042146
213 26.5556596830449
214 26.5964025106979
215 26.6988461814989
216 27.1089731257829
217 27.1647371014176
218 26.6723184363672
219 26.3402155636844
220 26.2969165909565
221 26.5202336256942
222 26.722526487901
223 26.5631067989715
224 26.8248307332403
225 26.6560855341238
226 26.2116265378927
227 26.0996764372093
228 26.2042134722719
229 26.2947068560427
230 26.1169435604177
231 26.338940028916
232 25.7824578539638
233 25.6394902935244
234 25.6259213633839
235 25.86561572506
236 25.722890365538
237 25.8989859448354
238 26.6575203876865
239 26.3477251154892
240 26.3735952899141
241 26.6863383711891
242 27.0760170717926
243 27.186580834394
244 26.7982554997841
245 27.0341120953663
246 27.0696161794151
247 27.1981567314931
248 27.6243215281376
249 27.6537791521236
250 27.9000264046937
251 27.4535807335935
252 27.5803003403158
253 27.8055915299589
254 27.8195827127898
255 27.5204744669573
256 27.8798327548404
257 27.9682403480498
258 27.482181869354
259 27.694569236034
260 27.2509495936378
261 27.2752913803139
262 27.2513748909205
263 27.4059779802881
264 27.5239710198393
265 27.8181844600348
266 28.4417386738174
267 28.443191552518
268 28.2123093630523
269 28.3962978966148
270 28.592537165236
271 28.7284722577447
272 28.6951967691762
273 28.7024307861095
274 29.0557883972629
275 28.8495932920521
276 28.367081351314
277 28.4214041275891
278 28.4079781920347
279 28.3470125941517
280 29.3240222908325
281 29.3475233119784
282 28.9310069363569
283 28.4884988331798
284 28.4761302969211
285 28.5210072731206
286 28.2705828722334
287 27.7309096091051
288 27.3344393830042
289 26.8825321704884
290 26.1466639670275
291 25.7558935994393
292 25.5447871041974
293 25.8228057502829
294 25.452919851803
295 25.4658887154448
296 25.5311458735609
297 26.5629540619635
298 26.7328895905326
299 27.2456814911288
300 27.0582656007458
301 27.599309326824
302 27.866851993637
303 27.5909075283323
304 27.7624235085935
305 27.8805095292342
306 27.7811800492381
307 27.3538376751223
308 28.5811121895213
309 28.6529112583028
310 28.7477982375372
311 28.3770853646792
312 28.3891429677198
313 28.7737748431142
314 28.6452050270752
315 28.731815947198
316 28.5841961081779
317 28.52820831166
318 27.4610665265101
319 28.1306597677019
320 28.5634042000598
321 28.4994328624431
322 29.2322926157932
323 28.8936997261302
324 29.9420346412164
325 29.8421564180432
326 30.1748221856351
327 29.8950805368319
328 29.9573344074642
329 28.8995251634124
330 28.2820044354635
331 28.5204725581476
332 28.4582743115975
333 29.216103170301
334 28.4921563021233
335 28.1002497482635
336 28.1014144209686
337 28.0015803100488
338 28.0485594985887
339 28.1853338759322
340 28.3637687050251
341 28.9538167005194
342 28.8673382662474
343 28.5941782440473
344 29.2299858818879
345 29.6264054188572
346 29.703753111654
347 30.2406462337731
348 30.225949864099
349 30.5972803087681
350 31.7026105081514
351 31.4836138856157
352 31.1935180544461
353 31.3231805843256
354 30.6519626266412
355 30.2276632057916
356 30.2143060962156
357 30.0368781878022
358 29.8153381113535
359 29.4847699477258
360 28.5303264746168
361 28.5188714343841
362 28.5273696307844
363 28.2332072405257
364 28.2776486695183
365 29.2435978771227
366 29.1933934565402
367 29.4753277092188
368 29.4165321473825
369 29.5631568458702
370 29.4998952277011
371 29.6154107691746
372 29.5925461748022
373 29.8412081846235
374 29.5620647896026
375 28.8923501668964
376 28.8524457026659
377 28.4025089532979
378 29.4534582600968
379 29.1320809892221
380 28.9980699358742
381 28.6077202391535
382 28.6698829224126
383 28.2670290475807
384 28.5708929672658
385 28.6549767685887
386 29.1150595716467
387 29.5209073621191
388 29.2032384741528
389 29.6968001585326
390 30.1497311903579
391 31.0725513561842
392 30.8783060467063
393 30.4566567117314
394 30.8227796560178
395 30.7374585718071
396 30.0611944288945
397 29.3225619773306
398 28.4037335310771
399 27.9087912169355
400 27.1169569327373
401 26.1770392286661
402 26.2471999267332
403 27.0432225118501
404 26.9233724172775
405 27.0961108357576
406 27.246457956611
407 28.010861564871
408 28.7265247200803
409 28.7548296380464
410 29.5730223338062
411 29.7872033342192
412 29.8163559306798
413 30.0834266861635
414 29.5015380474022
415 29.700991744171
416 30.1518404760267
417 29.837533582671
418 29.6906017427655
419 29.9585961226528
420 29.702852475932
421 29.8000145280861
422 29.7702090627142
423 29.2715912447651
424 29.5042294744316
425 29.0803592282648
426 29.1216764913895
427 29.5992917347254
428 29.4819896670352
429 29.2839234911999
430 29.0551484216936
431 29.0268808361816
432 29.4199606044322
433 29.7470347606651
434 30.1345390260568
435 30.1559597828802
436 29.6860041966387
437 30.2570421630655
438 30.3586077746833
439 30.2235762510193
440 30.1666403059274
441 29.6457182024585
442 29.3197780785329
443 29.2649141201353
444 28.8380613991061
445 29.1159446920732
446 29.3905938652916
447 28.2429744103092
448 28.7352422337657
449 28.960756251794
450 29.3579992928925
451 29.3589522539241
452 29.1335663298018
453 28.557997201653
454 28.4315456902012
455 28.116554787344
456 27.8906233131834
457 28.0639004717228
458 27.7796706140265
459 28.2906373420552
460 28.202889890286
461 28.3704528404697
462 28.618122473596
463 29.0772273089361
464 29.4354583010257
465 29.660704370284
466 29.4749189527194
467 29.5915558279586
468 29.4502393290999
469 29.289515078768
470 30.1621685273273
471 30.9827402326402
472 31.2093862406696
473 30.8540836253414
474 31.0999336022669
475 31.0586260603588
476 31.0947046772564
477 31.0604333911179
478 30.7846927717502
479 30.2533495173867
480 29.5839057915002
481 28.5858234289981
482 28.2322760270246
483 28.5605103182698
484 27.8598574352906
485 28.5501891316457
486 29.3187483825386
487 28.9484143370449
488 29.1595675270763
489 29.845284116817
490 29.5806229807115
491 30.9776255700143
492 31.1712972856537
493 31.4461147734668
494 32.8699029187993
495 32.7903630064408
496 32.4424437153282
497 32.7631829295712
498 32.5608613714649
499 32.1439769189238
500 32.123429916156
501 30.8373563064984
502 30.7426432860616
503 30.1146365680899
504 28.8488085428399
505 28.0472543070982
506 27.8907465463976
507 27.9824900944067
508 27.6472254274967
509 28.2766033830739
510 28.0837689049469
511 28.0575870877419
512 28.5826431174837
513 29.4886403916804
514 29.7399567968193
515 30.2866211015074
516 30.2826252511466
517 30.5463590494617
518 30.6361039070593
519 29.9107726853781
520 30.1173701799026
521 30.4169504041621
522 30.0892865602621
523 29.3530794302313
524 28.9573954264775
525 28.4963369367927
526 29.8713445572556
527 29.2615754552282
528 29.6074258402776
529 30.3630443799836
530 30.6629063494171
531 30.9482030933141
532 30.6929015630811
533 31.0208154031526
534 31.7901609921599
535 32.3798461867392
536 31.1649289987601
537 31.4396345301891
538 31.6804969807352
539 31.5125249021007
540 31.0687085184119
541 30.9386320556143
542 31.0210588174304
543 30.8537892034293
544 30.204734405419
545 30.1473024885652
546 30.568148136572
547 30.7278345750583
548 30.9379668064459
549 30.7044370291856
550 31.2193411869645
551 31.332433346427
552 30.8819132753508
553 31.6814933025061
554 32.3940126022182
555 32.2766900158185
556 31.6999994487476
557 31.2175926642677
558 31.2839540576257
559 31.0096294066768
560 30.5909352357322
561 30.4471276687984
562 30.4920884100327
563 29.8613689535276
564 29.73181851344
565 29.7291859079821
566 29.5590886725098
567 30.0361131738391
568 30.1338586962732
569 29.9961465014272
570 30.3725802003951
571 30.2034815200704
572 30.7865675342822
573 30.5955417381403
574 30.0287246186972
575 30.605123258517
576 30.5104362302377
577 30.4357032136221
578 29.6279691216192
579 30.8827770024036
580 31.3674970167083
581 32.4681718619026
582 33.2446387252837
583 33.0042675955708
584 33.2435392791782
585 32.9161204975824
586 33.0766668461587
587 33.3572327033934
588 33.7643308314069
589 32.4930992636417
590 31.8811063158614
591 31.2543893675372
592 30.1796561850878
593 30.3008814839098
594 29.9842456276205
595 29.4868344088807
596 28.9447580055036
597 29.18136486234
598 29.2606064296231
599 29.2090558216528
600 28.6524019666705
601 28.5041028247864
602 28.5829444004774
603 28.7421483741657
604 28.503361418585
605 28.4743307685601
606 29.4700695389115
607 28.8409248553801
608 28.0097724466858
609 28.5460856501204
610 28.8112064985255
611 28.1938764246696
612 28.1227535562485
613 28.7629679617939
614 28.8166720603246
615 28.7234816593249
616 28.1835586677123
617 28.4399185132763
618 28.8803621252621
619 29.1107326421374
620 29.1705583602616
621 30.5224991119798
622 31.5567989957488
623 30.4505747487534
624 30.4372492720817
625 30.597182366467
626 30.9038538760783
627 30.5601282553595
628 30.7342206919317
629 30.2801030357066
630 30.1944022052288
631 28.3711075074971
632 27.648277209197
633 28.2945184884792
634 28.6600862624544
635 28.3969686100991
636 27.5116046787121
637 28.1123192989284
638 28.4604053739412
639 28.4252782465556
640 28.3949175821651
641 29.5906619549807
642 29.0590472655016
643 28.5320748592291
644 28.673195220845
645 29.1099687061113
646 29.99248382176
647 29.9125731284632
648 29.5295321232999
649 29.3387295812821
650 29.0268179974814
651 28.39827836617
652 28.8695733594176
653 29.2303075170664
654 29.0477510556493
655 28.3857828717762
656 28.3989957548754
657 28.4086983732134
658 28.5362165218693
659 29.2162294911139
660 29.7070155434158
661 30.4693840708692
662 30.5467457030059
663 30.4735098356081
664 31.0621197446764
665 32.8668276005526
666 32.7497245164971
667 32.1218166026703
668 31.9594879775727
669 32.0054019881916
670 32.0544067696278
671 32.2334209388873
672 32.6179741841875
673 32.2495373210977
674 31.5607666637259
675 30.3709967321962
676 30.4402946306908
677 30.7521919243961
678 30.6015286040758
679 29.967551773976
680 30.0428270869611
681 29.6928350467366
682 29.0615114020492
683 29.7691086744519
684 29.9691047406232
685 29.7242366833622
686 29.2426920619376
687 29.4466036326797
688 30.0546499620244
689 30.2927844664215
690 30.5596508287225
691 29.8230243028578
692 29.6424576761372
693 29.1661107042071
694 28.9190661603118
695 29.3973026592028
696 29.6214288253756
697 29.1223489062683
698 28.5323777008767
699 28.5280433415955
700 27.8663469765299
701 28.0352313078299
702 28.0974700411758
703 28.1893275636956
704 29.072314330332
705 28.7308574110267
706 30.2022129801579
707 30.3787760591405
708 30.5154098016545
709 30.9955136767799
710 31.1076435353932
711 31.7551643028112
712 32.7512712972899
713 32.8844763661653
714 32.2069942029205
715 32.4310799544579
716 31.0546606902527
717 31.0885286609307
718 31.3529432535555
719 30.3156772156295
720 31.0213487443255
721 31.0375782713264
722 30.5060220649841
723 30.2006256757417
724 31.2069168680794
725 31.2389850100975
726 31.5953311214752
727 31.7584163463334
728 32.2076761817944
729 32.6131128602981
730 32.422327867984
731 32.0821743641845
732 31.6646665308157
733 31.9988591469621
734 31.2319311802581
735 31.4990105036525
736 31.3973936347717
737 31.5379150303582
738 31.1873986164343
739 31.2555860453922
740 31.1156425222097
741 31.369874239856
742 31.7099838106759
743 32.1065785190708
744 32.6870156828153
745 32.1490876917119
746 32.0697195694333
747 31.6754055663348
748 31.9586741477096
749 31.905266049369
750 32.2972240636977
751 31.8767949495218
752 32.2564514133415
753 31.6367815602875
754 31.0472482867482
755 31.5789535797382
756 31.3625498050429
757 31.7365111484481
758 30.7338814161338
759 30.34805165971
760 30.273135468391
761 29.7197266465546
762 28.9456162344489
763 29.2374203241569
764 29.3418655687996
765 28.9537413875269
766 28.7035602083339
767 28.0830063268391
768 29.2108473702439
769 31.045794620323
770 30.3959044150749
771 30.5999820002648
772 30.3764111569527
773 30.0642866000162
774 29.8083875918531
775 29.259055488092
776 29.4505426175027
777 30.8149006833139
778 30.4541661949037
779 28.6472936776567
780 28.4195887543378
781 28.1239751929744
782 27.9390078912081
783 27.7459162390036
784 27.826557608769
785 27.8569783891414
786 28.0754155861331
787 27.3670479562474
788 27.3299940577158
789 27.8619151495621
790 28.2773700075516
791 28.8756508750873
792 29.2528681463046
793 29.2172524609331
794 29.1367116106709
795 30.5223547017447
796 30.8847143309
797 30.6818951520636
798 29.9639884785774
799 29.6993211157952
800 29.4972921613404
801 29.3289270752779
802 29.4754438920149
803 30.0186994009853
804 29.6196855802507
805 28.8336930778196
806 28.3335895268845
807 28.3716737377908
808 29.1389283814936
809 29.2575596286243
810 29.1084560379287
811 28.8228218181564
812 29.2757272805013
813 29.3634379805404
814 30.2353200702078
815 30.1085617773895
816 30.1844596731393
817 31.1211675079552
818 30.4321816068136
819 31.6595935867324
820 32.5355708014349
821 32.7417811574723
822 32.5260417055425
823 32.1913135752287
824 31.7646261817049
825 31.517481254356
826 31.452385393594
827 30.3208126821071
828 31.4618258267388
829 30.551278717782
830 30.0388575867266
831 30.5404448990462
832 30.4971588564425
833 31.6952207880153
834 31.4524411990788
835 31.8188820186413
836 32.0336691269111
837 32.7881026666424
838 31.5804881318298
839 31.5008839917133
840 32.3866331280843
841 32.6733307154597
842 32.440741666259
843 31.2277985260241
844 31.1507459071715
845 31.0023698652739
846 31.0965614449066
847 30.0588669380468
848 30.2020565446878
849 29.8386937154886
850 29.2945247019524
851 29.0692604807662
852 29.6752822965753
853 29.898311852164
854 30.1183283640198
855 30.6820568149065
856 30.5468439835058
857 30.3978357030618
858 30.3618353454474
859 30.3952030802364
860 30.0670678894284
861 29.6826446066595
862 29.5885620144043
863 29.5948584175072
864 30.5051825173882
865 30.2967628770999
866 30.0957268530359
867 30.3499823866517
868 30.6242170603786
869 30.6272774987435
870 30.5591610784514
871 30.4854920186362
872 30.4618089584341
873 30.7124177135467
874 29.6059523050727
875 29.5854682754856
876 30.5246551901984
877 30.8895949607404
878 31.6019860802851
879 32.2562640729219
880 32.5990113151152
881 32.6630247958235
882 32.2462686787523
883 31.9242181848955
884 32.3028337783889
885 31.6365736223817
886 31.0405990017625
887 31.0954418369313
888 30.3942364459564
889 29.8612786198621
890 29.3933942885341
891 29.4909038162404
892 29.5117268469397
893 29.3882529964898
894 29.0010840230678
895 29.3246081857543
896 28.6525355670817
897 28.2789615250546
898 28.418296048429
899 28.4815906622937
900 29.0164622048757
901 28.8398757934246
902 28.7884285740815
903 28.6821513745191
904 28.7785464927862
905 28.3865093106272
906 29.0656963618751
907 28.7874455077711
908 28.2046614871616
909 27.8586500272431
910 27.6279263211646
911 27.7374471056025
912 27.4304453573751
913 27.1738355439299
914 26.788901787669
915 27.176098624249
916 27.1102418148072
917 27.6449756320263
918 28.9251674394611
919 29.3655869784529
920 28.9399361245187
921 28.7525317907162
922 28.8769465859669
923 29.5821285479183
924 30.0881284095572
925 30.6105023381947
926 30.5639773324683
927 30.8479168147029
928 30.6104002110809
929 30.5502653335676
930 31.5209290239263
931 32.0798292325549
932 32.9636740305131
933 32.5493862218494
934 34.1095496294733
935 33.1650398713657
936 33.5219826460592
937 33.3589784117566
938 33.048239391109
939 33.0391441891197
940 32.388681263977
941 31.9183982770908
942 31.6519612274201
943 31.9324970382325
944 31.1522670213148
945 31.3185999056667
946 31.4229806494863
947 31.1270067525954
948 30.6565824149709
949 29.9789201884116
950 30.5357543039964
951 31.1789971957645
952 31.2696323974699
953 31.3124229018512
954 30.8883281226601
955 31.3547424647716
956 30.9371219509982
957 31.0184796298531
958 31.8631149104795
959 32.7171573866661
960 32.4484957483176
961 31.9809998651159
962 31.8342260307887
963 31.7710862855008
964 31.1267791555102
965 31.2920536868874
966 31.1782358798144
967 31.3553332169813
968 30.5365307225805
969 30.6324171305901
970 30.8782332295191
971 31.3975282044797
972 31.5726353771374
973 32.876411839418
974 33.7483883348419
975 33.5789253183775
976 33.8301688378019
977 35.0102930508734
978 35.2559259030764
979 35.8231578145085
980 35.5156835932324
981 36.0546081425404
982 36.2237438086849
983 35.3341120609204
984 36.4743007856284
985 36.535513413043
986 36.3953225341477
987 35.6142602976774
988 35.5998509324923
989 34.6125167564833
990 35.2314543133829
991 34.4176145515229
992 34.2182488996412
993 34.166155370819
994 32.5670641306179
995 32.8361963241517
996 33.4160377129558
997 33.3674959004057
998 33.2684733024739
999 34.2187292180093
1000 33.3334591196142
1001 33.142738429914
1002 32.3802706473527
1003 31.7118972068406
1004 31.2186607141338
1005 30.6545843217079
1006 29.8458249046744
1007 30.1671280227911
1008 30.4055570679443
1009 29.2213696740062
1010 29.0333408867203
1011 28.9095497415126
1012 29.7919429400961
1013 29.4169377757907
1014 29.4751779363883
1015 29.1396535930956
1016 28.9493318013434
1017 27.8680185479688
1018 27.2932682123978
1019 27.4517424770187
1020 27.8727931637882
1021 27.966472895323
1022 27.5074518104252
1023 28.3662786584461
1024 28.8418344593764
1025 28.9060561815576
1026 29.1604825953564
1027 29.2461040468921
1028 29.6495523796592
1029 29.8810429618838
1030 30.7654976989939
1031 31.0974271276562
1032 31.5527005736462
1033 31.4529983431247
1034 31.3487192177862
1035 31.7154083798759
1036 32.2649081755735
1037 32.9366402071336
1038 32.7329632164294
1039 32.8108131547642
1040 31.9412271514677
1041 32.0629748510638
1042 31.8039133305964
1043 32.155661665102
1044 33.0051765386069
1045 33.7517616867109
1046 34.4660497668851
1047 34.188917925931
1048 34.2778177336279
1049 33.843469305779
1050 33.6168162052249
1051 33.393497791967
1052 33.415588692225
1053 32.7351307401128
1054 31.8054968351506
1055 30.8839930242556
1056 30.0175267309085
1057 29.768172026028
1058 29.6161417673949
1059 30.4603551614626
1060 30.4746305262218
1061 30.3636255882905
1062 30.3442592825548
1063 31.5334348040563
1064 31.5084905456934
1065 31.0448541694361
1066 31.7832110411479
1067 31.826345650647
1068 32.4454270752273
1069 31.8158777913461
1070 32.0000503895427
1071 31.9755671844632
1072 31.8433900922411
1073 31.0838136739351
1074 31.3094278276139
1075 32.0437315326128
1076 30.8551748793962
1077 30.8196804984939
1078 30.2170885173612
1079 29.9261122355979
1080 29.9041252495857
1081 29.696007221989
1082 29.3887504651469
1083 29.4771521392951
1084 30.4223532108255
1085 30.5179301776002
1086 30.9727706872866
1087 31.0211272166042
1088 31.199347767884
1089 31.8459861764888
1090 31.36439052271
1091 31.6706131629446
1092 32.3845318317522
1093 32.2203229336098
1094 30.9958941460802
1095 30.8455022953033
1096 30.8713851475602
1097 30.9245506748961
1098 31.4583893456854
1099 31.1839162869553
1100 31.0111236118271
1101 30.7440648836893
1102 30.3251916135957
1103 30.6728604205001
1104 30.7716524895254
1105 31.7196183458353
1106 31.1458951049805
1107 31.0853203324778
1108 31.2267746933722
1109 30.9452342780744
1110 32.0167900408651
1111 31.6245572847073
1112 31.6905393776518
1113 31.3183155371043
1114 31.030606283707
1115 30.3097174049446
1116 30.3923474752083
1117 30.3849820693259
1118 29.6556846824397
1119 30.5936128393158
1120 30.0802126376482
1121 30.1674082088766
1122 29.8696050516213
1123 29.1285719547519
1124 30.3609507275896
1125 30.002736092298
1126 30.7558959451167
1127 30.7486287918206
1128 30.6093237469525
1129 30.6888471139641
1130 31.4501382775507
1131 31.9303923598049
1132 31.8820515484887
1133 32.9183189686899
1134 32.512674271817
1135 32.9427122218705
1136 33.616664513414
1137 34.2567060053604
1138 35.0863358701599
1139 33.8989754312557
1140 33.6902929384482
1141 33.2256281535825
1142 33.3658046219118
1143 33.2676394791296
1144 32.7727086004868
1145 32.7740089123884
1146 31.3345973119602
1147 30.1724273604851
1148 29.0555027943271
1149 30.3678909240137
1150 29.9338606888563
1151 30.4188395778372
1152 30.4863204802366
1153 30.3843420244527
1154 30.4256372067581
1155 30.2842534134327
1156 30.2835723029976
1157 30.9176346920813
1158 31.2319966587084
1159 30.0509428975215
1160 30.6816766527099
1161 30.5001423099763
1162 30.4937237866146
1163 31.2819832202178
1164 31.7968943256852
1165 31.6757998685256
1166 32.4249685542058
1167 32.6046976798889
1168 34.5010653933081
1169 35.0008125375585
1170 34.2966623107515
1171 36.0851232793392
1172 36.1114120638917
1173 35.1929965338395
1174 34.5696744020115
1175 34.1850714701131
1176 33.5577229964267
1177 33.140700442621
1178 31.6315895225983
1179 31.3314310604449
1180 31.0955593559694
1181 29.7996104434813
1182 30.0574096338984
1183 29.9893950991277
1184 31.5353988641288
1185 32.2703183281541
1186 32.5842958386248
1187 32.8925313437583
1188 32.9303882529636
1189 33.3081547425622
1190 33.3777895620214
1191 33.4199914942848
1192 33.2460391350217
1193 33.4434855648569
1194 32.1562452601655
1195 32.1888976142412
1196 32.0695412992043
1197 31.9728986085205
1198 32.4350978696511
1199 32.4779474276135
1200 32.9446788901561
1201 32.6889290723635
1202 32.7623741265236
1203 33.0922309861129
1204 33.2432961929928
1205 33.3792146490673
1206 33.2275351776644
1207 33.688393676553
1208 32.7752761207022
1209 32.1643975472394
1210 31.6888771106601
1211 32.1072667849042
1212 31.9868163805642
1213 31.8207493553091
1214 31.4745583763258
1215 30.6747042492161
1216 31.0847951974488
1217 30.8168994559777
1218 31.2917777327889
1219 31.6682560942155
1220 31.5826418118596
1221 31.0001005265333
1222 30.8271268433304
1223 30.6086426681861
1224 31.0690332265413
1225 32.1419105130752
1226 32.3460263633945
1227 32.4639272964622
1228 33.1038677318476
1229 33.2321118351673
1230 33.7843603406203
1231 34.2018035802372
1232 34.7014109826394
1233 34.8878257723736
1234 34.7428943868725
1235 33.8411828015164
1236 33.4953677575837
1237 33.9417988783801
1238 34.5253900164429
1239 34.245659877618
1240 34.3749420827057
1241 34.3971475272191
1242 34.1298893524887
1243 33.7841260339297
1244 33.5106973792397
1245 34.2258311147786
1246 34.3028701509031
1247 33.7585742142124
1248 32.4538780112642
1249 33.2578725572167
1250 33.1678414793798
1251 33.9092535590694
1252 35.1837597784661
1253 35.4880879843875
1254 36.5413458571333
1255 36.205485862119
1256 35.8921681477953
1257 35.9760516368728
1258 35.6189867482961
1259 35.1942777299218
1260 34.3647320177744
1261 34.1870355008752
1262 32.6513754095193
1263 32.039451106475
1264 30.8290596692153
1265 30.7047511015081
1266 30.4335295656894
1267 30.6434399978374
1268 31.037353882346
1269 31.0970829000733
1270 32.0876575884129
1271 31.1016458843874
1272 31.6502636200713
1273 32.0972165523211
1274 32.6734041483401
1275 32.88752909139
1276 32.9501000242048
1277 32.3024954752394
1278 32.8480412223883
1279 32.9003244206581
1280 32.3683613784247
1281 32.6558962091323
1282 32.3343925395906
1283 32.5415132201376
1284 32.4259141421745
1285 32.0217539506041
1286 32.1895387999112
1287 33.4912797151609
1288 32.7927508473206
1289 32.6013626380261
1290 33.1624478261541
1291 32.7696331844469
1292 32.8782103833414
1293 32.4379218024637
1294 32.2578087976447
1295 34.0655968391249
1296 34.624267067832
1297 33.3679124991805
1298 33.6930768754007
1299 34.4198632068752
1300 34.1043878685804
1301 34.0367207701284
1302 34.0638654977927
1303 34.2555697743352
1304 33.8269894132185
1305 32.1750693513923
1306 31.3448484985661
1307 31.5204025632293
1308 31.1614570903696
1309 30.1528317865936
1310 30.1408050487449
1311 30.4419730441138
1312 29.7273241514245
1313 29.9323294861876
1314 30.7665421338171
1315 30.0860432121829
1316 29.9409195187468
1317 30.0343613913523
1318 29.9123736839031
1319 30.3602840205642
1320 29.6686322859924
1321 29.182376184682
1322 29.8331973527125
1323 29.2661575557952
1324 28.5521221923641
1325 29.071472013669
1326 29.6089886532905
1327 29.9339167387002
1328 29.9515154152997
1329 29.1933282948608
1330 29.2190570707774
1331 29.3669413124198
1332 29.0726328002968
1333 29.763738450707
1334 29.7352810427565
1335 29.6351481072437
1336 30.1208546766469
1337 29.6987029631835
1338 29.9832536051446
1339 29.8998665822053
1340 30.7668358420679
1341 30.6407508716743
1342 31.4475713434603
1343 31.0332231470614
1344 31.3711501467819
1345 30.7081432899603
1346 30.1248670389727
1347 29.6998277927816
1348 29.6801960823486
1349 30.6393577860685
1350 30.9761915159963
1351 31.3686726793177
1352 31.4455705935867
1353 31.7043506606293
1354 32.2599103183787
1355 33.1489941479035
1356 33.428566504165
1357 34.2652105520396
1358 34.4679930981318
1359 33.6006877253061
1360 32.5491156890766
1361 32.677003051419
1362 32.0530290596042
1363 32.2549858704048
1364 31.5663358049362
1365 31.7818742195116
1366 32.0721398441624
1367 32.003523213648
1368 31.9156799598102
1369 32.3968349018305
1370 32.3018923192121
1371 32.7820870037249
1372 33.5865841194519
1373 33.4568241581198
1374 33.7193895386915
1375 33.8019265173887
1376 32.9405205212688
1377 32.8439458465007
1378 32.560362765813
1379 34.4463997837208
1380 35.3917329110465
1381 34.9548141177043
1382 33.9957695881494
1383 33.7459215372232
1384 34.0608295483731
1385 33.5497438537758
1386 34.2613646324379
1387 34.0684932935792
1388 33.8825154214492
1389 31.917295090658
1390 31.5536171644201
1391 31.1773675678157
1392 31.1767202612849
1393 30.9033032330597
1394 30.4264414419265
1395 30.0627114448198
1396 30.2149925644039
1397 29.9966856116822
1398 29.8126807256947
1399 29.9234203330188
1400 29.2003716118954
1401 29.4759583570616
1402 29.4945077730043
1403 31.002580104201
1404 31.3806568473571
1405 32.245214723032
1406 31.8231564082394
1407 32.4295961363134
1408 32.6281927294928
1409 32.647044904832
1410 33.3323127465027
1411 33.0187161949032
1412 33.9215484157672
1413 32.5233877668698
1414 32.1942324714771
1415 33.5479841668228
1416 32.8445328777395
1417 32.2068449171709
1418 32.2970237169827
1419 32.7360236071565
1420 33.2750487343946
1421 33.9645914272151
1422 32.7843988265179
1423 33.1990050370661
1424 33.3392829349687
1425 31.2757104565057
1426 32.488329017247
1427 33.2405662273033
1428 33.5572213112356
1429 33.026109302981
1430 32.5720959988401
1431 32.2076785049028
1432 32.9180012219319
1433 32.793871778974
1434 32.5127802323974
1435 32.9629802265977
1436 32.9016189998799
1437 33.173502480948
1438 33.1901519355024
1439 33.2636379882583
1440 33.180582131946
1441 34.1300549991425
1442 34.3944698931164
1443 34.2114162686361
1444 35.4816175522493
1445 35.4115496767951
1446 35.308227000411
1447 34.1572831516664
1448 34.0823620895947
1449 34.3266135462246
1450 34.9698627631194
1451 33.7015367563585
1452 34.4693106785016
1453 35.2187564563371
1454 35.2324041517978
1455 34.7986697299029
1456 34.2962045240983
1457 34.3002355574598
1458 34.2103473381148
1459 33.6719135780201
1460 32.8311322808738
1461 32.986972073169
1462 31.5992383626914
1463 31.0890430303811
1464 29.8068201752139
1465 29.7064720270449
1466 29.9064949484084
1467 29.9968256365226
1468 29.346078452647
1469 29.6586795658915
1470 29.5451047550831
1471 29.4411237442366
1472 29.6102893645704
1473 29.0325800870531
1474 28.8568829102347
1475 29.3632118346718
1476 29.3038107708316
1477 30.1348319524918
1478 31.0876518006787
1479 30.6778104899519
1480 30.6773253968327
1481 31.3615978545932
1482 31.0592513059078
1483 30.8863788393543
1484 31.2517787420775
1485 30.9951134379377
1486 30.6305936887052
1487 30.2955481452693
1488 29.6676805673306
1489 30.7956583966666
1490 30.5694031354687
1491 30.0198247351267
1492 30.0068251345727
1493 30.6606763213811
1494 30.5316798034624
1495 30.4364919548465
1496 30.5857177373153
1497 30.6538673300098
1498 30.6155111351126
1499 30.0791715551524
1500 29.8925121626031
1501 29.3335264611616
1502 29.7069338905119
1503 29.3623601896721
1504 29.3935433779777
1505 29.6612239777029
1506 29.8829407412782
1507 30.1637446983185
1508 30.2635537423384
1509 30.1906891025315
1510 30.8890058932477
1511 31.8633832658881
1512 31.8587871886527
1513 32.4756187088812
1514 32.2677541542777
1515 32.1298709207287
1516 32.0495651310315
1517 31.6163629036952
1518 31.5246519317535
1519 31.5398841579895
1520 31.3357304502321
1521 30.8284739791921
1522 31.1358606827071
1523 30.6933600546725
1524 31.3947744851502
1525 31.5177331501541
1526 31.7912897600758
1527 31.2682328404493
1528 31.2340181355286
1529 30.8158433324854
1530 31.2513104565398
1531 31.2657744625558
1532 31.1886574670753
1533 31.2685114186473
1534 31.7849383338726
1535 31.8077553996849
1536 31.8063780861909
1537 32.4947293853621
1538 32.5076813172342
1539 32.5103249409231
1540 32.3695604619437
1541 32.7460239564646
1542 32.2616838973596
1543 31.8869916369791
1544 31.1824741997847
1545 31.0846459434531
1546 30.4787498616447
1547 29.7408776283144
1548 30.1740596867763
1549 30.5178190672357
1550 30.2578132814491
1551 30.4243160923633
1552 30.7876891756195
1553 30.9861822602927
1554 30.8002304070042
1555 31.1173958623073
1556 31.1187148501302
1557 31.3284090062989
1558 31.2878523892843
1559 31.6757568659991
1560 31.6033553139565
1561 31.4823875573361
1562 30.9786214586788
1563 30.9549034321798
1564 30.9648538624158
1565 30.1901291241112
1566 30.3644798845044
1567 30.2056943772209
1568 29.9356302833856
1569 29.6163071448161
1570 29.5995675152844
1571 30.0765358074271
1572 30.1921991875419
1573 30.3711249645958
1574 30.5155503893516
1575 32.0794819537516
1576 33.6994798420219
1577 34.0584091326394
1578 34.3082164656192
1579 34.052675587063
1580 34.4427854005318
1581 33.9018651915993
1582 34.1292064256126
1583 34.8915296268259
1584 35.0160789140596
1585 34.1767200142354
1586 33.1120618130713
1587 34.0603969263666
1588 34.252392827548
1589 34.2097912444602
1590 33.4015255245013
1591 33.324291328919
1592 33.3469831282669
1593 32.3995188204945
1594 32.9066787039621
1595 32.8933830023042
1596 32.5747343089221
1597 32.0372552720075
1598 32.0387877293369
1599 32.2212148031538
1600 32.8297886414628
1601 34.0090308700151
1602 34.1522575459134
1603 34.9279658481625
1604 34.9426698767931
1605 35.237246808144
1606 35.3472747370029
1607 35.4189165407123
1608 36.1429511050855
1609 37.7863531888056
1610 37.9593868048732
1611 37.6167202278432
1612 37.8176096634849
1613 37.8694979062601
1614 37.1993023124356
1615 36.9959571667363
1616 37.8457029316958
1617 37.7114968728884
1618 37.0583725200666
1619 36.019950275338
1620 36.2103857867854
1621 35.9408548636737
1622 35.7692612575035
1623 35.6453554138886
1624 35.5146264862943
1625 35.9508599468889
1626 35.706557835907
1627 35.5809950919734
1628 35.1590191716476
1629 35.7653132551214
1630 36.9541112566255
1631 37.3030518378956
1632 37.5408857780583
1633 37.9257885848402
1634 38.406332087929
1635 38.4821488873855
1636 39.2213071808095
1637 39.3584104982193
1638 39.4726347134258
1639 38.1744823248116
1640 36.1746604971798
1641 35.8082842160636
1642 36.1065797350618
1643 35.6376143697237
1644 35.7673627879998
1645 35.7036402565723
1646 34.9085411442256
1647 35.0708355050981
1648 35.1406219440413
1649 35.7335094224175
1650 35.9491979077685
1651 35.4469300738664
1652 34.3659224715151
1653 33.9774697068693
1654 33.0374944674635
1655 32.4151379134657
1656 32.5153254591277
1657 32.3217192644907
1658 32.1136360271851
1659 31.6482826082279
1660 32.499448927753
1661 32.2735059262393
1662 31.9315607108393
1663 32.1344891698168
1664 33.0115235588734
1665 34.1158359901452
1666 33.1491581123587
1667 33.177578926866
1668 33.5757670821449
1669 33.4535310098638
1670 33.0484027966736
1671 33.290288528664
1672 33.7375688446471
1673 33.7285653175718
1674 32.8803394256119
1675 32.232846888959
1676 32.3975735844564
1677 31.969249214526
1678 32.1932048720272
1679 32.3667007230467
1680 32.3498407732776
1681 32.6284459196254
1682 32.8272714237811
1683 32.5449602002659
1684 32.6189930526982
1685 33.1818479897907
1686 33.1682089466301
1687 33.851010817525
1688 32.93601680956
1689 33.2149762713257
1690 33.0270937155342
1691 33.0529620109882
1692 33.1601963663669
1693 32.8185319495242
1694 32.7996589019449
1695 32.792317637909
1696 33.534440286972
1697 33.3630017936369
1698 33.8525781062647
1699 33.4947854540852
1700 34.194698221044
1701 35.7613039459055
1702 35.1607266155883
1703 36.3167958729988
1704 37.3372778354219
1705 37.8826232064214
1706 37.8230323503555
1707 38.0613814277286
1708 37.9319387832748
1709 37.5470345661457
1710 37.1789315993071
1711 35.2836731947159
1712 36.6758797447091
1713 35.8457329826854
1714 35.1024617013733
1715 33.8710097748756
1716 33.0103393843042
1717 32.7079174398732
1718 32.8392269815153
1719 33.8073883119669
1720 33.5516868561816
1721 34.1331366367284
1722 33.5007719341642
1723 33.41511003766
1724 33.4596413838431
1725 32.5803639775549
1726 32.8703163988887
1727 32.4795032422369
1728 32.1742584354468
1729 31.7769159447383
1730 31.2635921461886
1731 30.0440720325185
1732 29.6711976194957
1733 29.53582277381
1734 29.1607622281961
1735 29.4509610463943
1736 29.8247378161506
1737 29.9198089199533
1738 30.0592420034614
1739 29.408033686549
1740 30.2390254569448
1741 30.8918936053904
1742 31.1005578562495
1743 31.112699295002
1744 31.1699772659074
1745 31.3805612593629
1746 30.9204477424082
1747 30.614651096922
1748 30.2085159949459
1749 30.4171139220582
1750 29.6077630593124
1751 29.0778522226003
1752 30.0131756596239
1753 29.8951203598952
1754 30.0249717281693
1755 29.6359494872273
1756 29.7475514530029
1757 30.4006390633264
1758 31.6206883807051
1759 31.5764792338513
1760 31.7835916230011
1761 32.8700179323453
1762 32.0634147753471
1763 32.6866690542183
1764 32.4277488659361
1765 33.0965534493427
1766 33.4246906031388
1767 32.5639286357879
1768 31.5600180116673
1769 32.0425773337296
1770 32.3182346585612
1771 31.3472678623351
1772 31.2632868867101
1773 31.4272402485872
1774 31.8095218836518
1775 31.3943888330528
1776 30.8264236257716
1777 31.1102658225246
1778 31.8747216808138
1779 31.8588268781423
1780 31.6408055569569
1781 32.2149110085801
1782 31.353251547985
1783 30.7875946775816
1784 30.1955273409321
1785 29.9567729164519
1786 29.7536496075156
1787 29.5269473662574
1788 29.7365944860127
1789 29.357983475505
1790 29.6094852283417
1791 29.4176483270935
1792 29.9408479624889
1793 29.7105609502554
1794 30.663149847148
1795 31.1149455924224
1796 31.4492429899768
1797 31.4587880301202
1798 30.6267959554641
1799 30.9019834767054
1800 30.4376260466568
1801 31.3764444547422
1802 30.8913788248852
1803 32.2682297673273
1804 31.9374257060928
1805 31.905521148873
1806 32.5157797844561
1807 34.3039281590811
1808 34.45597467547
1809 34.300925782321
1810 34.6481092089019
1811 33.6627140313607
1812 34.6702226310811
1813 33.241150632306
1814 32.9683508650003
1815 32.8192562252568
1816 32.7032678612052
1817 31.5816166077198
1818 31.3726078209889
1819 32.499955652405
1820 32.3278219024765
1821 31.9897092673829
1822 31.6754897027729
1823 32.5515912487113
1824 32.5931585358697
1825 33.3650560053038
1826 32.6324882061162
1827 32.3362664765398
1828 32.865928018519
1829 32.669885567793
1830 32.5349238117596
1831 33.0060208885829
1832 32.4697810559749
1833 31.5768153883673
1834 31.4784782923235
1835 31.4908040162119
1836 31.5470129770556
1837 31.3065320955177
1838 30.8191542229185
1839 29.9926998597379
1840 30.3321838307944
1841 29.9958500683575
1842 30.2913248048155
1843 31.1794290897394
1844 31.2697901360366
1845 30.6708220319616
1846 30.5571963094689
1847 31.689762496076
1848 32.708059707443
1849 32.679046826025
1850 32.3095468849416
1851 31.9644325060476
1852 31.8332947070141
1853 31.2365967681448
1854 31.0985157721004
1855 31.5772927352163
1856 31.2509594324359
1857 30.2707703457968
1858 29.3338207078255
1859 29.8439956092684
1860 29.5977969515182
1861 30.4627213357505
1862 31.3956802927321
1863 32.009024346184
1864 32.6090754296868
1865 32.7686843113896
1866 34.2764969191071
};

\nextgroupplot[
tick align=outside,
tick pos=left,
title={fc\_layer\_0},
x grid style={darkgrey176},
xmin=-93.3, xmax=1959.3,
xtick style={color=black},
y grid style={darkgrey176},
ymin=21.1242095989767, ymax=89.7367894783568,
ytick style={color=black}
]
\addplot [semithick, steelblue31119180]
table {%
0 36.4213372834289
1 34.7132012165698
2 32.6860659322945
3 31.8847820386484
4 30.9119954538345
5 29.6313371260392
6 28.4353792945378
7 27.6754129416319
8 27.1899795115361
9 26.5497096480772
10 26.0645312527415
11 25.4889668066099
12 24.8779630171309
13 24.3698408638774
14 24.2429632298576
15 24.4690265943802
16 24.7350362672842
17 24.931546417625
18 25.2093961653609
19 25.6465764081251
20 26.1321043450345
21 26.553715268415
22 26.9411767283572
23 27.1333383262036
24 27.4039328632479
25 27.5979504822454
26 27.9187609450206
27 28.3187376231745
28 28.2436799093099
29 28.5244857491285
30 28.8634437359659
31 29.2035676168996
32 29.2140581899182
33 29.5329566136315
34 29.5870155124264
35 29.8639719219418
36 29.7797671455684
37 29.5384018585493
38 29.7733238443068
39 29.7532348467053
40 29.9640489465044
41 29.9257731829212
42 30.452271430011
43 30.4836198854716
44 30.8603482959768
45 31.1566063594164
46 31.8189015099557
47 32.1684986646561
48 32.1471041104332
49 32.5765539247806
50 32.9260046494891
51 33.149104211212
52 33.0488406688136
53 33.4240830075406
54 33.3338417137488
55 33.6616559398138
56 33.39184699891
57 33.7091885752327
58 33.7567043834667
59 33.9265763416129
60 33.3438427081412
61 33.4913061011517
62 33.7649658493124
63 33.6419181711116
64 34.0032238558394
65 33.250617050299
66 33.2770900742649
67 33.0932235970614
68 33.318888369496
69 33.1057605930844
70 33.2067394634202
71 33.4874482602511
72 33.1337748418669
73 33.0047177981613
74 32.791246727227
75 33.1642791523926
76 33.1719701564147
77 32.9971076956821
78 33.0330265419075
79 33.0126007694918
80 33.1721013223055
81 33.0628167471358
82 33.3747579003928
83 33.3700954566045
84 33.2923224886569
85 33.2674474512016
86 33.2578932500557
87 33.6470956940608
88 33.4356776433357
89 33.2886801487274
90 33.1142258167024
91 32.7998072601642
92 32.5238671348115
93 32.7310489315509
94 32.7391329583552
95 32.8147937932354
96 32.5826007934473
97 32.4497050444105
98 32.6138270913073
99 32.8746837478081
100 33.0084000865911
101 33.3336448244638
102 33.4273730527369
103 32.9913395373911
104 33.1920371148386
105 33.4721486399761
106 33.7935657111914
107 33.70422005836
108 33.8849024620701
109 34.0302811662652
110 34.2815293870412
111 34.4485521114309
112 34.5554387465609
113 35.0144540087215
114 34.834429723352
115 34.4465953661304
116 34.3625658777646
117 34.1995600352151
118 34.4598300051334
119 34.2317966985953
120 34.0623786493352
121 33.9934309845025
122 34.0789931502202
123 34.4247438770501
124 34.970379227822
125 35.046127371654
126 35.1700428848302
127 35.3476805773754
128 34.9766829988479
129 35.1088139328075
130 35.6305586746244
131 35.5892939133453
132 35.9480763066986
133 35.9393949669657
134 35.8336386823614
135 36.206028319778
136 36.2746389530012
137 36.6360962234602
138 36.800189560487
139 36.7156472682763
140 36.7833569598679
141 36.8251469149488
142 36.6130674404819
143 37.1258991599927
144 37.0029771503357
145 37.0595068452812
146 37.3484456545491
147 37.3783299079252
148 37.3205442387824
149 37.6133537061666
150 37.3327999597309
151 37.4533043772372
152 37.6867742631753
153 37.2512687481486
154 37.4699440751712
155 37.2819455695989
156 37.1880187771414
157 37.2870756853602
158 37.2293401641897
159 37.141959009278
160 37.6338068132556
161 37.6246333362641
162 37.8357204024609
163 37.8610941986168
164 37.8865469594956
165 38.0606127469288
166 37.8216410568921
167 37.5869810848065
168 37.5041474988173
169 37.4391455300539
170 36.7248758512939
171 37.0382917531281
172 36.7175358737424
173 36.6174665040403
174 36.1831436540519
175 37.0789417455351
176 37.421049886126
177 37.0629328122236
178 37.2911936223629
179 37.0133400478099
180 37.23633003357
181 36.8748886754756
182 36.9266907089658
183 36.521023734438
184 36.7006125986932
185 35.5677670206565
186 35.6917489235841
187 36.4669650870603
188 37.0172700812796
189 36.9850568990633
190 37.0623703016337
191 36.826256208551
192 36.8445161583085
193 37.6026312120293
194 37.7674778832952
195 38.1150056787267
196 37.8082945612095
197 37.5020793712589
198 37.4034530128978
199 38.4969058714319
200 38.7321889283008
201 39.1276690164217
202 39.3326588117119
203 39.5089126058517
204 39.9537213209625
205 40.1282828026217
206 40.3038166756945
207 40.0805392078789
208 39.817842706739
209 39.2381203453365
210 38.9595677555952
211 38.8280842469175
212 39.0904369105539
213 38.7626453124831
214 38.6972856001203
215 38.4602138043624
216 39.0589811723346
217 39.4657772736328
218 39.3073689668286
219 39.5154551720081
220 39.4125221714381
221 39.4683262764694
222 39.0878428950587
223 38.9878414615658
224 39.1761309291952
225 39.088006316529
226 38.4224723839469
227 38.9195245656701
228 38.9264026259548
229 38.6578117519467
230 39.2010949089197
231 39.9317745031469
232 39.8891229619791
233 39.8210103628329
234 39.2752874570176
235 39.7811283219189
236 39.5510784878031
237 38.4292408510706
238 38.4041422402798
239 38.5883711982668
240 38.2589563978421
241 37.8858765128082
242 37.9496737353529
243 38.1660363704367
244 38.5843866920957
245 38.3313013998127
246 38.8707905706063
247 39.6579739644083
248 39.6760421072319
249 39.6713614550165
250 39.8572485919796
251 39.670761697681
252 40.0968387626388
253 39.9109763955076
254 39.8108018997839
255 39.7660852708796
256 40.0129582460183
257 39.7756313018749
258 39.9694129023266
259 39.7623502739212
260 39.6410212566593
261 39.6036875846706
262 39.2441982757072
263 39.178365893351
264 39.1708244443705
265 38.5234210124357
266 38.0762041129654
267 38.0150284716975
268 38.2406295683174
269 38.6048795003197
270 38.9674912587344
271 39.1756160024303
272 39.4110310179065
273 39.8796469732091
274 40.1994750324428
275 40.808299592314
276 40.9993322916002
277 41.1952020465669
278 41.4627230451785
279 41.2521523346004
280 41.1544828407768
281 40.6298288980857
282 40.3364171693396
283 39.7528427897903
284 38.9738090813721
285 38.8085412129832
286 38.472650013599
287 38.0699862724561
288 38.0002570058578
289 38.5149496412635
290 38.7790571260882
291 38.6059493507552
292 38.6898821635224
293 38.8705867840028
294 39.0212184504883
295 39.3023548619571
296 39.8475533884462
297 40.411416169264
298 40.0654798564115
299 39.4981655260759
300 38.8953433542478
301 39.6555422933644
302 39.8460791677096
303 40.0744229114611
304 40.560888751534
305 40.8056948980473
306 40.5543122358702
307 40.3820753134918
308 40.7704517147498
309 41.0259874480341
310 40.9778556304289
311 40.8320305588457
312 40.4518956906248
313 40.1928497218078
314 39.8707802816505
315 39.4329695228911
316 40.0077859461648
317 39.7704122702468
318 39.6177725112287
319 39.6060842441537
320 40.0408912122907
321 39.8083544132324
322 40.6738403521259
323 40.8952728207901
324 40.8078567235553
325 41.1722013306079
326 40.8532384204414
327 40.738571425254
328 40.7272588209584
329 40.5142529075066
330 40.4012808158868
331 40.8004666091564
332 40.02212783029
333 40.2789639528454
334 40.4586648967611
335 40.9794426119398
336 40.9136719356768
337 41.3707301408568
338 41.3341916059341
339 41.9092888061729
340 41.7091691845365
341 42.0590677856857
342 42.1477288451542
343 41.9503755779186
344 42.053878538984
345 41.6128958513632
346 41.6869103969776
347 42.4138785549358
348 42.4540481163118
349 42.2141284192697
350 42.6539857771575
351 42.0983671371548
352 42.4917650408813
353 43.1356733952346
354 43.52531540197
355 43.4919331804667
356 43.4766651569864
357 43.0577371444435
358 43.6148649771439
359 43.4955351650169
360 43.054734944251
361 42.7353854018683
362 42.5922588851544
363 41.6841955505203
364 41.3489508756191
365 41.4365044655296
366 41.4104289401835
367 41.7329455901719
368 41.244481578808
369 41.4833930244684
370 41.9274207097952
371 42.8270554533352
372 42.9385529073278
373 43.3000341739862
374 43.3749368644817
375 43.375380766772
376 43.4909888337071
377 42.2004046557364
378 41.9769412809524
379 41.5219627144791
380 40.978040803676
381 40.6070226279534
382 40.3146873131082
383 40.3293372643498
384 39.7471506967597
385 39.1387454665186
386 38.5769384297229
387 39.2056650611295
388 39.5042125347931
389 39.7154990071743
390 40.0004354790664
391 39.7694235158444
392 39.5169176251114
393 39.618467528868
394 39.9794816517111
395 40.1793540825727
396 40.0529902339323
397 40.4086819987007
398 39.8015289333962
399 40.2326059577946
400 40.3296804221223
401 40.1852227011602
402 40.4555824514245
403 40.3075171742725
404 40.4037766736379
405 40.7730604657104
406 41.4820039758888
407 41.2183132464802
408 41.6973302918302
409 41.1826204277217
410 41.458007391386
411 41.8938804616889
412 42.059024698624
413 42.0681548910247
414 41.9431423740773
415 42.0185557080129
416 41.4040065991033
417 41.5026954083061
418 41.7142590729939
419 41.8439306930833
420 41.5894831981239
421 41.8938896927711
422 41.3775507620836
423 41.0964861973129
424 41.7162547294247
425 41.5017857483162
426 41.614686589555
427 41.4013501488804
428 41.1275234384523
429 40.8447214277551
430 40.9915960644146
431 40.4078469361212
432 40.7057788318798
433 41.8296193759597
434 41.241462483937
435 41.2959673466423
436 41.3262853426352
437 41.5870682058716
438 41.776871186703
439 42.1538034599801
440 41.8287245760199
441 41.9226594607345
442 42.0117507344541
443 41.3492204177469
444 41.2908775966202
445 41.0799395883978
446 40.9939428706072
447 40.7190340011147
448 40.4626957858675
449 40.1272240192116
450 40.3275665965242
451 40.2787180041907
452 40.2481303759454
453 39.943191146644
454 39.7122004181489
455 39.7110048260555
456 39.7198406303144
457 40.2166045717625
458 40.8278788807376
459 41.2226900280688
460 40.8153870764827
461 41.3372701738202
462 41.6393858084014
463 41.4702766777834
464 42.0143917243737
465 42.1613852946472
466 42.1982684777765
467 41.9422172877442
468 42.0062310376574
469 41.7334298656271
470 41.9926141076535
471 41.7811257808786
472 41.3418220417613
473 41.5095399731388
474 41.5222268527666
475 41.5117883708455
476 42.082959606684
477 41.7995872262802
478 41.0705775292594
479 41.3116508774333
480 41.1798258031193
481 40.9401162150559
482 41.4925453544149
483 41.5330596021145
484 42.2722001366206
485 42.4181887150169
486 42.2301541043073
487 42.3231069792746
488 42.4027789852137
489 42.2338362677053
490 42.8666725337688
491 42.7503349830585
492 42.4475175211303
493 43.0158195901446
494 42.3084138443545
495 42.2742896886224
496 42.0939091236227
497 42.5036026270562
498 43.2201672542466
499 43.2121679064228
500 42.3298120733961
501 42.6717856882242
502 42.469963879744
503 42.1761640506099
504 42.435831855481
505 42.6510004996807
506 42.6837752376514
507 42.5155231710965
508 42.2116404840026
509 42.7650107883481
510 43.6622242507978
511 44.6224169937695
512 45.2300393265821
513 45.0469405524174
514 44.825329820655
515 44.9468783929269
516 45.1804767494187
517 45.6363299245822
518 45.6662563293505
519 45.1475384875266
520 44.7487794333699
521 43.988674571943
522 43.5936719051306
523 44.1157500880808
524 44.3598791215699
525 43.8746038232226
526 44.2281315782907
527 43.868817262619
528 43.4980795859731
529 43.8935506148788
530 45.0010547946916
531 45.3475487962709
532 45.8987837568939
533 45.3795525500635
534 45.4031837692838
535 46.0211228274275
536 46.0614734782823
537 46.1635256253433
538 46.3973320525163
539 46.1063163809918
540 44.7886229782778
541 45.4160779928727
542 45.0958858043596
543 45.335132915902
544 45.2201080576527
545 45.0339757978411
546 44.9488880232567
547 45.1085111885374
548 45.2876577394376
549 46.167103882805
550 46.3947942007902
551 45.5215344981437
552 45.5172090308731
553 45.2828301321665
554 44.829352799647
555 44.2753101263716
556 44.3617635205489
557 43.8976122384163
558 44.1116328428799
559 43.4703319011855
560 43.3686221023077
561 43.4905166998951
562 43.1400619328651
563 43.5232538408654
564 44.307195452549
565 44.4877086443651
566 43.7674492085851
567 43.6572983664056
568 43.9540555109804
569 43.8644575003193
570 44.0802480306973
571 43.6087107169996
572 43.9242712385818
573 43.6125028438418
574 43.0780078645332
575 43.3117404228833
576 44.1414468616735
577 44.5330897724175
578 43.8215671897025
579 43.7808553049902
580 43.5793212245334
581 43.7016315281559
582 43.6529079880711
583 44.1814552417788
584 44.0064508795035
585 43.9033810633066
586 43.166692724973
587 42.9663765353742
588 43.6634434442762
589 43.5879525451719
590 43.9264239108553
591 44.4105844904253
592 44.8717630410139
593 44.4588495143057
594 44.5727756553613
595 44.8664690079475
596 45.4391903484915
597 45.4106607028506
598 45.4820832994826
599 45.6705904050395
600 46.0478917595829
601 45.354197903818
602 45.0854641982798
603 45.5087542661343
604 45.429400738502
605 45.2051680101303
606 45.5608766553689
607 46.2509056899531
608 45.7336811323419
609 45.5968357141538
610 45.2330208895723
611 45.9119646270575
612 45.9184982143216
613 46.184069660977
614 46.3718047698866
615 47.0158628819314
616 46.5996513016563
617 46.1197010418476
618 45.7918654420979
619 45.6932991458987
620 45.6669437804382
621 45.1063132269516
622 44.9764568554207
623 44.8784991735982
624 45.1807404437245
625 44.6121641570455
626 44.3946605186781
627 44.6818454487672
628 44.5367462275081
629 45.1665353816403
630 44.9925072127046
631 44.9256279887931
632 45.1155997893094
633 44.2004189827546
634 44.2235745067721
635 44.0860803940084
636 44.3793585356885
637 44.1464353804422
638 45.1005389359902
639 45.0952044402919
640 45.343234535725
641 45.5466802802731
642 45.1848236654404
643 45.9694550346764
644 45.9322895343087
645 46.0779946259079
646 45.693481255776
647 45.8736157255981
648 45.3900510449012
649 44.7825810808433
650 44.5852116540564
651 44.5011187476964
652 44.7545512689867
653 44.0557759845807
654 43.655600803441
655 43.2710945095849
656 43.2424824122876
657 42.732551773397
658 42.8896013706264
659 43.2072038197275
660 43.7380505455559
661 44.1810158971389
662 45.1638775331018
663 45.3317305582644
664 45.5268039467411
665 46.4560215712158
666 47.072766566856
667 47.4700578940933
668 47.5736558925601
669 47.3736082377914
670 46.7916420310363
671 47.0662785538052
672 45.9313931501177
673 45.7386426812364
674 46.145446138178
675 45.7421756853205
676 44.9104765389284
677 44.5518814108061
678 44.0811498767413
679 44.146145008982
680 44.552022845004
681 43.8565756324686
682 43.9659413181015
683 44.1561066633919
684 43.7051535713268
685 43.9354842170302
686 44.2426110742302
687 44.3338804144745
688 44.7737942997516
689 45.1158760406016
690 44.8342831179004
691 45.1944804408179
692 45.1314754950239
693 45.8565214899072
694 45.5515184465415
695 45.3101455167618
696 45.4599030418518
697 45.5988755047074
698 45.1545994737799
699 44.9153162186644
700 45.1287473058258
701 44.632389210411
702 45.0894091965803
703 44.6355610504928
704 45.3777682282993
705 45.4690789134535
706 45.4409561414405
707 45.8953390058958
708 46.1630083737769
709 46.3394037700513
710 46.4192034521548
711 46.409732578227
712 46.3408989317693
713 46.4997215353577
714 45.8307776780582
715 45.4983928507254
716 45.4444837725853
717 45.6934208912567
718 45.6765973648186
719 45.1539366044013
720 44.7953775960712
721 45.2688514264182
722 45.1611118357302
723 44.6642096987027
724 44.7767315753021
725 44.9509551748424
726 44.6744648665126
727 43.9328390284066
728 44.2694300508817
729 43.8311309779526
730 44.672094911942
731 44.7454913057445
732 44.2599860388333
733 44.0110652616642
734 44.4257407245563
735 44.604167192419
736 44.8783500036712
737 46.1032715496996
738 46.2117923933839
739 46.8712451218262
740 46.6013098594522
741 46.2856019878247
742 46.8343819535397
743 47.4904561661743
744 47.1771962141672
745 47.1488822663485
746 47.3651349625463
747 46.2958386074164
748 45.676242787465
749 45.3460535219298
750 45.6929753634827
751 45.7087046395319
752 45.6715994182799
753 45.8108393722175
754 46.1650096051031
755 45.9915935882784
756 45.8289753476024
757 45.6780015788866
758 45.862648012313
759 46.1804025013454
760 45.6777420120325
761 46.182535358005
762 45.8813770033657
763 46.0383219177467
764 45.4753581087323
765 45.7093818129334
766 46.3247902042546
767 46.1564893425821
768 46.8884100134648
769 47.0503441429233
770 46.5085682524499
771 46.1540445909925
772 46.7585616417086
773 46.3072684058657
774 47.4762793307743
775 47.3828874556931
776 46.8616152379068
777 47.7754919384624
778 47.4410335566154
779 47.6969533517252
780 47.6992902520546
781 47.4131097446507
782 47.3923045914543
783 47.1824030201075
784 46.7634927004924
785 47.3585954547354
786 47.3285941946579
787 46.9725657524742
788 47.1326374227602
789 47.1022849690144
790 48.2034691594463
791 48.8623066914288
792 48.699160066186
793 48.8705630395894
794 48.4812835368076
795 47.9426325441439
796 48.0631132726319
797 47.7126430834762
798 47.7368439277321
799 47.9782025114016
800 47.212400324863
801 46.7254360746758
802 47.0106789397462
803 47.3973727312421
804 47.6883701990343
805 47.8839401601281
806 48.4543577634152
807 48.3750147877912
808 47.7961494560363
809 47.1274684503296
810 47.0750767044573
811 47.4642729137914
812 47.4171017806016
813 47.7192436566391
814 47.1081872725906
815 47.503886446778
816 46.977470387681
817 47.0211373017446
818 47.1636852992863
819 47.7030782058376
820 47.8509846273223
821 47.7940143462354
822 47.3859906727138
823 46.8505319131657
824 47.9106285735285
825 46.9242753159907
826 46.8917912700391
827 47.3419189156395
828 47.6050347248886
829 47.9214470172512
830 48.0292701578668
831 47.8796520413409
832 48.2577323433412
833 48.6669535833481
834 48.2052905768333
835 48.8411837974305
836 49.0960437177996
837 48.8172425333486
838 48.2645653670249
839 48.0173637685183
840 47.9703281475103
841 48.1645587176842
842 47.8072341663492
843 47.3692429255363
844 47.1735865017314
845 46.9428286642022
846 46.2899693728472
847 46.8311866939533
848 46.9088107812183
849 46.2076166906147
850 46.6464170023583
851 46.7383110745502
852 47.0736240175165
853 47.2835851118969
854 47.45299180463
855 47.9091948873722
856 48.7792872081892
857 48.7775765734489
858 49.0015294613946
859 49.5215694846198
860 49.1283587685363
861 48.6478038391212
862 48.3716148799927
863 48.8817842903048
864 48.8973648789027
865 48.2186366739874
866 47.9568736650551
867 47.7993326137108
868 48.0323227035733
869 48.1630667487601
870 47.9020389104068
871 47.9865507516998
872 48.7627001857696
873 48.2760782236688
874 48.139219115727
875 48.2356617537325
876 47.7618295681011
877 48.0416007188033
878 48.2228116389729
879 48.5527691544508
880 49.5080921134756
881 49.7148786850256
882 49.9142481058296
883 49.6516052700884
884 49.9216598104773
885 50.6599466685936
886 51.0584713701404
887 50.5605709084822
888 50.1177622091512
889 50.0968139388208
890 49.6082874043663
891 49.9008389185396
892 49.1542533844981
893 49.6224698786314
894 49.8221113535456
895 49.0879514845336
896 48.8681409049502
897 48.954184979379
898 48.6821079139706
899 48.049417026945
900 47.8121868125176
901 47.3343938692832
902 47.382873777767
903 47.2228172941705
904 47.2297629519898
905 47.8000211321335
906 47.7262165616175
907 47.5356529027574
908 47.6638503229984
909 48.034946980802
910 48.0590608291504
911 48.1420327054652
912 47.5178336251027
913 47.3960212042592
914 47.0091838996939
915 46.5110782763137
916 47.2253691493532
917 47.5177191412078
918 47.7235529657724
919 47.5372675002981
920 47.6005436776462
921 47.8443016917911
922 48.2569036621596
923 48.3214020128941
924 48.4192637254463
925 49.0226950120174
926 47.9992380727886
927 47.7694167877942
928 48.1153983944464
929 48.9263131410618
930 48.8222922781562
931 48.7713767038183
932 48.8300955627056
933 48.8079835659597
934 49.1620246814684
935 49.4191911228779
936 50.2222498887136
937 50.4545659849485
938 49.9501113241876
939 49.996283878806
940 49.6323915714896
941 49.5438705695844
942 49.403920148235
943 49.5818008392235
944 49.0915516179383
945 48.1482730430564
946 48.497037451441
947 48.1274778935466
948 48.0481451336818
949 47.4822223838812
950 47.6193837368481
951 48.1011424210953
952 48.3714953744813
953 48.3743932727928
954 49.4030165590419
955 49.8170235696644
956 49.7184014909384
957 50.1984066025494
958 50.5702838491522
959 50.0314701637065
960 50.3651368323385
961 49.7758722088205
962 49.8791040886266
963 50.080276754552
964 49.1913460489477
965 49.5866416907711
966 49.0548325999026
967 48.850122611835
968 49.468494089774
969 50.7584909511416
970 51.0960474092119
971 52.054557486432
972 52.8306127758914
973 52.4610630099736
974 52.6352075754973
975 52.4500448165736
976 52.5665826872748
977 53.1998052720096
978 52.4152847853168
979 52.211451386167
980 52.5772200777745
981 51.7624040687878
982 51.3596086188435
983 51.5552513867449
984 51.637877989376
985 51.9559040552743
986 51.8522442904525
987 52.1581049807016
988 51.9845704379199
989 51.1196706366984
990 51.05921689052
991 51.4276903463736
992 50.9743762076976
993 50.9434419412206
994 51.6504493658406
995 51.3402462674708
996 52.0089148678223
997 50.9639440519599
998 51.8425815586745
999 51.9292920707775
1000 51.4787864693166
1001 51.514295542392
1002 51.5244923483754
1003 52.0206827036587
1004 50.4956539175453
1005 50.048747408756
1006 48.8598861198746
1007 48.8806851517497
1008 48.4703582119996
1009 48.9298779036433
1010 48.8171137822247
1011 48.4441796476271
1012 48.9863397196893
1013 48.7989633150759
1014 49.2771948872034
1015 49.1424272881539
1016 50.2760667749494
1017 50.5716569786387
1018 50.4878554998346
1019 50.2824729452781
1020 50.3508843835301
1021 50.7859134052176
1022 50.3306571022951
1023 50.7826864101709
1024 51.4265659822515
1025 51.8055059658367
1026 51.1257454978951
1027 50.8354741671011
1028 51.2552488996975
1029 51.3287949716161
1030 50.9990406223192
1031 50.727584511112
1032 51.0985559686176
1033 50.4677216389301
1034 49.7668047477483
1035 49.9117923141088
1036 50.4030040180974
1037 50.5028373854546
1038 50.4588888086866
1039 50.1726855692171
1040 50.4378778208581
1041 50.7630009552512
1042 50.5393580079473
1043 49.6457779393407
1044 50.0407550759423
1045 49.8556815007095
1046 49.8569155196428
1047 49.9166627154858
1048 49.4422683811012
1049 49.2229091927201
1050 48.9365502927188
1051 48.7502827135702
1052 48.2285164134218
1053 48.7899386016949
1054 48.5544437681663
1055 49.2147664310838
1056 48.7847634173551
1057 48.9347064058627
1058 48.9194455361783
1059 48.8065558620654
1060 49.1377815460679
1061 48.6288383799479
1062 49.0887264631785
1063 49.1865749564446
1064 49.1384086037056
1065 48.2222281997311
1066 48.6774882952436
1067 48.0178159749124
1068 48.8870086360641
1069 49.2857421351113
1070 49.629478135147
1071 49.5295254475711
1072 49.6591256254136
1073 49.7910747804393
1074 49.8358265973634
1075 49.9248105156748
1076 49.4906262588655
1077 49.5510084629402
1078 48.1475299983245
1079 48.8215612186523
1080 48.008427012134
1081 49.16353855225
1082 49.7042239238843
1083 50.091510283725
1084 50.333521680679
1085 50.5571691555045
1086 50.9220239681169
1087 51.2037046869562
1088 52.3460099162608
1089 52.283249300504
1090 52.4479763850031
1091 51.7746035222218
1092 51.5145633951645
1093 51.7012289055693
1094 51.7035815069386
1095 51.8431581138088
1096 52.0766182722782
1097 52.268391307498
1098 52.3248820571444
1099 52.227056219648
1100 52.4421375834732
1101 52.5860183070321
1102 52.322233409356
1103 51.6433116045421
1104 51.9589981483789
1105 51.9537892998437
1106 52.0926764572084
1107 51.7235825786671
1108 51.0128314084003
1109 50.3197846958229
1110 50.9582924694892
1111 50.4449058543726
1112 50.0455327496395
1113 51.8457176090114
1114 50.952661185666
1115 50.9351765758942
1116 50.4628958585031
1117 51.0690744907583
1118 51.1190027504574
1119 51.0478514962873
1120 50.8424304781971
1121 50.9659183979236
1122 51.0834735982973
1123 49.0247107183465
1124 49.0316171843683
1125 49.1846377012632
1126 48.9823984638706
1127 48.7460701806887
1128 48.6718583939524
1129 49.1864386281267
1130 49.617463224101
1131 49.5663182241542
1132 49.9090804250766
1133 50.5449340077
1134 50.9017722806094
1135 51.2773515040169
1136 51.3067402959867
1137 51.4507708653942
1138 51.4631776540654
1139 50.8753074661742
1140 51.1653241727506
1141 51.221189153797
1142 51.5289635008555
1143 51.5653850300626
1144 51.2496645902831
1145 50.7184243980334
1146 50.7767661084951
1147 50.7267295294638
1148 51.1784267559871
1149 51.3227586037351
1150 50.0268421679697
1151 50.4612798090735
1152 49.6084695427342
1153 48.5565838982441
1154 49.1007823902972
1155 49.1969235384904
1156 49.2213074010309
1157 48.8303559124484
1158 48.7711871444977
1159 48.6656105674424
1160 49.3104110856515
1161 50.4556186349196
1162 50.723049121898
1163 52.3265044547765
1164 52.3959803937466
1165 52.0415405971375
1166 52.1870673660657
1167 52.7372831925306
1168 52.4560443298178
1169 52.7420823391383
1170 51.8304512582346
1171 51.4720697820843
1172 50.6237053198217
1173 50.7355201945087
1174 50.1035651780398
1175 49.9637913698503
1176 49.8358644947339
1177 49.776912876964
1178 50.1303576474334
1179 50.1103155622758
1180 51.1556342808301
1181 50.1919677310854
1182 50.7212556771332
1183 50.2017701134203
1184 51.1804348898631
1185 52.0825634193211
1186 52.7916667911988
1187 52.6358169316696
1188 52.1624235524437
1189 52.1701999848671
1190 51.7179729222038
1191 51.8741851056147
1192 51.8826647934236
1193 51.3453045435618
1194 50.9071061472907
1195 50.3893337684943
1196 49.9143109104048
1197 49.5518244746333
1198 50.1785654562345
1199 50.4635726039276
1200 50.4660839750726
1201 50.6256128094887
1202 50.7343161979841
1203 51.577296576981
1204 51.8345820741625
1205 52.053743319515
1206 52.1138793025888
1207 52.6214656240513
1208 52.9642786940524
1209 53.2097541141744
1210 53.3891108490891
1211 53.2929798232981
1212 53.5285886322266
1213 52.980726752805
1214 52.7901296486452
1215 53.3451240462392
1216 52.7322030012697
1217 52.5558185674778
1218 51.8504887103675
1219 51.0879449420227
1220 50.6462877049926
1221 49.9954477051808
1222 50.0165937385831
1223 49.7040460851497
1224 49.3750626014965
1225 48.7659672379215
1226 49.2853424723295
1227 49.6274893934422
1228 50.0444514875202
1229 50.3723931843173
1230 50.3212446624615
1231 51.3121073063888
1232 51.4458951037394
1233 51.841271169209
1234 51.7011034838302
1235 51.1340052816409
1236 50.5225667227229
1237 50.157425231869
1238 49.5357784845944
1239 49.956943475109
1240 50.8995645117894
1241 50.4123981936471
1242 50.3882870594346
1243 49.9328662274543
1244 50.3482998537788
1245 50.1764934794044
1246 50.8927723191714
1247 50.9753851784005
1248 51.1846886635687
1249 50.7137469973497
1250 50.1356276748485
1251 50.6718535277067
1252 50.5559836775662
1253 50.8808291047659
1254 50.9343870365897
1255 51.4232196125195
1256 51.1648509924688
1257 51.2919590398984
1258 51.6579290564698
1259 52.0344960049853
1260 51.9341969163
1261 51.9925471514347
1262 51.397505530339
1263 51.8402450166317
1264 52.1562578131206
1265 52.195502448845
1266 52.5779104641898
1267 52.8195785214831
1268 52.7129909105701
1269 52.8934489386267
1270 52.7074941885356
1271 51.8553694637393
1272 53.5033346102913
1273 53.5476846092415
1274 54.562544326225
1275 54.6740748318593
1276 54.5081818451538
1277 54.5092823992985
1278 54.8495741184307
1279 55.1482603947213
1280 55.184312786186
1281 55.4076301605368
1282 54.8026452086172
1283 54.4720103089282
1284 53.2043956414178
1285 53.4777108688153
1286 53.3419040506599
1287 53.9968726624793
1288 53.2405371233311
1289 52.6097061731655
1290 53.1488967238461
1291 53.2116594380656
1292 52.781834201874
1293 53.3082592492748
1294 53.2559029947209
1295 52.4516019550994
1296 52.7079007218226
1297 51.8000298260597
1298 51.8881174306208
1299 52.6265729798882
1300 53.2484875262611
1301 53.0953430863201
1302 53.0893877097276
1303 52.7170054752419
1304 53.2795634052886
1305 53.3647300544259
1306 53.7038885475007
1307 54.2001662536743
1308 54.4874893967169
1309 54.0326992591534
1310 53.1895038004699
1311 53.5925652093268
1312 53.1675978021984
1313 53.5666109518728
1314 53.2382597147483
1315 53.7011381101137
1316 53.0709086583233
1317 52.1502753001851
1318 51.9093051076849
1319 52.18517004898
1320 52.5292903634299
1321 52.4571289129601
1322 52.4420916058675
1323 51.3822798238254
1324 51.2083098468162
1325 51.2077824914771
1326 50.9087482702545
1327 51.3570167238098
1328 51.3703933391497
1329 50.0958016989604
1330 49.8757954602832
1331 49.2029215418154
1332 49.4171647384526
1333 49.7041255547329
1334 49.7276544828824
1335 50.1731158784799
1336 50.4816018450545
1337 50.1815385994945
1338 50.415931619674
1339 50.6416642253805
1340 50.2893812477162
1341 51.0171135725262
1342 51.2113376613571
1343 51.3978186111679
1344 51.5459790955819
1345 50.9014484912357
1346 50.8238101870979
1347 51.5880354866904
1348 52.0471887145404
1349 52.7196628040536
1350 52.7530407170636
1351 52.4923822063862
1352 52.2926778871581
1353 52.4232151845327
1354 51.996129503324
1355 52.1429435212753
1356 52.3500239650175
1357 51.6988242106203
1358 50.9442047470843
1359 50.5190360449844
1360 50.4549496485532
1361 50.5400256666906
1362 50.4384671975628
1363 50.4707331762301
1364 51.0040047655371
1365 51.5828241168845
1366 52.4190131685994
1367 52.6583708494689
1368 52.9900136317427
1369 52.9486234225163
1370 53.3106931999351
1371 53.3594881742129
1372 54.7290730343351
1373 55.1222155155566
1374 55.0105084747641
1375 54.3721658850562
1376 53.8398891513743
1377 53.5680730742412
1378 53.6476092818642
1379 53.8310487832669
1380 53.7936156359284
1381 54.4068071802217
1382 54.1410409871934
1383 53.8511794112086
1384 53.6755316665976
1385 55.0895810413912
1386 55.219417419515
1387 55.144585552785
1388 55.8092509980987
1389 55.6006098193786
1390 55.163695905839
1391 54.1620674220113
1392 53.2182119626439
1393 52.2455390258026
1394 51.9864365591837
1395 50.4710521972815
1396 50.3218258542566
1397 51.5990879537717
1398 50.3722034739347
1399 50.6448215346386
1400 50.8443422607551
1401 50.9409533069741
1402 50.8868842855798
1403 53.0043591653551
1404 52.6940267877607
1405 53.8160163612687
1406 53.5301778214903
1407 52.4994528103045
1408 53.1565639965058
1409 53.0546731254112
1410 52.8916499625229
1411 52.7667593376312
1412 53.2370346758767
1413 51.6928012784634
1414 53.2836226694686
1415 52.5678186376927
1416 52.6966365868462
1417 52.3441665590506
1418 52.2743754272547
1419 52.4631582978483
1420 53.6432108118944
1421 54.3458501851045
1422 54.3192251799285
1423 54.21683500295
1424 53.2221084128742
1425 53.5882528650496
1426 53.9267677402761
1427 54.2371917645928
1428 54.1861669377212
1429 54.2381452053977
1430 54.0942279398236
1431 54.1088614926342
1432 53.4890862414808
1433 53.5266161710212
1434 53.8243210540079
1435 54.3738552308282
1436 53.548275594717
1437 53.9882281414819
1438 54.2673808999629
1439 54.6082514265062
1440 54.4668057860773
1441 54.680770001426
1442 55.2038704116704
1443 56.4826705550695
1444 56.2491938035346
1445 55.1449661005121
1446 55.6187308913915
1447 55.32643128919
1448 55.2464548698438
1449 54.8250789276069
1450 55.061899956469
1451 54.1969159172063
1452 54.3347060803401
1453 53.5117728444458
1454 53.8015815036691
1455 54.0702752386804
1456 53.7844382608857
1457 54.2276484513826
1458 54.9151926399867
1459 55.0993666885474
1460 55.0661063202524
1461 56.1024837496037
1462 55.9017503694704
1463 55.8116775985152
1464 55.5538809423953
1465 55.3445816121793
1466 55.8935039221382
1467 55.3397147339396
1468 53.8823205969819
1469 53.7824230293475
1470 52.9474964087703
1471 51.6714306141579
1472 52.497171412571
1473 52.1970501919002
1474 52.4136120710518
1475 51.8513157093229
1476 51.8510943630036
1477 51.7688737283883
1478 52.3291003969427
1479 52.5712374060057
1480 52.4543421212697
1481 53.3481229075531
1482 52.759881796727
1483 52.9148532323555
1484 52.4362707974257
1485 53.2192513696578
1486 53.0892274289758
1487 53.2762286925975
1488 52.8782244937689
1489 52.7163618733814
1490 52.9974065150922
1491 52.5939988287462
1492 52.2551943836579
1493 51.8051343542731
1494 51.5611162157743
1495 51.7670620219604
1496 51.4127942170907
1497 51.6065867167711
1498 51.9706898961434
1499 51.7046507187837
1500 51.8069682336796
1501 52.0525144410716
1502 52.1598853184358
1503 52.5541386666629
1504 53.3376918041841
1505 53.7815611242191
1506 53.5601373830453
1507 53.9476665331537
1508 53.7828018861951
1509 54.3662426864278
1510 54.5646748275542
1511 54.5070417773666
1512 54.6783689552413
1513 54.7192064409966
1514 54.6606761542471
1515 53.9193562170067
1516 53.8875734737879
1517 53.2738070640733
1518 53.0117888712957
1519 52.3208500735166
1520 51.9141222095262
1521 52.5749738414358
1522 53.0779986978048
1523 52.9385436046618
1524 52.5655074374116
1525 52.535287950039
1526 52.867931077962
1527 52.9190272516107
1528 52.8919084412607
1529 53.0584874866906
1530 53.5649219434137
1531 52.9049922664149
1532 52.4908455365021
1533 52.7402111696674
1534 52.8999495802581
1535 52.4950502446012
1536 52.2592923002602
1537 51.9065725328616
1538 51.9200935932766
1539 51.9554203336265
1540 51.428580562306
1541 51.9250639220838
1542 51.5855249067332
1543 51.6593823714914
1544 51.2901568953379
1545 51.5015889942212
1546 52.0396217470359
1547 51.7761616004709
1548 52.071877476377
1549 51.4032469311039
1550 51.0663279131435
1551 50.8627902145718
1552 50.6482138841564
1553 50.6374582914816
1554 51.3510002146483
1555 51.3606394082757
1556 51.0442776459447
1557 51.4684078623432
1558 51.9350219642849
1559 52.1921708824607
1560 52.3526946786446
1561 52.0893302169308
1562 52.20825000759
1563 51.4309352044829
1564 51.8228454263761
1565 52.2591140870682
1566 52.1463502686838
1567 52.2028709523295
1568 52.5942496985162
1569 53.0558487917338
1570 53.2115235863033
1571 53.1423900252186
1572 53.0685898975031
1573 54.0818447320696
1574 53.3919083358393
1575 53.270601486287
1576 53.8813656133972
1577 53.6042673619157
1578 52.7575974415454
1579 52.9391267711828
1580 53.5969490277819
1581 53.1474846892364
1582 53.7638883834792
1583 54.3005737429222
1584 54.031217582052
1585 54.2820731263006
1586 54.4180366807456
1587 55.3933506038977
1588 55.6645191647897
1589 55.3330750367325
1590 54.7614447834879
1591 55.2321936288099
1592 54.7305644585358
1593 54.2937332843568
1594 54.163708950413
1595 54.5729515526301
1596 54.1048922247846
1597 53.9561009167361
1598 53.7215360548696
1599 54.0713601282989
1600 54.3274590509895
1601 55.7019955131435
1602 56.6827799610822
1603 56.3287902194636
1604 57.1443582975318
1605 56.7509855569415
1606 56.8244907508157
1607 56.149625260081
1608 56.2407021378463
1609 56.4301467049157
1610 56.2237356798685
1611 54.8410417408189
1612 54.5246067079073
1613 54.6078556460025
1614 53.5741337256454
1615 53.4836470362082
1616 53.8380552475354
1617 54.4183073913363
1618 54.3821590761879
1619 54.218031724582
1620 54.9220105313572
1621 55.136849763183
1622 54.4889587157751
1623 54.1596963979094
1624 54.740294365593
1625 54.4507385023932
1626 54.2776785585002
1627 54.3217563987752
1628 54.4984086785869
1629 54.4840505018933
1630 54.3749930938802
1631 54.8637442693997
1632 55.7699721823884
1633 56.2512690806172
1634 56.1789121757202
1635 56.4000395624852
1636 56.0113595020078
1637 56.0230834664389
1638 55.7359138786672
1639 55.8727882958441
1640 55.5922480094474
1641 54.6631773588933
1642 53.7507965774223
1643 53.6864274172792
1644 54.8040798181641
1645 54.6689494744401
1646 54.9016348665586
1647 55.2797826336757
1648 55.9289713418064
1649 56.16022268826
1650 56.0675424530841
1651 56.3294926759952
1652 56.3844597513394
1653 56.5685573353554
1654 55.6158820808023
1655 55.6670332246304
1656 55.5042523308942
1657 55.2305854290913
1658 55.6253289316628
1659 56.0781805335787
1660 57.0377169788968
1661 57.1233852391718
1662 56.8232042146918
1663 57.0077844805543
1664 56.802673986962
1665 56.894656854587
1666 57.7020528462642
1667 57.7322506543332
1668 57.7751343164109
1669 57.0975331898791
1670 56.4845913644553
1671 56.3750631015805
1672 56.7115543200445
1673 56.2970716674943
1674 55.9737715350046
1675 56.0293534479735
1676 55.9662153685753
1677 56.1518535546221
1678 54.7959629611991
1679 54.4963514542274
1680 54.8657146909692
1681 55.5126066264721
1682 56.0666173994311
1683 56.7760999173927
1684 57.1221977599399
1685 57.0405958365105
1686 56.4355105726221
1687 56.3525526366389
1688 56.1107278463248
1689 56.7176040896127
1690 56.2249010011997
1691 55.3564308933789
1692 55.6770386046685
1693 54.5519080847859
1694 54.5732529063568
1695 54.4096850886212
1696 54.2395643355163
1697 53.7925113886657
1698 54.6028188935678
1699 54.0457533652357
1700 53.8283582775109
1701 54.6929366491695
1702 54.6044089278881
1703 55.2398204514076
1704 54.8833552468931
1705 55.1130274124337
1706 55.0469154040896
1707 54.9056927628074
1708 55.1557928065379
1709 55.3812010894352
1710 55.0148046538415
1711 54.1898045570258
1712 53.7280119760223
1713 53.1689796876325
1714 53.4326762641678
1715 53.2523691441847
1716 53.7690659885428
1717 53.6920015278321
1718 53.2137479628591
1719 53.0198445835278
1720 53.4351778465075
1721 53.8613006391822
1722 53.4699348143045
1723 53.2216595233626
1724 53.8327469320113
1725 53.5934599500663
1726 53.2653351745665
1727 53.9649705995276
1728 54.0509218741187
1729 54.7047342844102
1730 54.7941934403535
1731 53.9798847007302
1732 54.1515151034936
1733 53.9721928263347
1734 53.7282263201433
1735 53.8939732428766
1736 54.3002763823894
1737 53.2509821138981
1738 53.2331122769873
1739 52.2437932578962
1740 52.2008245870944
1741 53.2035514886668
1742 53.2908433402462
1743 53.7856938322664
1744 53.5595567416569
1745 53.5128480186964
1746 52.8700034146105
1747 53.741443567488
1748 53.978382332962
1749 54.4306152461868
1750 54.1701929512643
1751 53.3867608898473
1752 53.9290658093557
1753 53.8282644739951
1754 53.6509015725641
1755 53.3466078810971
1756 53.3276718231662
1757 52.6740567359278
1758 52.5340803614346
1759 52.1501844548869
1760 51.7185971261967
1761 52.2332696970741
1762 51.8659405982319
1763 51.835963518525
1764 51.6632048318233
1765 51.922554811844
1766 52.7371751239326
1767 53.4651642426159
1768 53.1759704762895
1769 53.2807676225472
1770 53.8401015058397
1771 52.9477489931499
1772 53.1273404360377
1773 54.0272706333794
1774 54.6086660691031
1775 54.8307629920991
1776 53.6215602812812
1777 53.5455331353566
1778 53.8890492372949
1779 53.9803638391905
1780 53.790280942177
1781 54.1138192939246
1782 53.4217641640448
1783 52.3356527826388
1784 51.7998877318606
1785 51.272092905791
1786 51.655914177776
1787 51.0349356605512
1788 51.1290035951442
1789 50.8050742009644
1790 51.3928634148979
1791 52.4695819611901
1792 52.8950072590147
1793 52.8921949407655
1794 53.6336495365084
1795 54.0424046748571
1796 54.165501297899
1797 54.4781694717493
1798 54.3091674625239
1799 55.0186771776358
1800 54.8399633941178
1801 54.9404122316282
1802 55.1681879501416
1803 56.2529763638219
1804 56.0217417399392
1805 56.4960751774641
1806 57.2378759766546
1807 57.6055747954491
1808 57.1965211487546
1809 57.03196775922
1810 56.6662479064371
1811 56.0089343581457
1812 55.6217905818559
1813 54.7951559205725
1814 54.1697192915703
1815 53.3832136206711
1816 52.6364189142463
1817 51.9706975576409
1818 52.9336638617229
1819 52.7221804274916
1820 52.6201712127266
1821 52.3507675473605
1822 51.963679452442
1823 52.3017257968729
1824 53.0643819955208
1825 53.5280093213856
1826 53.1771324807732
1827 52.8252338009101
1828 52.2214978347597
1829 51.8293604706986
1830 51.852795402416
1831 52.0457423461238
1832 52.1112747524964
1833 51.7228757683226
1834 50.7727161109496
1835 50.8872834230719
1836 51.6037451344073
1837 52.5731880583824
1838 52.5478346229878
1839 52.917659646494
1840 53.1809923228595
1841 52.7314696352355
1842 53.101525196355
1843 53.4695238821124
1844 53.9544312126464
1845 53.2029986555709
1846 52.470799873297
1847 52.3577352793768
1848 51.8000545989598
1849 51.515923131754
1850 51.163737691342
1851 51.5324485007523
1852 51.5660952964954
1853 51.1733663812853
1854 50.6848239956163
1855 51.5144023943165
1856 51.8171200930366
1857 51.358800143461
1858 51.8963655270108
1859 52.3782801492398
1860 53.2496961414881
1861 53.1171922146572
1862 53.6564323773844
1863 54.5172734874138
1864 55.1201683387518
1865 55.1388974461426
1866 56.0921232156036
};
\addplot [semithick, darkorange25512714]
table {%
0 66.0503198045257
1 68.1773435120408
2 72.3527222549693
3 76.2653414275697
4 80.256739177315
5 83.6929433227722
6 85.6091847658655
7 86.5267540827597
8 86.6180358474758
9 86.4847435789183
10 86.2566111050158
11 86.0222127691635
12 85.7157557135629
13 85.2077430772454
14 84.4460343715225
15 83.5706509526774
16 82.6139768897406
17 81.6305249499575
18 80.3482883756479
19 78.5721725098444
20 76.3370270239031
21 73.2254429248967
22 69.9686069949547
23 66.7428601394413
24 63.3865830655476
25 59.866791539384
26 56.1978840886245
27 52.5607917865524
28 49.0680708344985
29 46.0313184789451
30 43.5033741189245
31 41.9060969458878
32 40.4705705055889
33 39.221843542056
34 38.1146290058545
35 37.3433158562807
36 36.7817910775835
37 36.2071092191992
38 35.8371692063442
39 35.2670114649052
40 34.7393601243769
41 34.2931518568434
42 33.8088788544125
43 33.3988377725987
44 33.1871663523401
45 33.0770553774687
46 33.1663732372611
47 32.9621968554501
48 32.8497449135629
49 32.9269198275177
50 33.1637879888529
51 32.973519087964
52 32.7542796884009
53 32.4171703055061
54 32.1855544201105
55 31.9709792820502
56 31.6215336705487
57 31.5971746024998
58 31.2663703441077
59 31.0949813183688
60 30.6271677806729
61 30.4598751266666
62 30.6018530304703
63 30.7296062496154
64 30.5994067146241
65 30.2151921796336
66 29.8538766805944
67 29.7165790864226
68 29.6852995522022
69 29.7064709779948
70 29.790996083659
71 29.8878890467054
72 29.7578306539318
73 29.6951288483835
74 29.4287205480314
75 29.4775906830303
76 29.4010600887504
77 29.1762286765233
78 29.2875071945127
79 28.9009601572809
80 28.5191547660409
81 28.2162145081091
82 27.8868166651582
83 27.6971881298714
84 27.7679818739578
85 27.8017207393656
86 27.7550006714904
87 27.9906325003709
88 28.0087973398306
89 28.1713174230222
90 28.2196359808074
91 28.3640978338002
92 28.5023625009165
93 28.5526911808186
94 28.7271502403763
95 28.9810283419791
96 29.11775521466
97 28.9980885844584
98 29.2583421587785
99 29.3225577362934
100 29.3467786804262
101 29.4119609459985
102 29.4474890328651
103 29.3052953963507
104 29.2784927573518
105 29.0998446061621
106 29.1759214142473
107 29.2113624714094
108 28.7405531022825
109 28.6542795187196
110 29.2457186278062
111 29.2099176140929
112 29.278701764287
113 29.3870075004794
114 29.3592283352434
115 29.2598148794114
116 29.4198172649725
117 29.3117883055578
118 29.626926314487
119 29.6531386765863
120 29.2792788922395
121 29.0571391019645
122 28.9888323409676
123 28.7657041314588
124 28.6416994132705
125 28.6615549130174
126 28.689695634436
127 28.8583525804617
128 28.7583018837486
129 28.6283509356915
130 28.4287849591895
131 28.5202319963452
132 28.3707187313431
133 28.4905845366097
134 28.7288190691632
135 28.7985185088492
136 28.7079867140809
137 28.7888262155069
138 28.8833779046952
139 29.1565511344701
140 29.2058818612087
141 29.5472051498464
142 30.1319295459145
143 30.3970112712633
144 30.5315693167715
145 30.6542379216135
146 30.9695512371602
147 30.9889804050494
148 30.849554657541
149 30.7363589745175
150 30.7334297804959
151 30.6822397660623
152 30.6155106671436
153 30.5220407878183
154 30.6260915530127
155 30.6219066086703
156 30.3972185716271
157 30.5982793217327
158 30.6542468287893
159 30.6771930094932
160 31.0034025394941
161 31.3242601801454
162 31.2277713077877
163 31.4140226944786
164 31.4803081804707
165 31.9327045681447
166 31.8367393253397
167 31.9748117384099
168 31.9492365557431
169 32.0625549329276
170 31.9469368869222
171 31.5707051747722
172 31.6248084534022
173 31.5567087102616
174 31.3413295855361
175 30.8259770104601
176 31.3714624747262
177 31.4155033945975
178 31.5165089652588
179 31.4222511501188
180 31.3122279597566
181 31.5652446956569
182 31.3021327085943
183 31.1456768918891
184 30.9399721335851
185 31.2169262150045
186 30.7282190747706
187 30.4397139256059
188 30.393209447593
189 30.6103532583341
190 30.8073031577546
191 30.760953976562
192 30.7324559776354
193 31.1282986782818
194 31.5785897669837
195 31.3528767789958
196 31.6173408972089
197 31.2537812882807
198 31.0421213273678
199 30.939576431111
200 30.9602184335573
201 31.0392182751782
202 31.6491598629443
203 31.8309699579627
204 31.8938485867503
205 32.1611703133994
206 32.16691447731
207 32.1551686769335
208 32.6526761699492
209 32.4040709364753
210 32.02845417272
211 31.6783271966582
212 30.9988347721906
213 30.3079118556357
214 29.7689124665952
215 29.7769393932862
216 30.0457568149911
217 30.2631247088898
218 30.0889984888648
219 30.5410656420926
220 30.8519946145408
221 31.0764352995709
222 31.560514076228
223 32.1558039294938
224 32.7801222350571
225 32.5517498901712
226 31.9584523199265
227 32.2335834479898
228 31.9064859385313
229 31.6430725302538
230 31.6594559320974
231 31.4348879958644
232 30.8076182326698
233 30.7137268492284
234 30.2916573258792
235 30.3332992506322
236 30.7291552140872
237 30.1778643210358
238 30.171864127382
239 30.0677138716364
240 29.8216346456731
241 29.9415828723028
242 30.2790490128482
243 29.8717785081702
244 29.840179641587
245 29.6902040740313
246 29.180785811184
247 29.3490102460505
248 29.0788127572804
249 29.0877637514493
250 29.0279323353693
251 29.1978618948058
252 28.9667145366123
253 28.8764304976575
254 28.5502800505068
255 28.631509929615
256 28.3822480057419
257 28.0327651753024
258 28.5451941462325
259 28.4285274711517
260 29.0900019494449
261 28.4969104444593
262 28.7068922818338
263 29.0011236605561
264 29.4211342181446
265 29.3400028955064
266 29.9795050763149
267 30.7078221779209
268 31.253466967276
269 31.054195880187
270 31.1346053433343
271 31.9347879673482
272 31.6036494517786
273 31.5440278042663
274 31.5455730352239
275 31.2710813203454
276 30.9188875174715
277 30.8614960948046
278 30.1424782708843
279 30.5427056574431
280 30.1417440508121
281 29.2096576686956
282 29.3024559435774
283 29.3620909531068
284 29.5666078523198
285 29.8637277738325
286 30.0980446593871
287 29.8137514425112
288 29.7519475331286
289 30.0310581828515
290 30.1476112337069
291 30.8062981864971
292 31.4409583533015
293 31.6840926030193
294 31.6010586321553
295 31.6017636470869
296 31.1761373276821
297 31.273284478964
298 31.3875430541593
299 31.6619617962446
300 31.6850158535061
301 32.2161614249793
302 31.831527308639
303 31.663116579155
304 31.6335681916126
305 32.0812783967483
306 32.5902615853642
307 32.4736958706605
308 32.5782505298677
309 32.2869335084696
310 31.7240717243057
311 31.0541133426903
312 30.9676866072769
313 30.7123323573453
314 30.9236735545394
315 30.5208788605507
316 30.4997601318418
317 30.6683015443425
318 30.5792443977551
319 30.1868928983529
320 30.4355678051781
321 30.3150693155881
322 30.4635220299229
323 30.7965275062475
324 30.1107927672426
325 30.1745224425273
326 29.6917604356359
327 29.3046570444324
328 29.0748480470222
329 29.4309230994281
330 29.8194317099689
331 30.2838792514947
332 30.1521012330871
333 30.2036708847338
334 30.2733867003978
335 30.6466674610307
336 31.3068792994878
337 31.4256843826579
338 32.1566269213832
339 31.6062479333619
340 31.2218800301696
341 31.6618643166678
342 31.5545727757334
343 31.3724534954093
344 31.7126135262873
345 31.0597975887234
346 30.8779564577841
347 31.1371284550384
348 30.3400978073607
349 30.1623403905219
350 30.1204247131976
351 29.2284496556433
352 29.1759081594129
353 29.1662248393594
354 28.7808589002163
355 29.0128471628277
356 28.8766054398534
357 28.5545073930345
358 28.7296526108016
359 28.9493589154393
360 29.1678946056409
361 29.3082459204283
362 29.2633492305121
363 29.0453937574021
364 29.039737135251
365 28.8049928425053
366 28.794402004427
367 29.3234241955321
368 29.5824491093289
369 30.1134332395019
370 29.9072475948719
371 29.9952414476908
372 30.3521438449998
373 30.5493542281157
374 30.6867841368098
375 30.8541520323577
376 30.7280278173197
377 30.5376090402726
378 30.5063578443412
379 29.9865844852703
380 29.8682706932453
381 30.2547355203021
382 29.9791412125196
383 30.2227567264899
384 30.4041273893349
385 30.2515815510973
386 29.999065729341
387 30.4963711293353
388 30.4725437588609
389 31.0635117103836
390 31.7530400234198
391 31.1211847194874
392 31.3377194557821
393 30.8555038002961
394 30.5592675993054
395 30.7584208935461
396 30.6396980575837
397 29.8948551505922
398 29.4367198505025
399 28.7210791423639
400 28.3498342545491
401 28.2483576280176
402 28.404069130236
403 28.3663265051311
404 28.8658594372637
405 28.6911905895951
406 28.7924204004558
407 29.0234572187554
408 29.469496978313
409 29.9995909329288
410 30.2090278514821
411 30.7531743618538
412 30.5389903883032
413 30.7223007314576
414 30.5018380192531
415 30.1805025768375
416 30.2787384703396
417 30.298989843312
418 30.484760816571
419 30.0439038625126
420 29.7454333910039
421 29.5851284058404
422 29.5427051676606
423 29.7742632591992
424 29.836646723305
425 30.5026830899379
426 30.4185751228198
427 30.1270792076619
428 29.7204248551479
429 29.8914011137904
430 29.9794389342532
431 29.2138221682067
432 28.8023333521321
433 28.9347023598302
434 28.5898580638734
435 28.3413518533772
436 28.534829254593
437 28.4373207719598
438 29.1655948856117
439 29.4105975763779
440 28.8554185074569
441 29.5842847860834
442 30.3778061755084
443 30.4373662168326
444 31.3218771923014
445 31.4028349884203
446 31.3937761533571
447 32.0485946415977
448 31.2478768946389
449 30.8320939861199
450 31.1751773535152
451 30.6183557202596
452 30.1654197767781
453 29.9448535077714
454 29.6062673066324
455 29.1580374107138
456 29.208374048175
457 28.7940086180809
458 28.644388393573
459 28.8060874572129
460 29.163678117681
461 29.5386031016087
462 29.4820567709423
463 29.5329902657769
464 29.03162519528
465 29.6712860136992
466 29.3653376515222
467 29.2107816094923
468 29.4769263106
469 29.8920432703666
470 29.9100480439511
471 29.7709193040277
472 29.6953522700729
473 29.2202827144422
474 29.2236547532262
475 28.4594131044569
476 28.7146390705929
477 28.7629916854104
478 28.8831336899398
479 28.2999501421296
480 28.1489216721662
481 27.6745348374605
482 28.226818895934
483 28.4114425502203
484 28.1483612751744
485 29.2392896128305
486 29.0814679092948
487 28.9545737547446
488 29.1589639460377
489 29.4826872952383
490 29.390457138983
491 29.770444245793
492 29.7169997581866
493 30.5639757178038
494 31.8548602067194
495 31.4071323265742
496 31.5770912931261
497 31.8929064392356
498 31.2170083208681
499 30.9109868538682
500 30.7331221220022
501 30.6665058205391
502 30.796269219734
503 30.5042811580509
504 29.7573690672786
505 30.2501513533163
506 30.6733473483497
507 30.2187422863363
508 30.8789664648608
509 31.3847170745061
510 31.3338556726762
511 31.7599687636464
512 31.266986831344
513 30.3467465148
514 30.451753698945
515 29.4617554936423
516 29.2381125875839
517 30.4966296619174
518 30.3107378991452
519 29.7473932262601
520 30.0972458197714
521 29.8308940239067
522 29.9642583380059
523 30.6395758218287
524 30.9516375169635
525 31.4840830652764
526 31.4732751423473
527 30.6505910901804
528 30.1528801702659
529 30.5566283950258
530 30.3173561400444
531 30.4262469837548
532 30.307284703723
533 29.6669146406925
534 28.9395062930153
535 28.5859461023734
536 28.0934445987606
537 28.0514343916354
538 28.9211861795145
539 28.6889770078269
540 28.2284513605376
541 29.1491352023904
542 29.5816063328107
543 30.7596205790093
544 30.4133416653345
545 30.6883607063549
546 30.7668005997539
547 31.4537980064184
548 31.0395322238262
549 31.0584284789663
550 31.4139426424129
551 30.5992725214343
552 29.8606877607826
553 28.8065320946939
554 29.6369383820279
555 29.5882027447641
556 29.5344192080085
557 29.1332695349495
558 29.8739817209652
559 29.8645938445247
560 30.3312025169913
561 30.1107355677115
562 30.135113353225
563 30.0943039149957
564 29.6305288264173
565 29.6521787707896
566 29.5474886558491
567 29.3891278834686
568 28.4029708624253
569 28.5555571666242
570 28.2035030746841
571 28.3969352545987
572 28.7653441440258
573 29.2082834236899
574 29.2562920657959
575 29.4443876975165
576 29.4278295726928
577 29.8663262322911
578 31.0122267243575
579 31.1993954989295
580 31.0420424672705
581 31.0169946528028
582 31.5601104759725
583 31.3844202918419
584 31.3520532764077
585 30.8792746888821
586 31.4413731617629
587 30.6753304712662
588 30.0902430791907
589 29.6750703201023
590 29.8307702371689
591 29.9268963288854
592 29.6324208174206
593 29.7566250817203
594 29.4764740306367
595 29.384636565544
596 30.1071449188118
597 30.1742308340886
598 29.712377571749
599 29.8325335606415
600 30.2346880672356
601 29.7507728318659
602 30.1086495814558
603 30.2510062006342
604 31.2195592351758
605 32.242596846118
606 32.3054530447077
607 32.6885902234287
608 33.0751735053332
609 33.4826815779309
610 33.3721511742103
611 33.2385684141014
612 32.7274130113694
613 32.0683260009347
614 31.8979028486804
615 31.1003224935168
616 30.0748766320127
617 29.4851234058563
618 29.0556824618571
619 28.7384032537591
620 28.3317128644063
621 28.7359027037409
622 28.4659495007883
623 28.6225380658206
624 27.7563497953361
625 28.343924273018
626 28.7978120213496
627 29.1801198843927
628 30.0199731087105
629 29.9822856392417
630 30.5298608094936
631 30.9620005888669
632 31.2746763032143
633 31.0809503755176
634 31.1955123833665
635 30.78891465985
636 31.0980638119646
637 31.3323432896059
638 30.8297905419133
639 30.9532507466515
640 30.10723531435
641 29.4376184242722
642 29.075010456241
643 29.6236508025184
644 30.6176950288142
645 30.97271558791
646 30.4824158623277
647 30.3981165743162
648 30.3455627237964
649 30.2394563241969
650 30.8431456501683
651 31.3944914637467
652 31.5406005878265
653 31.0643264320802
654 30.0839076847648
655 29.4273479223392
656 29.1602207291713
657 29.2068349160742
658 29.0836437059079
659 28.2633125509742
660 28.5944876009898
661 28.0025973423259
662 28.1257314518095
663 28.7851227913874
664 28.744974475675
665 29.3683454547039
666 29.2754064416696
667 29.1746832575035
668 29.0058647888501
669 30.2331962536595
670 29.8557423417244
671 29.7810485814288
672 29.5340230040076
673 28.8323143600266
674 29.5567922799502
675 28.7578600002687
676 28.4059400580501
677 28.9580100677792
678 28.8552623619492
679 28.2677487354354
680 28.1408012485265
681 28.1222659589173
682 28.1445506660051
683 28.0544056469757
684 27.2459362393619
685 27.4086673672615
686 27.5666037377883
687 26.7264114219055
688 26.7466315251791
689 26.8048678190479
690 26.9371069548855
691 27.5040749527239
692 27.4191813918651
693 27.9891602941571
694 28.6282254892742
695 28.6448382315487
696 28.987294916038
697 29.406156715729
698 29.3954476292588
699 29.0744040356876
700 28.6990286881847
701 28.4890027477679
702 28.1283992580877
703 28.0888360710025
704 27.77061089641
705 28.0667302974166
706 27.8406891515729
707 27.7330086994873
708 27.4298974271981
709 27.8241218057062
710 28.2308826130793
711 28.0395742449994
712 28.4578933019292
713 28.3158434425929
714 28.2811154211585
715 27.9579497592373
716 27.6112976372665
717 27.126286018974
718 27.8913667063619
719 28.0113945718048
720 27.5442435161182
721 27.616063985542
722 27.87525070415
723 28.4761396072347
724 28.5526106744036
725 28.7523756211545
726 29.2288998568761
727 29.6539822830493
728 29.3366426506548
729 29.1768460390293
730 29.1234500438304
731 29.6262224884245
732 29.612116703816
733 28.9738459292577
734 29.0970593942763
735 29.3414823099486
736 29.0667613014766
737 29.2624457617149
738 29.5570229876627
739 29.5530044494832
740 30.4735506950374
741 30.2493832615633
742 29.8077144807352
743 30.1243485640401
744 30.7180658659876
745 30.5565276573467
746 30.9309380937494
747 30.3369006548194
748 30.051594980259
749 29.6899267832828
750 28.7499842179761
751 28.7293961145807
752 29.3767356693547
753 29.2703631482191
754 28.5553995494664
755 28.744506762222
756 29.3271602539005
757 29.503713556064
758 29.8185707856056
759 30.086980001857
760 30.1990288733387
761 30.6826482097996
762 30.3439498718288
763 30.3027569294106
764 30.0144441306128
765 29.9791819461791
766 29.9201997989443
767 29.7053851603174
768 29.1666169120539
769 29.7528848373746
770 30.0526197252565
771 29.1848311710705
772 29.3414987196088
773 28.9913177428633
774 29.4321556774051
775 29.1432442901298
776 28.3339273445202
777 28.6136301514943
778 28.7305532570103
779 27.9612455177915
780 27.5035564410962
781 27.8111290632411
782 27.6314991542834
783 27.7103504858658
784 27.7147124050185
785 27.3891825897902
786 27.9481498735876
787 28.4023046631577
788 29.1918718970002
789 29.4810620069267
790 29.892123031321
791 30.0279294508017
792 30.3492209593911
793 30.8373169368803
794 30.8730451969932
795 31.2684045781629
796 30.8025317807529
797 30.5571440579011
798 30.2543519503444
799 30.1936761816827
800 29.9441556407158
801 29.7966473642983
802 29.5155668951606
803 29.6709732757772
804 29.7652392746879
805 30.0033906168824
806 30.6009138698066
807 29.9341496186362
808 29.4261644962039
809 29.3637102557983
810 29.4576091343081
811 29.8283588257522
812 30.3145503468817
813 30.0568565589918
814 29.7730132584851
815 29.3779564434533
816 29.4917527174279
817 29.8825722219401
818 30.104811772644
819 30.5155637297668
820 31.0001267299656
821 30.5294900550917
822 30.3032662693607
823 30.2498582669587
824 30.5557393702101
825 31.0031342583012
826 30.6344154859183
827 30.6285557047023
828 30.3053983827534
829 29.8205922372222
830 29.508990093453
831 29.6832833249902
832 29.6953714372522
833 29.9771759620257
834 29.9995376563104
835 29.6078604141247
836 29.4717782822363
837 29.9843596934257
838 30.2352064406363
839 30.5816706556282
840 30.9483719490389
841 31.1838493758802
842 31.2530565325348
843 30.7698211079938
844 30.9277970764459
845 30.9153590800137
846 30.9153717946382
847 30.9630422673843
848 30.8144494739904
849 30.2125096242175
850 29.7152583337353
851 29.4285291601566
852 29.5757485972362
853 30.3486909902872
854 30.0966219331803
855 30.5128741088557
856 31.0332409637211
857 30.7763748893175
858 30.9634918193113
859 31.6497465765183
860 32.1532263789835
861 32.0631670821952
862 31.47877419483
863 31.6709239033743
864 31.4110129524969
865 30.9736889507161
866 29.9566576865216
867 29.6307574448317
868 30.5529982063632
869 29.9212528693161
870 29.5477296389418
871 29.8961635167038
872 30.9685434056996
873 30.1618376138421
874 30.2743099094416
875 30.5167562212208
876 31.1426443647954
877 32.2239291184249
878 31.4402192174578
879 31.9965448660821
880 32.3422696255406
881 32.1376984021215
882 31.9344586731864
883 32.2719659691924
884 32.5870827234688
885 32.5298405375092
886 33.0081124172597
887 31.5526469512698
888 31.2286303770946
889 32.0309060379098
890 31.1918153737305
891 31.6956908782542
892 31.3435588402562
893 30.5948041273464
894 30.0426501831046
895 29.6245794930151
896 28.86460048835
897 29.329509264764
898 29.5962052826809
899 28.6288110670163
900 29.3261347441825
901 28.0279224820659
902 27.6039736348468
903 27.9065111451997
904 28.1135771590837
905 28.0196515290302
906 27.9298307292355
907 27.8587599885063
908 27.5815054483992
909 27.7995017286929
910 27.2266864056654
911 27.9820919075647
912 27.6807399696123
913 27.6348860878005
914 27.4503620051766
915 27.4973195018631
916 28.0201355119274
917 28.3007635104981
918 28.4749677922284
919 28.7484462089146
920 29.2359538411838
921 30.4002729892606
922 31.0966038277773
923 31.1519116820688
924 31.4417358335259
925 31.4072640462468
926 31.2974517318764
927 30.8616102490981
928 30.9134272591113
929 30.8354995845539
930 30.7819482565592
931 29.585936818827
932 29.2074743257212
933 29.329596761134
934 29.5708955873135
935 29.5694052134692
936 29.5409002768245
937 30.0061939102626
938 29.8302947191937
939 29.8707323256947
940 29.9588229973083
941 29.9534276048376
942 29.8180663930738
943 30.0744171560245
944 29.3374903974987
945 29.7875161346456
946 29.8526242565893
947 29.5010573041387
948 29.5074273978483
949 28.7519467586885
950 28.3446184272343
951 28.8738159084219
952 29.1751520090341
953 28.4134645516268
954 29.5626205173913
955 29.1918321302119
956 28.5294161839534
957 28.855618681687
958 28.580510513723
959 28.5246068602698
960 28.3101420073387
961 27.7598661910113
962 27.4273710620178
963 28.1296285156335
964 27.0699923176699
965 27.8546436103816
966 28.3735942732257
967 28.1107611222188
968 28.4574565603437
969 28.9564876995216
970 29.5821604869276
971 29.2558671446327
972 29.4141457431959
973 29.4207756235406
974 29.7175172809438
975 29.5536739617952
976 30.3257177565419
977 30.7700947137737
978 30.6613000201201
979 30.8262268732493
980 29.9566639574702
981 30.3938320658002
982 30.6779955132376
983 30.6960863070189
984 30.2770033791495
985 29.6990023524266
986 28.6578387870703
987 27.6839218411127
988 28.2818246743445
989 28.2497914798157
990 28.7026148271025
991 28.1850513404132
992 27.6492554334928
993 27.3728790842287
994 28.382949942907
995 28.7433466700164
996 28.795715763353
997 29.5982605214565
998 29.5419435615604
999 28.815290852097
1000 28.9150279920826
1001 29.6350745507128
1002 29.6920017602051
1003 29.0405567068496
1004 27.8840417259698
1005 27.0855768925275
1006 27.0310285500132
1007 27.2190767524349
1008 26.966572079297
1009 27.1752790421915
1010 27.0010701702625
1011 26.6127798381238
1012 27.2054338069256
1013 27.6543250856398
1014 28.434735322242
1015 29.2280353101854
1016 29.5068241452139
1017 28.4707221164919
1018 28.3663948929934
1019 28.7495953843508
1020 29.6989435768047
1021 30.317202895497
1022 29.3840771073494
1023 30.8730593186757
1024 30.7912878039735
1025 30.5117094631162
1026 31.0631714476884
1027 32.0948281465014
1028 31.9464088692608
1029 31.4045464981135
1030 31.1821203144406
1031 31.1049486221021
1032 32.1385285013972
1033 30.7856942249534
1034 31.3868208229173
1035 31.7570777451346
1036 31.0017155965922
1037 31.3239136674616
1038 31.2382482463298
1039 31.6632642210895
1040 31.3865265085146
1041 30.9282565324691
1042 30.3367356975144
1043 30.2754344571402
1044 29.9502083352897
1045 29.5056896963357
1046 29.7149085122247
1047 28.9659688065549
1048 29.1251484749762
1049 28.9777546306323
1050 29.0090770543575
1051 28.7671641881974
1052 29.4169851518966
1053 29.0373135049311
1054 28.7162114184196
1055 29.1171467099509
1056 29.4624060808968
1057 29.5542990183347
1058 29.6310223973968
1059 29.9085004912927
1060 29.5971003302387
1061 29.8058921351039
1062 29.2285981426346
1063 29.4920874614323
1064 29.2158105888547
1065 29.7033242894781
1066 29.5598658744559
1067 29.3705602973175
1068 29.1644147241008
1069 29.1045712374321
1070 29.2852572956239
1071 29.3127363425934
1072 29.5828161359088
1073 29.739822802366
1074 30.3402952676272
1075 29.4922425941467
1076 28.9623195526203
1077 29.2388804095949
1078 29.4981649821957
1079 29.265945384433
1080 28.8860922898805
1081 28.9746379331826
1082 29.3292277496075
1083 29.6627501851233
1084 29.1933991471121
1085 29.518622504018
1086 29.1928110771743
1087 28.5545233703178
1088 28.6373641091294
1089 28.8630926571242
1090 29.2231114501602
1091 29.418412213485
1092 28.9570512648066
1093 28.9793972628658
1094 29.2964872525999
1095 29.1746212389594
1096 29.6226769585323
1097 29.7961309458099
1098 29.6451688810785
1099 29.2782641913442
1100 29.9289701095117
1101 29.9852635764946
1102 29.8378069813424
1103 29.4088740425313
1104 29.5767387552614
1105 29.3713843077491
1106 29.7497449906485
1107 30.4439522648005
1108 30.4210816834687
1109 30.7720601504438
1110 30.4248969723965
1111 29.6237729261383
1112 29.8342152253934
1113 29.7358185837193
1114 29.4464655036676
1115 29.711706473393
1116 28.9466874060935
1117 28.4984608796652
1118 28.4973219914838
1119 28.8183419407021
1120 28.0908609255406
1121 28.5012285583597
1122 28.4113576442968
1123 28.8600418608886
1124 28.3079464088625
1125 28.4188996579734
1126 29.010374868137
1127 29.14563429178
1128 29.089019501027
1129 29.1472331308918
1130 29.0273210833297
1131 29.0192040483238
1132 28.546124361868
1133 27.9925733035853
1134 28.09306913739
1135 27.8198364548787
1136 27.6177571838818
1137 27.7792406637005
1138 27.5824987766923
1139 27.1454597227378
1140 27.8192857024551
1141 27.6738337197173
1142 27.6825412948166
1143 27.8591450674684
1144 27.8444667639436
1145 27.884949734241
1146 27.9273126605207
1147 28.0822066808234
1148 28.2735338565705
1149 28.2387346363991
1150 28.1970979630063
1151 28.3685734235317
1152 28.732051932437
1153 29.1636109295456
1154 29.6025990131522
1155 30.0519341450532
1156 30.4331207868063
1157 29.4057915598935
1158 29.4245671529629
1159 29.2656124192494
1160 29.0320959216664
1161 28.4682126607867
1162 28.3187221657736
1163 28.2435896706421
1164 27.8659646288095
1165 27.6257172903035
1166 27.5151767320083
1167 28.4566581227497
1168 28.3050484518358
1169 28.5946283557252
1170 29.108396429184
1171 29.8891290313458
1172 30.3895964413958
1173 29.9818785185132
1174 30.063774790447
1175 30.298885134162
1176 30.2086092210057
1177 30.1735423870385
1178 30.3493353901151
1179 30.2486785954776
1180 30.2063116919453
1181 30.2798910960642
1182 29.7665308705613
1183 29.9865727478213
1184 30.314444377074
1185 30.2216782134732
1186 30.4018058080099
1187 30.3286643423723
1188 30.1323771145179
1189 29.8551382985974
1190 29.6051370796331
1191 29.7856637912215
1192 30.1967470879076
1193 30.266135896989
1194 30.4375341506237
1195 29.9829160966921
1196 29.7426471353249
1197 29.4434590457521
1198 29.8790003162221
1199 30.0921105491984
1200 30.2138268965906
1201 29.9935532950653
1202 29.3277498411592
1203 29.5414372088195
1204 28.5771393543257
1205 28.8635713777267
1206 28.677311934056
1207 28.9032582581129
1208 29.3085668881412
1209 29.9262545331784
1210 30.3245622969659
1211 29.9421969475514
1212 30.3058619775822
1213 30.1280080430378
1214 30.458275434771
1215 30.304361722297
1216 30.6569148205148
1217 30.4598552533832
1218 29.3604799406484
1219 28.6614552450914
1220 27.8705505527301
1221 28.5858081255634
1222 28.752881336262
1223 29.0968528085123
1224 29.3926203119057
1225 28.8797254270704
1226 28.2405898205577
1227 28.0369983921059
1228 28.8796168878713
1229 28.7815621321784
1230 28.9912992268225
1231 28.1846040585695
1232 28.0159988336416
1233 28.3977580320469
1234 28.8149747430347
1235 29.0895652563684
1236 29.5139809099782
1237 30.2606882934999
1238 29.6605487110855
1239 30.1242631431646
1240 30.1665628385535
1241 30.8878716865739
1242 30.3691667301122
1243 29.4824495215193
1244 29.0500525096631
1245 29.2698971833169
1246 29.5112943456618
1247 29.227870836092
1248 29.6842537791711
1249 29.6934989333231
1250 29.6998239939973
1251 29.7942818690787
1252 30.2887420242858
1253 30.1230790254317
1254 30.5563218487401
1255 31.0113652664345
1256 30.6963528922959
1257 30.6495824777803
1258 30.3934832675021
1259 30.0756088886891
1260 30.0848518451882
1261 29.8634599321004
1262 29.6421991577194
1263 30.0590721413941
1264 29.8111089541662
1265 29.7056931004315
1266 29.4888739565487
1267 29.0732140620156
1268 29.8257642794797
1269 30.7846093601567
1270 30.8541052324323
1271 30.9589311655246
1272 31.8129064826955
1273 31.3234981884271
1274 31.1987055147347
1275 30.930440138056
1276 31.5526071035682
1277 31.2696259897473
1278 30.7825832915256
1279 30.0309095350525
1280 29.7030147228494
1281 29.3004835021187
1282 28.678585611775
1283 29.0359227386016
1284 29.0658869973
1285 29.7152322191013
1286 29.7896155541003
1287 30.1927449409646
1288 29.7683975294709
1289 29.7505798459348
1290 30.4810539539353
1291 30.7400502340322
1292 30.6414360985125
1293 31.2942636315149
1294 31.9280698528613
1295 31.5338249673117
1296 30.9878544778296
1297 31.1050262852846
1298 31.4642447298513
1299 31.4435539636718
1300 31.2526194231755
1301 31.0238409938012
1302 30.7850919845458
1303 29.6068602637273
1304 29.4449789137839
1305 29.4961349168575
1306 29.3806716510457
1307 29.2977865158594
1308 29.2454178344971
1309 28.8380690553382
1310 28.1050910040854
1311 28.7676663701346
1312 29.3486384621982
1313 30.0081512031262
1314 29.3518578130253
1315 29.2561751314057
1316 29.0017449956781
1317 29.0998516045907
1318 29.0254180664925
1319 29.8393752132982
1320 30.4782739459286
1321 30.0957753698449
1322 29.9956319773332
1323 29.7765214586077
1324 29.8757161556493
1325 29.6637237384171
1326 30.0353803754767
1327 30.0231698714289
1328 30.1387221903219
1329 29.5075729605186
1330 29.7277163421081
1331 29.5624712047982
1332 29.7523687147865
1333 30.4546118366639
1334 30.7924241493835
1335 30.5299251473691
1336 30.8721550171942
1337 31.4555929144146
1338 31.4556306607322
1339 31.8357775825314
1340 31.4973608205873
1341 31.0451502581629
1342 30.7149224013147
1343 30.7226065592659
1344 30.7331774000617
1345 31.1675488023371
1346 30.3832010365025
1347 30.8865098284506
1348 30.6193694561076
1349 29.888806842712
1350 29.7296643012604
1351 29.8982137366215
1352 30.0540598380871
1353 29.4320225450754
1354 28.6063687994815
1355 28.5257896650765
1356 29.3316339264854
1357 28.9233083624495
1358 29.4781073562592
1359 31.0342063811986
1360 31.1621703866116
1361 31.1126776013323
1362 31.5044946130671
1363 31.3630580924504
1364 32.1197602709773
1365 32.0996769282239
1366 32.0984297802172
1367 31.5120498442552
1368 31.5272354276982
1369 30.4770783472263
1370 29.9163070619713
1371 30.3203240914063
1372 30.2463049603284
1373 30.5659583444969
1374 30.7027382680328
1375 30.3591342121476
1376 30.3421205009904
1377 30.1349387309742
1378 30.1955717438677
1379 29.8724488987623
1380 30.2211495738228
1381 31.0369575919706
1382 31.0220939767464
1383 31.2820908468673
1384 31.3212384606549
1385 31.9088839341184
1386 32.010939779749
1387 32.4147942277164
1388 32.0463308314319
1389 33.0401418064917
1390 32.6608468279791
1391 31.9913639586758
1392 31.850644621873
1393 30.9995442535544
1394 30.4960192020012
1395 30.3926686082612
1396 30.1088489983137
1397 30.2921890019128
1398 30.3280649837866
1399 29.6115320691999
1400 29.8241660024175
1401 29.4948997742714
1402 28.8413793097425
1403 29.8360379503577
1404 29.9662300082387
1405 29.7760331683081
1406 29.2294068455407
1407 28.7643179709438
1408 28.6469599225002
1409 28.9423834649625
1410 29.6845319756864
1411 29.9916848406809
1412 30.5054150252393
1413 30.0165464244491
1414 29.852731484239
1415 30.6539619243147
1416 31.2517160600878
1417 31.2502550586407
1418 31.2341677026074
1419 31.5872780308326
1420 31.6681272906332
1421 31.432701748051
1422 31.649328584675
1423 31.6792238746581
1424 32.2335888678348
1425 31.2480753357735
1426 31.6404202403885
1427 31.751345915552
1428 31.7491291511588
1429 31.3019601110065
1430 30.741129003304
1431 31.0524731352937
1432 30.7544897754087
1433 30.3928884328651
1434 30.1669652029378
1435 30.9084279617352
1436 30.5313564380211
1437 30.0152218815758
1438 30.4634496586837
1439 30.9106360618065
1440 31.4128326181395
1441 31.4252684055984
1442 32.102505893642
1443 32.0857009416064
1444 31.9841056545399
1445 31.4063476492161
1446 31.6951193142768
1447 32.5758446743862
1448 31.8219768352135
1449 31.4741594105124
1450 31.5595389791636
1451 30.8361811591202
1452 29.9415701456101
1453 30.3701660005622
1454 30.8140077190081
1455 30.9710008429357
1456 30.8328616766342
1457 30.5222779585999
1458 31.1178565888045
1459 31.4317185089919
1460 30.9567106367642
1461 31.3587467488954
1462 31.7959450391703
1463 31.6052927426759
1464 31.4021745560863
1465 31.2086805686401
1466 30.617048281326
1467 30.604819986257
1468 30.0966880129261
1469 29.2639290372611
1470 28.9217663315639
1471 28.8361286766646
1472 28.8954691031428
1473 29.3828292373086
1474 29.138876668154
1475 29.3576513584874
1476 30.0510266334282
1477 29.4879710307768
1478 29.4919101050851
1479 29.3540464768224
1480 29.4535094087613
1481 29.939911315512
1482 29.8281487073753
1483 29.1446993463358
1484 29.554328052092
1485 30.019045229718
1486 29.9128085686257
1487 30.6938855980698
1488 31.2299146742794
1489 31.9561393884188
1490 32.2879495769269
1491 31.8990491647807
1492 32.226486863416
1493 32.1278212260341
1494 31.6702354757991
1495 31.4907612726094
1496 31.9246515660553
1497 31.9946454139555
1498 31.7187670677895
1499 31.6903361122214
1500 31.3191949767496
1501 30.6751762320022
1502 30.3730967954878
1503 30.5923469717686
1504 30.5058447152511
1505 30.7532777646254
1506 30.0652195330419
1507 30.2444711202969
1508 30.0058853849157
1509 29.5590065736939
1510 29.6965336691667
1511 30.673630845612
1512 31.3788904509636
1513 30.9475098321832
1514 31.5814441939593
1515 30.9793520150705
1516 30.3481183551654
1517 29.9186243625268
1518 30.1110736376316
1519 29.8241304501267
1520 29.7337173173466
1521 29.3273719827963
1522 28.7062923565401
1523 29.5970134025131
1524 29.0803279042937
1525 29.5785226546217
1526 30.5748051718639
1527 30.1609192805663
1528 30.0692182352555
1529 30.7575473960863
1530 30.8905808458428
1531 30.7277784329989
1532 31.0377180951633
1533 30.8667946976706
1534 30.7814958107794
1535 30.058726879476
1536 30.0441328565547
1537 30.0687408722675
1538 29.8878920574564
1539 30.2035454705122
1540 30.2728566637163
1541 30.9312629712452
1542 30.3964299451008
1543 30.5959599650924
1544 30.2441533363356
1545 30.5726867091252
1546 30.044488543324
1547 30.1213382528304
1548 30.519260345884
1549 29.7321875952929
1550 30.0559552116978
1551 30.0383472703406
1552 29.9485462204325
1553 29.5446484928182
1554 30.0408296045995
1555 29.2947197689772
1556 29.6617310778764
1557 29.6586868379925
1558 29.7396157978303
1559 30.0426812233237
1560 29.8872089907198
1561 30.4332453577438
1562 30.124658823827
1563 29.70346315278
1564 29.954174761765
1565 30.1116873492486
1566 29.9556507837887
1567 31.2674275737376
1568 30.9006749925377
1569 30.7515822335343
1570 30.2396147865472
1571 29.1897784674816
1572 29.7473645149603
1573 29.6813996934799
1574 29.2873047297985
1575 29.8502830697407
1576 29.4917147795254
1577 28.3653544720148
1578 29.1836654678295
1579 30.2326632896425
1580 30.5318069264059
1581 30.3851167475417
1582 31.1889774310148
1583 31.8938867563654
1584 32.2774832300915
1585 31.9560613628192
1586 33.5761005966998
1587 34.2733075624499
1588 33.1730251113863
1589 32.4679463024918
1590 32.6569913527935
1591 33.2237062595086
1592 31.7423180781572
1593 31.952883094802
1594 31.4118190335538
1595 31.6570288647991
1596 30.8563393683284
1597 30.5771949561103
1598 30.8888445652992
1599 30.4402935061993
1600 30.2533003027473
1601 29.8842056112792
1602 30.2811622367473
1603 29.7925359144882
1604 29.9790899108415
1605 29.9669412213005
1606 29.5319653640518
1607 28.8867330179343
1608 29.2886337645861
1609 30.3047947008473
1610 30.9513345484226
1611 30.6804066361924
1612 31.3146268560548
1613 31.8385666728365
1614 31.7539915160991
1615 32.4497321356648
1616 33.4020391880864
1617 33.8995559838793
1618 33.653135645817
1619 32.9378527615474
1620 31.8872502512667
1621 32.4742133896606
1622 32.3862599687089
1623 32.3008829811493
1624 32.2421608497315
1625 32.5239898625245
1626 31.7892872160951
1627 31.2680953957884
1628 31.0579734791265
1629 30.7181073722106
1630 31.0361510372037
1631 30.4564001812214
1632 29.8695295761224
1633 29.3890335878039
1634 29.5470075349535
1635 28.5410333908058
1636 28.7764441753336
1637 28.9854519250457
1638 28.9360421619728
1639 29.2407453316907
1640 29.5489354371769
1641 30.5717941374522
1642 31.3569848443882
1643 31.3474250010364
1644 31.6071350729797
1645 32.0749868914406
1646 31.4531122925107
1647 32.4985554165563
1648 33.1472161105616
1649 33.7025483979145
1650 33.335131008725
1651 32.7886798950121
1652 32.0108869998999
1653 31.946187521983
1654 32.4194428444531
1655 31.7899386400721
1656 32.48198319437
1657 31.5999583946338
1658 31.8354982852929
1659 30.6988476494033
1660 30.5535051053565
1661 30.0253820758076
1662 30.3217966636314
1663 30.653606734533
1664 30.0194667781492
1665 29.9447388546034
1666 29.5573391924041
1667 29.8590410270311
1668 29.9456028153107
1669 30.611550475743
1670 31.4197112159214
1671 31.8279061056226
1672 32.064087360175
1673 32.2136707908225
1674 32.2984200201394
1675 32.3902911552349
1676 32.7007346179457
1677 32.3368643978847
1678 31.7061910377993
1679 31.8240028702023
1680 31.5150673829577
1681 31.7291505364806
1682 31.2466833813887
1683 30.8946128175739
1684 30.3214480390531
1685 30.7318855020285
1686 30.4508925475626
1687 30.721302603532
1688 30.5876165777239
1689 30.2430384795711
1690 29.5404966977842
1691 29.2361199301426
1692 29.5283510751112
1693 29.2146996237378
1694 29.8914680703834
1695 30.5106345269271
1696 30.3750223835906
1697 30.4958944216416
1698 30.5153547978402
1699 30.5511354657256
1700 31.4109246223053
1701 31.356318013805
1702 31.8795834288342
1703 32.4945564308777
1704 31.9039872544416
1705 30.687412460514
1706 30.414036354856
1707 29.8724702764086
1708 30.0760809722176
1709 30.3722963343809
1710 29.7046326052362
1711 29.8024723296339
1712 29.3237224457555
1713 29.1284429851911
1714 29.1778230279002
1715 29.5138107840894
1716 30.0467021170392
1717 30.6254701252088
1718 30.4368327287006
1719 31.1191727755992
1720 31.8484270672279
1721 31.849609877813
1722 31.6291211384579
1723 31.7175576312099
1724 32.6216958911964
1725 32.4074138144371
1726 32.5812542897926
1727 32.2009274686951
1728 32.2082038511761
1729 31.7826267369747
1730 30.9668392213442
1731 30.9464822290813
1732 30.9413657208725
1733 30.3862243808007
1734 30.6057526142928
1735 30.6860277333233
1736 30.4105658165136
1737 30.4258287250151
1738 30.1739967590485
1739 29.7382329090586
1740 30.350085640008
1741 30.5559693745685
1742 30.8724217242508
1743 31.038778000026
1744 29.91391598711
1745 29.9076396995354
1746 30.0307556215749
1747 30.165035139221
1748 30.7063408608938
1749 31.2187509360646
1750 30.9830733141141
1751 30.6665289288699
1752 30.7519320161766
1753 31.5300179436273
1754 31.7248324069021
1755 31.9965903431691
1756 31.9981208768441
1757 32.1566227049935
1758 32.3431178269209
1759 32.2280273274915
1760 32.0547073256776
1761 32.1845099851928
1762 32.3105370513037
1763 31.8102635787582
1764 32.1008196682167
1765 32.337399563738
1766 31.7844723739244
1767 31.2020286919747
1768 31.1754650018556
1769 31.3562327543376
1770 31.1711653104599
1771 30.7867592275787
1772 30.6068001130805
1773 30.6587188817196
1774 30.4624113193607
1775 30.0734923369612
1776 29.8954914115647
1777 30.0721009471024
1778 29.5356613689428
1779 28.9326821488057
1780 29.2014414617231
1781 29.5654744445888
1782 29.1505683909747
1783 29.2376528067043
1784 29.6798096549138
1785 29.1415692730789
1786 29.5229251003839
1787 29.2550221810323
1788 29.8736577562386
1789 29.825882440305
1790 29.7865347822463
1791 30.0870069418311
1792 30.2819240867808
1793 30.0643354907453
1794 29.8459009818882
1795 30.810863703082
1796 31.2981969370319
1797 31.7500789878446
1798 30.7587479927321
1799 30.9531386500582
1800 30.7844297270004
1801 31.0338376347327
1802 31.4257339759151
1803 31.5294007250251
1804 31.3924671805704
1805 31.3991109178766
1806 31.6476206814211
1807 31.4844794305404
1808 31.5930184580495
1809 31.5738469609714
1810 31.8710719641301
1811 31.4210881710196
1812 30.8683376757974
1813 30.5472964445014
1814 30.0992443436634
1815 30.1285503439237
1816 29.9911803870334
1817 30.203723276151
1818 30.9342810246171
1819 31.1940817551342
1820 31.9499461787362
1821 31.73585898674
1822 32.3026832575062
1823 33.1793395693406
1824 33.9032332715522
1825 34.2146196620906
1826 33.6025417566123
1827 33.1844392325089
1828 32.4922985151779
1829 32.1797298422123
1830 31.5444529201739
1831 31.5753810584942
1832 30.7886755207023
1833 29.7181337897427
1834 29.4825870934833
1835 29.6362179477974
1836 30.3933532427363
1837 31.0078676072147
1838 31.6732640183263
1839 31.459512645846
1840 31.6883110187178
1841 31.8274127849489
1842 32.0544804110046
1843 32.748201234489
1844 32.6996374082123
1845 31.6526448052217
1846 31.4964579801694
1847 31.1060225587954
1848 30.9006894890075
1849 31.4132372716976
1850 30.8585545190646
1851 31.2898486197182
1852 31.3341640279519
1853 30.711403673385
1854 30.1241607764985
1855 30.8250795202305
1856 30.0094383024156
1857 29.9827018826616
1858 30.3531879109972
1859 29.6689254676408
1860 30.1247041718279
1861 29.8921089461936
1862 30.8887813594428
1863 31.9766765442432
1864 32.2719514542996
1865 31.5724856097752
1866 32.4540529435123
};
\addplot [semithick, forestgreen4416044]
table {%
0 72.8666932984395
1 73.086360823187
2 73.2288625196614
3 73.4043294594282
4 73.5128479579163
5 73.5078464641949
6 73.5015558434101
7 73.4086277772658
8 73.1197473990069
9 72.8325013007871
10 72.5926721597248
11 72.3021391870081
12 72.0733426847384
13 71.8981392787274
14 71.7700205711448
15 71.7296919339756
16 71.7100544070192
17 71.6709962569061
18 71.6845047892837
19 71.7361012409444
20 71.8067164192781
21 71.9535774006312
22 72.0528447318198
23 72.1849387602529
24 72.2746536107127
25 72.358290018914
26 72.393183191197
27 72.4752648839462
28 72.5779361494371
29 72.6280243551473
30 72.6774950047621
31 72.6541552149751
32 72.7082229057258
33 72.6622821561378
34 72.5832947982533
35 72.4702391897134
36 72.4516735814564
37 72.3069717967396
38 72.2897528056657
39 72.4127622751885
40 72.45629008299
41 72.5317790503727
42 72.5685980060073
43 72.8028340887766
44 73.1586685099303
45 73.4680828610624
46 73.7687405138171
47 74.1134183101559
48 74.3097105141549
49 74.3861841423146
50 74.5688018909367
51 74.7717772655805
52 74.910426359049
53 74.9016021637058
54 74.7718181133011
55 74.7439073745481
56 74.6231995480558
57 74.6124177847828
58 74.5209035912328
59 74.4822808086712
60 74.3730809527327
61 74.2494347630826
62 74.1569859390906
63 74.0617765512203
64 74.0612986558755
65 73.9127641010367
66 73.8539909884471
67 73.7925604067924
68 73.7856378637663
69 73.7636300422085
70 73.6767245643999
71 73.711228383673
72 73.6808366631128
73 73.6450152725832
74 73.566989997087
75 73.6258913612351
76 73.7246628226695
77 73.7220428155034
78 73.7395324481527
79 73.7717102304871
80 73.8621191962113
81 73.8710208550252
82 73.9455341443328
83 73.9842765508232
84 74.1073576323118
85 74.205406006313
86 74.2112475667503
87 74.2316903482623
88 74.2454343612029
89 74.1772579348172
90 74.071851860951
91 73.9768727432857
92 73.9353326606122
93 73.9221271373656
94 73.8088102154663
95 73.7269924749507
96 73.6296007213019
97 73.5172712939194
98 73.4646345161235
99 73.4299154713863
100 73.5168665810465
101 73.6022557915772
102 73.5687292154025
103 73.4845450640758
104 73.4334837701257
105 73.4278756348586
106 73.3987915904502
107 73.3411473913374
108 73.4234415228603
109 73.5688743924042
110 73.5553358691159
111 73.5238065062181
112 73.5260216701982
113 73.5320667543765
114 73.5715169473425
115 73.5450729147096
116 73.5682023403764
117 73.5667438974372
118 73.4514467420512
119 73.3490630736404
120 73.3426562733875
121 73.2890106234177
122 73.2302428631865
123 73.3120617592547
124 73.2837809213764
125 73.2090804337768
126 73.1205829154418
127 73.1262384745525
128 73.0487501562458
129 72.9117740734703
130 72.8537271204857
131 72.7553267360478
132 72.6776404154274
133 72.5090810753984
134 72.4292387366437
135 72.4527518545309
136 72.5249694535221
137 72.5678461318715
138 72.6450347936464
139 72.6485535148016
140 72.7061745072816
141 72.7307784111156
142 72.7979957239574
143 72.8137458886359
144 72.8314506790869
145 72.6733072040696
146 72.60726483559
147 72.5826093763134
148 72.5375866541361
149 72.5021519089008
150 72.4543609624123
151 72.383048248191
152 72.3677760137014
153 72.3524389663423
154 72.3254689969782
155 72.3282538003741
156 72.1764677835878
157 72.0732909292148
158 72.0034035875001
159 72.1469199146057
160 72.2153163475561
161 72.2165499029318
162 72.201658285199
163 72.272102703958
164 72.2233180686755
165 72.2430105979003
166 72.3143806824762
167 72.3560730892692
168 72.3125499741666
169 72.2272877324623
170 72.1335871025441
171 72.1905358127728
172 72.1105024034524
173 72.1205724917971
174 72.1189035849413
175 72.3526983225767
176 72.3609944502042
177 72.3950124347955
178 72.4073403095386
179 72.3920904272518
180 72.4474568003884
181 72.4462857180759
182 72.3995639559431
183 72.2490341586019
184 72.2104224683035
185 72.0174036730688
186 71.9900712104111
187 71.9725708971961
188 72.0970767275669
189 72.0712231658105
190 71.9239273634251
191 71.7606459493385
192 71.9035908766648
193 72.0360940072352
194 72.1241445873672
195 72.1014741627426
196 72.0555985743311
197 71.9825586349145
198 71.9713292618199
199 72.0382379320169
200 72.0699003382269
201 72.1295264583609
202 72.0707648625342
203 72.0638239180713
204 72.1289540379761
205 72.2063608988398
206 72.3159453461272
207 72.2860539014535
208 72.2320421905334
209 72.1451404190697
210 72.0668872580383
211 72.1508894390851
212 72.1796735351242
213 72.1467182728744
214 72.1852634186881
215 72.0247223614037
216 71.995948782164
217 72.0850968616167
218 71.9311595207637
219 71.8627511774475
220 71.8788185421081
221 71.7782169402128
222 71.7005343824778
223 71.5861108592651
224 71.3757396842267
225 71.4337335882559
226 71.4336022017993
227 71.3516998101958
228 71.4126936653272
229 71.4226543579565
230 71.4216321538549
231 71.5602415294119
232 71.530204475247
233 71.5228315922857
234 71.5763720743931
235 71.6147075301739
236 71.5898844657726
237 71.4921785478467
238 71.4825775372383
239 71.4971058812912
240 71.4523965710961
241 71.2835434793092
242 71.3550081874753
243 71.4879912244799
244 71.5899425063859
245 71.4843197529184
246 71.4471180133943
247 71.5831329071354
248 71.5895692895343
249 71.433222705982
250 71.3697418458431
251 71.4225983587985
252 71.4062621498421
253 71.2924254442043
254 71.0974908523816
255 71.1005798341782
256 71.1527396047367
257 71.133852005806
258 71.1425944009741
259 71.2779277692246
260 71.2623807440615
261 71.0922001718608
262 70.8536438312637
263 70.7213964337709
264 70.5557039662728
265 70.3091718003201
266 70.0520122221128
267 69.8374281987423
268 69.7838977674106
269 69.6417038849826
270 69.55516710686
271 69.5325757718941
272 69.5309185019354
273 69.4881411831035
274 69.5740424772327
275 69.6216226959342
276 69.5964297362418
277 69.4996493108718
278 69.2056342004347
279 69.0702044835366
280 69.0096356387271
281 68.7936632677214
282 68.7921830978148
283 68.7530593803118
284 68.6777058973568
285 68.6113215475035
286 68.6121206739734
287 68.639478841071
288 68.7405705493703
289 68.7754266666965
290 68.82824404658
291 68.9517860151425
292 68.9417102568477
293 68.9183803242261
294 68.8392897377952
295 68.8923111853667
296 68.9304264983619
297 69.0256298430361
298 69.1368484457071
299 69.0442294421963
300 68.9304024669839
301 68.8473983412842
302 68.7335912495448
303 68.6384901728894
304 68.6851915218998
305 68.7861818950592
306 68.625644377621
307 68.3815827298789
308 68.2966323576946
309 68.3647536082424
310 68.3915887911028
311 68.4316076993807
312 68.4629258837853
313 68.5355390742291
314 68.5890933993301
315 68.5189523503178
316 68.65879094407
317 68.8669721572382
318 68.8450597127196
319 68.8429048028001
320 68.9352455309466
321 68.8198013995989
322 68.8528100784268
323 68.815905788191
324 68.6479313511728
325 68.5726406543789
326 68.5218838063326
327 68.3227802114632
328 68.3126229604672
329 68.2792144667024
330 68.2823495767399
331 68.3461675402841
332 68.2629899371881
333 68.2191196907413
334 68.2162808310468
335 68.270964346663
336 68.2079707073772
337 68.2094410567681
338 68.0703617806223
339 68.1410912576133
340 68.0383231214207
341 68.0966198511027
342 68.0301639798371
343 67.94454389367
344 68.0148738718336
345 67.8776316734574
346 67.7727934709185
347 68.0222588961139
348 68.0318382807544
349 67.8946341591269
350 67.9209724391974
351 67.7847157629842
352 67.8095204141036
353 67.8205191001227
354 67.6062189941525
355 67.6123068082604
356 67.664141745549
357 67.3182250261274
358 67.366870179827
359 67.3237271044349
360 67.1556836009198
361 67.1611460962801
362 67.0018632711784
363 67.1046100689401
364 67.2084789657034
365 67.1655364886995
366 67.0473963640083
367 67.0498226595635
368 66.8870449633935
369 66.9589874607993
370 67.0383678794803
371 66.9364660545018
372 66.9322312633434
373 66.7293094359031
374 66.7238616574936
375 66.6711439560496
376 66.7955922941731
377 66.8594844245143
378 66.8568288953495
379 66.6399600754126
380 66.5475394846631
381 66.5666311803885
382 66.6529138863602
383 66.6503989390909
384 66.4537093050395
385 66.2988638828888
386 66.134815303483
387 66.1152146324021
388 66.0004438763848
389 66.004287628169
390 65.8778961256998
391 65.8680310957186
392 65.7210561726573
393 65.5357658079805
394 65.5486765830521
395 65.4876993622386
396 65.372111310771
397 65.2154461986102
398 65.2628369253208
399 65.2058796979375
400 65.1168625655429
401 65.0077881378378
402 65.1506007941307
403 65.1958527362348
404 65.1915340295306
405 65.1440359762512
406 65.1706237327865
407 65.2043663981357
408 65.1922745296373
409 65.219321444075
410 65.3249840855857
411 65.4236078687044
412 65.2375630287781
413 65.2713677816239
414 65.0439966481941
415 65.1626873201118
416 65.0005408226465
417 65.0404645953928
418 64.9566435049417
419 64.9850956182785
420 64.9573914475083
421 64.7736617998405
422 64.6954817444244
423 64.7079878395871
424 65.0534825289618
425 64.9557520987721
426 65.1764727267529
427 65.0694691329996
428 65.1198118928629
429 65.0378182413493
430 64.9403964621183
431 65.2796696834897
432 65.3681240328719
433 65.445535594812
434 65.3721586972024
435 65.5425922094292
436 65.4481066011651
437 65.4945421368245
438 65.7115343744736
439 65.8399932330863
440 66.0200336695241
441 65.6992192419024
442 65.6667732547387
443 65.5906678029475
444 65.6254695283677
445 65.530634318668
446 65.4917756763295
447 65.3902425231755
448 65.14996250495
449 64.9525027541654
450 64.8169758853674
451 64.8320786461674
452 64.8117897531771
453 64.7753794347893
454 64.5892697615829
455 64.633254977397
456 64.6101057183777
457 64.6222296328453
458 64.5745008716227
459 64.6779266496769
460 64.5869959798337
461 64.5672876551922
462 64.7547596116623
463 64.7381087412785
464 64.8766775412179
465 64.8569968951415
466 64.9545362287977
467 64.896294492577
468 64.8964726739445
469 64.7102663589601
470 64.7579373200757
471 64.6818862509073
472 64.459860184779
473 64.3807998793657
474 64.3038719069167
475 64.3139371113685
476 64.1154050140874
477 64.151841302633
478 64.0891119552309
479 64.3269590868392
480 64.3474782334207
481 64.6398018480179
482 64.6868997724807
483 64.6095640638394
484 64.6505444975779
485 64.4455596594593
486 64.4506699630108
487 64.3759006291799
488 64.3683743722983
489 64.3090426185881
490 64.2878599367475
491 63.9842323094594
492 63.9923984160781
493 64.288855779691
494 64.3633499000732
495 64.3607907710215
496 64.333875302236
497 64.2506975655955
498 64.4217159473914
499 64.3463089922585
500 64.2125714591536
501 64.364077050678
502 64.3004548396954
503 64.1186477097628
504 64.0783683304615
505 64.2370055819772
506 64.2706847088815
507 64.4459264931172
508 64.3918924140782
509 64.7761581341265
510 64.8093104589445
511 64.8929792907217
512 64.9419662714499
513 64.8546891130658
514 64.7748792136145
515 64.849935518275
516 64.9902659200615
517 65.1459873653475
518 65.040185604046
519 64.6606766397852
520 64.6885294216614
521 64.6442480857732
522 64.4580348872473
523 64.5553459368627
524 64.5927803481115
525 64.3717865314285
526 64.2521119145457
527 63.9982772152
528 64.0354333970872
529 64.0410958507623
530 64.1084022228212
531 64.0128611286087
532 64.0648146710164
533 64.0114238290689
534 63.8721562187079
535 63.7920789053443
536 63.8061154741907
537 64.0268638638289
538 63.9723929953398
539 63.8316094829058
540 63.7271326462725
541 63.7093036623393
542 63.5128462282262
543 63.4840632128818
544 63.5684006640794
545 63.7763192801285
546 63.8886651807859
547 63.782796281922
548 63.7167476886956
549 63.8824907913409
550 63.7846078796903
551 63.6794976928205
552 63.93041322036
553 63.8312134520243
554 63.6283170856921
555 63.2940440952577
556 63.2170144407483
557 63.1570319578443
558 63.1619078510878
559 62.9684831152747
560 63.17253589448
561 63.0413463334793
562 62.8872848492455
563 62.9467209139542
564 63.0337355392502
565 63.0054342819383
566 62.7909796640938
567 62.6078508009505
568 62.6618156788312
569 62.7299097858947
570 62.5963044756026
571 62.7744647297806
572 62.8330870478238
573 62.8082582300679
574 62.931328628245
575 63.0471523113015
576 63.2167400322696
577 63.4733578689256
578 63.6124437838107
579 63.6548379120959
580 63.6558546011922
581 63.5607799520125
582 63.6471820165276
583 63.7012275051439
584 63.6583024166486
585 63.8086607061387
586 63.9381153955994
587 63.8300449217904
588 63.8326648963701
589 63.8536000481453
590 63.7641633421736
591 63.7882224720257
592 63.9499333441251
593 64.0066633875172
594 63.8901053325956
595 63.8248760840932
596 63.8163699936573
597 63.8131485113607
598 63.6229593226956
599 63.5649445253626
600 63.8240859745722
601 63.8291463976625
602 63.7717918513396
603 63.6076273993532
604 63.5224544282881
605 63.4388315921665
606 63.3184055181294
607 63.5215279670403
608 63.5757374788287
609 63.5163063744727
610 63.6224205323444
611 63.5412230349813
612 63.3740410084931
613 63.5748444096341
614 63.6632133112048
615 63.6946909699875
616 63.454802010072
617 63.217835620903
618 63.3227606554935
619 63.2848761041507
620 62.9471906264338
621 63.118378393816
622 62.945428663051
623 62.7645453824389
624 62.7600998564897
625 62.7422639130565
626 62.8993534135159
627 62.7552212595935
628 62.4364572594033
629 62.5457097485573
630 62.6017746619364
631 62.5016380311633
632 62.6029349594067
633 62.5714769137311
634 62.3811380248951
635 62.3442205761591
636 62.3268591479008
637 62.3936638046555
638 62.6100178914232
639 62.6763246293012
640 62.6257016826449
641 62.5843459932084
642 62.5705939253133
643 62.6731430350012
644 62.8125673584321
645 62.8428302572891
646 62.9195726152423
647 63.1047533504325
648 62.9479172567683
649 62.605831039111
650 62.5126771379892
651 62.4147356223178
652 62.5696275170234
653 62.4615178571308
654 62.3784998889613
655 62.4027832658389
656 62.3046612553697
657 62.1160938200545
658 61.9643685602525
659 62.1018839245894
660 62.1786995890984
661 62.282626826592
662 62.2286160582847
663 62.2142261236245
664 62.2767402227394
665 62.3357749811798
666 62.297179579848
667 62.28867552094
668 62.4912889677678
669 62.4096706826845
670 62.5092183271876
671 62.7792958843214
672 62.7167042739671
673 62.6424930842842
674 62.8148015180133
675 62.7521824119522
676 62.7607615476144
677 62.9183817228696
678 62.9300809671852
679 62.9750579957084
680 62.8889587203523
681 62.4400658246391
682 62.3239124886025
683 62.5157161696436
684 62.5489654333199
685 62.4947068913821
686 62.547892945612
687 62.5571140967992
688 62.5229797131109
689 62.5687414554889
690 62.5809600418554
691 62.9696252249801
692 62.9360882993271
693 62.9620082818114
694 62.8036182474397
695 62.8955047067892
696 62.9825488934923
697 62.7294826568609
698 62.5778604781956
699 62.4663515461918
700 62.4897129301584
701 62.1819683465983
702 62.4481836712624
703 62.2945453716173
704 62.2877556583145
705 62.1137272204973
706 61.9850821013139
707 62.0500877061589
708 62.0400970680415
709 62.1890134407379
710 61.9496074135884
711 62.1954397575026
712 62.0808139533103
713 62.1847692614859
714 62.0226335586753
715 62.1113899956884
716 61.9656436604728
717 62.1093988953136
718 62.3726079128227
719 62.3128263356328
720 62.4445136211211
721 62.3541318581695
722 62.2942960464084
723 62.2232167115665
724 62.2148535106083
725 62.0703597248476
726 62.1103370232203
727 62.0357673571286
728 62.0664930406389
729 62.0316751722441
730 62.0897657125646
731 62.0331363856524
732 62.0698428919569
733 62.028094986101
734 62.1098646789523
735 62.3197880752757
736 62.4303779723775
737 62.6275718018222
738 62.2865771231172
739 62.1002089884836
740 62.1448689585438
741 62.0586329950966
742 62.0241948036298
743 62.0372144840083
744 61.9551535129488
745 61.9563324694975
746 61.9383946059888
747 61.4772536132406
748 61.5925363821398
749 61.7563927469999
750 61.5984772320628
751 61.6098092282467
752 61.7303889197307
753 61.7645847074566
754 61.652810386375
755 61.5648336458944
756 61.590797802459
757 61.8178166298195
758 61.8795201598539
759 61.8794518310432
760 61.8015731960121
761 61.6794113064234
762 61.5056057706199
763 61.5431082997488
764 61.8686909686999
765 61.9310335128652
766 61.946566521126
767 61.7913740510938
768 61.775064684527
769 61.7422043612
770 61.7842180234216
771 61.9365548925501
772 62.0237824719135
773 61.7851278682379
774 61.6587879275573
775 61.5787932756863
776 61.6226897947573
777 61.7190540644525
778 61.7778327256981
779 61.8474681545219
780 61.8608738360915
781 61.6835185620224
782 61.6891383421636
783 61.7892749262367
784 61.9126489477554
785 62.0526977348361
786 61.7983628954999
787 61.7703273913137
788 61.7722018365487
789 61.5818791268045
790 61.8640663158543
791 62.113356428294
792 62.1106386405833
793 62.1820077250673
794 62.0334311978632
795 61.9710274057613
796 62.0709399314978
797 61.9670976478122
798 61.9291488254613
799 61.9986805976546
800 61.5552719982763
801 61.4820934182691
802 61.4509838342155
803 61.4163435832809
804 61.3670333180628
805 61.4006783765507
806 61.4754794268315
807 61.6769891090049
808 61.6024514488115
809 61.599725919945
810 62.1023062235182
811 62.1187214495693
812 62.1275206942302
813 61.9626311094613
814 61.7891477760095
815 61.642118956157
816 61.7185253100613
817 61.7259371571877
818 61.6766354833803
819 61.8467118807355
820 61.4523875103263
821 61.2363438244102
822 61.0954363616337
823 61.1090553455148
824 61.0870198543794
825 60.7992469621667
826 60.7082720886319
827 60.4986669363464
828 60.5079085529626
829 60.4931427826664
830 60.4461952837437
831 60.4685918223062
832 60.536569136052
833 60.7794423560609
834 60.7890279497681
835 61.0316185214036
836 60.9574441314687
837 61.1068535919835
838 61.0248888298779
839 60.790193498666
840 61.0065940052536
841 61.3161233270295
842 61.3043095234716
843 61.1561276350678
844 61.3098867449558
845 61.2716746190214
846 61.2404156511312
847 61.208604397931
848 61.1960878527422
849 61.0447241962654
850 60.9048815759495
851 60.8181059797967
852 60.7904891306844
853 60.8667012934353
854 61.0195157396294
855 61.2823496100483
856 61.1146814532023
857 60.9923650145498
858 60.9505498374387
859 61.131075214402
860 61.4185682342187
861 61.1819854657348
862 61.0742982102239
863 61.2783734446802
864 61.1571970578145
865 60.9548831258397
866 61.1142855965545
867 61.1113038714112
868 61.2917506012821
869 61.5743948667154
870 61.1442883164648
871 61.130468858095
872 61.4839889044804
873 61.1339216433586
874 61.0968518204356
875 61.1241648505874
876 60.8983681717281
877 61.2439862254947
878 61.4797711819389
879 61.0872344846009
880 61.2431017195007
881 61.2364231720631
882 60.9370426332207
883 61.1387492484454
884 61.0817603086154
885 61.0631225424861
886 61.383126991918
887 60.9867534021401
888 60.67454446163
889 60.8951605909385
890 60.9733832122792
891 61.2103965810411
892 61.3090607941377
893 61.1105778514294
894 61.2320208680722
895 61.1966315490891
896 61.0345114911761
897 60.9852893920446
898 60.8950322508378
899 60.7943959474696
900 60.7602203942941
901 60.6355563172771
902 60.4776860144156
903 60.5987487432429
904 60.6242250743273
905 60.7229809642809
906 60.7900083164082
907 60.8407671670987
908 61.0337546679707
909 60.8856990348178
910 60.7518636426604
911 60.5975180118576
912 60.7065008909564
913 60.3390808097792
914 60.2219372652744
915 59.9489170592675
916 59.8549918871476
917 59.9582196578556
918 60.0129417194178
919 59.9691963798829
920 60.1277693452719
921 60.1771123911117
922 60.0131329454688
923 60.2995490350677
924 60.2913202105823
925 60.2425151570305
926 60.1037266123836
927 59.8442365149215
928 59.8474682073984
929 60.485474611862
930 60.1220475706629
931 59.9643413511661
932 59.8772824713524
933 59.6752774239332
934 59.6703025160488
935 59.7562635494357
936 59.8466237999215
937 59.9891760408372
938 59.8483300888927
939 59.5701636342738
940 59.5744549895285
941 59.7347710956427
942 59.8757381391599
943 59.9744351381347
944 60.0203623461766
945 60.3203898186274
946 60.300283394545
947 60.3085842357026
948 60.0563537417647
949 59.7736402351749
950 60.006988957848
951 60.1083389887096
952 59.8817082138149
953 59.7889525651631
954 59.9852332844471
955 59.554968621186
956 59.7436198157643
957 59.6981723610939
958 59.8229719297488
959 59.7873766844838
960 59.6019697327573
961 59.4153972182092
962 59.5212113684622
963 59.5565699399893
964 59.1430435994749
965 59.4057014001969
966 59.1663049410425
967 59.1118488789692
968 59.2594956906006
969 59.6081776158345
970 59.7530504814666
971 60.1949965469868
972 60.2762338532528
973 60.155572084137
974 60.3210937494305
975 60.2184720177567
976 60.3853585709449
977 60.5477007806272
978 60.3343536972773
979 59.9621866303175
980 59.7873345874953
981 59.5387928732407
982 59.4966226597568
983 59.5533002952156
984 59.5489657268015
985 59.4680106230706
986 59.3632280025865
987 59.3276727178977
988 59.3511703240685
989 59.500734644174
990 59.7056302522366
991 59.5219054960912
992 59.4893772988515
993 59.8227923965166
994 59.7145958558012
995 59.9653006128314
996 60.148170732683
997 60.041025577504
998 60.264806067629
999 60.2034511501855
1000 60.2074548915595
1001 60.3946749986336
1002 60.4368622531863
1003 60.3484615931514
1004 60.3852025293153
1005 60.2668794160875
1006 59.9235043608784
1007 60.0919624618341
1008 59.9417969533188
1009 59.9827018655523
1010 60.0852051403979
1011 59.9312625635024
1012 60.0415910708105
1013 59.8457714350875
1014 59.8843814810789
1015 59.5445017621224
1016 59.6994494695544
1017 59.571687117057
1018 59.3709679119239
1019 59.4902046087046
1020 59.3413104388367
1021 59.6495101417103
1022 59.4177675823877
1023 59.8518820186214
1024 59.9142410375964
1025 60.3127086902154
1026 60.2201597489113
1027 60.3578669999079
1028 60.5547132282635
1029 60.6450182180989
1030 60.4315000862743
1031 60.1333905735639
1032 60.1384538895672
1033 59.7565637471577
1034 59.984366418532
1035 60.1917415628444
1036 60.593856328001
1037 60.5740422065243
1038 60.6077997361234
1039 60.5568797627036
1040 60.9932438662178
1041 61.160557396314
1042 61.3289868225936
1043 61.1901213649919
1044 61.0662038925368
1045 60.86291426904
1046 60.5361673177852
1047 60.4881353605032
1048 60.3616480232206
1049 60.1289394923951
1050 60.0108975009063
1051 59.916644100203
1052 60.0115248106936
1053 60.2437788889699
1054 59.8960299972543
1055 59.9488140953286
1056 60.061572866554
1057 60.1791402518774
1058 60.2065603310067
1059 60.1175596889846
1060 59.8742090593456
1061 59.7251762143258
1062 59.5843872619881
1063 59.2590010881633
1064 59.2571170418474
1065 58.7551312719984
1066 58.46228559358
1067 58.1812587517345
1068 58.1701801988819
1069 58.1358457475911
1070 58.1201435516183
1071 58.157452239804
1072 58.1891721590809
1073 58.2987114731418
1074 58.3829302804814
1075 58.4256976440994
1076 58.3840332835729
1077 58.2055669050892
1078 58.1068580579898
1079 58.237129352595
1080 58.2047117753826
1081 58.3921318297156
1082 58.5480889949961
1083 58.4947415007504
1084 58.6009400832761
1085 58.7116603521795
1086 58.8435776073541
1087 59.1439751041127
1088 59.2262185611637
1089 59.0651056120932
1090 59.0722473681151
1091 58.6860775162639
1092 58.5173966865508
1093 58.6453653891131
1094 58.646503677508
1095 58.7968452916955
1096 58.8631351437391
1097 58.6630728007011
1098 58.6869234539625
1099 59.1140781904897
1100 59.0455788455874
1101 59.6329084259641
1102 59.7123059488817
1103 59.6296349089272
1104 59.5109833672741
1105 59.4185022270398
1106 59.2515885904109
1107 59.3084790655495
1108 59.416815846146
1109 59.2800346873447
1110 59.5933091435893
1111 58.9438413274255
1112 58.698167994226
1113 59.212268821561
1114 59.184519789118
1115 59.2924836684529
1116 59.3241978175039
1117 59.4698094675945
1118 59.4669297630575
1119 59.5212604713103
1120 59.458891460719
1121 59.8825791944586
1122 60.1001640278197
1123 59.4554533494481
1124 59.5111978683872
1125 59.6188411757326
1126 59.6657634576733
1127 59.5387747332897
1128 59.4713144927938
1129 59.4554403685191
1130 59.4116708845227
1131 59.4878610305133
1132 59.2591169940051
1133 59.4246846837114
1134 59.2008101580094
1135 59.2056446706476
1136 59.3947871470971
1137 59.5306812411761
1138 59.2546751761731
1139 58.8282479363405
1140 58.8424069666433
1141 58.3560622339786
1142 58.2510202285574
1143 58.2815920802611
1144 58.7461310764918
1145 58.6412113729662
1146 58.4372706852434
1147 58.5808385946306
1148 58.6444805810068
1149 58.8247574727751
1150 58.6832472553278
1151 58.7643098674175
1152 58.845367033302
1153 58.7305947420346
1154 58.4585778980611
1155 58.3490291581545
1156 58.3242946441026
1157 58.1295086211236
1158 58.106521061624
1159 57.909818144867
1160 57.7601657104582
1161 57.9029017404503
1162 57.7341672851246
1163 57.9010512503025
1164 57.7541965271502
1165 58.0028802983712
1166 58.0248758752004
1167 57.9159137287264
1168 58.3120481840708
1169 58.6940379184903
1170 58.9000073431238
1171 58.9731685129252
1172 59.0820772912299
1173 59.0374867470336
1174 59.1009670327064
1175 58.8591828722449
1176 58.7551643571313
1177 58.7667881445091
1178 58.4552681497158
1179 58.1746094031546
1180 58.1515516457369
1181 57.8780976782499
1182 57.7314880586839
1183 57.7895309620851
1184 57.6408802673055
1185 57.9307239934821
1186 58.0141953041993
1187 58.2046746311451
1188 58.022067970179
1189 57.9101568859777
1190 57.9808693486635
1191 58.3355169745916
1192 58.6095627167744
1193 58.2779748143144
1194 58.3502641829713
1195 58.0570619707305
1196 58.152373206563
1197 57.777799268982
1198 57.9825374893331
1199 58.156284624016
1200 57.8454564374666
1201 57.7722910926953
1202 57.616377562781
1203 58.1891934823079
1204 58.3278910721641
1205 58.4977346648153
1206 58.4785598495539
1207 58.6711488884765
1208 58.6168695200936
1209 58.4721018441251
1210 58.9022470556787
1211 58.8612490551953
1212 58.8704331235366
1213 58.7195379154048
1214 58.8033222692975
1215 58.6000332646209
1216 58.4853103466497
1217 58.5173019329884
1218 58.77054159996
1219 58.8160878694328
1220 58.5464353154219
1221 58.5567105436542
1222 58.7867298056807
1223 58.7053896255499
1224 58.5768048318655
1225 58.5403532200131
1226 58.6663610661
1227 58.6790904311803
1228 58.7857972295588
1229 59.0077391741732
1230 58.845578500095
1231 58.8605228335323
1232 59.0887867269463
1233 59.3036039382267
1234 59.1031862235762
1235 58.8898930725626
1236 58.6495950615569
1237 58.4923677087184
1238 58.002202733253
1239 57.9395294169545
1240 58.3554116191535
1241 58.2989281042404
1242 58.1318372299515
1243 57.8032539711558
1244 57.7716217335347
1245 58.0333022008693
1246 58.5746866968892
1247 58.6787110838622
1248 59.0972341608206
1249 58.9732124573386
1250 58.9232204707696
1251 58.8027065398608
1252 58.5691992602915
1253 58.5335848683831
1254 58.6294113049721
1255 58.5548069363206
1256 58.3415796139073
1257 58.4170085627469
1258 58.1194264236519
1259 58.4667961120681
1260 58.2035164708178
1261 58.3445033586731
1262 58.2719852770856
1263 58.4009304520867
1264 58.7384445327818
1265 58.671014082416
1266 58.4997678767162
1267 58.2687206942338
1268 58.1733111127187
1269 58.0120070428976
1270 58.0637910171561
1271 57.7807177321127
1272 57.8744748436796
1273 57.6965593924672
1274 57.6831734012271
1275 57.8614195293234
1276 58.2492313779568
1277 58.3071881940005
1278 58.705794502109
1279 58.7501939701643
1280 58.8614940013689
1281 58.8906875699937
1282 58.8799146677649
1283 59.0892988796071
1284 59.2107928928213
1285 59.119743485165
1286 58.7214788534216
1287 58.7701107316064
1288 58.3345486192938
1289 58.2586623076267
1290 58.1258748075501
1291 58.0768479032113
1292 58.3793266671988
1293 58.132035977928
1294 57.8298960131267
1295 57.8444988772908
1296 57.6943494919515
1297 57.8358330290099
1298 58.0399853022043
1299 57.9067866131802
1300 58.4119886911209
1301 58.6536621407296
1302 58.472510516934
1303 58.5560363125692
1304 58.9985607867763
1305 58.9723385954269
1306 59.0241936546799
1307 58.8455401658054
1308 58.8769543867324
1309 58.8805613001584
1310 57.9920153391509
1311 58.1945671431742
1312 57.7483971639105
1313 57.9650950611928
1314 57.4624102055982
1315 57.395351489837
1316 57.2618869900628
1317 57.1195228035712
1318 56.7625047289277
1319 56.8843058657205
1320 57.1191782166138
1321 56.628084701473
1322 56.7534740576012
1323 56.35639775771
1324 55.9770532478032
1325 56.2015599371612
1326 56.3277523200466
1327 56.3026423827922
1328 56.4506893372705
1329 56.1371458956449
1330 56.5111558685497
1331 57.0538371442187
1332 56.8881853199593
1333 57.0262592004326
1334 57.5792809210576
1335 57.4247650369827
1336 57.7016501452413
1337 57.6277361137708
1338 57.5305496668094
1339 57.3184805056914
1340 56.8612332246465
1341 56.342348698242
1342 56.6596605166385
1343 56.2949701787632
1344 56.2783648556152
1345 55.9563183326831
1346 55.5376263832899
1347 55.6303323061829
1348 55.7392454689453
1349 56.393369775173
1350 56.6780306388042
1351 56.6540267540749
1352 56.6994671262704
1353 57.0898044698995
1354 56.9885344566252
1355 57.3216271067396
1356 57.4155546704992
1357 57.3150674804587
1358 57.0453954137733
1359 56.7179737740497
1360 56.6132928511794
1361 56.6068150345491
1362 56.4691189617327
1363 56.1652078297771
1364 56.0897784820707
1365 55.8992560792289
1366 55.9243017463447
1367 56.0656012566457
1368 56.6692994419657
1369 56.6403945908178
1370 56.566297978836
1371 56.7497333851027
1372 56.9986319038488
1373 57.0436980474913
1374 57.5618394639586
1375 57.6005676205541
1376 57.5641327286095
1377 57.5179200276753
1378 57.2921399712266
1379 57.4787947797297
1380 57.5166162154369
1381 57.4121568881227
1382 57.2566957503147
1383 57.3158728935351
1384 56.708777188088
1385 56.9035455703804
1386 57.0723577905493
1387 57.3177722084072
1388 57.6343372549118
1389 57.5506909744356
1390 57.5206617675491
1391 57.3443128988384
1392 57.34245543466
1393 57.141224714121
1394 57.6892567750538
1395 57.5384345089553
1396 57.2852544405322
1397 57.3379598058112
1398 56.8576573856605
1399 56.7589461602979
1400 57.1753126188538
1401 57.4830300472421
1402 57.2528046794299
1403 58.0731047174302
1404 57.6244885411233
1405 57.3802480335557
1406 57.5737215896568
1407 57.6724993708351
1408 57.4262942410938
1409 57.5430269459974
1410 57.0718961480402
1411 56.9727819317452
1412 57.3236998389739
1413 56.7043718053048
1414 56.9044579872125
1415 57.0798322543332
1416 56.8701912829464
1417 56.3149713767894
1418 56.5588510405178
1419 56.4139413470346
1420 56.6250424046399
1421 56.9551158510734
1422 56.7105546138578
1423 56.8169040008615
1424 56.474585203648
1425 56.2560903993015
1426 56.3448375675082
1427 56.630594082888
1428 56.6697747070521
1429 56.5551694982607
1430 56.3439070783486
1431 56.2089095195904
1432 56.1640010209137
1433 56.4887803819806
1434 56.4840382833896
1435 56.9545775118755
1436 57.0347986290895
1437 56.9386141582439
1438 56.8151208276093
1439 57.2304366128332
1440 57.269104363026
1441 56.9640710581965
1442 57.3354515454664
1443 57.0493177267734
1444 57.0996964772921
1445 56.8086429716366
1446 56.7678599529804
1447 56.773755197188
1448 56.9132470317623
1449 56.3950294551401
1450 56.6023974679559
1451 56.5200048879044
1452 56.3685418812394
1453 56.1663280945076
1454 56.3154572443593
1455 56.4615237923608
1456 56.7087546256476
1457 56.9311706632404
1458 57.0679794194121
1459 57.257502547846
1460 57.0381883195961
1461 57.0431015742256
1462 57.262964881472
1463 57.6050639542191
1464 57.5664591048208
1465 57.4755525357946
1466 57.3449103183341
1467 57.2359687737928
1468 56.7552133151673
1469 56.7777688489635
1470 57.0685835302447
1471 57.0108035183957
1472 56.6604572029645
1473 56.4158897654405
1474 56.3907544827637
1475 56.2676300806463
1476 56.3909456419306
1477 56.3388314432518
1478 56.5623040074387
1479 56.4747182213336
1480 56.1795507412628
1481 56.4740607441925
1482 56.4773583469482
1483 56.4066172469265
1484 56.3618158064627
1485 56.81028927652
1486 56.4762617327836
1487 56.7413819498849
1488 56.7317887767685
1489 56.8109028623175
1490 57.1322292250233
1491 57.0732867283319
1492 56.9735970030378
1493 56.8997554180584
1494 56.8123994778184
1495 56.4638081111882
1496 56.6390579267455
1497 56.4761449938204
1498 56.690655456878
1499 56.6046061405697
1500 56.2892425982431
1501 56.2141773725291
1502 56.2625071988299
1503 56.3769023580701
1504 56.6973261466979
1505 56.7935246741216
1506 56.4990231431346
1507 56.2577355665185
1508 56.2250000384311
1509 56.4116277783733
1510 56.2046720224121
1511 56.2263213046238
1512 56.4757321783718
1513 56.3964920719845
1514 56.331579418561
1515 56.1479210956764
1516 56.2898250483948
1517 56.8478174873793
1518 56.5265254170996
1519 56.4371365714317
1520 56.9351868017343
1521 57.0654934170311
1522 57.0329092561188
1523 57.2797779335414
1524 57.2380968762466
1525 57.6907668845412
1526 57.8568599766185
1527 57.3351962104912
1528 57.4871678752868
1529 57.4673905254882
1530 57.6696040680523
1531 57.6560426741271
1532 57.7660204226616
1533 57.5422043424586
1534 57.3031349047692
1535 56.8800191194362
1536 56.665500111503
1537 56.3807971574771
1538 56.1579400468237
1539 56.7243727775746
1540 56.4226749593311
1541 56.3724420734454
1542 56.0677037008234
1543 56.3034047891771
1544 56.272095769362
1545 56.3765697170911
1546 56.641316834021
1547 56.8189097867164
1548 57.188067483483
1549 56.8085674747618
1550 56.9827546822244
1551 57.0739827427559
1552 57.0557945608873
1553 56.8237396018415
1554 56.7518050288033
1555 56.7305061982707
1556 56.6114334923758
1557 56.7381653259268
1558 56.7151857798841
1559 56.6216614771859
1560 56.3762658207068
1561 56.7513835957806
1562 57.0711802996695
1563 57.0231951191511
1564 57.3277094193433
1565 57.3287777118392
1566 57.3541764393
1567 57.3690234846719
1568 57.6319496384051
1569 57.5406127357599
1570 57.6999772863429
1571 57.3170678671805
1572 56.9967531916851
1573 57.3685089402923
1574 57.3453222464324
1575 57.2867953434397
1576 57.6852127958769
1577 57.8094116358535
1578 57.8727951097752
1579 58.3493852948517
1580 58.813770477617
1581 58.8653673535701
1582 59.3051176242544
1583 59.2657833003415
1584 59.2215881254629
1585 59.7075565499233
1586 59.7614003012123
1587 59.8372309147632
1588 59.3946643641421
1589 58.8689583091207
1590 58.0621376226399
1591 58.2640579116373
1592 58.1943157795695
1593 58.549796223275
1594 58.6360522107907
1595 58.4007884147725
1596 58.2282237887534
1597 58.173816760517
1598 58.4460800811188
1599 58.5472752241539
1600 58.6429682792614
1601 58.5464659538215
1602 58.3426741525768
1603 57.8418840257835
1604 57.9035140726912
1605 57.940275745813
1606 57.7862798471764
1607 57.9654321330196
1608 57.9097139221008
1609 58.4273818798722
1610 58.5007584188424
1611 58.2430578931758
1612 58.2495564091308
1613 58.2294881263163
1614 58.0223760120925
1615 57.6752890444576
1616 57.6126124015414
1617 57.7078817795456
1618 57.7193620624288
1619 57.4595856860209
1620 57.3355416271364
1621 57.7120164643516
1622 57.7436555312447
1623 57.8527343496831
1624 57.9043590531709
1625 58.5090795017433
1626 58.5676541523852
1627 57.9394692623345
1628 57.7352522425324
1629 57.800658359605
1630 58.2014694720155
1631 57.973174834731
1632 57.7750224589972
1633 57.8894248909096
1634 57.8128905759851
1635 57.5263853709137
1636 57.596572373966
1637 57.7844453348413
1638 57.8716245418198
1639 57.5410591092123
1640 57.3942418547162
1641 57.5385514526693
1642 57.4189747634107
1643 57.1999170158106
1644 57.1927806141134
1645 57.1334806826932
1646 56.6854880287832
1647 57.0186063710586
1648 57.0471787979563
1649 57.1199935587098
1650 57.0480822715507
1651 56.6038819744523
1652 56.9070722271288
1653 56.9186763677071
1654 57.0501883729779
1655 57.0245222949331
1656 57.4622407612515
1657 57.0995694276122
1658 57.3296255396254
1659 57.460237530273
1660 57.4523279857118
1661 57.5526748894078
1662 57.3613412896277
1663 57.2846896427864
1664 57.8421890399943
1665 57.8638638588552
1666 57.5889399660174
1667 57.6925503516623
1668 57.7996573310633
1669 57.6608594373354
1670 57.8833334217282
1671 58.0164020200986
1672 58.3808742261802
1673 58.5125013493319
1674 57.7017481901887
1675 58.0126245516442
1676 58.6257300817761
1677 59.1110678712427
1678 59.0758188684304
1679 59.1344258147345
1680 59.1912720423286
1681 59.2997581190776
1682 59.3536227839086
1683 59.6534239264636
1684 59.8553547497352
1685 59.7473755416783
1686 59.6504276222039
1687 59.348259733819
1688 58.9081554011127
1689 58.8377072096169
1690 58.3729339812108
1691 58.3638811526523
1692 58.262339079083
1693 57.7393875416711
1694 57.6192838838952
1695 57.4215395682253
1696 56.9917531124165
1697 57.0421349385272
1698 57.2675698790442
1699 57.3364985991402
1700 57.7457034909161
1701 57.8371445927543
1702 57.6969282804278
1703 58.0986190569337
1704 58.4730967922945
1705 58.6623556392302
1706 58.7661553849628
1707 58.9130290491811
1708 59.0741543639598
1709 58.9749216914365
1710 58.8483369097483
1711 58.7771034937914
1712 59.0693559885476
1713 58.8817679210488
1714 58.9744453438887
1715 58.5293891589454
1716 58.5375959624534
1717 58.1978877666649
1718 58.0368938379821
1719 58.3662855042738
1720 58.221481454186
1721 58.320949378772
1722 58.3202550142912
1723 58.3770915338822
1724 58.3484374716124
1725 58.6605692604729
1726 58.5427505042629
1727 58.736057297345
1728 58.8794680865255
1729 58.8439823445773
1730 58.8198809646939
1731 58.5657744922662
1732 58.3911168828384
1733 58.4193327134025
1734 58.6041552819047
1735 58.5643276064407
1736 58.5956470096515
1737 58.4273747191865
1738 58.3188854041697
1739 58.1158884723503
1740 58.395936717305
1741 58.4555420178309
1742 58.1721932214062
1743 57.8747756347415
1744 57.1240265980386
1745 57.2241240340469
1746 57.2161743410411
1747 57.015514598184
1748 56.7446520553571
1749 56.6569468722552
1750 56.4968051277281
1751 56.5939602096239
1752 56.923664243069
1753 57.2034680004688
1754 57.2035107134669
1755 56.7998593769901
1756 56.8219287954591
1757 57.0262638655617
1758 57.3145722794233
1759 57.7277168934765
1760 57.5935516719625
1761 57.7078846493482
1762 57.6293854560946
1763 57.5322983506061
1764 57.853524338833
1765 58.5891714473728
1766 58.6302807793826
1767 58.7002927051376
1768 58.6506006312998
1769 58.0589916765056
1770 58.2739008469758
1771 57.8367577284251
1772 57.6643749378417
1773 57.4692676767588
1774 57.2652496554836
1775 56.6062420864194
1776 56.4517650698221
1777 56.2219581123844
1778 56.369988675992
1779 56.7150854775254
1780 56.4561012621611
1781 56.5282112053723
1782 56.9486508892619
1783 57.3163679964394
1784 57.3736527338018
1785 57.3526397281764
1786 57.4657727585581
1787 57.2442966372623
1788 57.2247313138339
1789 56.9725617189551
1790 56.9101244897888
1791 57.203185732383
1792 56.8302044243665
1793 56.6435612613369
1794 56.6501512402767
1795 57.0234404578658
1796 56.9409342052642
1797 57.1059419750955
1798 56.5633896131211
1799 56.4862147411143
1800 56.7123269194905
1801 56.6600190008607
1802 56.5958796011764
1803 56.5152803473312
1804 56.6008468372174
1805 56.3812189773798
1806 56.7736243457461
1807 56.9949321000871
1808 57.1915054919208
1809 57.9815256508979
1810 58.0227412003536
1811 57.8493776868949
1812 57.9253753757117
1813 57.5076360168059
1814 57.3384383409434
1815 57.1180920168931
1816 56.719147165626
1817 56.3254913542188
1818 56.523294301237
1819 56.230634489607
1820 56.0571237780707
1821 56.0744523662994
1822 55.8069964542934
1823 56.4632201463313
1824 56.9075460668246
1825 57.2695705401573
1826 57.1580145734418
1827 57.4069024622475
1828 57.1395069256715
1829 56.5418713952323
1830 56.5132791228312
1831 56.5204737371722
1832 56.4228215041873
1833 56.1835803774193
1834 55.7459143020833
1835 55.9824523895599
1836 56.2568539041003
1837 56.2379062231207
1838 56.7278530564959
1839 56.8565405667391
1840 56.9288217168828
1841 56.766500217145
1842 56.9701176685458
1843 56.9746163146692
1844 56.8068154675512
1845 56.122762320401
1846 55.819273154158
1847 55.9721400012094
1848 55.6477608985613
1849 55.9291016272873
1850 55.9376786318582
1851 55.9601720336338
1852 56.0634582402031
1853 56.0111185255377
1854 56.3174646119772
1855 56.8702630495487
1856 56.7668314592426
1857 56.6990164823057
1858 56.5439046379476
1859 56.1833245924171
1860 55.9126688478782
1861 55.8060617039563
1862 56.0704375414675
1863 56.6404085595645
1864 56.3241531365691
1865 56.1304836376374
1866 57.3121398750789
};
\addplot [semithick, crimson2143940]
table {%
0 35.6993726451725
1 34.187195293676
2 32.2182040182592
3 31.4581040561471
4 30.3668740945967
5 29.1048589647476
6 27.9918550796153
7 27.5614750262626
8 27.3177011705838
9 26.8934611667554
10 26.6215714563225
11 25.9989022975841
12 25.7616406289475
13 25.5980843130244
14 25.6734042390933
15 25.9515498063104
16 26.1528712104019
17 26.2451107158782
18 26.5125494485078
19 26.9777057641255
20 27.4878715553317
21 28.0503967496281
22 28.4173366429869
23 28.7150543806851
24 29.048295538427
25 29.4407720017713
26 29.7464206249252
27 30.2883421817251
28 30.347654602648
29 30.6966209587164
30 31.1896111910638
31 31.5292785110316
32 31.3421643005877
33 31.635597973289
34 31.9782978929718
35 32.1672991054843
36 32.375604254487
37 32.2720873336997
38 32.6698834936773
39 32.5598439174901
40 32.7754193570229
41 32.6796443756531
42 33.0705093262151
43 32.8264376836705
44 32.951791383629
45 33.099046865724
46 33.5658557574574
47 33.7211920681917
48 33.5719910009294
49 33.9576285640408
50 34.1764091862535
51 34.3308426058408
52 34.3472244115252
53 34.8384678166729
54 34.7000824375924
55 34.848438428803
56 34.479673335074
57 34.554560249955
58 34.4184751601705
59 34.5808206202089
60 33.8544948818453
61 33.8319864880844
62 33.9917067837729
63 33.57399112816
64 33.8234290065501
65 33.2464949873885
66 33.3377528675556
67 33.3981925226324
68 33.6584470634275
69 33.625729809224
70 33.9197899222321
71 34.3763489906895
72 34.1075794149797
73 34.1055188136732
74 33.8108735326953
75 34.1532702512656
76 34.3575082508205
77 34.2569421231776
78 34.341315363048
79 34.4213013382466
80 34.5299504907579
81 34.399547726617
82 34.684822984246
83 34.7750469559594
84 34.8410761095546
85 34.8973814968205
86 34.9435738959864
87 35.3795772693973
88 35.3286785713292
89 34.9579016103458
90 34.9959922955048
91 34.5924087464316
92 34.4075910258961
93 34.5722088045719
94 34.5203628121928
95 34.7928067757317
96 34.5026755062952
97 34.4565962722747
98 34.6672079666176
99 35.0936293131917
100 35.0975282115909
101 35.6566486848179
102 35.784683081835
103 35.4152539739191
104 35.4792111661407
105 35.4881701365419
106 35.8245709167718
107 35.5478512217139
108 35.568481859142
109 35.6016853581764
110 35.7632291582905
111 35.7295875505667
112 35.8932249899568
113 36.4221539616356
114 36.3939605134794
115 36.0748725840071
116 35.9470019389023
117 35.7776076758476
118 35.9776799466044
119 35.6784401005897
120 35.4467533712955
121 35.2240299349694
122 35.1487497168645
123 35.2343921572903
124 35.6810150657123
125 35.7561784362092
126 35.8405270159717
127 36.1703813049978
128 35.9033494754559
129 36.1602413818779
130 36.818491516964
131 36.9600492747582
132 37.1933729664758
133 37.2514383697218
134 37.2916299060024
135 37.4859143552935
136 37.4353858264067
137 37.5697687774855
138 37.7061361269789
139 37.4609602122929
140 37.6207055298995
141 37.7490372508948
142 37.5360417586855
143 38.002461940417
144 37.9439484946864
145 38.3266644031363
146 38.7422846957675
147 38.9297928059382
148 38.857156010497
149 39.1858005864266
150 38.6516964201381
151 38.6145080497192
152 39.0108384724778
153 38.6719318976494
154 38.5797172461565
155 38.3155225216512
156 38.2356811643267
157 38.241644903312
158 38.1885719909186
159 38.0108655665197
160 38.7040197926593
161 38.8263598642927
162 38.9029182112835
163 39.0931272022343
164 39.4678758615259
165 39.5950688455563
166 39.451278078981
167 39.3906590460717
168 39.2708362813562
169 39.2134755608664
170 38.3220566820842
171 38.7070345453788
172 38.3983279018316
173 37.9763551425321
174 37.3789384745414
175 37.9341046449588
176 37.8878244670736
177 37.387553656907
178 37.4187892240985
179 37.2810192032566
180 37.5709837108459
181 37.1082575837126
182 37.2655694975127
183 37.016277013887
184 37.1984872706103
185 36.3967313349215
186 36.3350434056641
187 37.0087731014257
188 37.8470205004951
189 37.8850028648928
190 38.0234296481497
191 37.8937341968379
192 37.6304128955893
193 38.7292655838441
194 38.8683784073002
195 39.080601324968
196 39.2338767127193
197 39.2927052760694
198 39.2328490932287
199 40.1395799385197
200 40.2641667602621
201 40.5234838812744
202 40.9600041555207
203 40.5472313462425
204 40.937900752765
205 41.24967799831
206 41.499810544182
207 41.0289985463415
208 40.7306973652956
209 39.9590818708891
210 39.6258104806069
211 39.503508075587
212 39.5619613992721
213 39.502859112229
214 39.4188375933853
215 39.0271990993635
216 39.5542452605462
217 39.798699846402
218 39.2806885806953
219 39.3058130608687
220 39.4139238364557
221 39.43345325982
222 39.0840913392466
223 38.7243590299531
224 38.8007515902171
225 38.8271673573592
226 37.9213626780111
227 38.0628445653706
228 38.3967248641493
229 38.5941025997863
230 38.8443260037521
231 39.4274625176307
232 39.6582046952726
233 39.8023863276913
234 39.5220603733328
235 40.0119688474249
236 40.0346888551075
237 39.5594611225976
238 39.564618145395
239 39.6943663091899
240 39.5406228241459
241 39.4170667871761
242 39.4526977748984
243 39.7505533863862
244 39.9963366972788
245 39.7106272655386
246 40.2010030621594
247 40.9312756237495
248 40.8585074121311
249 40.7823343756932
250 41.0322514464219
251 40.8986563175954
252 41.1979202078805
253 40.9850716825102
254 40.8690729258475
255 40.8685667853092
256 41.0846863742772
257 40.7512802350932
258 40.9924377899746
259 41.0365916074216
260 40.7067885776057
261 40.464186366518
262 40.0595115070802
263 40.079637469402
264 40.0555394509801
265 39.7006230767068
266 39.3906410076824
267 39.5217037491049
268 39.8983601986725
269 39.9738706000072
270 40.5182098674325
271 40.757842803227
272 40.8297901253899
273 41.0869795411964
274 41.5938482701944
275 41.9470548901623
276 41.8959884485903
277 41.7193878517585
278 41.7787104910233
279 41.8024931388198
280 41.9513643941653
281 41.524593272052
282 41.4327593447328
283 41.1900455792871
284 40.4692398272682
285 40.4470239790236
286 40.509523930778
287 40.6406058977633
288 40.6512317238378
289 41.377041194603
290 41.2676231287321
291 41.2028024853867
292 41.3916652907167
293 41.102292176166
294 41.2303064468391
295 41.3592760490038
296 41.3224910521303
297 41.5448391522239
298 41.1467807289517
299 40.1792796194213
300 39.6128515269855
301 39.7423397339291
302 39.4027853619239
303 39.7605400213456
304 40.4383717312578
305 40.5209760852124
306 40.5529553689777
307 40.1957938062072
308 40.2137075628242
309 40.5577372168345
310 40.5789714111828
311 40.8999569372433
312 40.9420862491526
313 40.678715806351
314 40.0686255102644
315 39.8066482592852
316 40.3585125586895
317 40.1001057940808
318 40.2615227709644
319 40.1952020914552
320 40.6148739421616
321 40.1275424702551
322 40.9735867785373
323 41.1269522842788
324 41.0892234867046
325 41.1865179770019
326 40.7155363515104
327 40.7368888052061
328 40.6547346954336
329 40.4149116019476
330 40.2218630328917
331 40.7814286766166
332 40.4312771289592
333 40.6774802838303
334 40.6783365291541
335 41.164506472282
336 41.2884445952157
337 41.8748338055244
338 41.8365042259471
339 42.4991052864124
340 42.3140665353603
341 42.4972797346706
342 42.2772373594893
343 42.3639954617806
344 42.5136775940951
345 42.5085913522137
346 42.1958796760681
347 42.8213471267833
348 42.7302805926498
349 42.3860942355362
350 42.867330092119
351 42.6271166982647
352 43.0407354691217
353 43.3389992747627
354 43.4014192998528
355 42.8827830013375
356 43.0143824351745
357 42.6988229231695
358 43.3226407831373
359 43.3440425395593
360 42.7350128469276
361 42.5357806803517
362 42.3402942431154
363 41.6977565957965
364 41.5018465513456
365 41.8968158633964
366 42.0085411658903
367 42.0322418191808
368 41.6230121088644
369 41.8640394688059
370 42.3003480411871
371 42.81153459306
372 43.1232473538521
373 43.416520435665
374 43.7660932430299
375 43.5871105961643
376 43.4943128424808
377 42.5796205045545
378 42.3869950455978
379 41.800892181801
380 41.3133169248865
381 40.9987799157658
382 40.4820150514491
383 40.9555474991751
384 40.6023487892416
385 40.0919865447603
386 40.1562921239282
387 40.3649796608159
388 40.7472716877993
389 41.0674759416857
390 41.5411561581108
391 41.5225075698077
392 41.0968675591445
393 40.8622480009977
394 41.2284380681852
395 41.830410869
396 41.4141693180971
397 41.5134778324831
398 41.0697397843273
399 41.1106928449836
400 41.0363249317838
401 40.7221682168846
402 41.1908332497993
403 41.2063953988052
404 40.8788826372248
405 41.0175319850512
406 41.6003884074198
407 41.9402524067774
408 42.204707089848
409 42.1893680952208
410 42.4979370443464
411 42.7925913721545
412 42.8436246941425
413 42.5717392995162
414 42.3531858607881
415 42.2477024549165
416 41.6737060885122
417 41.3645933430537
418 41.7615951708514
419 41.7283433531036
420 41.4790052130379
421 41.8573084384877
422 41.4693913119515
423 41.4126180725918
424 42.1390208763263
425 41.9166911569574
426 42.0991364594407
427 42.0799007832552
428 41.6518547781381
429 41.3254391710713
430 41.1135724887415
431 40.6218479488029
432 40.818795180027
433 41.0387891992425
434 40.5073342335668
435 40.4257983091108
436 40.3207796812417
437 40.5456454152897
438 40.3884480310294
439 40.7312918881556
440 40.7264666477688
441 40.5839825838567
442 40.6441965367782
443 40.4044811114352
444 40.2771640872242
445 40.0349996113481
446 39.9207019908366
447 39.7624755094043
448 39.8190895632406
449 39.608954546909
450 39.7526219059685
451 39.7729224112965
452 39.7589680757705
453 39.8733813014974
454 39.8373680943552
455 40.0726775730487
456 40.3268352436156
457 41.0358194277778
458 41.683675468198
459 42.1072197363004
460 41.8721819527527
461 42.1280298196602
462 42.2728940078025
463 41.8982783825215
464 42.1120036948042
465 42.1530332890397
466 42.1056057723902
467 41.7538267412659
468 41.234818817022
469 40.7498678671666
470 41.4846284203458
471 41.9417292938482
472 41.6023975829588
473 42.0232462767058
474 41.828235829779
475 41.9805749934976
476 42.3733525269263
477 42.1145933982704
478 42.0003650736013
479 42.4407263338256
480 41.7466195307888
481 41.5706950696251
482 42.0663609933848
483 42.1194104179036
484 43.6031038678279
485 44.0051307259271
486 43.9979358097694
487 44.7472499756357
488 45.10252715088
489 45.2199810123079
490 45.7446507303209
491 45.4211107327303
492 45.0939569157748
493 45.4380073502113
494 45.0473329092521
495 44.8995621114255
496 44.6664671540467
497 44.1436128707
498 44.2700889922062
499 43.7930849657643
500 43.1126193200397
501 43.2204311876106
502 43.0012006469447
503 42.3475799134925
504 41.787096680483
505 41.5806940353243
506 41.4492687201574
507 41.3867582421942
508 41.1557722413203
509 41.7462421962467
510 42.2695296998616
511 42.7756096935784
512 43.5738828692143
513 44.1504320292049
514 44.0225611573244
515 44.2096719441522
516 44.6775700065541
517 44.912733906209
518 44.8322639706021
519 44.4865809472816
520 44.395897815589
521 44.2110119969781
522 43.6493176964786
523 43.2050547956288
524 43.0095278323972
525 42.4658516478802
526 42.358032881282
527 41.9331133044843
528 42.0970275353727
529 42.4461445256144
530 43.7914895307792
531 43.7347861042343
532 44.2614925067139
533 43.9605137621878
534 44.4464799613715
535 45.0781516828901
536 45.3144040582806
537 45.942936813904
538 46.0296540235382
539 45.6365738648844
540 44.147432783289
541 45.3082858749805
542 44.6830630962419
543 45.1160046705906
544 44.8631024877666
545 45.0169966427008
546 45.1057641790471
547 45.5852452559832
548 45.9159206374782
549 46.4932769112421
550 46.7930466211709
551 45.5545335333297
552 45.957558540857
553 46.0469717181165
554 46.0075719009817
555 45.2356440780753
556 45.1664852789307
557 44.0609792376082
558 43.66739058561
559 43.3393851371172
560 43.0914265019809
561 43.2870681143753
562 42.7875723322231
563 42.9393348662274
564 43.6761202824794
565 44.4250330341176
566 43.9808275357109
567 44.0879433686681
568 44.2891281601193
569 44.3244287614055
570 44.8904044427825
571 44.7057594160844
572 45.1742007837836
573 44.6336633740235
574 43.8167814244336
575 43.3429449652912
576 43.625710323695
577 43.8529562502958
578 43.4575282850529
579 43.40425976597
580 42.8771799329732
581 42.9797237245126
582 43.158185360331
583 44.0205619787174
584 44.1120816133063
585 44.4186779201985
586 44.5348445852942
587 44.337338542751
588 45.1889685112996
589 45.0672441091384
590 45.6337604179646
591 45.815290737455
592 45.6207308074671
593 45.4569352021348
594 45.4777147988116
595 45.2425605082563
596 45.1666957142208
597 45.1649572867647
598 44.958370530874
599 44.9286899128172
600 45.0623237597848
601 44.7677755647861
602 44.8426046271492
603 44.4491001134154
604 44.7627111079798
605 44.9462407182219
606 45.4679715787947
607 45.6527567241037
608 45.2628713104647
609 45.2731644578378
610 45.2253415585049
611 45.6938089152935
612 45.6334331509912
613 46.5280957836135
614 46.5153670153619
615 46.7988709327061
616 46.449891123194
617 46.7261057741829
618 46.8676954562164
619 47.0503944393099
620 46.6383183988908
621 46.4387368281301
622 46.7283160916716
623 46.5514212173482
624 47.1985384253315
625 47.3934745114889
626 47.323663010938
627 47.5655575422189
628 47.9312770749498
629 48.4399505322835
630 48.6697834162515
631 48.647988518137
632 48.9476953904107
633 47.8959703000575
634 46.7033540403858
635 45.7892539207238
636 45.9918416672489
637 45.9542277408364
638 45.8540427571899
639 45.2761667743415
640 45.3916022934855
641 45.6212843536948
642 45.049967372762
643 45.5409030890604
644 46.1353445964926
645 46.5505683769568
646 46.1452663305475
647 45.9104162780195
648 45.5815558694772
649 45.844156996469
650 45.4685954643903
651 45.0053371075276
652 45.0466836597861
653 44.6394805330814
654 44.3116219137661
655 43.8206684153847
656 43.3950112218091
657 42.9883538609998
658 42.4861111145103
659 42.3415782060553
660 42.9643559001493
661 42.6259492200703
662 42.9739513430108
663 43.4955685772874
664 43.4047018151695
665 44.3078102380915
666 44.6800150840335
667 44.6685500982539
668 45.3289380003871
669 45.6498117604634
670 45.0124783704678
671 45.815272890943
672 45.689043997681
673 45.5579329730227
674 46.3407457803261
675 45.9728656714746
676 45.8004865087854
677 45.785648192148
678 45.504512013261
679 44.9357596288403
680 45.4086137356763
681 44.8256371463661
682 45.0811243278886
683 44.8766879220898
684 44.4376713515187
685 44.9419037787135
686 45.0091041493626
687 45.1272019790021
688 45.4336935832447
689 45.9807835011426
690 46.1102275866987
691 46.5666775594697
692 46.1385328212051
693 46.7466315565863
694 46.056347052529
695 45.608221156688
696 46.0346816111026
697 46.0931500923945
698 45.6018628427294
699 45.4188540845233
700 44.9248406909991
701 44.5486899792834
702 44.7170468761966
703 44.4158386922013
704 45.4785417914541
705 45.6166104316449
706 45.3316616622517
707 45.9574497332246
708 46.001565567599
709 46.3129669134792
710 46.8225262658387
711 46.976135005724
712 47.2962580617858
713 48.1545964774574
714 47.5650945251853
715 47.8249862840102
716 47.9796028773464
717 47.9321752198394
718 48.2129963709441
719 47.4757983005807
720 46.9393512790458
721 46.9068403356239
722 46.4037244913808
723 45.3752987505852
724 45.1595724806004
725 44.6455035501682
726 44.9072377745711
727 44.3622352880497
728 44.0017773008186
729 43.7899295850734
730 44.4150423500497
731 44.4070056907847
732 43.8111334569492
733 43.6740519517333
734 44.0311505914843
735 44.3521515059058
736 44.4017175591944
737 45.2740901052787
738 45.7212949544652
739 46.541829768722
740 46.0409161126699
741 45.9784468209068
742 46.9970367076444
743 47.6380147113605
744 47.6362917767916
745 47.782433960751
746 47.462163586277
747 46.8490109241173
748 46.4572870004816
749 45.9963046134533
750 46.3769665859977
751 46.7304291637142
752 46.8314197234151
753 47.1736308036728
754 47.5197861842934
755 47.4680913056522
756 47.7910734127681
757 47.5214096968718
758 47.6418487220262
759 47.5558515443906
760 47.8990682027444
761 48.2158419849274
762 48.0316133849699
763 47.6504424243441
764 46.8916071000836
765 46.7356301806504
766 47.1707504324805
767 47.1583124173279
768 48.6540316701166
769 49.3942061435222
770 48.5342035874119
771 48.1248362532524
772 48.2920820619472
773 47.8854080765368
774 49.7555206446689
775 49.499470417539
776 48.9314704104564
777 49.6632202825856
778 48.6978858879965
779 48.318684779268
780 48.4753815391725
781 48.2614668482239
782 48.2747739915014
783 47.9325309424654
784 46.5444982097631
785 47.044263871993
786 47.2061352124296
787 46.6648887039362
788 46.6596457160779
789 46.9558932040065
790 47.4330422771104
791 48.0280886973195
792 47.6230979861402
793 48.2609687421586
794 48.0497658607112
795 47.6054197035425
796 47.5909934610493
797 47.3733297998162
798 47.2419130106062
799 46.9219857241804
800 46.1799707567027
801 45.4299783083776
802 45.8326583574073
803 45.5964014697283
804 46.045637501534
805 46.0808025266756
806 45.8774195222757
807 45.7626786720004
808 45.7765047543893
809 45.3324473325214
810 45.4893006428931
811 46.0286490637144
812 45.6662660957133
813 46.0883872379119
814 45.4533338881646
815 45.6557776882313
816 46.3475025046081
817 46.9968363939023
818 47.0003915597689
819 47.5211482296333
820 47.7699058709263
821 47.5891828197102
822 47.5844591266143
823 46.8812259569999
824 47.3883697115493
825 47.3684997030674
826 46.4126845689976
827 45.8556039710831
828 45.823962361667
829 46.2496560294809
830 45.7707203206392
831 45.5990039966444
832 46.1817506332376
833 47.1333043542242
834 47.643053127506
835 47.9643795111316
836 48.2207394016808
837 48.3910898632539
838 48.2863619157536
839 47.6504261959467
840 48.202162241067
841 48.5126195163501
842 47.8460796459891
843 47.1834525717317
844 46.5063749020726
845 46.1747330194226
846 46.0734987250096
847 45.9609626928387
848 45.6628635739209
849 44.9811718789516
850 44.4813669038006
851 44.7060368156188
852 44.9441355742315
853 44.9935915727849
854 45.168497701921
855 45.4261848102345
856 45.4399862829212
857 46.0742194308028
858 45.914620812288
859 46.5992237363653
860 47.0749374290682
861 46.4007504305387
862 46.4457520536331
863 46.6472720336713
864 46.7385614279203
865 46.0910445524456
866 46.1217450996027
867 45.6508600071542
868 46.3504300752786
869 46.2746812973438
870 46.1236560903255
871 46.0868098688501
872 46.3818156490681
873 46.3879779897578
874 46.1915410936353
875 46.3058283035994
876 46.1109806132513
877 46.6559335737214
878 46.9914397590996
879 47.4032757613513
880 47.8513406802178
881 47.7847481602231
882 48.4131250496449
883 48.0444622013467
884 47.9117556934413
885 48.2345547414346
886 48.7447794800166
887 48.4042112698504
888 47.6573584689155
889 47.6259249012402
890 47.7697127943753
891 48.3464371801572
892 46.9650557708465
893 47.0736183300431
894 47.7293633421429
895 47.3732984763526
896 46.8730325967667
897 46.749889469477
898 46.5285193297763
899 46.1060021337649
900 45.6188474878695
901 45.5747721310819
902 46.1217776584657
903 46.6397331688839
904 46.6681701535139
905 47.2767816728477
906 47.4070294299432
907 47.3311856650759
908 47.529353968478
909 47.796115439563
910 47.9421163263669
911 47.4078700461494
912 47.1442731973256
913 46.7455333902942
914 45.9694297433201
915 45.2845294939189
916 46.1757078346823
917 46.5255095142366
918 46.9275971801212
919 47.3158007133743
920 46.874993714352
921 47.6015798203411
922 47.5693529235561
923 48.2048359695441
924 48.556590881877
925 49.5181772594606
926 48.4627703139718
927 48.0357587019013
928 48.5447401634314
929 49.0921934410215
930 49.3260318332665
931 49.3519051974749
932 49.0923618420385
933 48.818272714412
934 49.0563795406601
935 49.5130095062379
936 50.2271315697562
937 50.6675672936376
938 51.085707056059
939 50.570067909553
940 50.2517921111475
941 49.7863285222624
942 50.1444221365262
943 49.6194584498023
944 49.4006339522835
945 48.2894650989921
946 48.1326682308431
947 47.515107360506
948 46.0577263514174
949 45.4866864781317
950 45.5048698470361
951 46.2320748955322
952 46.7032137100654
953 46.8025261838245
954 47.9992673643079
955 48.148869778974
956 47.8410432108955
957 48.3405548421398
958 49.1801095895675
959 48.8259885343335
960 48.9073800166145
961 48.2132067045714
962 47.7599118389622
963 47.663573414331
964 46.5046779013034
965 46.6078310027118
966 46.7908900470447
967 46.5361553824412
968 47.1042040871236
969 48.6419449481425
970 49.2702824645592
971 50.180039305367
972 50.6587251478156
973 50.7481102162524
974 50.8412841415966
975 50.9539679134226
976 51.4499761747628
977 51.9303761763394
978 51.1638635274721
979 50.9997393875504
980 51.016157050935
981 50.3472560814969
982 50.4999183152345
983 51.4221980216289
984 51.8836461547566
985 52.2608976828545
986 51.7072800343829
987 52.1776118102171
988 51.7961191123812
989 50.8119197195933
990 50.7321389991876
991 51.3241118786039
992 51.1740563455788
993 50.3844549469809
994 50.2374510011387
995 49.7840099851956
996 50.0867138010828
997 49.2045272579598
998 49.5681700593727
999 49.8574645167754
1000 50.043707399525
1001 49.3756799290599
1002 49.0376999276604
1003 49.7264669854607
1004 49.2015536329375
1005 48.8998501823697
1006 48.2002463106676
1007 48.2107457873071
1008 48.1166807265776
1009 48.2694475862955
1010 47.8501764115037
1011 47.7669163438799
1012 48.1436844609158
1013 47.234510634875
1014 47.37920195474
1015 47.4337879088838
1016 48.0744743168485
1017 48.5254908693452
1018 48.6147947422987
1019 48.8185706635444
1020 49.2438482653431
1021 49.7961852153476
1022 49.031609838987
1023 50.0702307385063
1024 50.2427913412297
1025 50.1215758491237
1026 49.7049498852004
1027 49.2249048798251
1028 49.6486622224089
1029 49.4905183318908
1030 49.102029025693
1031 49.1555370412652
1032 49.94648516766
1033 49.7794498138727
1034 49.5078066399407
1035 49.8814412363438
1036 50.6857505322219
1037 50.5745234556517
1038 49.9121582252992
1039 49.3326376534096
1040 49.7569598220954
1041 49.407128488003
1042 49.2340212992379
1043 48.1991903301257
1044 48.4258197872742
1045 48.6401771240968
1046 48.3621261802732
1047 48.425568011007
1048 48.4520743787037
1049 48.7014999225974
1050 47.9492257985042
1051 47.9935263772435
1052 47.9466172414157
1053 48.337391950743
1054 48.1118321002065
1055 48.3107506008189
1056 48.4020453972053
1057 48.9868703100364
1058 49.383507848013
1059 49.4129022264113
1060 49.5833494189468
1061 49.7124731001594
1062 49.8543004492002
1063 50.1453557348015
1064 51.0307199573884
1065 50.1776796437384
1066 50.6450021367064
1067 50.2308370273424
1068 50.2509631011798
1069 50.4842970960746
1070 50.776114507929
1071 50.4031073694872
1072 51.0752871713502
1073 50.688067963214
1074 50.244254251843
1075 50.8490737032243
1076 50.390201325162
1077 50.8794116679721
1078 50.5544636643947
1079 50.742065707344
1080 50.9082607844142
1081 52.1678554498531
1082 51.5016608252891
1083 52.7185157685448
1084 52.7168190134618
1085 52.6401499240512
1086 52.2434496516946
1087 51.7123707753902
1088 51.7839012116994
1089 52.0845172294592
1090 51.4729988569321
1091 50.5911811287882
1092 50.4036231960944
1093 49.5591988297221
1094 49.5641556913472
1095 49.5104345655248
1096 49.9338746878309
1097 50.3299640434399
1098 50.7013428089312
1099 50.3016208048675
1100 50.5462767899783
1101 50.4260468880149
1102 50.0659676008612
1103 49.8027773961396
1104 49.7235657227937
1105 49.6607531770764
1106 49.6868144042577
1107 49.2013370569817
1108 48.8473727086573
1109 48.4971415348589
1110 49.2111841741654
1111 48.8130317922895
1112 48.6544259093744
1113 49.285797364154
1114 48.6092235407998
1115 48.7193728895863
1116 48.6852150579322
1117 48.7486977267862
1118 48.8559174801505
1119 48.8324278264262
1120 48.4112539306721
1121 48.6692554455761
1122 49.5595665420712
1123 48.8201255726355
1124 49.3516822570107
1125 49.3854518984529
1126 48.8261199006548
1127 48.6889882027756
1128 48.1707580153352
1129 48.6367149812509
1130 49.1667163683035
1131 48.92828033249
1132 48.9071364322626
1133 49.3543184784479
1134 49.3282669887676
1135 50.1454754749002
1136 50.1814680670253
1137 50.5939556847061
1138 50.590893856145
1139 49.8482289031622
1140 49.3895828820965
1141 49.3824565820758
1142 49.4820361523886
1143 49.4394376304908
1144 49.7154925512589
1145 49.5216111399141
1146 49.4711858520644
1147 49.2714034899618
1148 49.1290517543769
1149 49.8077407137411
1150 49.4811960717527
1151 49.872446505661
1152 49.1167186430922
1153 48.364722889697
1154 48.2509681556679
1155 47.3669631920464
1156 47.7482077878946
1157 48.0218368505234
1158 49.0609455614214
1159 48.9833699847416
1160 49.7601010684303
1161 50.4150595917105
1162 50.9959652720628
1163 52.2433226016244
1164 52.4263330637866
1165 52.3435075961438
1166 52.0928848946343
1167 52.0764179427427
1168 52.2609647339282
1169 52.4417077376815
1170 51.4881536584388
1171 51.9728337378337
1172 51.3554212226395
1173 52.1360957201792
1174 51.6812167208744
1175 51.8042246664453
1176 52.7624679234141
1177 52.6366438377991
1178 52.1085523354985
1179 51.6637539417912
1180 52.6900487329617
1181 51.7435668381096
1182 51.8926427911152
1183 51.3325704733826
1184 51.5978949842899
1185 52.6624118472321
1186 52.1811683841368
1187 52.2427417891103
1188 51.9712596468989
1189 52.0997626618244
1190 51.2399028587877
1191 51.1837166201322
1192 51.0946761691005
1193 50.0667892134185
1194 50.7145870728425
1195 50.6611381728812
1196 50.4813832482515
1197 50.4841559688365
1198 51.2438143872397
1199 51.5474993295274
1200 51.4522737324491
1201 51.4853471662257
1202 51.7998990855754
1203 52.2416343888271
1204 51.5993348670999
1205 50.6998770870046
1206 51.3406994155357
1207 51.549860690073
1208 51.3493275648748
1209 51.183345860164
1210 51.2853789580545
1211 51.1375280532727
1212 51.0371831447181
1213 50.800814701684
1214 50.6398984732508
1215 50.7529865109856
1216 49.6808747480244
1217 49.0381015360196
1218 48.684796522948
1219 48.4285705171167
1220 48.6003433544609
1221 48.0551966648132
1222 48.1182652042068
1223 47.9328920900856
1224 48.1056350358357
1225 48.5350648419644
1226 49.1433925263546
1227 49.670944588513
1228 49.8436182822838
1229 49.9564950271135
1230 49.795969971471
1231 50.8245269887715
1232 51.2801472159614
1233 52.0683461242116
1234 51.6145500577569
1235 50.5602970559369
1236 50.5090270229959
1237 50.4408150323684
1238 50.0403568228694
1239 50.0727063932603
1240 50.7311184137492
1241 50.4627058278128
1242 50.1665454437117
1243 49.5514779349651
1244 50.0276626869687
1245 50.7048131249238
1246 50.50690990713
1247 50.6794708050953
1248 50.7889009086131
1249 50.8103148094085
1250 50.7840191209943
1251 51.7505482558254
1252 51.5661086119493
1253 51.8942958680814
1254 51.5199175525727
1255 51.6042706395554
1256 51.3386861858315
1257 51.9264777854299
1258 51.6487388727486
1259 52.0372576908949
1260 51.768904840287
1261 51.0826490872559
1262 50.9041534352497
1263 51.2324660597509
1264 51.8789007790302
1265 51.6145783756999
1266 52.4759370184557
1267 52.0707726542624
1268 52.8045299308046
1269 52.9034676317498
1270 52.7586531460956
1271 52.0138709191191
1272 52.2312190534595
1273 52.0339826345688
1274 52.5500339189475
1275 52.4562840008721
1276 52.0134275345989
1277 51.8724566509015
1278 52.2141609050647
1279 52.2105511309809
1280 52.4328605800044
1281 52.6704747492801
1282 52.8161572229443
1283 53.1853034100383
1284 52.8290100061623
1285 53.7165789000585
1286 53.8000472302845
1287 53.892950198135
1288 53.0491435441606
1289 53.1187436281193
1290 53.355848316284
1291 53.3802837168109
1292 53.3570955554408
1293 52.6478857341081
1294 52.7388805888842
1295 52.1585786371619
1296 51.7469927759133
1297 51.3656293301808
1298 51.2860632661036
1299 50.7476150933872
1300 51.1493626659716
1301 51.4517178487746
1302 51.5411998292174
1303 51.9234339409936
1304 51.7991738068292
1305 51.7672790512199
1306 52.8920087557082
1307 53.0518837848682
1308 53.9163601351117
1309 54.1299330281165
1310 53.6566897007924
1311 53.8150916011645
1312 53.2346142245903
1313 53.7776513502247
1314 53.5797318945824
1315 53.2670505778685
1316 52.418552927537
1317 52.1743133170695
1318 51.54460154987
1319 52.0827825056215
1320 52.0886054088786
1321 52.1133906619865
1322 52.0310066818313
1323 50.6983176285523
1324 50.5585996309074
1325 50.5626610110215
1326 49.8542648715206
1327 49.956495082881
1328 49.9061771513434
1329 48.9237286771819
1330 48.6833928325723
1331 48.0072137449923
1332 48.3204378029229
1333 48.7662591756558
1334 48.8095815976094
1335 48.9595705469729
1336 49.8690629782064
1337 49.7314690198673
1338 50.2940528809466
1339 50.4226209902419
1340 50.4781937785332
1341 50.8124176038326
1342 50.8503133204753
1343 50.8071772487706
1344 51.1128799159909
1345 51.1188647432522
1346 50.9018976848989
1347 51.2628432861558
1348 51.5448438066018
1349 52.2729710584892
1350 52.412670451589
1351 52.2596143013448
1352 52.6331575032087
1353 52.8460383648933
1354 52.6779538842574
1355 52.9886557689464
1356 53.1182975897883
1357 53.460999906834
1358 52.8289535629603
1359 52.3989154529796
1360 52.0437304706948
1361 52.6794747700648
1362 52.56835827323
1363 52.6199065275139
1364 52.0566050306007
1365 52.489137917702
1366 53.4783362799246
1367 52.925923583193
1368 52.8750766032417
1369 52.3435556161971
1370 52.5708337054776
1371 52.4239640401369
1372 53.5880777354567
1373 54.0519867498665
1374 55.1063053889079
1375 54.5497920357606
1376 53.7826285565297
1377 53.8702956357
1378 53.8719913550826
1379 54.6933052298217
1380 54.715867218622
1381 55.3699673798648
1382 54.5239005378145
1383 54.5018448450275
1384 54.0187891936388
1385 54.7889871338129
1386 54.8549094964377
1387 54.997271162747
1388 54.9582530907284
1389 54.7367990139835
1390 54.4964337145751
1391 53.2313216807409
1392 52.5368706814186
1393 51.4888642721602
1394 51.0769553411869
1395 50.2894803480656
1396 50.6674115540236
1397 51.0317587224278
1398 50.8610444123239
1399 50.9772583894705
1400 51.3557173123419
1401 51.6783350315597
1402 51.9082368588094
1403 53.6081112275844
1404 53.6512089780841
1405 54.712197459362
1406 54.0211148588863
1407 54.1818736413693
1408 54.3924358654919
1409 54.2402979761392
1410 54.4183888528715
1411 54.1260822239929
1412 54.4714742971193
1413 53.4114205666019
1414 54.2621214297123
1415 53.6939727867764
1416 53.5695693242478
1417 52.4527850141069
1418 53.0296388789984
1419 53.4279889167325
1420 53.956181419992
1421 54.4412954734055
1422 54.3116718922466
1423 54.13843986138
1424 53.5170530525702
1425 52.9676100504283
1426 53.3851720056473
1427 54.5671427252113
1428 54.2667898005456
1429 53.4020404574792
1430 52.819789428985
1431 52.9829295209368
1432 52.3033358798646
1433 51.8925754281776
1434 51.7620353001809
1435 52.7077953385214
1436 52.1398619609163
1437 51.4225806973867
1438 51.5080752172874
1439 52.1598720849314
1440 51.5493924775638
1441 51.6679036457494
1442 52.278933035708
1443 52.7139625054575
1444 52.8537859353759
1445 51.9383626781069
1446 52.2073975250401
1447 52.0389421122818
1448 51.7060903257559
1449 51.536587185512
1450 51.7220467808321
1451 50.8663793801872
1452 50.8848097095601
1453 50.9642215591548
1454 51.2257405299793
1455 51.079369595641
1456 51.0333297680167
1457 51.447798901856
1458 52.2708246803884
1459 52.6056493447382
1460 53.1196282930029
1461 53.4421657579718
1462 53.5392661753705
1463 53.7651107704318
1464 54.1256477718812
1465 54.2997508042498
1466 54.9988250463005
1467 54.621835217485
1468 53.4995148204896
1469 53.4429995693103
1470 53.3353162610613
1471 52.8614489596712
1472 52.8055863392626
1473 52.3666168565859
1474 51.1706164732358
1475 51.0034093972005
1476 50.7278459350185
1477 50.9155096815298
1478 51.0467012203999
1479 50.6485922021346
1480 50.8095043387867
1481 51.5659154507381
1482 51.481643990054
1483 52.1526936627644
1484 53.4643368233309
1485 53.9619738675114
1486 53.4521136009527
1487 53.731975719053
1488 53.7029154760446
1489 54.0897171973185
1490 53.4477467096735
1491 53.2743344432286
1492 53.0778444733163
1493 52.4051173416176
1494 50.9602910024856
1495 51.1274323236546
1496 51.6864248287423
1497 52.0553316810259
1498 52.4876713270096
1499 51.8162269833182
1500 52.1695887818824
1501 52.0908082494818
1502 52.2574414642991
1503 52.3516498006166
1504 53.6098738802481
1505 53.4753961387064
1506 52.8735010203426
1507 53.6233338566128
1508 53.2288981820411
1509 54.4793720163611
1510 54.7361004495832
1511 55.0866466921583
1512 55.5643713998981
1513 55.9447576680752
1514 55.743200608101
1515 55.8815272241377
1516 55.9441981550081
1517 54.35879715414
1518 54.220698369187
1519 53.323625791867
1520 52.9582889598601
1521 53.081341276806
1522 53.0121204000844
1523 52.3798304040812
1524 52.571809158832
1525 52.3523464305112
1526 52.6071565456221
1527 52.7399511152957
1528 52.6810994453394
1529 52.6266650676346
1530 52.7095314801808
1531 52.2087666402338
1532 52.4876669498823
1533 52.9402567680611
1534 52.3888297478083
1535 51.6097540206439
1536 51.1909919225334
1537 50.8357529361938
1538 51.1639930626229
1539 51.322804997038
1540 50.7379545293905
1541 51.298274349343
1542 50.4402375033446
1543 50.5530857255882
1544 50.6586035821872
1545 51.3866115601966
1546 52.6359014575609
1547 52.5044147762293
1548 51.9619289273967
1549 51.160522235746
1550 51.2295535092339
1551 50.7444994805915
1552 50.7199263500959
1553 50.4787063823488
1554 50.045692107998
1555 49.9565271696681
1556 48.8609875411944
1557 49.3319548629049
1558 49.595375921933
1559 50.4885463168779
1560 50.3850795163525
1561 50.687002742647
1562 50.741295961634
1563 50.040717548023
1564 50.9210002089683
1565 51.1298140879527
1566 51.7199052409721
1567 51.4381193871149
1568 52.4052682226298
1569 52.1729755164883
1570 52.4474609416245
1571 52.0429479267501
1572 51.9332142508345
1573 52.8117177296667
1574 52.5306498467157
1575 52.8902307591772
1576 52.8481492789261
1577 52.3432210334333
1578 51.7020761666983
1579 52.4038455003737
1580 52.6219597120702
1581 52.5904251920545
1582 52.2704419650214
1583 52.3757673228838
1584 52.3734224670421
1585 51.0517157855097
1586 50.94704640097
1587 51.9391555811645
1588 51.7108469796307
1589 50.4650957885152
1590 50.0927128561221
1591 50.0594640654989
1592 50.3517785269927
1593 50.3146705832164
1594 50.0487637705925
1595 51.2825338657107
1596 50.9316028304
1597 51.602552247589
1598 51.853880293925
1599 52.3582365854274
1600 52.4462965867365
1601 53.8408726083092
1602 54.485769309391
1603 54.5018304484321
1604 54.781460400376
1605 54.4721675344437
1606 54.5727426416006
1607 53.7451640141976
1608 53.9924386276215
1609 54.9274306247777
1610 55.5934176560614
1611 54.268571600665
1612 54.6689777302332
1613 54.7167332898823
1614 54.8598233254943
1615 54.6036912996226
1616 54.9231351038929
1617 55.8364810678424
1618 55.7339010708318
1619 55.3320120766925
1620 55.4803224681406
1621 55.586605203548
1622 54.859185615652
1623 54.1584589964016
1624 53.892120623943
1625 54.2098058560284
1626 53.8620476644952
1627 52.7958600737332
1628 52.1793342389819
1629 51.9135975094252
1630 51.5992456524655
1631 52.2091928014093
1632 52.1156937499879
1633 52.7379942709411
1634 52.315952790485
1635 52.3513574360838
1636 52.7027291837113
1637 53.0669514776902
1638 53.6858009867508
1639 53.9666208944991
1640 53.7680865264758
1641 52.8455493293662
1642 53.3140520028848
1643 53.3478449370988
1644 53.8837607137566
1645 53.81209855066
1646 53.702013745973
1647 54.3202716195783
1648 54.4150747058706
1649 54.5695654000905
1650 54.7272014855721
1651 55.3652033789028
1652 55.8178286943874
1653 55.8533580363894
1654 56.1304612630884
1655 56.7780366436825
1656 56.8247249427964
1657 55.8323908732209
1658 56.0414748686487
1659 55.9157153646053
1660 56.4119695608657
1661 56.3241021429054
1662 55.1188990932367
1663 55.5778491533209
1664 55.64927394288
1665 55.2385295552375
1666 55.4722589251036
1667 56.5754939038669
1668 56.4709808286136
1669 56.6029840148628
1670 56.5042822991239
1671 56.2751869699818
1672 56.5762413716579
1673 56.0845768838136
1674 55.3200373185249
1675 55.3330744650395
1676 55.5267195599725
1677 54.7914947213532
1678 54.3611061226382
1679 53.3310351077507
1680 53.3560261836567
1681 54.0327098910626
1682 54.0708613783132
1683 54.8268725358925
1684 54.6695019995674
1685 54.7625697875373
1686 54.0652729170213
1687 54.5335702795285
1688 54.1468218544827
1689 54.277222699128
1690 54.4091665730206
1691 54.2025272072125
1692 54.1715131686845
1693 52.2169406388623
1694 52.20573595313
1695 51.9970246962223
1696 51.9790486282983
1697 51.7709031990376
1698 52.2322364862796
1699 52.5189137408426
1700 52.0950014056389
1701 52.5369486728194
1702 52.8435353567979
1703 54.1131508543227
1704 54.0536269571513
1705 54.1666852112463
1706 54.364482384133
1707 54.0286434196968
1708 53.6787981367874
1709 54.4455815346497
1710 54.4170006766365
1711 53.3810847709639
1712 53.4925479667755
1713 53.1446964123391
1714 53.3959816755693
1715 53.5993804260897
1716 53.969332744105
1717 54.1009887288156
1718 54.7609584404852
1719 54.5967972742492
1720 54.4887213721439
1721 55.6490168929859
1722 55.3398681813996
1723 55.1799352891788
1724 55.8694095173725
1725 54.871034723708
1726 54.5500439506814
1727 54.7755439051978
1728 54.5279168000341
1729 54.6895318576994
1730 54.5637193092858
1731 53.263320655108
1732 53.2579953170742
1733 53.2259616572899
1734 53.2432875272308
1735 53.7771758411365
1736 53.6334604676081
1737 53.3797800166539
1738 53.3499882584799
1739 52.6726721376725
1740 52.7895559693065
1741 53.3473358594025
1742 53.584834986189
1743 53.4260958824867
1744 52.9208147846693
1745 52.5739383988178
1746 52.4328563063392
1747 52.1989089413205
1748 51.7099850344719
1749 52.0520138082108
1750 51.9468882759303
1751 50.8907284758989
1752 51.2818496025603
1753 51.6364944908675
1754 51.269038348457
1755 51.1523458670576
1756 50.7544465925644
1757 50.9949354950963
1758 51.8232893842455
1759 51.9195044858687
1760 51.4502431617473
1761 52.7032306387613
1762 52.0707698649644
1763 51.7747126085265
1764 52.1322336126085
1765 52.4503800004175
1766 53.3504815948814
1767 53.2902235856979
1768 52.942404977749
1769 52.8588819863049
1770 54.513212834648
1771 53.4360130306179
1772 53.3072786895125
1773 54.1398177077316
1774 54.6084733466375
1775 54.7641616628396
1776 53.8209214638414
1777 53.945524147668
1778 54.7330446427653
1779 54.2357649772761
1780 53.0181381120808
1781 52.9358915778796
1782 52.8942588196305
1783 52.4594621905954
1784 51.493615225453
1785 50.8777146702885
1786 50.9077494438263
1787 50.1138663046297
1788 49.7055692645589
1789 49.9610077020693
1790 49.9859521368727
1791 51.2817984000568
1792 51.7485106327089
1793 51.8013039978338
1794 52.5137618403672
1795 52.8573133637194
1796 53.3846529661086
1797 53.4620862001517
1798 52.7801324475265
1799 52.4660303334922
1800 52.2625514715203
1801 52.4901731561761
1802 52.3392992211483
1803 53.0230057363547
1804 52.5608592360746
1805 52.6825878825352
1806 53.0748562646032
1807 53.9953950521517
1808 54.2056442998554
1809 54.7657709678348
1810 54.8214183538338
1811 54.0976772698978
1812 54.0045401099969
1813 52.7245575530216
1814 52.6482981868303
1815 52.8452725514895
1816 52.5269505981908
1817 51.9896895269739
1818 52.5889861195806
1819 52.5141580091839
1820 53.1299392737948
1821 53.5809219940542
1822 52.9688711336872
1823 54.0490417952809
1824 54.8939158676101
1825 55.2819832269138
1826 54.9982361460135
1827 54.6010912849916
1828 54.0334753454868
1829 53.9425073325842
1830 53.9700894229145
1831 54.2008712617708
1832 54.6968807596106
1833 54.0839031683307
1834 53.1245521465948
1835 53.229731348903
1836 53.5973593890509
1837 53.6429541937109
1838 53.6462393972952
1839 53.3024066432466
1840 52.4053820161403
1841 51.1629893322429
1842 51.2765940497261
1843 51.5767195149761
1844 51.8494051586456
1845 50.6880832969174
1846 50.0048307808529
1847 51.5987042282486
1848 51.2972617331162
1849 51.3312365164845
1850 51.5971725054561
1851 51.353559716732
1852 51.2850178783825
1853 51.3715349883144
1854 51.230253519015
1855 51.9711444472236
1856 51.8183703057767
1857 50.6005589894173
1858 51.3539676874181
1859 51.6515673671731
1860 52.390766682181
1861 53.8973136722871
1862 54.4332900439337
1863 54.9968629378918
1864 55.6683260605475
1865 55.7518800626443
1866 57.1588602759621
};

\nextgroupplot[
tick align=outside,
tick pos=left,
title={fc\_layer\_1},
x grid style={darkgrey176},
xmin=-93.3, xmax=1959.3,
xtick style={color=black},
y grid style={darkgrey176},
ymin=18.0808203276397, ymax=81.2113842173934,
ytick style={color=black}
]
\addplot [semithick, steelblue31119180]
table {%
0 35.6779885057938
1 34.8099490375307
2 34.4157658200297
3 34.5143912403513
4 34.7403636062139
5 35.3423280287817
6 35.5894525410346
7 35.5788450444525
8 36.1899296385041
9 35.5576968687566
10 34.4834385219878
11 34.4947902446022
12 34.1878318070064
13 34.3599779598545
14 34.3318716207236
15 34.0214840951296
16 34.0644993355972
17 33.4987690605453
18 32.860580234709
19 32.799447779085
20 32.9920650990359
21 33.2902847957658
22 33.7671224391336
23 33.855099960412
24 34.5826934231677
25 35.1993451428702
26 35.1249519044317
27 35.475829303438
28 35.2805455895986
29 36.1505599059534
30 36.3613725424257
31 36.0853075763837
32 35.4652012638969
33 35.3428340119767
34 35.1019744055295
35 34.802966183072
36 35.1009757672006
37 35.0409993060142
38 35.5933497341972
39 34.5636400773765
40 34.4469757906985
41 34.1090569959661
42 34.1731328641319
43 33.7768568232299
44 33.5959905301776
45 33.1856039217149
46 33.4361058019006
47 32.9611836506903
48 32.2427701183237
49 32.1854383385446
50 32.1986349014658
51 32.3682930840066
52 32.0691238876988
53 31.6859159878757
54 31.3988918588026
55 31.7119473696003
56 31.1777998373426
57 31.3968646384662
58 31.3146484247994
59 32.3021539843931
60 31.7128024745935
61 31.585351218733
62 31.8016248503473
63 32.1076860714426
64 31.9280791251641
65 31.1118842917817
66 30.6882098321271
67 30.7535932086659
68 30.9343049756711
69 30.4314152854143
70 30.5273754939287
71 30.7383907496086
72 30.6814448011626
73 30.535215717003
74 30.9645406732973
75 31.5099370102843
76 32.1169494497578
77 31.5143810093059
78 31.55520238013
79 31.7659489847544
80 31.642719382439
81 31.2828119113862
82 30.7498240023502
83 30.5811227393138
84 29.8442714423931
85 29.8872466855135
86 29.8781749050516
87 30.9356918889979
88 31.1171055566384
89 30.3569329795274
90 31.331484843201
91 31.5216466084588
92 31.7122234752878
93 32.0325066408467
94 32.6911321496311
95 32.9387319444523
96 32.5003896503122
97 31.9029298760207
98 32.1918404487222
99 32.2689374578497
100 31.5269523478988
101 31.8313657630263
102 32.2796058407166
103 31.9827009206064
104 31.8003774973149
105 31.6852101127225
106 31.9113068211378
107 32.2568607131856
108 31.7575749226345
109 32.3783400864023
110 32.8596290367539
111 32.944824479256
112 33.2709049210657
113 33.858043792366
114 33.9050806398819
115 34.1177360864995
116 34.4093336636464
117 33.906200203492
118 34.2900471136741
119 33.6507187757528
120 33.3945652772215
121 33.2750576474564
122 32.9638293651006
123 32.7277741539061
124 32.830280698107
125 32.4359784080685
126 32.1907001226072
127 32.2556789321514
128 32.024387699493
129 32.104209032081
130 32.2762007721553
131 32.1823281645594
132 31.9275592234845
133 31.5902633578648
134 32.1791550528938
135 32.3007869370362
136 32.4697103199306
137 32.7180680914868
138 33.185813828287
139 33.2333995506172
140 33.9781766676435
141 34.0669620663207
142 34.4896439403606
143 35.7580393424669
144 35.217306143133
145 35.4728820598549
146 35.7854757070213
147 36.2836399002663
148 35.5984630053423
149 36.5608526423806
150 35.610873128458
151 35.9932021810626
152 35.9314919333809
153 35.0708489267254
154 34.7632795141361
155 34.3529236702077
156 33.7000091810527
157 33.563758217864
158 33.8303975625668
159 33.7385412337175
160 34.6205036658972
161 34.9231100635888
162 35.7915540243956
163 36.0420350036217
164 36.6665221698104
165 37.5495091214258
166 37.6049504118058
167 37.3464364666262
168 37.3329448846113
169 37.293408488087
170 36.4690883283155
171 36.4501797092882
172 35.9079848542213
173 35.8572428313332
174 35.8183613422151
175 36.2109364100108
176 36.3217364186941
177 35.9615629627701
178 36.1618521278152
179 35.5934779403935
180 35.8455613504716
181 35.5967940675387
182 35.4426926387864
183 35.8242701294993
184 35.7498631270293
185 34.9781926034561
186 35.2235844446677
187 36.1884646006182
188 36.3532708722756
189 36.6967692147704
190 37.130011748975
191 36.9935094493745
192 36.6143860668776
193 37.0088707699823
194 36.6629065408434
195 36.4347130348928
196 36.2720830528663
197 35.7198764447641
198 35.2606992629175
199 35.6063807760611
200 35.166236147781
201 35.0219450295146
202 35.5260220515109
203 35.6160568705365
204 36.0723530960512
205 36.5761040533756
206 37.1315499791871
207 37.3437693062277
208 37.5413274764993
209 37.3938865357023
210 37.276669879427
211 36.9091234059424
212 36.8423164347766
213 35.8785638628315
214 35.5385951149811
215 35.0157019644279
216 34.62697129856
217 34.3494687628728
218 34.603674793317
219 34.7691834603082
220 34.8956225192625
221 35.3367747545009
222 35.4548876074403
223 36.2372728919955
224 36.8119971508592
225 36.9745423484007
226 37.0108407284836
227 37.8846327660176
228 37.870305988533
229 37.1640666515245
230 37.5482459913778
231 37.6275635879132
232 37.8731544007758
233 37.3585512914122
234 36.8444532355209
235 37.5981862805624
236 37.2252355019449
237 36.183959441861
238 35.8962773375675
239 36.0304249496136
240 36.2689631754098
241 36.4739983694329
242 35.8142316559305
243 35.9851433735214
244 36.5047398996249
245 35.8413611403199
246 36.3586571898658
247 36.6550891653811
248 36.7657658202561
249 37.4255364457771
250 36.9825233869731
251 37.2836899294841
252 38.0196715555905
253 38.5266995660148
254 38.6229053253561
255 38.1048129801167
256 37.9643340099267
257 37.9756106107627
258 37.902842647321
259 37.263596696852
260 37.4717283455562
261 36.7185023528009
262 36.0680888653826
263 35.277809963327
264 35.2476820306774
265 35.1548512808868
266 35.0232895914347
267 35.327004133579
268 36.0238007157454
269 36.1260825607573
270 36.0422231330523
271 36.5692517939655
272 37.3742398258871
273 38.6321825532781
274 38.7267158256517
275 38.9523223832782
276 38.9578235053348
277 39.1623485989589
278 38.7057451817825
279 38.7569784218274
280 38.5508591288926
281 37.9553627921787
282 37.4001818979647
283 36.2128515636494
284 35.5822808756841
285 35.9566700463098
286 36.3197335821431
287 35.8359052905239
288 35.8945036079453
289 35.4834563502361
290 35.7388639654054
291 35.9132891940973
292 35.92301320063
293 35.7849974086213
294 36.3402622524115
295 36.8596171981029
296 36.3766878274988
297 37.2301671226338
298 37.2980222699777
299 38.0518040889719
300 38.4847156918483
301 39.0929881729236
302 39.8953719338895
303 40.2475927487807
304 39.9186538614874
305 40.518352673925
306 40.6788547426961
307 39.8199031024497
308 40.0107810807924
309 40.0723168152187
310 39.2689401427901
311 39.0962028300865
312 38.4965144945516
313 38.1517810666589
314 38.405292183535
315 37.5895089364582
316 38.2146192299576
317 38.1421611741148
318 38.8151217359552
319 38.7827611172449
320 39.3701446663895
321 39.1434221349147
322 38.9281747730495
323 38.8102258691503
324 38.1138245370237
325 37.557707890394
326 36.8549363745005
327 36.4155821616993
328 35.6145997074785
329 35.0591692825318
330 34.3813428392189
331 34.435399226027
332 34.416953710937
333 34.8810723275046
334 35.1947366160286
335 36.0305184836292
336 36.503729094147
337 37.173946018807
338 36.9947225151389
339 37.4720993774912
340 37.884001374663
341 38.702314088222
342 38.8982035929795
343 39.4624487402357
344 39.7682480403947
345 38.9823946742666
346 39.44687728989
347 40.8158184447294
348 41.0506448683797
349 40.8527959061773
350 41.1825765125121
351 40.7995926703219
352 40.8931859857479
353 41.2624190440842
354 40.8075111722005
355 41.0120081707114
356 40.70185606852
357 39.2755440881846
358 38.9693360374495
359 38.7355670438552
360 37.6939441380098
361 37.067450762278
362 37.7805515439143
363 36.2892326806281
364 36.5823025663297
365 36.8971129132038
366 36.6333760422313
367 37.2826820234954
368 37.2201007494214
369 37.4897947212781
370 38.0651595206529
371 37.939157518622
372 37.3026011089073
373 37.6160329535765
374 37.9802052153401
375 38.0623719508954
376 37.9258117628284
377 36.7202131877259
378 37.0790351831174
379 37.4302726340344
380 37.7026028143015
381 38.4515706278326
382 37.715625041179
383 37.5041316809868
384 37.0256627673178
385 36.6458047135828
386 36.7420425137763
387 37.5032839913113
388 37.3162016258719
389 37.3067902320922
390 37.8807872150669
391 37.5296029088303
392 37.9459301689483
393 38.4190071786533
394 38.3278978368029
395 38.1996447309464
396 38.4987845187882
397 39.1481297841524
398 39.1726533044586
399 38.9761702424244
400 38.0633151910634
401 38.0010873301818
402 38.0459627975332
403 38.5741790775749
404 38.9014450443214
405 39.6741670850698
406 40.1921712912014
407 38.9851342292492
408 39.3767266347994
409 39.8244006195264
410 39.7710436166565
411 39.4315206502107
412 39.2392311359336
413 38.8372933685004
414 38.4109390610857
415 37.9738089519012
416 36.7625178771725
417 37.2828866353769
418 36.9948453695783
419 36.2223796970247
420 36.4694765262515
421 37.1196781066333
422 37.6751299597315
423 37.8782963538906
424 39.1121502583181
425 39.9702323434292
426 40.6101362539473
427 40.6392670742639
428 40.9873731339181
429 41.4579977465887
430 41.9137396646575
431 41.4762769586048
432 41.1902928568883
433 41.5871426479242
434 40.9198765315222
435 40.2874716919477
436 39.7412642773761
437 40.0000612032085
438 39.9409418632363
439 39.6644787835539
440 38.7252109990676
441 38.4621736310344
442 38.7622730079389
443 37.8369422428479
444 37.6933960142075
445 37.3908708136594
446 37.5468608229519
447 37.350387888026
448 36.9779298626444
449 36.6420398318446
450 37.3963979424423
451 38.2912478193506
452 37.8844246060181
453 38.2825962418336
454 38.507768721441
455 38.4157475418669
456 38.2600445835906
457 38.5218893370579
458 38.385266384421
459 39.2955739697037
460 38.9294776559589
461 38.8050405342314
462 38.6926747306819
463 38.4157072108094
464 38.7970825322913
465 38.5893107099752
466 38.9544516700574
467 38.1526673085819
468 38.7193656705431
469 38.683048959465
470 39.225178340396
471 39.0091563376006
472 39.4470506225586
473 39.1280322870468
474 38.7243407642997
475 38.883407769576
476 39.5352068249033
477 40.2119424825826
478 40.1849995016236
479 39.6069527977352
480 39.0607535070799
481 39.0202995354756
482 39.4512490217105
483 40.1418501269495
484 40.750104002935
485 40.794243874651
486 40.426690961682
487 40.5363873166398
488 40.1851489012727
489 40.3868548577273
490 41.3545290358372
491 41.1168111618129
492 41.1511396573825
493 41.5356067969055
494 41.6146462659765
495 41.7151937614566
496 41.2855344349854
497 40.9409781221474
498 42.0459477098848
499 42.7548885964502
500 41.6551168018666
501 41.7463459918888
502 41.0194004995416
503 40.9212078860495
504 40.0247117756923
505 40.8229665993572
506 40.5124355364791
507 40.249731079458
508 39.332247797501
509 38.7215505195232
510 39.1210326543971
511 39.3230194400959
512 40.7523111593239
513 40.4300663503113
514 40.6564895406111
515 40.2513031009932
516 40.8358054249371
517 41.4181015384142
518 42.3800995613757
519 42.0661782848376
520 42.2573187371252
521 42.7561667496345
522 41.3006109094171
523 42.2989889811833
524 42.9527676473886
525 42.751913138112
526 43.1841194114511
527 42.6293843744028
528 41.5517109015691
529 41.9783183955944
530 43.4914956119165
531 43.9850819296122
532 44.0951517628869
533 42.6684550760097
534 42.4669105929429
535 43.3033162403509
536 44.783263078682
537 45.4355850212607
538 46.6228005822268
539 46.5695132754782
540 45.6547006603554
541 45.7823739448936
542 46.2805661047268
543 47.220205267614
544 46.576365717075
545 46.0515589212156
546 44.5826839996869
547 44.1742255407562
548 43.6770995527955
549 44.0732577822211
550 43.1700253424786
551 42.3261591528433
552 42.1595260734571
553 41.6399311561798
554 41.5203842328227
555 41.0722553290566
556 40.6770892649249
557 40.8528688440509
558 40.0813068034323
559 39.3774747136948
560 39.2018259110059
561 39.8648274348969
562 39.5624341382596
563 40.2790792480433
564 40.6341438058801
565 41.7942009463633
566 41.5194130035112
567 40.9117413747533
568 41.8861522368174
569 42.6948943040521
570 43.1934879699971
571 42.407501258171
572 42.7165691022378
573 42.3308201720739
574 42.7239283065275
575 42.4070043303522
576 42.7527734405551
577 42.9352709650559
578 41.6031417126027
579 40.5272884976701
580 41.2524007126139
581 40.8729897708419
582 40.7230319952553
583 40.3189182778465
584 39.7606994529605
585 38.8148402530096
586 38.8316205285381
587 39.0910674482661
588 40.5599567073057
589 40.4482902694999
590 40.1065148924338
591 40.6320319377592
592 40.6209783803986
593 40.7354718993733
594 40.4528415310784
595 41.7034212575541
596 41.9669044685817
597 41.6660720927646
598 41.1284194090358
599 41.26340660565
600 41.049976208436
601 40.6575029970373
602 40.6494194449967
603 40.4400040136535
604 40.5228326587997
605 39.8804746690063
606 39.7911030672101
607 40.6821480589174
608 40.6406377654732
609 40.6037555303585
610 40.5292232332913
611 40.8805523723268
612 41.167232438642
613 41.7927388148729
614 42.5032627080286
615 42.5948682500328
616 42.7677653080892
617 42.3598699835261
618 41.9308248917827
619 42.4468599733719
620 41.7293056547299
621 41.3681582771333
622 41.3416390285633
623 41.4202449769563
624 41.8344922962947
625 41.0371350298137
626 39.7611026108744
627 40.0759827039229
628 40.3437343421219
629 40.9267690833584
630 41.39068655234
631 41.6716670053425
632 41.8175000602621
633 41.0707550473634
634 40.6365026357407
635 41.3153163373807
636 41.833980233916
637 41.8872239631382
638 41.850935001341
639 41.8079050407381
640 42.3112775502876
641 42.5825363292168
642 42.2078912116551
643 42.9521292970049
644 43.0744012760775
645 42.8984150021845
646 43.6780325095935
647 43.0342136108024
648 43.6249095396493
649 43.096111946566
650 42.8002427817399
651 42.4862229862001
652 42.5528929271495
653 41.7330618581496
654 41.5809147381943
655 41.2922901635605
656 40.9168725322741
657 41.0225938488324
658 41.1781725811293
659 41.0205280189957
660 41.8426106921118
661 41.5684613914946
662 41.3005403406799
663 41.7175849501206
664 42.1880598327062
665 42.1541928229579
666 42.5027334322881
667 43.1318246436783
668 42.6932108021854
669 43.0224877443207
670 41.5654406376298
671 42.6426250549902
672 42.6613592147143
673 42.26394773393
674 42.4210572057923
675 42.7626562605824
676 41.7587551999001
677 41.1519877664601
678 40.648146023066
679 39.8173405357688
680 40.42580111802
681 38.9335865128587
682 38.9942665293656
683 38.7028690596059
684 37.4447875396769
685 37.5523879773302
686 38.111386334883
687 37.8523337076897
688 38.8253915253901
689 39.9943522261202
690 39.5864420687531
691 40.413913370351
692 40.0024253190987
693 41.2885770774207
694 41.1492203217373
695 41.0251857922851
696 40.8725763994346
697 41.5539211291206
698 40.6295272898513
699 39.8566630151029
700 41.2769359736669
701 41.416720156029
702 42.237899726418
703 41.2725993772254
704 41.5191412475132
705 42.2884215196177
706 42.6929689257138
707 43.2790876557895
708 43.541343089453
709 44.1348886470865
710 43.3624791429436
711 43.2583815086136
712 42.9583526221667
713 43.3626041424595
714 43.6753032536921
715 43.02090120058
716 43.0632309354055
717 43.1493784564216
718 42.8300310286213
719 42.12958874653
720 41.3479028794629
721 41.6720626353947
722 42.8777766354588
723 42.4684839307331
724 43.0269966961646
725 43.0635487284199
726 43.4623949290423
727 42.6468545186231
728 43.6048085257485
729 43.9326339266345
730 44.5185404841903
731 44.4953112663917
732 43.2140282139138
733 43.9223374830892
734 42.8077906497226
735 42.9636179536781
736 42.0701614259817
737 43.3099834565713
738 42.5226160673421
739 42.6078417885671
740 42.4074166898107
741 41.7182731962695
742 41.5166487666254
743 42.1627137026691
744 42.1560821621992
745 41.8209325076785
746 41.9819005504312
747 40.8614304729882
748 41.0993656536781
749 40.6392719756905
750 41.4441617521963
751 41.9218036523176
752 41.8735229563348
753 40.702805156115
754 41.377109573386
755 41.5344695606076
756 41.7183732667859
757 41.348004214421
758 41.2620922373567
759 41.48393003022
760 40.4775117662121
761 40.2350361869793
762 40.5511474213299
763 41.4345914503753
764 41.1494388215239
765 41.2854388448267
766 42.1066400987365
767 42.2683859810988
768 42.9848511673382
769 42.9218608543953
770 42.6919194979262
771 42.6586521072611
772 43.1753002340123
773 41.8113211742533
774 41.6934533487159
775 41.3460433878391
776 40.4816391406052
777 41.0377307052595
778 40.1756199346378
779 40.1928316580214
780 40.5307888712764
781 40.0559343694664
782 40.3282846999999
783 40.5818616814322
784 41.6835681647685
785 42.7384952745273
786 42.5690574392758
787 41.9183338792715
788 43.4792474139361
789 44.4739170675409
790 45.3885024781946
791 47.637023558556
792 47.4191801629767
793 47.6797688311955
794 46.8781067738393
795 46.1611337450776
796 46.8194556581282
797 47.1459575777747
798 46.6572423352257
799 46.0359559430226
800 45.9171954589717
801 43.7498524073477
802 45.0375881748661
803 45.1255287185127
804 44.660936748739
805 44.682056211122
806 44.9942238036774
807 44.8162845209726
808 43.7319428986555
809 42.8431915251878
810 41.742826998478
811 42.6827166745424
812 40.6179512341975
813 40.8869352861142
814 40.4898662544995
815 40.696543740664
816 40.0331770328321
817 39.4575994417131
818 39.6022491171076
819 40.4883519560918
820 41.0166517064067
821 40.6460931825042
822 40.6665744238558
823 41.3181833014987
824 42.6768007473145
825 42.1628623366618
826 42.158138212677
827 42.5111512678584
828 43.3188965081729
829 43.9217191592597
830 43.754976028916
831 44.1639607361723
832 45.0664715099685
833 44.2058468424042
834 44.0980380028453
835 44.7108326564617
836 45.2149993729201
837 45.2175352778085
838 45.0262506851963
839 44.6863822574419
840 45.3602760714976
841 45.7334300439517
842 45.8101379587127
843 45.895577729217
844 45.462046400967
845 44.9141358194114
846 43.9769942766736
847 45.150746646974
848 44.2529419699255
849 43.8172299548176
850 43.2646977798026
851 42.6138574963167
852 41.917948894684
853 42.4018028057797
854 42.362843873128
855 42.8137343741719
856 42.9219192169199
857 43.1522674101847
858 44.1542760360195
859 44.358878262076
860 44.6447816685065
861 44.2989233492005
862 44.943334065987
863 45.3104025239102
864 46.2042707750628
865 45.7243362012361
866 46.2190841927311
867 45.757944329712
868 46.3328149385637
869 46.0277669181492
870 45.7742323301248
871 45.8248392799419
872 45.6361582287555
873 44.8050622919694
874 44.1341740906683
875 43.7772530927892
876 43.4060814454976
877 43.3372983571574
878 42.7294408860536
879 43.278555553509
880 44.8009837793836
881 45.555853628563
882 46.4544817661221
883 46.5134136944359
884 46.6136117726458
885 46.9836180196468
886 47.3308098904804
887 46.6807205931915
888 46.1857031696136
889 46.4994240331171
890 44.6688273171213
891 45.1101769148446
892 43.7776995893141
893 44.2427444498215
894 43.8890593919178
895 43.9563709976446
896 43.4136637830964
897 43.0757024143721
898 42.712633082096
899 42.5198096568858
900 43.5070990144888
901 41.8201944535764
902 41.9074718440284
903 41.7023090078349
904 41.8719317636744
905 42.2448936178342
906 42.0627976838208
907 42.7919572532841
908 42.8845528345339
909 43.2323200766587
910 42.8005193590657
911 43.5191111705866
912 43.3678468546964
913 43.3086742380467
914 43.2442832563595
915 42.8464748592782
916 44.2315005118151
917 43.876377822494
918 44.1071769291553
919 43.5372119065323
920 43.9820522955596
921 44.3433539635686
922 44.6201606667744
923 44.4382231598397
924 45.0442543911502
925 45.3038270068856
926 44.5717232888292
927 44.7512197464578
928 45.3726209672493
929 46.2865202951886
930 46.4383463620763
931 46.0618835223007
932 46.7603492731634
933 47.0869425688775
934 47.0807676904065
935 47.0865458343465
936 48.0089294258507
937 47.8755777572226
938 47.229446652518
939 47.5148962494338
940 46.5742109227385
941 46.2452800659026
942 45.7307936782787
943 44.8949689031475
944 44.8275392294812
945 44.7556339304866
946 44.0118247142514
947 44.5045939170085
948 44.8360799828952
949 43.3472588917984
950 43.3385632556885
951 43.9118304421861
952 44.4674467502822
953 44.8875850755544
954 45.0579507273003
955 44.720571450042
956 44.5779686992357
957 44.5760872408737
958 44.6289824261066
959 45.5198886573847
960 45.6684403050697
961 45.039683322557
962 44.4216332566446
963 45.4520226348508
964 45.105200397333
965 46.1846169733336
966 46.3805699463501
967 45.9641772777348
968 46.4023533080719
969 47.2754587071893
970 47.3924427896116
971 48.6703545421624
972 49.4669868086308
973 47.9595494360511
974 47.6751613786495
975 47.0770275080455
976 47.4266386240571
977 47.877796659163
978 46.9725207268022
979 46.6175882680443
980 47.1675747143101
981 46.1677945805257
982 45.6412003873044
983 46.0019232702465
984 46.5531325213553
985 46.4422333446819
986 46.0423327402102
987 46.1278939092428
988 45.8895117700353
989 45.2666054864836
990 45.0534971300422
991 45.0291469449413
992 44.5401008056289
993 44.376576733938
994 44.7833749779114
995 44.8360761679682
996 45.1825247380839
997 44.6576394622275
998 45.0322154884599
999 44.8520043627573
1000 44.2623516720208
1001 44.6713127638262
1002 44.7561410886062
1003 45.5126546397036
1004 44.4806977607782
1005 43.8792893038445
1006 42.6282756429081
1007 42.9209295063824
1008 42.8146893062476
1009 43.2504582335345
1010 44.0514144194823
1011 43.7029674479458
1012 44.2702580784871
1013 44.0961027336536
1014 43.9912209853843
1015 44.4526451054621
1016 45.8774017485763
1017 45.6161601760486
1018 45.8115352859886
1019 45.2056339863881
1020 45.0009773830861
1021 45.7196869777954
1022 45.3753158438658
1023 45.6189253237706
1024 46.5938042042071
1025 46.2424444284307
1026 45.3236764923358
1027 45.0333591177506
1028 45.6092720385877
1029 45.2193677955363
1030 44.5534528164502
1031 44.5188286475275
1032 45.0992881867896
1033 45.1360189057621
1034 43.822606831896
1035 44.2729151479477
1036 44.925358232797
1037 45.4285832756526
1038 45.21880410033
1039 44.9824288076589
1040 45.1268980412847
1041 44.4791900164041
1042 44.0062870220979
1043 42.7862060492173
1044 43.6529975413602
1045 43.881624569066
1046 43.6071502481702
1047 43.1113083258578
1048 42.6569827101742
1049 43.3283823273204
1050 42.8222548043601
1051 43.3639843841454
1052 42.2516990094693
1053 42.619959792847
1054 42.4547854583914
1055 43.0303748992081
1056 42.7934039350377
1057 42.9987267983825
1058 42.9058501686921
1059 42.2438402881034
1060 43.2788031875292
1061 42.5323803166292
1062 43.9882071753094
1063 43.7390580376793
1064 44.8563219009216
1065 44.0067978855754
1066 44.106759535208
1067 44.2405159048087
1068 45.5885773969624
1069 45.9551339541634
1070 47.1186559366267
1071 47.359863278698
1072 47.0092543689809
1073 47.7489156464264
1074 46.5524148431915
1075 46.8461979035105
1076 46.7839114646315
1077 46.0497701767716
1078 45.1046044309401
1079 45.5522208927101
1080 43.7261681621157
1081 44.4531523012378
1082 45.7388957741564
1083 45.7051516626579
1084 45.5940664433006
1085 45.6753345577309
1086 46.2902873810405
1087 46.821481393589
1088 46.8160437261269
1089 46.5736144985899
1090 46.6171516869032
1091 45.7661028537959
1092 44.8766821036811
1093 45.6945719379844
1094 45.6873688368785
1095 44.9470306814742
1096 44.5009734449269
1097 44.9617251663609
1098 45.0534779875106
1099 45.8992935247003
1100 45.8617951929714
1101 46.5875508974018
1102 45.7051814505758
1103 45.0738588305821
1104 46.0290228978623
1105 46.0415700847771
1106 46.6114244413095
1107 46.331728598742
1108 45.9595388717134
1109 44.7805532040383
1110 46.2166439608082
1111 45.2041863355279
1112 45.1061869780287
1113 45.8253814137708
1114 44.9246940832784
1115 45.3882780289298
1116 45.2212994868044
1117 44.995286117699
1118 44.9118633424219
1119 45.0276196793749
1120 44.3780849826503
1121 44.6727179483463
1122 44.9015746978411
1123 43.9899816282264
1124 43.9759629192184
1125 44.4662849526627
1126 44.0042566682318
1127 44.6970403982483
1128 44.5794446135501
1129 44.5222268135896
1130 45.2366733566212
1131 45.5456540360865
1132 46.2091610455895
1133 46.6244122436538
1134 46.6491548589823
1135 46.4357007743398
1136 47.1073109298
1137 46.8512502991048
1138 46.7660200896356
1139 46.9041612756049
1140 47.2005938117767
1141 46.9509685748821
1142 46.5249758454835
1143 46.1238466109101
1144 46.4572413576787
1145 46.3023134780117
1146 45.7055715659857
1147 45.210440378921
1148 46.8247336822761
1149 46.8931840388186
1150 46.1586609538716
1151 46.9188897189118
1152 46.3708641746941
1153 45.3734022037797
1154 44.6886105537794
1155 44.9143052828342
1156 44.8269644641098
1157 45.7029474284755
1158 44.6406457476852
1159 44.3819332502305
1160 43.9944764093142
1161 44.4912740147866
1162 45.1518056549878
1163 47.3754458772564
1164 48.4198718926603
1165 48.1821465042093
1166 47.8913081913707
1167 47.551773329042
1168 46.8288679412353
1169 47.2717653791116
1170 46.3684829173786
1171 46.5796533010282
1172 46.0277575240017
1173 45.898006758455
1174 44.5917709216986
1175 44.0859950931731
1176 44.2379879087288
1177 44.9266577074076
1178 45.9716474430278
1179 46.0343163776116
1180 46.815681179904
1181 45.4815001569585
1182 45.6644735076413
1183 44.5746124928107
1184 46.2056155751113
1185 47.0178034594066
1186 47.6878282453621
1187 47.2536822557564
1188 46.7476612763018
1189 46.2850948113805
1190 46.0652230277599
1191 45.7503945419327
1192 45.9086614086221
1193 45.2099703970687
1194 44.2369745395277
1195 43.9099430770305
1196 43.5451761099122
1197 43.0296712441342
1198 42.7872414518214
1199 43.6267337657671
1200 43.6270654949105
1201 44.1528451932922
1202 44.5837000908719
1203 45.9284193019719
1204 46.9562788064309
1205 46.9767623594992
1206 47.0991394926607
1207 47.2464280922434
1208 48.6893550450356
1209 48.5293656522487
1210 48.5448093376074
1211 49.1443581947658
1212 48.5944746885225
1213 48.6750863612292
1214 47.6660248346579
1215 47.8469413985831
1216 47.3534902166081
1217 47.4095333737683
1218 45.6276892634798
1219 44.8985074477498
1220 44.5444737891451
1221 43.8711536444119
1222 43.9079519808736
1223 42.8506056456809
1224 42.6127939203263
1225 42.1935033953985
1226 42.8455418273613
1227 42.5433699525272
1228 43.1872475673763
1229 43.5828904557881
1230 43.5724782167334
1231 44.185831269009
1232 44.4939448331489
1233 44.1196534818681
1234 43.8683194592146
1235 43.8334020274765
1236 42.9380542551919
1237 42.9661491915237
1238 43.1279677976237
1239 43.5394890856208
1240 44.7573614682627
1241 43.9143139919584
1242 43.8013979796561
1243 43.8223674649109
1244 44.1782971729435
1245 44.2889280495635
1246 44.8759704054792
1247 44.5262867203843
1248 44.695227812195
1249 43.5197409201722
1250 42.7000651875946
1251 44.032926883961
1252 44.4706117530784
1253 44.5946275006155
1254 44.6553190091744
1255 44.3624144726785
1256 44.9299885030575
1257 47.2700450410592
1258 47.2137674204313
1259 47.717575070255
1260 48.0893910806896
1261 47.9433980961028
1262 46.8979855913885
1263 48.5285156677411
1264 48.5635567293782
1265 48.6643770188395
1266 47.9002023587595
1267 46.013507409242
1268 45.494729115194
1269 46.1731003603146
1270 45.1095994564717
1271 43.8489528521489
1272 44.8094465817563
1273 45.6027807597062
1274 46.6697014111206
1275 46.9327150412106
1276 47.4989907465062
1277 48.108836354185
1278 48.63155242125
1279 48.788852571774
1280 49.2383316075452
1281 49.4923078207802
1282 49.6818336370093
1283 47.7916592886802
1284 46.9360010040566
1285 47.0920130037224
1286 46.5507723520442
1287 46.545603160092
1288 44.9916560368486
1289 44.4213632386712
1290 45.1080506906268
1291 44.8663791268046
1292 44.1446216292303
1293 44.958221101768
1294 45.4084518766892
1295 44.7122121519205
1296 45.1237672282481
1297 44.3677407129176
1298 44.8350369055785
1299 45.5598801661498
1300 46.1572881817875
1301 46.2431638831682
1302 46.8930336544257
1303 45.6880279119914
1304 46.0748085342146
1305 46.5259229570423
1306 46.4535722996632
1307 47.1816563431498
1308 48.3559215507299
1309 48.7835671777886
1310 48.295190729301
1311 49.4997647464188
1312 48.6782213545688
1313 49.8308243965008
1314 49.1676132889294
1315 48.4380903155434
1316 48.8406238166492
1317 48.5523642511111
1318 47.7282426162985
1319 46.6811957923466
1320 46.3099929872506
1321 45.5511272299347
1322 45.9747768139494
1323 45.3820806793068
1324 45.0411244584399
1325 45.4993125825553
1326 44.7277533415109
1327 44.5933062919333
1328 44.5308048072784
1329 44.3577529806245
1330 44.755750315156
1331 43.9291579998733
1332 43.4153364098275
1333 43.0132722809303
1334 43.1887968173768
1335 44.8866542499419
1336 45.3471275917672
1337 44.9510753876901
1338 45.9583002390549
1339 45.5698793806259
1340 45.279556905257
1341 45.4508873609701
1342 45.908319533142
1343 45.6334396490934
1344 45.3179421086951
1345 43.9937505593805
1346 44.2311423555987
1347 44.0626558688042
1348 44.1629002807036
1349 45.7831868492266
1350 45.3699834222951
1351 46.1126149513887
1352 45.7137884602777
1353 46.1165011138005
1354 45.9433886878878
1355 45.9381646061228
1356 45.3006175331971
1357 45.7246725235096
1358 44.5218815169958
1359 43.0761403264154
1360 43.3122229927108
1361 42.680947556572
1362 42.8543985438531
1363 43.1742845371868
1364 43.3246083173483
1365 43.3659020752755
1366 44.0379258706989
1367 44.5777081567533
1368 45.2851704938201
1369 45.6705337399466
1370 45.6551127634716
1371 45.4923959184209
1372 46.5155878190572
1373 46.7898438038272
1374 48.1350896106016
1375 48.3891406943848
1376 48.518230288931
1377 47.8056031289031
1378 47.6390859505279
1379 48.7452489992544
1380 49.3823239128773
1381 50.3150907703585
1382 50.3507379572746
1383 50.4151624215238
1384 49.8981014036253
1385 50.8392788671646
1386 50.4360155601536
1387 50.246152696677
1388 50.6062576136608
1389 49.8798761718344
1390 49.4439887247555
1391 48.5617060478899
1392 47.5933680556703
1393 46.8214914999066
1394 46.3478499621262
1395 45.1609569050564
1396 45.3700843178354
1397 45.8252110437298
1398 45.2778649548858
1399 45.0073314633384
1400 44.7952966101248
1401 45.0032034819385
1402 44.9829804801737
1403 45.8613153269669
1404 45.3105049937077
1405 45.4318982844219
1406 44.947633831518
1407 44.901717860217
1408 45.3259933252389
1409 45.7971267375547
1410 45.8025286387855
1411 45.0810872166304
1412 45.9475226016624
1413 45.6728409953938
1414 47.095714179156
1415 46.9210228951069
1416 46.9092528321926
1417 46.997586350828
1418 47.4293402721114
1419 46.9025597524715
1420 47.8221132950564
1421 49.3120470576794
1422 48.3922564373494
1423 48.8943639749931
1424 47.4446410156009
1425 47.8468224572547
1426 48.0600113204702
1427 48.3166092923136
1428 47.4179856144205
1429 47.0733994962732
1430 47.097838126037
1431 47.4556308436133
1432 47.3881111317635
1433 46.0843140581325
1434 46.8298354173324
1435 48.0121459462176
1436 47.7855488187309
1437 48.2905489497804
1438 49.6613544498021
1439 50.8779606502847
1440 50.5996236600286
1441 50.3303148558077
1442 50.707373012654
1443 52.2401981841006
1444 52.4235171085981
1445 50.9282223895081
1446 50.694210744911
1447 50.7225254916739
1448 50.2695528276083
1449 48.9053875314541
1450 49.0408945635858
1451 48.4866722846027
1452 48.3331108573546
1453 47.8610157945678
1454 48.4240817529196
1455 48.711320199914
1456 49.2965000228437
1457 48.6646644642709
1458 48.9975566952533
1459 49.388029785041
1460 49.4464722303304
1461 50.1179122433808
1462 50.0578738144545
1463 50.0363846409326
1464 48.9353727892094
1465 49.1730271184058
1466 49.6528655823978
1467 50.1571278870748
1468 48.6447644828634
1469 48.5155842362032
1470 48.0994606369852
1471 47.0566367378351
1472 47.5600004236247
1473 47.6463639022255
1474 49.0769464760406
1475 47.949410247807
1476 48.4990553435816
1477 48.5386292630232
1478 49.0431186615074
1479 49.1444090759139
1480 48.5301348872411
1481 49.7562891438144
1482 49.3045054318366
1483 48.3536863965766
1484 47.0745742586026
1485 47.5378639125346
1486 46.8628274463259
1487 45.8107317249539
1488 45.2708894532361
1489 45.6527140328296
1490 46.2206078831164
1491 45.6615568250384
1492 45.9865094245499
1493 45.729295351324
1494 45.4604477958953
1495 45.8988830703077
1496 44.9492031040356
1497 45.8619608490129
1498 46.6670330849461
1499 46.4265324848645
1500 46.7300416468181
1501 46.5204722553968
1502 45.9088759047604
1503 46.724674992131
1504 47.2596522533954
1505 47.0492143633891
1506 46.9269325230596
1507 46.3081125267525
1508 45.8284664355807
1509 45.8391352105446
1510 45.4944645748084
1511 45.8830872502074
1512 46.5846141271416
1513 46.2935151338136
1514 46.270007059601
1515 46.7729121357568
1516 47.4676891018702
1517 47.598106502794
1518 47.6433081812525
1519 47.0949765241628
1520 46.3638849447652
1521 46.275438580356
1522 47.0822863075495
1523 46.9301864610748
1524 46.9923907642056
1525 46.8602838805078
1526 46.4973442179105
1527 46.4746304674577
1528 46.3473060579318
1529 46.839918457755
1530 48.0773045195752
1531 47.5562400002259
1532 47.292974067806
1533 47.1556956509268
1534 46.188973787907
1535 44.9567312582734
1536 45.416817357088
1537 44.9485999545245
1538 45.5772852115339
1539 45.7440770652961
1540 44.7877451239617
1541 45.5050610922034
1542 44.613368846166
1543 45.4129437433724
1544 45.7706992706322
1545 46.6420534926343
1546 46.5654175412555
1547 46.6507427675026
1548 45.9649390121752
1549 45.5125910599053
1550 45.8376784399572
1551 46.6499494444387
1552 47.1355699420438
1553 46.8776907375945
1554 47.7433766033542
1555 46.9778415687478
1556 46.5442460650755
1557 47.2109605330146
1558 47.7944234987481
1559 48.5361722118886
1560 48.7464259832071
1561 48.1129560041705
1562 47.2923305243627
1563 47.135829324373
1564 47.9162457654809
1565 49.277327195084
1566 49.0927312904977
1567 48.5271874799653
1568 48.3864548735654
1569 47.9133638767992
1570 47.7657701681528
1571 47.5241204922362
1572 47.4812945074329
1573 48.0236374593903
1574 46.8605941149086
1575 45.7623956176158
1576 46.3068292939928
1577 47.5951997327362
1578 47.8899915621206
1579 48.2575996216435
1580 49.1233716828582
1581 48.6035401500284
1582 48.9629990237699
1583 48.5728046076941
1584 48.6751376258448
1585 50.0510736533561
1586 50.302195501261
1587 49.8818085298392
1588 49.6377041683168
1589 49.5476079935565
1590 48.4293483619042
1591 48.5718664543517
1592 48.6291599294693
1593 48.6298520115688
1594 48.0127812717791
1595 48.0213694911486
1596 47.9028281085218
1597 47.1725695582316
1598 46.958170646047
1599 46.6951142760178
1600 47.0412862666814
1601 48.4244103508287
1602 48.650077075518
1603 48.4048386352801
1604 50.0117235631555
1605 49.5031123223152
1606 49.4790235199467
1607 49.8347747119364
1608 50.5809242003632
1609 51.187169009664
1610 51.5241183895899
1611 50.1672841765694
1612 50.1978372781666
1613 49.9597275414066
1614 48.4579375713264
1615 48.183568532739
1616 49.500208140116
1617 49.2811157275347
1618 48.4311290673437
1619 47.7257224497658
1620 47.9544350480312
1621 47.5891213299106
1622 47.1879800210882
1623 47.2709514712619
1624 47.9497398755507
1625 47.3478331325905
1626 46.6906536213442
1627 47.4030650634351
1628 48.0809223039488
1629 49.0583570393126
1630 48.67467530074
1631 49.0096273766775
1632 49.7058363411296
1633 50.8023001509856
1634 50.7136985191723
1635 52.2175906628678
1636 51.6685681797462
1637 51.5154557668506
1638 51.152004290147
1639 50.6454574012855
1640 51.3191902743775
1641 51.6512959514578
1642 51.4802858970217
1643 51.5708803330146
1644 52.2273883772016
1645 51.8024723254857
1646 52.2036055624927
1647 52.7226048301699
1648 53.2516353914769
1649 53.5115217992262
1650 52.3438256633827
1651 51.7389529446213
1652 52.0631971800669
1653 51.7292096894691
1654 51.0690207892803
1655 50.9504825185927
1656 50.1836060798197
1657 49.0610159441147
1658 49.7367496735563
1659 50.6542623390143
1660 52.0544427434603
1661 52.3175384089958
1662 51.0131942709471
1663 50.7379513414921
1664 51.011281751765
1665 50.5887330435595
1666 51.5200454352645
1667 52.4753311233571
1668 52.1095155533062
1669 51.0076519039123
1670 50.095767006155
1671 51.2472976882434
1672 51.8180739579958
1673 52.3751705703915
1674 51.2760122193545
1675 51.4336986811362
1676 51.3875746190803
1677 51.3116568922393
1678 50.0036799594166
1679 48.9478531244522
1680 50.0959196032158
1681 49.632528981598
1682 50.3571796986677
1683 50.6469112215636
1684 51.3395034344316
1685 51.1642167297049
1686 50.5070396528054
1687 50.4447968688083
1688 50.6139705014595
1689 52.0743567280159
1690 50.5942176107873
1691 49.9890263778106
1692 49.2127975668665
1693 47.8443478910096
1694 47.9378375678806
1695 47.855187678074
1696 47.9067279620916
1697 47.469511209804
1698 47.7009296796868
1699 46.7714713012472
1700 47.2075973485096
1701 48.4494074966623
1702 49.2059077483536
1703 50.1549361125013
1704 50.0536901532827
1705 50.0825846226495
1706 49.7097994234898
1707 49.9943561095325
1708 50.5349767590113
1709 50.9537632697348
1710 49.9860320271661
1711 48.2619525173523
1712 47.2788959572282
1713 46.1548984018267
1714 46.5480298373312
1715 46.0474326400453
1716 46.8462382310046
1717 45.9048961634046
1718 44.9913520557133
1719 44.5905606863482
1720 44.7943608430628
1721 45.4065915697177
1722 45.8530314888764
1723 45.817767943567
1724 45.2471483898284
1725 45.9030809494761
1726 45.966776548669
1727 46.3680932065114
1728 47.2472576925413
1729 48.6492701068732
1730 49.5485372448777
1731 49.3423650269144
1732 49.2269460417434
1733 49.9153254450684
1734 50.9404859512391
1735 50.7662753142895
1736 49.8228358568909
1737 49.7720101620647
1738 50.5084878506404
1739 48.4550223467327
1740 47.9729891058852
1741 48.1274443834192
1742 49.2234756769106
1743 48.5072480141441
1744 47.6606643915015
1745 47.2977826858812
1746 47.5511483391287
1747 47.2819509436005
1748 46.1478533288161
1749 47.3316580497497
1750 47.2023052875239
1751 46.7469692692281
1752 46.2087746659546
1753 46.4492715082592
1754 46.6302376672181
1755 47.4220119028403
1756 47.1586249259321
1757 48.3514953000579
1758 47.9328988648603
1759 47.3860175622206
1760 46.976076889134
1761 46.9318724977941
1762 46.6663527882978
1763 46.7651338537502
1764 46.5758074538702
1765 46.049733858016
1766 46.7857240126668
1767 46.3432478577078
1768 46.2772537910997
1769 46.3873790428159
1770 46.7596745661308
1771 46.8246986007304
1772 46.2491328544727
1773 46.9643495850743
1774 47.3947953466049
1775 47.9159557585681
1776 47.2089222399408
1777 46.9536075789713
1778 48.2640050295261
1779 47.7557833309912
1780 47.5203887602898
1781 47.2581712872939
1782 47.2596047996958
1783 46.1666234997045
1784 45.5185912599209
1785 45.0280441286816
1786 45.254844183158
1787 45.0423902605822
1788 44.1454345920075
1789 44.5873893594632
1790 45.0188724822122
1791 45.9293448266848
1792 46.7154667131215
1793 47.2595411414187
1794 47.5168463853816
1795 48.0870686899495
1796 48.1560373636927
1797 48.838508619419
1798 48.5524567226607
1799 48.5176854937726
1800 48.1740604055652
1801 47.8207110619025
1802 47.1212311827038
1803 47.9949251796284
1804 48.3026674582121
1805 47.789887156625
1806 48.7423479480164
1807 48.9186995502156
1808 48.8699832700345
1809 49.083833061419
1810 49.2484693432054
1811 48.8561208795242
1812 48.8723753730021
1813 47.7214360275095
1814 47.3570415818494
1815 47.3467961793977
1816 46.7322152360044
1817 45.6622159777429
1818 46.3809705277764
1819 46.3612873568198
1820 46.4443573827747
1821 46.1985258184977
1822 46.0874901733387
1823 46.2370581681646
1824 46.9781833078772
1825 46.5944024579435
1826 46.0662146109431
1827 45.8972893753609
1828 45.9781084887879
1829 45.8491410373404
1830 45.7741667125606
1831 46.1479760198885
1832 45.5217254179987
1833 45.3544273626851
1834 44.2811500685236
1835 44.7750783977757
1836 44.9561510740519
1837 45.4085913583801
1838 44.6148137718083
1839 44.2981872058059
1840 43.9242348317318
1841 44.3632705311042
1842 45.268515096767
1843 45.184342717446
1844 45.2243770490136
1845 44.0551147895768
1846 43.6673901347273
1847 43.6536540516948
1848 44.1163703026177
1849 44.4202538812507
1850 43.7793710562701
1851 42.6448844244074
1852 42.1979970153768
1853 42.7308522052484
1854 42.6822242217693
1855 44.5002035138936
1856 44.1984041699404
1857 44.0249076091538
1858 43.6199925194905
1859 43.7171103120253
1860 44.6649684600784
1861 45.1812574426713
1862 46.0425575198699
1863 46.1655945316898
1864 46.7737830645435
1865 46.4737020151641
1866 47.0254148392104
};
\addplot [semithick, darkorange25512714]
table {%
0 69.4412441480896
1 72.0172955892003
2 75.2404500199431
3 76.8596486750006
4 77.3624657980798
5 77.6706356296463
6 78.1649527203421
7 78.3418131314955
8 78.0404133879994
9 77.6578223268885
10 77.2486900173543
11 77.0057872501724
12 76.735925206751
13 76.3581927902846
14 75.8449873761855
15 75.3679639929517
16 74.7745249774254
17 74.0254975287016
18 72.7577078767424
19 70.8828949934223
20 67.5883740362527
21 63.994566722116
22 60.3924223206257
23 56.5952345686816
24 52.3533684128816
25 47.9772514769088
26 43.6027003201489
27 39.6935759495925
28 36.5072408836883
29 34.0723732288684
30 32.8688097608541
31 31.5077353404267
32 30.3863012916199
33 29.469723130386
34 28.9369624946916
35 28.2623943895056
36 27.8331591183364
37 27.4451495527171
38 27.3014256667221
39 27.3087936482551
40 26.9688973075024
41 26.5298790126095
42 26.2193746333759
43 25.6839005766013
44 25.4948101999053
45 25.5082891964937
46 25.9254983410487
47 25.5924342140069
48 24.9798082923661
49 24.426488913858
50 24.2733042581877
51 24.4488397755529
52 24.1054665695015
53 24.2175845589012
54 24.0373241326414
55 23.9522443284676
56 23.4112358463013
57 23.4933818374588
58 23.4378980220333
59 23.1297397005936
60 22.9836465464787
61 22.917402824165
62 23.1814490018207
63 23.0376243240931
64 23.045187709243
65 22.8894631665754
66 22.7904337255826
67 22.4996382006786
68 22.2145500759375
69 21.9589602226083
70 22.5752915851484
71 22.6095835050651
72 22.3668804414688
73 22.2916900438027
74 22.2655693117769
75 22.3488337295238
76 22.6782287066337
77 22.7697999179927
78 22.9421927273559
79 22.6862148353945
80 21.9005522581809
81 21.4919234520584
82 21.4360455843445
83 21.1997073333408
84 21.1859280242733
85 21.452297041293
86 20.9503914135376
87 21.0494710921958
88 21.3897591854431
89 21.6319054289117
90 21.628174376431
91 21.7812243104228
92 22.2888858930974
93 22.7580860296251
94 22.8333310837444
95 22.4865648658707
96 22.5388005801728
97 22.4919249037847
98 22.2188564232091
99 22.3906473928442
100 22.187287398864
101 22.4109161132902
102 21.7880672557433
103 21.6411823807137
104 21.3760500611645
105 21.681015455265
106 21.6631645986602
107 21.5121044999475
108 21.1568993006113
109 21.051122368806
110 21.4235073109874
111 21.2301133676259
112 21.465621514854
113 21.6421311876838
114 21.9266951269896
115 21.664080964924
116 22.2466889661641
117 22.0383632744391
118 22.4173855828806
119 22.5742567680463
120 22.4589273952996
121 22.4490752310381
122 22.4027926540737
123 22.3046617725955
124 22.4574655638495
125 22.5264096021387
126 22.2120859047366
127 22.4854268077795
128 22.4848278250223
129 22.4096511678825
130 22.3500343273167
131 22.4485610639758
132 22.4418789879659
133 22.6193287740986
134 22.3115906628847
135 22.4442167993495
136 22.5589016704421
137 22.6494193983746
138 22.6733835377572
139 22.871689989202
140 23.0160548517345
141 23.3261371453458
142 23.4160613667286
143 23.3887421111341
144 23.7209640286336
145 24.0087386502859
146 24.1219714431743
147 23.9517581622481
148 23.9438114112123
149 23.9647848247432
150 24.0590385594771
151 23.926311870829
152 23.9607214575899
153 23.7999217670078
154 24.2802135680746
155 23.9357711808971
156 23.3902325244519
157 23.5716112522103
158 24.1181020602759
159 24.2784781913376
160 24.6163113219027
161 24.9191763226057
162 24.5733465013182
163 25.0433820261247
164 24.7138362267466
165 24.8972369438122
166 25.2082844107891
167 25.5566683220625
168 24.8946117427895
169 24.8972749517512
170 24.6810680106907
171 24.2159134417455
172 24.6819922551276
173 24.5766528696444
174 24.4724620451456
175 24.5194781169047
176 24.9937694128643
177 24.9451483329521
178 25.2095739237788
179 24.8330015353595
180 24.9507070863005
181 25.0009395682298
182 24.9523927404691
183 25.1294128242212
184 25.4231558951503
185 25.074410035639
186 24.6042650360256
187 24.2411916970451
188 23.9477878809564
189 24.31805756025
190 24.3044398483051
191 24.6575760395255
192 24.9910904756524
193 24.8260708028537
194 24.5439147315539
195 24.9507759360319
196 25.7771442069556
197 26.0064069075719
198 26.5264216277741
199 26.4683950806445
200 26.4426801255405
201 26.6469064947277
202 26.8928952545295
203 26.8558665406725
204 27.2720670602218
205 27.3547907874316
206 26.8204316609988
207 27.2774342086777
208 27.1115647556865
209 26.8174474582151
210 26.5108576299585
211 26.1575456456827
212 25.793919655058
213 25.8733775746434
214 25.6677801741194
215 25.5061594194563
216 26.1260750627379
217 26.2900098904299
218 26.3292286775981
219 27.3552357939477
220 27.9615132155875
221 28.2082823284353
222 28.8843007608253
223 29.2087807309393
224 29.6191696581956
225 29.5964392003351
226 29.0089278255399
227 28.2206889544873
228 28.3615915361126
229 27.5351615809879
230 27.1676157959491
231 26.981991653416
232 26.3125258439882
233 25.8336564588081
234 25.7952249165213
235 26.8406701791337
236 27.5023578066608
237 28.0477192330403
238 27.9492774444966
239 27.8350257363333
240 27.9141226984939
241 28.4573368114545
242 28.4095470135212
243 28.7408278345491
244 28.4838924485182
245 27.9461797554328
246 26.9301450883188
247 26.5420779467234
248 26.638009391007
249 26.7666885214858
250 26.5131843812369
251 26.4099902552222
252 26.0015333478837
253 25.5792890959024
254 25.3978036589064
255 24.8001265819886
256 25.3831627779001
257 25.1549040046596
258 24.8289618500002
259 24.9732085311499
260 25.605029522068
261 24.9997617404662
262 25.0594340457725
263 24.9060172894893
264 25.0555947981492
265 25.1075840321423
266 25.4524347676447
267 25.6570908198359
268 25.7715880752111
269 26.4622822586854
270 26.0708500241745
271 26.3608144391028
272 26.5337232447478
273 27.0107800751534
274 26.8013735738266
275 27.1595500427696
276 26.1728898962288
277 26.2038014681345
278 26.3079573749642
279 25.9014833265649
280 25.9381365271037
281 25.7199524170179
282 25.7713749032112
283 25.279572295925
284 25.203380409348
285 25.2475274485691
286 26.048320965438
287 26.3047239925344
288 26.4195822899674
289 26.276409025074
290 26.2658986347577
291 26.5755887283202
292 27.9877515190886
293 29.3386385601965
294 29.669520485193
295 29.2947744311388
296 28.9264776298937
297 28.6414401173129
298 28.6066049656748
299 28.5298542275786
300 28.9325576410344
301 29.402579994058
302 28.6582700396011
303 27.5195746483264
304 27.8907147423796
305 28.6834064920839
306 28.8373752826937
307 28.7057830741428
308 28.3499047673006
309 28.0157985037583
310 27.5645364141283
311 26.4865340176085
312 25.965966406429
313 26.1593678131177
314 25.2453538575929
315 24.6585139163565
316 24.3106711912223
317 24.7752589499407
318 25.383663803509
319 25.36769390962
320 24.9266993631361
321 25.3916523859457
322 25.4527141176019
323 25.4866582641104
324 25.5263987509297
325 25.8085365949064
326 25.6093023297348
327 25.1300687832806
328 25.4078239264366
329 25.9685783158686
330 26.5499486954444
331 26.425735586279
332 26.298475104019
333 26.1101874308077
334 26.3659870656854
335 25.8082056645671
336 26.0834951306234
337 26.2365935903695
338 26.2847322105542
339 25.974006753131
340 25.743027875234
341 25.975902877251
342 25.7675384736514
343 25.8568991617646
344 26.0997367007902
345 25.6152290711001
346 25.8375927580223
347 25.8468443070282
348 25.0074969727085
349 24.6771984814145
350 24.2097172139952
351 23.5007793834525
352 24.2649563818821
353 24.1169083714489
354 23.954815097775
355 25.6491091866846
356 25.5600952385762
357 25.8432639116038
358 25.6278977791953
359 25.4727691265497
360 25.5565910838048
361 25.5805443745085
362 25.1861830620197
363 25.1285494637959
364 25.568897851679
365 24.1997722031677
366 23.8287307896488
367 24.0909161907235
368 24.7729327238901
369 25.1699820291365
370 25.1788545576662
371 25.3908983984584
372 25.2013326985434
373 25.194411291549
374 24.5489465674863
375 24.9707700939675
376 26.0966206695289
377 25.7352168919087
378 25.4843402332137
379 25.972930572372
380 26.0841077644966
381 27.0254356547504
382 27.1201459603975
383 27.963773932726
384 27.9515692906156
385 27.4871757703488
386 26.563546112849
387 27.0358709625519
388 26.6305321589725
389 26.3017623369476
390 26.3667366985635
391 25.8352498779383
392 25.5140686147549
393 25.1626927153033
394 25.1328306982874
395 25.777313886312
396 25.8000990259517
397 25.3852658818777
398 25.4322950151672
399 25.4425475881873
400 25.3166607925534
401 24.566983870588
402 24.7290863975227
403 24.388333918688
404 24.1595792738954
405 23.4858917279579
406 23.778343114775
407 23.5660992777092
408 23.8942087055296
409 24.1729077994389
410 25.4643468448483
411 26.8854948204222
412 27.0700153670577
413 27.1444456984305
414 27.3001882759015
415 27.580713281268
416 27.0205377037774
417 27.7505851335618
418 27.085275050682
419 26.1138422086232
420 25.5111799157573
421 24.7238019099764
422 24.3502767472002
423 24.7568053584155
424 24.6706323419894
425 24.9759233495913
426 25.2849790660987
427 24.231993250999
428 24.818067758352
429 25.5458746472324
430 25.4266538711717
431 24.9941763327075
432 24.8809474989524
433 24.5056609915142
434 24.4030725975245
435 24.2286735456706
436 23.7995751801645
437 23.8278073395927
438 23.9926527667814
439 24.0559889738414
440 23.5621173041277
441 23.6112928120009
442 24.3656703378313
443 24.5526378282134
444 25.97779596406
445 26.0033746140542
446 26.3243555731289
447 27.1153974006465
448 26.354372236771
449 25.4320540189853
450 24.9206348397522
451 25.0188135191794
452 24.3617999219423
453 23.8504882515259
454 22.5430987418683
455 22.7290286596393
456 23.2630825981525
457 22.6884572411686
458 23.4106689943242
459 24.2547457001779
460 25.6273448435441
461 26.2286860223519
462 26.940066105305
463 27.7684475487627
464 27.5957783592818
465 27.2434818371074
466 27.1537994234657
467 27.0101072813312
468 27.004016174755
469 26.5819407813756
470 26.1231708168744
471 25.4364493543269
472 24.6812074343101
473 24.0663428242155
474 24.7776043573768
475 25.317175315001
476 24.6488548453482
477 24.614072168039
478 24.310425979951
479 24.7469932849003
480 24.7237018212356
481 25.2764360659559
482 25.9150192755573
483 26.175511599905
484 25.8174159144177
485 25.406805607259
486 25.7377791873421
487 26.1873764669519
488 26.2531206479471
489 26.6719310331158
490 26.4604714573018
491 26.0640795555162
492 25.6855512526037
493 25.758090945721
494 26.6401140493614
495 26.6591984719202
496 27.2250000820552
497 27.1266994740996
498 26.7953648248865
499 26.2484478640815
500 26.3086321937269
501 26.8181359582885
502 27.1003381237374
503 26.9211393539015
504 26.6625439129401
505 26.5534976508967
506 26.2323531636351
507 25.4774954177884
508 25.4589449236996
509 25.9124481386006
510 25.7267563066427
511 25.0619029276102
512 24.8581773730192
513 24.9254489708234
514 25.112317361374
515 24.7975946556995
516 24.4810613575936
517 25.8201756642119
518 26.794536687189
519 25.8168732003205
520 26.3565897662178
521 27.2176160479587
522 26.7919908407608
523 26.8920653620744
524 26.7842637923086
525 27.1457642517051
526 26.9610215878517
527 25.9710704542254
528 25.3138598013135
529 25.7790245048033
530 25.1321808853189
531 24.5708914841184
532 24.7207430504009
533 24.7417819662032
534 24.0032420627622
535 24.1426968194784
536 24.2898146866866
537 24.3303473448675
538 25.4409985686483
539 25.2962518912691
540 25.5776011082168
541 26.2869148253384
542 26.7018780337782
543 26.5631329680344
544 26.7545518965743
545 26.6235265771631
546 26.5417338333604
547 27.6402987146242
548 26.9691238035726
549 27.3818500528838
550 27.3943364265113
551 27.1298298791233
552 26.9770807644129
553 26.9636082206188
554 27.7615735383586
555 28.1199275411727
556 28.4854695481393
557 27.6042887638307
558 28.3114781393316
559 28.1099378583846
560 28.509748853838
561 29.2867578182003
562 29.1082930016169
563 28.5260084458293
564 27.5244559417939
565 27.1789537614902
566 26.5034909845851
567 26.3514239044539
568 25.6010822783311
569 25.4485735096835
570 24.9835770975894
571 23.4830798726955
572 24.4227416492442
573 24.9758346853686
574 25.5524720859197
575 25.7313089236196
576 26.0431758124698
577 26.528866952023
578 26.6153306500386
579 26.6266080064163
580 27.2788951010007
581 27.2597972623174
582 27.0597746280601
583 27.5844441541259
584 27.2224868681887
585 27.0900381379739
586 28.047445855173
587 27.9075741462342
588 27.17843182239
589 27.217542770515
590 27.2652510743741
591 27.1861418819642
592 26.4112913390983
593 26.6120992954377
594 26.3226125059761
595 25.931403424542
596 25.6629270865065
597 24.9609199238164
598 25.5813098632117
599 25.4844824873653
600 25.5783486765194
601 26.2737745265061
602 27.6150388055825
603 26.8989002141097
604 27.5063972548956
605 28.1553940011768
606 27.9063816613699
607 28.6085165866575
608 28.6244681873028
609 28.7730214902653
610 28.8024055178458
611 29.4702562027159
612 28.7507048115095
613 28.512691391259
614 29.0238816187154
615 28.6916236399034
616 28.6277048662446
617 28.8094710873288
618 28.6939912062634
619 28.6682264687932
620 28.0108192207423
621 27.1461120162385
622 27.1040083176922
623 27.7048391106919
624 27.0867968352
625 26.7794742754996
626 26.5558598098155
627 25.9017354980014
628 26.5542798262601
629 26.1733461545155
630 27.2567463429863
631 27.3878476685432
632 27.5828551036496
633 26.8220242592893
634 26.7047021991203
635 26.8449983561248
636 27.5044401889162
637 28.6329351254566
638 28.572187260881
639 29.146589085134
640 28.4921492915631
641 28.4722162316876
642 27.6749565301012
643 28.2029462201431
644 28.656222732743
645 29.448942353356
646 28.3357104201916
647 27.9043323365798
648 27.8747129805507
649 28.56661922927
650 28.3605511101085
651 28.6513341558621
652 28.8030048594203
653 28.6285191866283
654 28.360021971622
655 27.5199713872304
656 28.0525829379621
657 28.1325965409026
658 27.0540962291168
659 25.9332264492234
660 26.5295950549955
661 26.7227707379021
662 28.0405182804906
663 27.836596407292
664 27.4073836503927
665 27.9902497956069
666 28.3436050325976
667 27.6209059900246
668 28.3224090613187
669 28.5022806860762
670 27.7327384670869
671 27.850789378007
672 26.701904384284
673 26.3024715664726
674 26.1687274956137
675 26.4032019845967
676 25.9290927955983
677 26.4089154083653
678 26.6156677411002
679 26.4686731546438
680 26.8974428902581
681 26.2400338103533
682 26.7159967286578
683 28.0691804545251
684 28.407935396308
685 28.7732876502594
686 28.9949998431943
687 29.4404956777182
688 29.1955720504397
689 29.628260759713
690 28.7004357204697
691 28.3049095101208
692 27.6898561226839
693 27.4965690054177
694 27.5025872961732
695 26.8182813120936
696 26.6779455612073
697 26.754909885456
698 26.085050790102
699 26.1497520923535
700 26.4853424446074
701 26.8410893882222
702 26.9203554476743
703 26.1639612999113
704 25.9809875470497
705 26.1475910973539
706 25.9458859592839
707 25.4910092307423
708 26.0637279945678
709 26.2677728060531
710 26.9374689939472
711 27.4333429840573
712 27.5918756212044
713 27.5294368825494
714 27.6235256511286
715 27.3598357644639
716 27.3757817544744
717 27.1848638811146
718 26.9836879293289
719 26.083748149808
720 25.0888069532545
721 25.1913094953382
722 25.3705950697362
723 25.7581150865005
724 25.7896375091805
725 25.9620484149767
726 26.8979875116577
727 27.8015763818186
728 28.4800335899177
729 29.0679631810601
730 29.428515331338
731 28.7571102190284
732 29.0751961887369
733 29.6905707803816
734 29.8990151986544
735 29.6713847457258
736 30.2682393342745
737 30.8775444128639
738 30.5493613951009
739 30.5436244416481
740 30.7206249411373
741 31.1366386942929
742 30.6323798966719
743 29.674001543946
744 29.2643795663636
745 29.1902963055959
746 27.8707206515389
747 26.2279016696201
748 26.3439239141352
749 26.8027057000483
750 27.5656707588019
751 27.2433004947937
752 27.0458164882011
753 27.1257075542648
754 28.1339041576198
755 29.3386523668595
756 29.717846981855
757 30.2844643884973
758 30.1642070305344
759 29.1086435208927
760 29.059271347044
761 29.0223823738612
762 29.0850627563683
763 29.477652683823
764 29.1075897597393
765 28.9905611039045
766 29.033575154351
767 28.4351896227944
768 27.8488531278104
769 28.5066329666287
770 28.4189844096089
771 28.3283243593329
772 28.5393424552841
773 28.5395682561099
774 28.8293678325722
775 29.0381276919421
776 28.9668100127815
777 29.2752210762189
778 30.3460209893827
779 29.8930963208316
780 28.8436310306777
781 28.4912061138579
782 28.6511867865123
783 28.4550424770615
784 28.131278691796
785 27.9493473531487
786 28.4683445247472
787 28.5478976117821
788 28.473167452768
789 28.6292641824497
790 29.6269448933119
791 30.3245602334203
792 30.1682123076661
793 30.634147476675
794 30.7970754471575
795 31.1155948710986
796 29.758783473166
797 29.6777167999726
798 29.1693089526996
799 28.6527949296463
800 27.9670635980162
801 27.1621423821865
802 27.2623165309678
803 27.2011158797192
804 26.5193845253185
805 25.4164215563092
806 26.3945779586004
807 26.511276815445
808 26.3922815864175
809 27.911008119986
810 28.3706270693693
811 29.8990265363023
812 30.6677201601083
813 30.5144961031803
814 30.1247211311183
815 29.8696687343998
816 28.9022217389594
817 28.8231040998035
818 28.5762420577547
819 28.2833793113446
820 28.4069171297779
821 27.0564259105581
822 26.4769850413821
823 26.7238992046766
824 27.2297951078888
825 27.4400642273249
826 28.1574111575668
827 27.9533707781737
828 28.7891129866594
829 28.0236661056438
830 27.499602166482
831 27.9331118190011
832 29.0035218541984
833 28.270836340223
834 27.9641946787485
835 28.1073828287592
836 28.9108117444449
837 29.27803501249
838 28.9569745993007
839 29.411536489311
840 29.9130610108897
841 31.0017138419391
842 30.3202164009281
843 30.3678086104704
844 30.8887724195027
845 30.5794160377816
846 29.5342021249019
847 29.6681198474896
848 29.992392010628
849 29.6425551767425
850 29.3158017978621
851 28.1653943683399
852 28.3836238046952
853 29.1071767044558
854 29.106319959419
855 29.3703178385526
856 29.5873900546036
857 28.8972259638986
858 29.1426682455119
859 29.082338102964
860 29.1567022923226
861 29.2827371078965
862 29.2164598609648
863 30.0761529361417
864 30.2875188046293
865 30.3911459616493
866 30.5136273072931
867 31.4024442212198
868 30.9791507676811
869 31.7386387696477
870 32.4675162880243
871 32.9170025317496
872 32.5200169288066
873 32.3174726632906
874 32.8547048192447
875 33.1482880806404
876 32.785419688843
877 32.7877774482025
878 32.7792682622279
879 31.7647387438139
880 31.1381759526833
881 31.1914415667768
882 30.321767363112
883 30.2573215458038
884 30.4370996721093
885 29.8816889348768
886 31.400654136676
887 31.4285612064883
888 31.0285141352544
889 32.4677770555801
890 33.0609985042906
891 32.7332377500225
892 33.1917725224826
893 32.3512194023605
894 31.1375756588524
895 30.9733445976415
896 29.6467965392697
897 29.3109245444388
898 30.0542561380407
899 29.157823745601
900 28.971875174994
901 29.4355866442774
902 29.40182063542
903 29.3518703887811
904 29.337748758748
905 29.8856198164719
906 30.1938193591863
907 29.9950469971729
908 29.848338862063
909 29.8053546183333
910 28.9066772133812
911 28.9894552743816
912 29.6074541835274
913 29.4257588194979
914 29.7344642558231
915 28.9923149358984
916 28.8538692721456
917 28.9615342929779
918 28.6235183112319
919 28.5795908878374
920 28.9073316874813
921 28.8493129749033
922 28.9655619462339
923 29.0250420222973
924 30.1157503183751
925 30.4788084977982
926 30.8412014038769
927 31.0636532621338
928 30.8951946969967
929 32.021391164263
930 32.4637423791796
931 32.0992891884501
932 32.0347851855964
933 32.5852092094035
934 32.3783051042443
935 32.5722233365684
936 32.5958103347314
937 33.209574663552
938 33.5735635170802
939 33.1243910011004
940 32.7391381575938
941 32.8393326769824
942 32.3954574642556
943 32.5644342898765
944 31.9397084150048
945 32.131780037688
946 31.9669312689698
947 31.2869285198068
948 31.219747723952
949 30.6672666447123
950 30.5510657823213
951 30.1790498669343
952 30.6355813022968
953 30.3726068321665
954 31.2745632847476
955 31.4104709946514
956 31.6371089322435
957 32.2786113876119
958 33.5569651074492
959 34.1404728301103
960 35.0412450184087
961 35.3168309534303
962 34.8149366417328
963 34.9164661919182
964 34.1405459894187
965 34.1190111265101
966 33.5785378496853
967 32.9531015227452
968 31.7570420953645
969 32.0980501094517
970 31.6819998070574
971 31.6244076230483
972 32.2142498140013
973 32.1128605191555
974 32.0933244329844
975 32.4334561355928
976 34.113669471451
977 34.2376640502274
978 34.2211506508253
979 33.966780828274
980 33.7136656117737
981 34.511573119686
982 34.6312309298123
983 35.4057678461438
984 35.8898899080972
985 35.5228513756913
986 33.8376442027976
987 33.6188025305354
988 33.6198352790258
989 33.6058313369831
990 33.9912478140256
991 33.7294791620561
992 33.299762282312
993 32.2659402941341
994 32.3071983149253
995 32.0105842109845
996 33.2592506315679
997 33.5556240226627
998 33.7080980429205
999 33.0316655949779
1000 33.004210905076
1001 32.7095564605561
1002 32.0979225693877
1003 32.113007070408
1004 31.3177905182658
1005 31.8467454542236
1006 31.5251035419436
1007 30.6769507245494
1008 31.3056058716185
1009 31.8382129425357
1010 31.7259863224098
1011 31.2699796856062
1012 32.0684711343246
1013 32.6665859191388
1014 32.2928550543885
1015 31.7052772063583
1016 30.7878570297818
1017 31.4197814599528
1018 30.75152385513
1019 30.7017427196199
1020 31.0880716108986
1021 31.561517196526
1022 30.8444238939083
1023 31.4950193991931
1024 33.2859219881344
1025 34.3857232962804
1026 35.753752854155
1027 35.9501221600186
1028 36.5742417595784
1029 36.0455978839505
1030 35.7177858296609
1031 35.8896837633912
1032 36.4796870686148
1033 35.1576428594379
1034 33.6104918003919
1035 32.7901387287352
1036 30.8869307506024
1037 31.174043752136
1038 30.6662024011429
1039 30.854792196549
1040 31.0547439468817
1041 30.9435419646964
1042 30.6086050749168
1043 31.3599461449315
1044 32.1235647735461
1045 31.8402372595018
1046 31.9854422749386
1047 31.3991462717561
1048 31.2279919783719
1049 31.5492019936507
1050 31.2338577500012
1051 32.0558814650328
1052 32.076834573989
1053 31.5583422652065
1054 30.9148056787039
1055 32.2557618313446
1056 33.1646648484398
1057 33.4035532516149
1058 33.3010337193025
1059 33.6474514646409
1060 32.8996029074278
1061 31.1966793180126
1062 31.9060036537401
1063 31.3083328493126
1064 31.2747169255427
1065 29.8273192706856
1066 29.5009406947749
1067 29.1412057061817
1068 29.0458837759732
1069 27.9935027326643
1070 28.5028475192519
1071 28.890405162889
1072 28.7653941777961
1073 28.9087881625471
1074 29.4679242664656
1075 29.3402685791829
1076 29.8953283428449
1077 30.157708131393
1078 30.9380835022514
1079 31.7869948336444
1080 32.6952148115282
1081 33.1404705663149
1082 33.4627347542553
1083 33.6588443407844
1084 32.9916380303141
1085 34.0214149163239
1086 34.033545246541
1087 34.2048313165049
1088 33.6073152070456
1089 33.2442447080496
1090 32.0241786006393
1091 31.4584542230293
1092 30.5327609665678
1093 32.3334839177685
1094 32.8517298990488
1095 31.9188130609024
1096 31.1981514130253
1097 30.98109908229
1098 31.0820136242468
1099 31.5502575719436
1100 32.6490321474719
1101 33.7510067582263
1102 34.4103974267343
1103 32.9520650224125
1104 33.4787587804922
1105 34.9445193506836
1106 35.3446202127742
1107 35.2780931319813
1108 35.4105246833319
1109 35.6816069732955
1110 35.4695783105287
1111 34.3475532618611
1112 34.0133283539606
1113 33.3711173843328
1114 33.3969300673774
1115 32.4524708261032
1116 32.3547352319844
1117 33.25832159209
1118 33.3017827250793
1119 33.1298715334541
1120 32.8632595901615
1121 32.672431767016
1122 33.4720486875243
1123 34.4198224915521
1124 34.0560132296015
1125 33.9176699069137
1126 33.7063938974453
1127 32.7759084623143
1128 32.3266549658432
1129 32.9639556443446
1130 32.7876186987026
1131 32.9887124901227
1132 32.5435413127371
1133 31.8002192068472
1134 31.5332918660468
1135 32.5329516797307
1136 32.9378471433916
1137 33.219925104064
1138 32.4986974837145
1139 31.0796160441609
1140 31.0227267562248
1141 31.265771868056
1142 30.3980280087947
1143 30.8844136157789
1144 31.3100791844273
1145 29.8403038907924
1146 29.4824249618395
1147 29.4764050195006
1148 29.6141651999906
1149 30.6132215334672
1150 31.6471537264123
1151 32.2894950162581
1152 33.1115251736524
1153 33.3507070542718
1154 32.8663293365143
1155 34.1345632964841
1156 34.1001039881436
1157 34.4974053089962
1158 35.7005933391123
1159 34.9172822424138
1160 33.8470492208276
1161 33.5106974736901
1162 33.0515532143729
1163 32.7249303023623
1164 33.0070318662974
1165 31.8416108475129
1166 32.0053401093828
1167 31.1261237146217
1168 30.9540905400771
1169 30.5479249242933
1170 30.9656560028163
1171 32.2406955761502
1172 32.2924052694846
1173 32.8445298125685
1174 32.050309065132
1175 32.3178553375727
1176 31.6445256817199
1177 32.217585275242
1178 33.1399299332635
1179 33.2482131898669
1180 32.7569981031411
1181 32.1135273473251
1182 31.5952484315203
1183 30.5015949716847
1184 31.4562038811796
1185 31.3277694739536
1186 31.8044153803222
1187 32.4183030258612
1188 31.3498878598754
1189 32.1432322571003
1190 32.4592900935112
1191 31.6600185327424
1192 32.2071924945039
1193 32.3489592520482
1194 31.3939123204538
1195 31.7199450926769
1196 32.3929339106855
1197 31.4278888379034
1198 30.9032664049198
1199 31.3160378111724
1200 31.600225182971
1201 32.0508266363469
1202 32.3062450393942
1203 32.2584271702292
1204 33.3624502522281
1205 33.9168672110073
1206 33.8218712724516
1207 35.0340942316658
1208 35.7940510327461
1209 36.0924121786034
1210 36.3379072837998
1211 36.1214989461298
1212 35.0044344453049
1213 36.1827115156213
1214 35.3514797498094
1215 34.5012430669119
1216 33.7887681392803
1217 33.3874395663591
1218 33.3013430579583
1219 31.948178510293
1220 31.3424630353248
1221 31.435297124333
1222 32.5457736850367
1223 32.0492847421696
1224 32.4488576156106
1225 33.2853311007415
1226 33.9950026616762
1227 33.6506003595065
1228 34.0637414689861
1229 34.5877157077908
1230 33.968816112184
1231 33.8278896329596
1232 34.3353882025168
1233 34.9576958513186
1234 35.0803087373005
1235 34.9757781233357
1236 35.3533598014943
1237 35.0989124705588
1238 34.3709748415838
1239 34.7522179815028
1240 35.9558245335735
1241 36.0305177555747
1242 35.5641297489938
1243 34.7445566563722
1244 34.7130842731364
1245 34.6650576456064
1246 34.261274766379
1247 34.0499303615218
1248 34.4025552121212
1249 33.7576145178273
1250 33.8368547963675
1251 33.6868378303374
1252 33.3675409194335
1253 33.454259750813
1254 33.2839369164801
1255 33.2736806725604
1256 34.0566306859436
1257 35.2162656124115
1258 35.436874999331
1259 35.4797421540096
1260 34.9530406099049
1261 36.0895277053223
1262 36.0196200726165
1263 36.9850027287589
1264 37.4970868357077
1265 37.4177861677564
1266 36.3302908653457
1267 35.4349469682921
1268 35.4073768648892
1269 36.780546799583
1270 37.3190435493881
1271 37.4757911665659
1272 37.7613927930363
1273 37.2664566507004
1274 37.0003347697845
1275 37.9106407951381
1276 38.8546510004315
1277 39.6659768022698
1278 40.0187785373136
1279 38.1665018292706
1280 37.5408086535515
1281 36.6563085404046
1282 36.381121343191
1283 36.4307070636904
1284 36.9141103028123
1285 35.8717413129272
1286 34.8244895844252
1287 34.4666367655405
1288 33.8400505190008
1289 35.1396317251044
1290 35.4491932214515
1291 35.3885458869063
1292 36.3295036488183
1293 36.1279622467843
1294 35.9876705057482
1295 36.0971767823574
1296 36.3844828978317
1297 36.609990981538
1298 36.999882480165
1299 37.3451603506855
1300 37.1751059980387
1301 36.6434592987224
1302 35.9149911128213
1303 35.8849687873461
1304 36.4773530204393
1305 36.5849180451936
1306 36.8002212421946
1307 36.443709973413
1308 36.3243826610287
1309 35.7767364599923
1310 36.0230237327981
1311 38.2434572009961
1312 38.5913156893676
1313 38.5861031625708
1314 37.8624825090999
1315 37.8726763119123
1316 37.7485656055169
1317 37.8179661426838
1318 37.6671397958018
1319 37.0710767683531
1320 36.9473784442913
1321 35.3173091404786
1322 35.3613804033625
1323 35.2824712633654
1324 34.8685169049826
1325 35.4337386252693
1326 35.9151397502767
1327 35.7063213237763
1328 35.885974857298
1329 36.4387289066048
1330 37.1753986712461
1331 37.0884803678784
1332 36.7823223563823
1333 36.2989965472055
1334 36.384936598403
1335 36.5076502034687
1336 35.705696973463
1337 35.8252019739077
1338 35.8407680802845
1339 35.8076981404278
1340 34.4681213500755
1341 34.8598095780287
1342 35.7035007326044
1343 35.9138021468037
1344 36.6884937571346
1345 35.2689563282925
1346 36.0040101781796
1347 36.1128186158668
1348 35.8180941913423
1349 36.582923697348
1350 37.1175344957295
1351 36.5212569989115
1352 36.4169741619527
1353 37.7325259637621
1354 37.5235863631744
1355 38.392268851303
1356 37.3170981862365
1357 37.7211189721924
1358 37.6744332493125
1359 36.9549936237832
1360 36.909501330982
1361 37.2299457374746
1362 36.6925955455503
1363 35.3687709882925
1364 35.4614689607137
1365 35.2585884041774
1366 35.8885226311698
1367 35.0827545345381
1368 35.9571577669856
1369 35.0276442516141
1370 35.214361707111
1371 34.8950201021024
1372 35.1971762909102
1373 35.2170231389133
1374 35.0793817033963
1375 35.5090556039611
1376 36.5111897465165
1377 37.3478076084097
1378 36.6207260359889
1379 37.1812123413242
1380 37.6257700158488
1381 38.8823506512528
1382 38.3081029569808
1383 38.6524183722934
1384 38.2636032924487
1385 38.1735523270039
1386 36.6641916390008
1387 36.7910443042576
1388 36.3933030706852
1389 36.5750432194663
1390 36.1666196552476
1391 34.8714863121361
1392 35.5034898243506
1393 36.3214228661514
1394 36.4233537537318
1395 36.4114826673737
1396 37.0318975061583
1397 37.2461138731961
1398 38.058689924817
1399 38.921227492484
1400 39.946867408383
1401 40.1429697239281
1402 39.869534751551
1403 38.9545570621188
1404 38.7810120885389
1405 38.6407740941747
1406 39.2226535455977
1407 38.786267747527
1408 38.4492317129072
1409 37.8410974795292
1410 36.7465576883711
1411 37.189355791462
1412 37.2206239335312
1413 36.7420716547039
1414 36.8531709670468
1415 36.4987687609483
1416 35.2756782034613
1417 34.8176348602597
1418 35.2021521262375
1419 35.0319623418817
1420 35.3280077763534
1421 35.0344426776162
1422 35.4637294223644
1423 36.4401634104115
1424 36.3891052993712
1425 37.5439513772202
1426 38.2716378556983
1427 39.8204795609536
1428 39.7630300325098
1429 40.2744604663471
1430 41.168126738335
1431 41.5921823556107
1432 40.6039799944127
1433 40.7379844299434
1434 41.0969182092841
1435 40.6832181940754
1436 41.122548088819
1437 39.9156111424073
1438 40.3164250825166
1439 40.0263455831717
1440 39.9408524014669
1441 39.8996025395905
1442 40.751995906144
1443 39.6372070245665
1444 39.727219009328
1445 39.3725672165538
1446 38.8151534475227
1447 39.0434748202995
1448 38.7173541218568
1449 37.9495319967232
1450 37.1369102851271
1451 36.9250175359436
1452 36.7792103809801
1453 38.0337916860232
1454 38.0788662480783
1455 38.0718335852288
1456 38.461068369304
1457 37.8762568580098
1458 37.9637622100569
1459 38.6235216163749
1460 38.0078071790898
1461 37.8894323952065
1462 37.5712220811295
1463 37.1729630030222
1464 36.8117062383321
1465 36.7390419886448
1466 36.5745543874697
1467 36.4633206949382
1468 36.7044350878221
1469 36.9463850207628
1470 37.8727016858127
1471 38.308340440201
1472 38.8226113617068
1473 38.9148282247119
1474 38.873731514867
1475 38.6761398374501
1476 38.8369963058707
1477 39.4458430780941
1478 38.2336537884384
1479 38.7325570667139
1480 37.9620430302623
1481 37.5760711042818
1482 36.5661489952975
1483 36.3291627302783
1484 36.5994202884072
1485 37.2864538592698
1486 37.6727860502912
1487 38.5345690391579
1488 39.6546912745599
1489 38.1034495298468
1490 38.768923110286
1491 38.0755548094397
1492 38.9099958392098
1493 38.7671984385857
1494 38.2563934079886
1495 37.4055836418169
1496 36.9877484455851
1497 35.5403823305275
1498 35.0212343452789
1499 35.0897557978232
1500 34.8566014915458
1501 35.8144276062901
1502 36.3178505334613
1503 36.1072339673305
1504 35.920553059174
1505 36.0411366580248
1506 35.6523206675707
1507 35.9856865007728
1508 35.6372447319926
1509 36.95918475021
1510 36.5438964324048
1511 36.5379557231379
1512 35.7662435927423
1513 35.8193375624137
1514 36.9371655425207
1515 37.1361022092139
1516 37.5340796869868
1517 36.8309042768389
1518 36.6012729023901
1519 35.9923613967679
1520 35.8746194779385
1521 35.559048383191
1522 35.6666873030633
1523 36.2759422642363
1524 34.8011629622522
1525 34.8775507209167
1526 34.8085190697053
1527 35.96162602498
1528 36.7610057278646
1529 36.8390734575446
1530 37.4518731063232
1531 37.4619457307094
1532 37.903645574692
1533 38.1978330759869
1534 38.4745659986246
1535 38.4574910474458
1536 38.0845233234207
1537 37.4209229480688
1538 37.0658437228625
1539 36.9029886821051
1540 36.6098978055944
1541 37.0429238764814
1542 36.6094413259078
1543 35.9483640192431
1544 36.3188941272416
1545 36.1293464730966
1546 36.2530595130128
1547 36.8861324363157
1548 37.6545260689872
1549 37.4577057582095
1550 37.2596134934646
1551 37.1080673984643
1552 36.7415681939062
1553 36.1532736274975
1554 36.2483966322656
1555 35.8700998710301
1556 36.1192763233377
1557 35.3557024989589
1558 34.4350785948324
1559 34.3760085613685
1560 34.8325486272371
1561 35.60572505041
1562 35.7014071643431
1563 35.7649512172814
1564 35.8314079786566
1565 36.0818905275768
1566 35.8852186959492
1567 36.1167587723086
1568 36.6482417289496
1569 36.6153336756345
1570 36.593970764669
1571 35.2764726455531
1572 35.5510848723239
1573 36.6232161930807
1574 37.1979801319201
1575 38.1881820036109
1576 39.089649861228
1577 40.0740931158957
1578 39.6590975246549
1579 40.8061773222064
1580 41.2595135774096
1581 41.556558100508
1582 42.3895590372357
1583 41.7306042127667
1584 41.7747611110975
1585 41.7579947067577
1586 41.6271089591731
1587 40.4988663318377
1588 40.7095185551924
1589 39.781860941593
1590 39.0112461530864
1591 38.9388562569706
1592 38.282245591202
1593 39.3102946503116
1594 39.7015897938915
1595 38.8501665995309
1596 38.2606653142551
1597 38.3804470076856
1598 38.325039369255
1599 38.9516669190032
1600 39.885149200925
1601 40.5550300956661
1602 40.737491766896
1603 39.7283989001426
1604 39.0913118408552
1605 39.9143896919997
1606 39.6742718774451
1607 39.9202142417468
1608 40.201540127213
1609 40.3502789535315
1610 39.6215398155027
1611 38.9282429618397
1612 39.3584627864425
1613 39.6619084068377
1614 38.3068428587262
1615 38.4001902432687
1616 38.3585061724167
1617 37.8788609927826
1618 38.0830873355767
1619 36.2973513198203
1620 35.8913193124592
1621 35.8198580448714
1622 34.8835765237889
1623 34.6625991980653
1624 35.4126430354925
1625 35.14506418287
1626 34.9937582525854
1627 34.8706378389712
1628 34.0519772106054
1629 34.884799293833
1630 35.7718129222896
1631 36.1617504027579
1632 36.055414589863
1633 36.8145714971457
1634 37.4213855764003
1635 37.2832703915502
1636 38.6231959877984
1637 38.9814494074902
1638 39.2501656729996
1639 39.3439187883833
1640 39.2676654742387
1641 38.7179539534352
1642 38.873088645893
1643 38.473897713017
1644 37.7540854147891
1645 37.3013339011828
1646 36.3092698287003
1647 36.6572857476772
1648 37.1023440259866
1649 37.3120294896298
1650 37.353933312244
1651 38.2963010561516
1652 38.5182559242052
1653 38.331035824369
1654 39.0560438456308
1655 38.501172164852
1656 39.1348537464431
1657 38.7612971009806
1658 39.2063368351869
1659 39.4157513087335
1660 38.6395790947297
1661 37.797380218092
1662 38.0113737803453
1663 38.3716260605372
1664 37.6101984309761
1665 37.9543081624758
1666 37.3549019311072
1667 37.786544295873
1668 36.9849110477613
1669 36.3862082754535
1670 37.5357624362908
1671 38.1186517933077
1672 37.5433227035432
1673 37.3563632453477
1674 37.4596576823775
1675 37.5824014488367
1676 37.6917576213096
1677 37.3128128540309
1678 37.8207824831274
1679 38.1153751202406
1680 37.2226184303036
1681 37.0032428354761
1682 36.8761503385371
1683 36.2410811908069
1684 35.9235838108009
1685 35.9728999998359
1686 36.6513984734375
1687 37.2828617810202
1688 37.0161748697216
1689 37.521061545421
1690 37.3501097262483
1691 37.3345385300021
1692 37.4639656470521
1693 37.1699033201163
1694 37.4523139973808
1695 37.2986940365535
1696 36.6229438286656
1697 36.0908012069618
1698 36.0984088078857
1699 36.1206916670085
1700 36.3462458858504
1701 36.1762167594154
1702 37.3347653897971
1703 38.6437629112309
1704 38.4089484125292
1705 38.9647973476527
1706 38.8233964414481
1707 38.6239921085765
1708 38.4714370566847
1709 38.3734865482277
1710 38.4106019797771
1711 38.4821118867143
1712 38.1817104026032
1713 37.9616302732503
1714 38.5329911625961
1715 38.5441413192578
1716 38.4678802399178
1717 38.7682719473286
1718 38.6645619997533
1719 39.151497569415
1720 39.5776424739096
1721 39.8876461549861
1722 39.2622209195965
1723 38.3308644103153
1724 38.5415840187483
1725 38.8135590979678
1726 39.1691775324897
1727 39.8905883263887
1728 39.7799613660293
1729 38.9644331403227
1730 38.7289381472786
1731 39.4238206837303
1732 38.7637818901961
1733 39.6277744854328
1734 40.6337979512907
1735 40.2498997725424
1736 39.6398738831151
1737 38.7094819310695
1738 38.9020223437781
1739 39.249059105329
1740 39.8980518404544
1741 39.6766442812993
1742 39.6431389247158
1743 39.8239681969578
1744 39.4781637191219
1745 39.2073701013622
1746 38.8715020812289
1747 38.7816206598904
1748 38.9132627388661
1749 38.1108619340391
1750 36.4963440231451
1751 35.6926558791114
1752 36.3208495773268
1753 35.8024624130217
1754 35.4708304722876
1755 35.6727353346493
1756 35.9815280880412
1757 36.3555631576621
1758 35.638531320128
1759 36.7614837569639
1760 38.2138655890799
1761 37.8505435708223
1762 38.0959660157501
1763 37.9653638748931
1764 38.2096963588691
1765 37.6814097174196
1766 38.0302084150563
1767 37.4749910347666
1768 38.021357526377
1769 37.6269714687154
1770 37.1589451101406
1771 37.9713739704627
1772 37.7577840211719
1773 37.402770746744
1774 36.6880550634855
1775 37.2515519166864
1776 36.9207285870731
1777 36.6203178780692
1778 36.4400868328481
1779 36.193574056945
1780 35.8074743606336
1781 35.4437211736276
1782 35.6970549847368
1783 36.6078119181471
1784 37.5573952143278
1785 37.3968402160726
1786 37.4320956410664
1787 38.4275445404232
1788 38.59467212081
1789 38.5231103956605
1790 38.3903122101277
1791 38.4922064568867
1792 38.685410084309
1793 38.7887186843382
1794 38.6791917676958
1795 39.5315907580173
1796 40.142444250525
1797 39.6929885476761
1798 39.6904304602551
1799 40.2498660338243
1800 40.8944025371998
1801 40.9784933747382
1802 40.5667802667772
1803 40.1683537553411
1804 39.3784383007279
1805 38.9012100502026
1806 39.12851022419
1807 39.9975604922514
1808 40.3646251873128
1809 40.6154238709244
1810 40.108592537485
1811 40.4427518172212
1812 39.9195764576549
1813 39.6949248726948
1814 40.3428275804453
1815 40.9251931974077
1816 41.6629553971168
1817 41.4085558172325
1818 41.6996716550913
1819 41.0869932410971
1820 41.7330136940175
1821 41.6744589320049
1822 41.6547725262395
1823 42.2588719264241
1824 40.9648819049788
1825 40.9600171238331
1826 39.5385443971164
1827 38.5747836015209
1828 37.7679131847755
1829 38.1579826593522
1830 37.6203698797831
1831 36.9148040523985
1832 37.2973783624097
1833 36.5059408521882
1834 37.2129664954782
1835 36.9338791489221
1836 36.5602084016216
1837 36.3750888392386
1838 36.3520474557636
1839 35.6409969331221
1840 35.9104635134049
1841 36.0179649862093
1842 36.0422451913311
1843 35.900377059909
1844 35.7004301613968
1845 35.1724658376105
1846 35.8130579909479
1847 36.454645940952
1848 36.3247878412447
1849 35.793505731681
1850 35.4529975052549
1851 35.4266232126675
1852 35.4859065696807
1853 36.1189538628872
1854 36.1896921238237
1855 36.5026932948119
1856 36.901975048764
1857 36.5217166902396
1858 36.0363587744639
1859 36.400697240301
1860 36.4428767208084
1861 36.994068770431
1862 36.9497232232164
1863 38.3013229149482
1864 38.0096848690337
1865 38.111099443663
1866 37.4953377784202
};
\addplot [semithick, forestgreen4416044]
table {%
0 53.0592802323905
1 52.6157612589388
2 52.2041962542044
3 51.7389135143019
4 51.2659301812763
5 50.9604601129585
6 50.8023034815183
7 50.621416332038
8 50.5953116509541
9 50.4530057608919
10 50.4784429230704
11 50.4942918620293
12 50.4311501750874
13 50.2430889533828
14 50.0337156862766
15 49.7126894368758
16 49.326717579198
17 48.868109123074
18 48.467527337003
19 48.1516420397991
20 48.2580639967792
21 48.3203905089066
22 48.6309157631357
23 49.149222598674
24 49.916871688859
25 50.8450703619146
26 52.236833102726
27 53.8871333695185
28 55.1889862173402
29 56.2465000928468
30 56.9748287521788
31 57.9959717881408
32 58.5592458852986
33 59.0306522693627
34 59.1090103767666
35 59.2658143705244
36 58.9636933282818
37 58.4431911098727
38 58.6294742473656
39 59.3302255488806
40 60.1369261104508
41 60.5296486142979
42 60.6521389261567
43 60.8808565782055
44 61.2545448863572
45 61.3935398345498
46 62.0225698413163
47 62.1993838501118
48 62.3261503232693
49 61.9604443363321
50 61.2151027616354
51 60.6367535507882
52 60.5991431208165
53 60.5010753984848
54 60.2707504751755
55 60.1391075605344
56 59.4396320292318
57 59.089454116961
58 58.5085436676825
59 58.2283178319435
60 57.896418488723
61 57.2977635968272
62 56.8294390815631
63 56.1016288982452
64 55.7203531703339
65 55.3988626237886
66 55.2503492204416
67 55.2212915870599
68 54.8623039129408
69 54.3706962003138
70 54.2125327863939
71 54.374102300568
72 54.4352777139227
73 54.823401818839
74 55.0811454590387
75 55.1676356548839
76 55.1919915052726
77 55.2155608157554
78 55.3632238154396
79 55.7381242605032
80 55.8004999529161
81 55.9500651120876
82 55.8975650924175
83 55.825724522776
84 55.8117322127427
85 55.9620223975907
86 56.144506358284
87 56.4510254715391
88 56.490063212323
89 56.453343710371
90 56.5429375337161
91 56.5013404308014
92 56.6755298386391
93 56.9427002831897
94 57.1030704819075
95 57.22070585708
96 57.375096823227
97 57.0836972521381
98 57.3225319464536
99 57.4102066969493
100 57.729804360636
101 58.0034455774648
102 58.0759549926012
103 57.9435465606692
104 58.0162512665368
105 57.9884622647772
106 58.0546996111765
107 57.9481340328092
108 57.9054926593996
109 57.9104366195957
110 57.6146413691477
111 57.3619695446664
112 57.4889600497596
113 57.3632189919842
114 57.0014989045029
115 56.7936044044549
116 56.4209782510283
117 56.2099347534785
118 56.1546210808482
119 55.6746293996751
120 55.3927353749722
121 55.2235100926251
122 54.793179444349
123 54.7806974827542
124 54.6484625717134
125 54.3003213879934
126 54.4271825564218
127 54.5967954555072
128 54.4895291371965
129 54.7819790366836
130 54.9454350299001
131 55.0572803449244
132 55.0884436670075
133 55.1985395492931
134 55.9611237058304
135 56.1703407734103
136 56.047266706787
137 56.0882996404156
138 56.1117165866305
139 56.1135125160825
140 56.3408050424745
141 56.2884021676093
142 56.3989022812394
143 56.1171876691278
144 55.2473723291271
145 54.9409940178023
146 54.6795378284491
147 54.5512382961142
148 54.1501937707013
149 54.0593286610239
150 53.8900226143039
151 53.889437064463
152 53.8089501661585
153 54.0686066865893
154 54.1996132286228
155 54.2502692545728
156 54.3679024530871
157 54.3869037476067
158 54.5596726986083
159 54.772015905914
160 54.7250733791967
161 54.8057262299551
162 55.0244541724261
163 54.8849808445119
164 55.1408331304941
165 55.6789773223863
166 55.6068436478274
167 55.5819307001914
168 55.5959426958033
169 55.7009174061782
170 55.5549440619039
171 55.8901595765476
172 55.6629929881552
173 55.4947297339713
174 55.2118287042895
175 55.0824952642798
176 55.4395256095752
177 55.3337259257728
178 55.5185397992427
179 55.2511045997855
180 55.6419502944148
181 55.5790298212543
182 55.4102638910549
183 55.8273301599103
184 55.9247317440218
185 55.723203496515
186 55.3377594674722
187 55.6728518765364
188 55.5057256388317
189 55.61668381633
190 55.6004515571354
191 55.8236169338859
192 55.8119152767612
193 55.8955014704697
194 56.0065239720486
195 56.1515022778185
196 56.4480327387628
197 56.594793118868
198 56.6366434347988
199 56.7834774952113
200 56.6929534157389
201 56.5616458330052
202 57.4066719128888
203 57.6915658303365
204 57.8785173169583
205 58.021251791438
206 58.1798122042322
207 57.9185841586696
208 58.0107964897123
209 57.8670242924318
210 57.7848569217646
211 57.3094608837443
212 56.9344753561799
213 56.6225416774763
214 56.4629077502955
215 56.477600661283
216 56.7555495772703
217 57.1193078045944
218 57.2175107612339
219 57.189139165039
220 57.1228464038572
221 57.3522090790257
222 57.7482698568481
223 57.8176097869947
224 58.3727258582547
225 58.0930948123185
226 57.5974716863966
227 57.6358229589845
228 57.3634140255948
229 57.1712149261982
230 57.2900152148668
231 57.1509870456686
232 56.4524867791622
233 56.0832193876777
234 55.4754709152093
235 55.9678456430629
236 55.9154107086715
237 55.2841725636686
238 55.865490494833
239 56.038454325048
240 55.9424933981743
241 55.8118546660006
242 55.7975502702534
243 56.007496139537
244 56.3530919983316
245 55.8798693066828
246 56.166012482836
247 56.6421248154739
248 56.2997906226439
249 56.1395197563409
250 56.5823661642648
251 57.1256965621388
252 57.4059966587207
253 57.2645056249528
254 56.8788067381029
255 56.889274534843
256 56.5690437567451
257 56.5838710865614
258 56.9134302611742
259 57.3078590642862
260 57.2398637007437
261 56.5770316738806
262 56.5415863938391
263 56.5762765503766
264 57.0301347407861
265 56.8121709127001
266 56.8286462275533
267 56.5992077466739
268 56.3848467077824
269 55.9091174851425
270 55.5269473198626
271 55.8851747483773
272 55.4839057707641
273 55.3054146800634
274 54.7866032917546
275 54.6803162466678
276 54.3972150238965
277 54.414329386963
278 53.9890611121924
279 54.1729278531097
280 54.0220549631168
281 53.3092222748532
282 53.3554031668152
283 53.3298599850817
284 52.995678547415
285 53.0349358495991
286 53.3683644031274
287 52.7249963094982
288 52.5598904651305
289 52.2943479112118
290 52.4601467329457
291 52.5605648179961
292 52.9015088306869
293 52.5202596329786
294 53.1804994095474
295 53.0479397123311
296 53.2434699073295
297 54.0878870555019
298 54.2449930104797
299 54.7912185579239
300 54.8229737137126
301 55.7506826631059
302 55.8758294754161
303 55.8129248030247
304 55.639771625038
305 56.1566189042996
306 55.6346648619972
307 55.1636419791577
308 55.358797626505
309 55.1902797762326
310 54.6111073189729
311 53.902827093854
312 53.0174602903198
313 53.1290652116038
314 52.8842670956448
315 52.271423145374
316 52.397512452453
317 52.4439271175185
318 52.6675763913876
319 52.7260865754376
320 53.0193453209456
321 53.3602245134646
322 53.7400080247907
323 54.0576944145381
324 53.8837009143698
325 54.1909786569449
326 54.4317063672459
327 54.1670668634664
328 53.7849185787381
329 53.3411950270558
330 53.6594852202918
331 53.4928802561018
332 53.2605958403112
333 53.283574393014
334 53.2716380172213
335 53.0821055892801
336 52.748050232544
337 52.7635486494791
338 52.6084807561934
339 52.509343322936
340 52.2059702405606
341 52.1589655789701
342 52.4153670707129
343 52.2609535000308
344 52.7661022355005
345 52.6691207735616
346 53.0035165482119
347 53.9283092685482
348 53.8090419822873
349 53.9615810669263
350 54.3511659129429
351 54.3581519267331
352 54.1758255708819
353 54.3229033259526
354 53.5433784777503
355 53.3767989173026
356 53.6074247393017
357 52.6889810092871
358 53.2117943876609
359 53.2719458335859
360 53.3066211052016
361 53.5498314864747
362 53.7301745268732
363 53.9099146162183
364 54.5654771456527
365 55.1161705782579
366 54.8748756727667
367 55.1751688981293
368 54.8719268590985
369 54.8977556598398
370 54.9540618766821
371 54.6555465190389
372 54.965185220416
373 54.7031556802279
374 54.7375686032462
375 54.3986039756284
376 54.1447456705654
377 54.0581812695542
378 53.9475691584817
379 53.8104232716734
380 53.2249783906924
381 53.3053007975174
382 53.0316661364543
383 52.8570238410088
384 52.5145878239943
385 52.7039045370969
386 52.5368164523297
387 53.314622058769
388 53.5762944138277
389 53.3214312700632
390 53.8745605318124
391 53.9981039328416
392 53.6841248731282
393 53.5050527155521
394 53.513804742494
395 53.154369585679
396 53.1533870446379
397 52.2441727628142
398 52.3073351454978
399 52.3599793039936
400 52.2500618202573
401 51.8881862018888
402 52.0422265841708
403 52.2025525320668
404 52.1553462610806
405 52.6386849445928
406 52.8006946198882
407 52.9069117579129
408 52.8038775661871
409 52.9575228502636
410 52.6539158688861
411 52.5034217355888
412 52.3521204171524
413 52.0568273205872
414 51.758636892423
415 51.4058564443258
416 51.0785746822975
417 50.5575198462269
418 50.1075876773031
419 49.583827217566
420 49.5958587517729
421 50.3478800113706
422 50.4724229149895
423 51.3754299827798
424 52.2205419035607
425 52.9106197404466
426 52.7765644390857
427 53.0837947518769
428 52.9918704825395
429 53.4620556409994
430 53.3877589370689
431 53.0478768954481
432 52.7432643707549
433 52.6980858294103
434 52.1762840466747
435 51.5985743685609
436 51.4599494972866
437 51.4503592576161
438 52.4657669885677
439 52.600244638788
440 52.6344782646598
441 51.8308184112549
442 52.2465458216878
443 51.1437979193322
444 50.6973714114945
445 50.7348871751619
446 50.7503354314252
447 50.8693048424014
448 49.846933395778
449 49.5509864763443
450 49.4814574449737
451 50.0800625011924
452 50.1619147273133
453 50.4096552460341
454 50.6896251102465
455 50.2847637545386
456 50.5664005024182
457 50.2083596448721
458 50.3883020374889
459 50.31914181797
460 50.144353489611
461 50.1146226738888
462 49.6498305629765
463 50.5250025942007
464 51.2770977842684
465 51.2040891773096
466 51.3924852827581
467 51.3538362099329
468 51.4992308539769
469 51.9272851707535
470 52.1210108531163
471 52.2250095883237
472 52.3597776286049
473 51.5332386702039
474 50.8686426910195
475 51.4826707401345
476 51.987257560825
477 52.4090511532963
478 51.9169147298205
479 51.232351767465
480 51.2876812625356
481 50.8412828323244
482 50.9699890186674
483 51.2298654629326
484 51.1304126086705
485 50.9553972814426
486 50.8042090900085
487 50.5530991381377
488 50.9884401604081
489 51.4861710137872
490 51.4934037945368
491 51.6163430933571
492 52.1749620644046
493 51.6781263208928
494 52.3989690293297
495 52.9188985230821
496 52.3221953345985
497 52.1243913933406
498 52.2167227192085
499 52.7321283685301
500 52.7762967324789
501 52.6828620449237
502 52.0695542696147
503 52.9248082997749
504 52.7909369928238
505 53.3440579691558
506 53.5689069463083
507 53.9483540434462
508 53.9249800050247
509 53.6825679238458
510 53.260141390772
511 53.7569454335989
512 53.3961806943975
513 52.6723789105515
514 52.1765513731933
515 50.9510160549173
516 50.6912200352835
517 50.9322689891366
518 51.2175958187326
519 51.0031915839924
520 51.3403540984937
521 50.549515326039
522 50.4567573311042
523 51.5934157174329
524 52.4490807751069
525 52.9505369777085
526 53.1217122959484
527 52.4697709895591
528 51.7542873029989
529 51.5274816241262
530 51.1548949188645
531 51.6095170205076
532 51.9116939796538
533 50.5387054214473
534 49.151354005973
535 48.5563839922783
536 48.0669160969334
537 48.4147612719078
538 48.3372365111018
539 48.3157894026661
540 48.2711299350736
541 48.2195766237976
542 47.7068936007924
543 47.9794881105358
544 48.7027498106228
545 48.4782642945779
546 48.9485064476906
547 49.057507682805
548 49.2563247132099
549 49.2678211347046
550 49.4120335324627
551 49.2926060382749
552 49.6412363011392
553 49.8341841082044
554 49.5419955965584
555 49.7499505408149
556 49.4074242555189
557 49.0304047455788
558 49.5579251774977
559 49.6999283370973
560 49.9650332935269
561 50.4559103207446
562 50.2634102830907
563 50.050524465174
564 49.9635171470472
565 50.3689938319583
566 50.411283530092
567 50.4061259644417
568 49.9941459635476
569 50.7743276295574
570 50.8360162354344
571 51.0085685657783
572 52.029545783406
573 52.3025905494037
574 53.1638050935011
575 53.156749428398
576 53.7807051458574
577 54.1583639543593
578 54.9376350928801
579 54.4659137920197
580 54.0227297616154
581 53.5547748284827
582 53.2273799187618
583 53.6096058082811
584 52.5150310357261
585 52.6648713892575
586 52.5077445982231
587 52.7631162946027
588 52.6794934736979
589 52.6472048943059
590 53.0850821738112
591 52.9948053396574
592 53.0682345122925
593 53.4816613968162
594 53.541493703347
595 53.2966858019344
596 53.7191691140824
597 53.0551634607305
598 52.5734910353047
599 52.3634016706476
600 52.980267716256
601 52.9832302608951
602 53.5397958749517
603 52.9001015381347
604 53.5980295185248
605 53.8191528557478
606 53.3487441363709
607 54.0264612884357
608 54.7866488949479
609 54.5352769051121
610 54.674498421182
611 55.3888297178269
612 54.6765688129592
613 54.8112483381631
614 55.111664720029
615 54.8761401981743
616 55.0122537913481
617 54.5005851440142
618 53.7518065169449
619 54.0158724530549
620 53.219846578791
621 52.1491744538752
622 52.4181399189321
623 52.3097984129476
624 51.665954035107
625 51.3900649225986
626 50.7895892536772
627 50.7254542027217
628 50.9014691832798
629 50.6716318124557
630 50.5977902264641
631 50.9521638718754
632 50.6932709852529
633 50.1351289315999
634 49.7257358480641
635 50.1326351811049
636 50.1091401144929
637 50.147755131796
638 49.194479355772
639 49.0671429915691
640 49.1602341017574
641 49.053754844836
642 49.0371335537059
643 49.2956131240035
644 49.7655926348949
645 49.6438385055967
646 50.4382997318256
647 51.2749345165094
648 51.7056282166339
649 51.6296873609392
650 51.4504040479002
651 52.6332657453517
652 52.68578337667
653 52.8225752233782
654 52.8182322195776
655 52.9701581891292
656 52.3192989305918
657 51.7854823042681
658 52.2021593987682
659 52.6792024548125
660 53.4847614550751
661 52.6944322349656
662 52.9669288249771
663 52.8475775947645
664 52.8334932212334
665 52.8201222544874
666 52.7880073290384
667 53.0448508113749
668 52.9325141802635
669 52.5125627849656
670 51.4132480806456
671 51.5542118640182
672 51.0484510751108
673 51.4154015008256
674 51.8598108990384
675 51.6618067507242
676 51.8909402564381
677 51.8212347993054
678 52.1711734013654
679 52.2124917532829
680 52.2001692130134
681 51.6033684598216
682 51.4332938647426
683 51.4305298497691
684 51.2326660559236
685 51.3705776634352
686 51.3414063423374
687 51.664146395007
688 51.4793239904275
689 51.8393282028329
690 52.4089801113203
691 53.083369515279
692 53.0846065490166
693 53.5787358081615
694 53.3230721199082
695 53.4272722684531
696 53.6450669451622
697 53.3259735994123
698 52.9562914976115
699 52.9270868937451
700 52.621675657352
701 52.0702989447894
702 52.3649056066027
703 51.774523542805
704 51.9390672952857
705 51.8606079623606
706 51.5892111673931
707 51.4089203107234
708 51.8667134822933
709 51.428277613536
710 51.5431141780081
711 52.1564016187859
712 52.5809214661222
713 52.3029981819363
714 52.0747477677711
715 51.7471808618993
716 51.8672171263326
717 51.7264734128313
718 51.2392943097038
719 52.0216490908715
720 52.5871536891843
721 52.6693170793702
722 52.3065711779271
723 52.9246956528378
724 53.5216842073875
725 53.8261621370033
726 54.3914573456413
727 54.8053930006555
728 55.5629084136348
729 55.2115648014119
730 55.0847627766183
731 55.1714861966654
732 55.210283970477
733 55.5593888071083
734 54.6872333136537
735 55.2449657333815
736 55.0765096394645
737 55.6909247339836
738 55.0482752440342
739 55.1160397193175
740 54.4077398970686
741 53.8099508099253
742 53.9423304812676
743 53.8863103632695
744 54.3411824484064
745 53.6501391797089
746 53.6086506503591
747 52.6687736764417
748 53.5817759987626
749 53.4003062865655
750 53.9161874597127
751 54.1443534874743
752 54.3416057006515
753 54.0534080441066
754 54.2956388278058
755 54.643458033317
756 55.0833930852509
757 55.5996781456255
758 55.4737988205393
759 55.7588704807314
760 55.509250001044
761 56.0106199029733
762 56.2563567199449
763 56.325262122015
764 56.0802625568382
765 56.7722522908116
766 56.9561877325785
767 56.3958849441042
768 56.1602337164217
769 56.0930156941976
770 56.8734737513533
771 56.4003113296934
772 56.3418700751804
773 56.0153225250862
774 56.0873256182225
775 55.3057869756412
776 54.6084080965211
777 55.2263929874305
778 55.5083298713763
779 55.5088540950087
780 54.7142312363402
781 54.3395307429258
782 53.9212709939753
783 54.1614292443913
784 54.1727341139291
785 54.2576958503018
786 54.1463807664094
787 53.9459231140965
788 54.5295218463654
789 54.5092920493469
790 55.0728674064409
791 55.5264977912485
792 55.5905682216001
793 55.623166382187
794 55.6463106251879
795 55.5936008284711
796 55.5694219631444
797 55.6933813957385
798 55.3576073660639
799 55.371393843897
800 55.0138848134892
801 54.7668498800655
802 54.8335271427965
803 54.9604579495145
804 55.0751057225801
805 55.7986554042194
806 56.6789326997525
807 56.0695231025221
808 55.2322479977232
809 54.8376632854822
810 55.13173582055
811 55.2729899438905
812 55.3703276794918
813 55.0235988142646
814 54.6701647613968
815 53.91386408665
816 53.1466051740411
817 53.3510277982732
818 53.6094617227845
819 54.1207751241425
820 54.0847083418818
821 53.6087879524404
822 52.9585000616773
823 53.0066717426428
824 52.6215099794841
825 52.3193162577835
826 52.303908361808
827 52.2537527633716
828 51.9398504391395
829 52.0343635387776
830 51.2544775055463
831 51.3833027474839
832 52.013448968465
833 52.1681172098855
834 52.2301605260264
835 52.0542158144241
836 52.5338154607314
837 52.5073649275385
838 52.4933835643561
839 52.0069974507127
840 52.8861948794795
841 54.0038549962717
842 53.4448587533208
843 52.5616475132444
844 53.0061147085328
845 53.5578450427752
846 53.425166901848
847 54.1176887369477
848 54.0926039910913
849 54.0873632247743
850 53.8921449454665
851 52.6600931461889
852 52.5770598184615
853 52.6987480120069
854 52.7951914394992
855 52.8630090980006
856 52.8040763719255
857 52.7292607912864
858 52.8386092409605
859 53.175512992774
860 53.9280481860283
861 54.2244237443639
862 54.8693244961945
863 56.5839572331144
864 56.8783952583204
865 56.723322185997
866 56.6441181085254
867 57.0760124277423
868 56.9124673258051
869 56.629820741811
870 55.5440132166155
871 55.4950324952251
872 55.3832155041404
873 54.5114907517444
874 53.8422626400874
875 53.7344692050332
876 53.6409893317783
877 53.55185653304
878 53.8537862803015
879 53.8828242812275
880 54.1391071762485
881 54.686837081957
882 54.4092565149045
883 54.4885858778871
884 54.3350485770308
885 54.2593645029598
886 54.5228799019696
887 53.6292103867164
888 53.355412950866
889 53.8120861921587
890 54.0757653934901
891 53.11705609763
892 53.5329104529774
893 52.6659149814651
894 53.1515941792652
895 53.3049885048537
896 52.6565250766714
897 52.5982401153913
898 52.4850745579522
899 51.8890828564583
900 51.3495335684523
901 51.8835056471799
902 51.3824714470208
903 52.3226436711982
904 52.3615432443092
905 52.1096794410727
906 52.5106981756798
907 52.6829158642984
908 53.1894298983069
909 53.9968819082812
910 54.3914152022186
911 54.127910429657
912 54.7387427466009
913 54.030404793445
914 54.1733934337229
915 54.3370904039828
916 54.5157525644881
917 54.7106344243305
918 53.9637850835735
919 53.3267179598622
920 52.9618719922755
921 53.3890006388719
922 53.0241018600841
923 53.3822201383892
924 54.2149363326848
925 53.9726704870429
926 53.5565689299568
927 52.986342101719
928 53.4042468578826
929 54.3805926460408
930 54.4675114642098
931 53.9828638688292
932 54.2799728907871
933 53.6467117507727
934 52.0887763882151
935 51.9706500194228
936 51.9341860394536
937 52.4352516244122
938 52.4949455083497
939 52.2874760586741
940 51.7723770299367
941 52.4359279332755
942 52.0940760110637
943 52.067286848082
944 52.5243623634648
945 53.1489718645934
946 53.1231647573148
947 52.8080071114543
948 53.0826923879029
949 51.4583876376224
950 51.6280373558473
951 51.1576743325081
952 51.7458165291843
953 52.2825508438523
954 52.5954197139557
955 51.8369174371889
956 52.0539641803235
957 51.8680643027213
958 51.5553801817946
959 52.7176204604966
960 52.954628652627
961 52.6300282426361
962 52.278944165186
963 52.655426601617
964 52.2538353340319
965 53.1009174462217
966 52.7134525893389
967 52.6350502772318
968 53.16909431501
969 53.3423286008752
970 53.692290417834
971 54.3070863403157
972 53.8783930334227
973 53.1998648339272
974 53.4060181476386
975 53.582512591537
976 54.4418671671446
977 54.7988413368916
978 54.2129336174528
979 54.0021887293858
980 53.2198437468507
981 53.1276302648588
982 53.4375214890243
983 53.9796406764694
984 54.1442257502645
985 53.2396742413846
986 52.0389201065551
987 52.3694359234362
988 52.590109080002
989 53.0081657895118
990 53.2824307461923
991 52.9837601189337
992 53.1244082283826
993 53.0163191837978
994 52.911194770563
995 53.8789550995733
996 54.4233833458124
997 54.0714494424906
998 54.0610143315458
999 53.3736842970781
1000 53.9364197317423
1001 54.6145443339604
1002 54.0133801960461
1003 53.8952888241681
1004 53.5804650565431
1005 52.8283557769795
1006 52.0182232210173
1007 51.93311108748
1008 52.3021345889171
1009 52.1867830467905
1010 51.7450092502607
1011 50.922821438564
1012 51.474648957985
1013 51.1325010473253
1014 50.9942981739226
1015 51.0027202951472
1016 51.4641935622172
1017 51.0794603710938
1018 50.2822489190859
1019 50.5546192251251
1020 50.7797791297589
1021 52.049040422169
1022 51.5454752538878
1023 52.002770332004
1024 52.0172908654506
1025 51.8010888672835
1026 51.7812605774329
1027 52.6880864844204
1028 52.9392841151119
1029 52.894865136817
1030 52.5265051496597
1031 51.2662881078282
1032 51.5194262540396
1033 50.6354963156689
1034 50.9351447959254
1035 51.7180298524128
1036 52.3989871790619
1037 52.1283743348843
1038 52.0407145043007
1039 52.1421373647863
1040 52.7917066984626
1041 52.4010651345954
1042 52.043552919516
1043 53.0444396963503
1044 53.3083595836121
1045 53.7107987309426
1046 53.1285571432017
1047 52.6835493274906
1048 53.3958632951128
1049 53.1953316152524
1050 52.6546886595474
1051 53.0220550384477
1052 53.4231373290661
1053 53.2582736959196
1054 53.2549959389117
1055 53.3753520771989
1056 53.8443496189491
1057 54.0577404159083
1058 53.6127003461402
1059 53.7010105538116
1060 53.868014180229
1061 54.271772654634
1062 54.2552248381642
1063 54.1533443579181
1064 54.3247960527874
1065 53.3591136885995
1066 53.9314965478612
1067 54.0111282432595
1068 53.6258788010833
1069 53.8084401729623
1070 53.7464351795877
1071 53.8765170162229
1072 54.165301136855
1073 53.9218862630651
1074 53.4382593363553
1075 52.9177217410588
1076 52.2799100971698
1077 52.3039322499267
1078 53.2173712516766
1079 53.3819647232023
1080 53.1158627352741
1081 53.4372920013396
1082 54.233546793217
1083 54.7089021144991
1084 54.0603947922416
1085 54.758035085158
1086 54.5372055043023
1087 54.8669477427483
1088 54.5388235001034
1089 54.5237519892708
1090 55.3972662128497
1091 54.8386933918407
1092 53.3754657047393
1093 53.6084628161898
1094 55.0331962217737
1095 55.2216288507082
1096 55.5922956165176
1097 55.2595259728011
1098 55.6227680698462
1099 55.6070350746383
1100 55.3883363175157
1101 57.2038723289147
1102 58.3763418584913
1103 58.2038744628325
1104 57.0590445549909
1105 56.5703347202172
1106 56.3599476518733
1107 56.110769723145
1108 55.4116233131599
1109 55.3158700395569
1110 55.4359156352982
1111 53.6706908064741
1112 52.9033290056051
1113 52.937985003565
1114 53.5244165859729
1115 54.1083018137206
1116 54.0565663544152
1117 53.974799635923
1118 54.4142918455369
1119 54.7690434440377
1120 54.3052676340601
1121 55.0168017758761
1122 54.4592716921779
1123 54.2989997794455
1124 53.8738953368699
1125 53.4610148537895
1126 54.0486866136188
1127 53.9024931947478
1128 53.0871886939042
1129 53.7250411612856
1130 53.6061770129476
1131 54.0376055172118
1132 54.8282925671777
1133 54.5600737387097
1134 54.7457214218612
1135 55.2536897946831
1136 55.3209186683926
1137 55.8588744204127
1138 55.7104892055526
1139 54.5174804277305
1140 54.8819063129599
1141 53.7009717375006
1142 53.0626418780609
1143 52.6733262999294
1144 53.1894392712062
1145 52.35253414033
1146 51.9449877445179
1147 52.5081218037158
1148 52.6587166880325
1149 52.765616825242
1150 51.9413956581298
1151 51.7150155044357
1152 52.0744232066032
1153 52.3766836614612
1154 51.7374888320701
1155 52.0113118877388
1156 51.4691003817691
1157 50.6180090625657
1158 51.0988187330772
1159 50.9323110396131
1160 51.4445664069355
1161 51.2411427663996
1162 50.9607549130488
1163 51.3023455435321
1164 51.3041173106348
1165 51.374056658166
1166 51.9818813435495
1167 52.1827882886031
1168 52.1634382294498
1169 52.1763451810431
1170 52.2246770881105
1171 52.7801134973353
1172 53.4280214716741
1173 52.954358818707
1174 53.0069633985658
1175 53.1093731599674
1176 52.5323777007569
1177 52.3779738489277
1178 52.1855444467546
1179 52.0405907389978
1180 51.2294108077652
1181 50.6885362227144
1182 49.9422431649012
1183 49.936455593762
1184 49.6947927059641
1185 49.3634239163616
1186 49.7493927400374
1187 49.9652278311761
1188 49.8093114547903
1189 49.3768491924549
1190 49.9310862914667
1191 50.5923938282462
1192 51.4810050223344
1193 51.5483609393689
1194 51.8928496969986
1195 52.0331800783392
1196 51.7385102543131
1197 51.0756552541128
1198 51.2967556073102
1199 51.7527920383501
1200 51.7725628649183
1201 51.8897768763268
1202 51.3333058870362
1203 51.1908257618324
1204 51.2048498335709
1205 50.7315445988685
1206 50.9926052429509
1207 51.9307106014916
1208 51.6741702384523
1209 51.8544313548023
1210 51.9666578570709
1211 51.7316381024249
1212 51.5481456036445
1213 52.3052851027768
1214 52.8201994083787
1215 53.1616131803505
1216 53.1174889208027
1217 52.6860774473568
1218 52.9034029158644
1219 52.5263481436667
1220 52.6741404251332
1221 52.6322876244089
1222 52.5353022215428
1223 51.9066494167552
1224 50.8293186304839
1225 51.0912175931202
1226 50.8642488441757
1227 50.8407855963121
1228 50.8595642067603
1229 51.0586972074736
1230 51.0909541001617
1231 50.9010286624419
1232 50.6480058375592
1233 51.7515896151708
1234 52.1273380814043
1235 51.8116991862229
1236 51.6483813828768
1237 51.4021515537941
1238 50.9104588186728
1239 52.0233847594126
1240 52.1465191938259
1241 51.9904729020722
1242 52.8654261982141
1243 52.0836428794978
1244 51.7912656389861
1245 51.7243988192364
1246 51.9214484506557
1247 51.638370337528
1248 53.0710819043535
1249 51.7740175408801
1250 51.6135146620909
1251 51.9055820716924
1252 51.3882277210065
1253 51.1117689485433
1254 51.8467146361102
1255 52.2263195221556
1256 52.3997811915142
1257 53.6218683560849
1258 52.1754095821352
1259 52.3774249870356
1260 52.3638833766402
1261 52.3019013688705
1262 51.9315181114777
1263 52.3859627778521
1264 51.9987234794995
1265 51.5563731883601
1266 51.8153907873628
1267 50.9615498768831
1268 51.5753087810601
1269 52.175395463773
1270 52.2771173580664
1271 52.5367171502348
1272 52.8603704330572
1273 52.5137598575452
1274 52.9386065660458
1275 52.9892621400991
1276 53.0292564409355
1277 53.1676380236472
1278 53.3165603382934
1279 53.1059506743188
1280 52.9651631102118
1281 52.452717656459
1282 52.2673974731086
1283 52.5602185438611
1284 52.1913244225375
1285 52.590532056961
1286 51.9851311616684
1287 52.1427166815162
1288 51.4319298127352
1289 50.5147638551284
1290 50.6062206444945
1291 50.6878432513954
1292 51.3188748777238
1293 50.4788409909892
1294 50.3343730680383
1295 49.8985162416545
1296 49.7757863457532
1297 49.2502345274427
1298 49.3980059791751
1299 49.9283810708038
1300 49.9842899245524
1301 49.8979871454045
1302 50.0130329192432
1303 50.411751563581
1304 51.3722881865775
1305 51.7805645419391
1306 51.9553640672871
1307 52.3715227556665
1308 52.7610132219808
1309 52.8847507406364
1310 52.7059298967837
1311 53.3896118732475
1312 52.3481351699172
1313 52.5293179946764
1314 52.034424739544
1315 51.6556834583819
1316 51.175972656171
1317 50.5951043237565
1318 50.3571909522875
1319 49.9339978827056
1320 49.4989039591131
1321 48.7247136193296
1322 48.9074016630148
1323 48.3287614365723
1324 47.4248783994577
1325 47.5785604581954
1326 47.4797751788583
1327 47.6010382487373
1328 47.4040466952384
1329 47.8384563913492
1330 48.7455198667232
1331 49.5115624330085
1332 49.933064390545
1333 50.5797996567765
1334 51.3584210436979
1335 51.5659981637139
1336 52.4357982959278
1337 52.8222479804219
1338 52.9066726750444
1339 53.1009016464099
1340 52.2576758005376
1341 51.2172878960635
1342 51.2474959109041
1343 50.5281531167987
1344 50.2702628425647
1345 49.4193104394805
1346 49.1056767878743
1347 49.0811017230808
1348 49.0854345073129
1349 49.0187584951692
1350 49.4761728552625
1351 50.1290830252592
1352 49.7882655074965
1353 50.1848151841024
1354 50.0908185580905
1355 50.987959687336
1356 50.9017051358938
1357 50.5497693349748
1358 50.3341434707766
1359 50.047864727343
1360 49.4127318334415
1361 49.5204590079197
1362 49.6459175090424
1363 49.7881104816794
1364 49.7420036249298
1365 49.4018903467687
1366 50.0011297381425
1367 51.0209152046371
1368 51.7832036192019
1369 51.7172623058739
1370 51.9651128109306
1371 51.5542370200279
1372 51.8191507454255
1373 51.7262869052732
1374 52.920927587588
1375 53.1739595313853
1376 52.7965535920917
1377 52.1963860938446
1378 51.8939520130183
1379 52.4866525517342
1380 53.089751317563
1381 54.0740128877908
1382 54.3224169225974
1383 54.8442218796173
1384 54.2877326022881
1385 54.3184334915381
1386 54.1863105047689
1387 54.0007427421855
1388 54.3289758616732
1389 53.9420855027703
1390 53.5729556979131
1391 52.6905968634672
1392 52.6851160641996
1393 52.2081971586241
1394 52.7119137035837
1395 51.8977951821487
1396 51.4122840987689
1397 51.7131902047366
1398 51.4152652379813
1399 51.2645567410462
1400 51.4624875447784
1401 51.2853361820583
1402 50.394726865237
1403 51.0277169276646
1404 49.7003253870663
1405 50.5676113319076
1406 51.2485415372751
1407 51.4825361650317
1408 51.220626357875
1409 51.0600969047993
1410 51.2673156585905
1411 51.8908971503057
1412 52.8881554647374
1413 53.0488572677945
1414 53.3665211858769
1415 53.1864030537276
1416 52.5509432089751
1417 51.7109621160792
1418 52.00352510623
1419 52.3601926764499
1420 51.9271318163554
1421 52.5679601166899
1422 52.5522677058474
1423 51.8213930185019
1424 51.9243503970225
1425 51.7225608014813
1426 52.1106732039238
1427 52.7455349359351
1428 52.7221579100908
1429 52.4267234166389
1430 52.2700179004392
1431 51.4158509906834
1432 50.5271837182889
1433 50.7288725148689
1434 50.6990437382349
1435 51.2405250171712
1436 51.1771200246498
1437 51.1874091503025
1438 51.5630385129184
1439 52.2774631176445
1440 52.4014942426229
1441 52.8237500846821
1442 53.3221519292914
1443 53.2543107550291
1444 53.1893249905656
1445 52.7252247349389
1446 53.351276562979
1447 53.6298132368975
1448 53.3351731747675
1449 52.8749868585465
1450 53.2100541264749
1451 52.3204941595129
1452 53.146990735065
1453 52.7232420427312
1454 53.2012065669033
1455 53.5036987408647
1456 54.0297338999979
1457 53.1646001121476
1458 53.301597195792
1459 53.3170160837319
1460 52.4671101912886
1461 52.7467501519646
1462 52.4413809564248
1463 53.2050391766593
1464 53.0417311358824
1465 53.1460039168801
1466 52.4218793544271
1467 52.8699310094673
1468 52.3961190184502
1469 52.1932585077889
1470 52.6486573118934
1471 52.3167790397792
1472 51.7684065820514
1473 51.1053152404026
1474 51.1215417330579
1475 50.3727043279023
1476 50.0753083744484
1477 50.0926973382857
1478 49.6065942435971
1479 49.1978312087116
1480 49.4312599447028
1481 49.9394421055806
1482 49.4313294233745
1483 49.3918466750488
1484 48.6143406228281
1485 49.4587433537072
1486 49.4576681414878
1487 49.7322615005485
1488 50.1582011666966
1489 50.6793967736744
1490 51.328794449211
1491 51.0082591607425
1492 51.6719260218323
1493 51.602331420003
1494 51.9336019979744
1495 51.4836977969126
1496 51.6387209744587
1497 51.1895404108625
1498 52.2546030832721
1499 52.1503227332356
1500 51.2124099161625
1501 51.0865051531586
1502 50.7781297206933
1503 51.2126438763346
1504 51.6081494612282
1505 52.2715437078396
1506 51.4593603698837
1507 51.2347596218621
1508 50.2550139718176
1509 50.5442532942602
1510 50.5834384980842
1511 51.2950858530285
1512 51.6881671325168
1513 51.4188037588039
1514 51.200078569639
1515 50.4308506390999
1516 51.1491257560131
1517 51.5863571157126
1518 51.156654944796
1519 51.1129901718969
1520 51.1086689278547
1521 51.6683642081444
1522 51.3742339948567
1523 51.1419021592334
1524 51.0068982199119
1525 50.9910120651015
1526 51.363239907372
1527 51.1547130693798
1528 51.5831852130594
1529 51.2352331881608
1530 52.0032413809755
1531 51.0607249350998
1532 50.8082803100475
1533 50.6250518747302
1534 50.4061688306414
1535 49.9487055120255
1536 49.5932430725202
1537 49.4287433536397
1538 48.9131197874971
1539 49.2525981848089
1540 48.8052236371243
1541 49.0095091426277
1542 49.0032964436549
1543 49.357214461155
1544 49.5016156917649
1545 50.252058051596
1546 50.2688243125349
1547 50.2517581119073
1548 50.877275749916
1549 50.7738103681333
1550 50.2409220937571
1551 50.7012513176806
1552 50.5421767328571
1553 50.8550202322048
1554 51.087097871536
1555 50.7297205091353
1556 50.8393887830136
1557 50.6056203185517
1558 51.1209212589998
1559 51.3540821172478
1560 51.4602648177351
1561 51.6665762982383
1562 52.2493434854132
1563 51.903693869158
1564 51.9220162854079
1565 51.9140806484692
1566 51.8802553551588
1567 52.3615749251172
1568 52.2152308532284
1569 51.4933543388346
1570 51.661622615508
1571 50.4768031650398
1572 50.4060004064792
1573 50.4294594707191
1574 50.0896288219733
1575 50.0830709571323
1576 50.3756854572797
1577 50.8969379075749
1578 50.5411531015006
1579 52.2878777521282
1580 52.94708847124
1581 53.4765468005172
1582 53.8530433690973
1583 54.0055949610227
1584 54.2609086049945
1585 54.6924782566007
1586 55.2073946432135
1587 54.8312312629766
1588 54.8685026223299
1589 53.5383308027167
1590 52.9740844906362
1591 53.1852322824381
1592 53.1743606027145
1593 53.7316825356417
1594 53.5136311742292
1595 53.4006448289888
1596 52.2433275372626
1597 51.9823443288687
1598 51.5589114680855
1599 50.8977281125848
1600 51.0694135637972
1601 50.4713625415377
1602 49.812009448786
1603 48.8430629603673
1604 49.3635335293449
1605 49.5626587864793
1606 49.8339226874636
1607 50.1923497637807
1608 50.4536361805248
1609 52.2367839043118
1610 52.6559947093411
1611 52.6958540649183
1612 53.5374922826447
1613 53.8582491897167
1614 53.1349758168007
1615 52.744125713143
1616 52.6928032351073
1617 52.5155030074096
1618 52.4741493572802
1619 50.9970794183902
1620 50.5155572804745
1621 51.4154768148738
1622 50.702839413184
1623 50.8026883325073
1624 50.9698730529087
1625 51.3586138188304
1626 51.3713844049274
1627 50.8957472859496
1628 50.7210784735864
1629 50.7327168162007
1630 50.5665844907829
1631 49.8331473110366
1632 49.4490745354227
1633 49.6714764004902
1634 49.6490499078287
1635 49.3965395613471
1636 49.3288715428178
1637 49.5295699404778
1638 49.575194960691
1639 49.966149125875
1640 49.7789712264038
1641 49.8107226500096
1642 50.3712313115289
1643 49.6980171182315
1644 49.4885333905068
1645 49.701014878074
1646 49.534790558749
1647 50.3494037794612
1648 50.792169824914
1649 50.668889723911
1650 51.5248724030932
1651 51.3951827080975
1652 51.4747976589382
1653 52.0080866415196
1654 52.4025316620707
1655 52.4386906557417
1656 52.7432340445973
1657 51.9973121294528
1658 52.1597692105755
1659 52.2520150451814
1660 51.2530030935926
1661 51.5326342748243
1662 51.0781591918162
1663 50.8332101072815
1664 51.4676306474183
1665 51.2170003942885
1666 51.1057149611452
1667 51.668064912902
1668 51.9409678298804
1669 52.1815832231323
1670 53.2210843179753
1671 52.8783029448133
1672 52.7451399350303
1673 53.2668197613397
1674 52.8972656890881
1675 53.8193217192557
1676 54.6972396777025
1677 55.229613493989
1678 54.700668799564
1679 54.3583300255431
1680 53.7892036240952
1681 53.8150358347198
1682 54.0231463443357
1683 53.9164908198517
1684 53.5306666224284
1685 52.6059463365728
1686 52.1337697475782
1687 51.4754967201785
1688 51.2146654881201
1689 51.1124503879505
1690 50.3277901409984
1691 50.9515997062721
1692 51.6489215985108
1693 50.7755651693076
1694 51.0048628958193
1695 51.0753834167167
1696 50.1480934495173
1697 50.5158695652424
1698 50.7589817079665
1699 50.6023569916674
1700 51.6687739852271
1701 51.5448005962121
1702 51.1040379906415
1703 52.3875817786748
1704 53.2100879162445
1705 53.2972470644364
1706 53.7395019472275
1707 53.5166987047342
1708 53.5507342022781
1709 53.8563215576345
1710 53.1967334075471
1711 52.6906118788386
1712 52.8581538318952
1713 52.7307444898748
1714 52.5846115863154
1715 52.4884069950134
1716 52.3644043439026
1717 51.4857414137978
1718 51.2628511545369
1719 52.0080291957679
1720 52.1458131766219
1721 52.4910464241371
1722 52.871879524584
1723 51.4107633186031
1724 50.7054226860116
1725 50.6526468777289
1726 50.340274072994
1727 50.9643414203281
1728 51.3487758601324
1729 51.0744228136512
1730 51.1450668296663
1731 50.7778805952488
1732 50.1904803174304
1733 51.0752765529198
1734 51.9245869650336
1735 52.3429124030391
1736 53.3345660943557
1737 53.7131682195163
1738 53.0489060740045
1739 52.4389030674274
1740 52.6435047267894
1741 52.6254322424634
1742 52.1925760824123
1743 52.1748092934922
1744 51.1221854436577
1745 50.6940649722403
1746 49.9713994625905
1747 49.7801206321771
1748 49.7465362562565
1749 50.0159533320891
1750 49.713409864977
1751 49.9372200932937
1752 50.6919188153411
1753 50.9400016084116
1754 50.8575567866704
1755 50.4191732711311
1756 51.0202897260165
1757 50.9825310164319
1758 51.715012796512
1759 52.1876212539064
1760 52.2881250838592
1761 52.016314294879
1762 52.4311172792633
1763 51.672105535211
1764 52.3429241041134
1765 52.4829734557039
1766 52.4993866325739
1767 52.6862322359465
1768 52.8889136533391
1769 51.9567655594379
1770 52.4443311034597
1771 52.3525854542009
1772 51.7715512876078
1773 52.2773720532248
1774 52.0145675018883
1775 52.4853269387166
1776 51.9899015989622
1777 51.6335900228854
1778 51.9304823123516
1779 52.6462290491234
1780 52.5098983467861
1781 52.6430391185816
1782 53.4677320473855
1783 54.3291017040877
1784 54.2817488809723
1785 53.692786644219
1786 54.7275488360171
1787 54.4627396473023
1788 54.1032766239132
1789 53.9649431654697
1790 53.1242914652288
1791 53.2811301777431
1792 53.004137001494
1793 52.2644562571108
1794 52.1728197124483
1795 52.7883543425051
1796 52.4421353249422
1797 52.6751522042047
1798 52.1751593706516
1799 52.0221630917753
1800 53.1978915339601
1801 53.6350366178384
1802 52.9814789648248
1803 53.2445038775426
1804 53.5128142282993
1805 53.7861855315114
1806 54.5969471432175
1807 54.8444976079993
1808 55.4018261520924
1809 56.8118002279917
1810 56.3309869202978
1811 56.0348869458279
1812 55.3177341412533
1813 54.8432168183718
1814 55.0355419418219
1815 54.4541426959478
1816 53.1714776696283
1817 52.5874820670117
1818 52.2519817634086
1819 50.7283077785461
1820 50.2287649487347
1821 50.0613592462771
1822 50.3934042799829
1823 51.5415033154502
1824 51.2697551668316
1825 51.9061297885927
1826 52.2913162070575
1827 52.5460459367094
1828 52.2377407951038
1829 52.6795498751537
1830 53.7308038127262
1831 54.1017145264167
1832 54.1814083411877
1833 54.0147396142059
1834 53.767451720456
1835 53.9486324168343
1836 53.9341232659128
1837 54.3673039944653
1838 55.3524140939375
1839 54.6736083061152
1840 54.3875444873053
1841 54.0085949518147
1842 54.5564427169884
1843 54.4512727832592
1844 54.4079972048405
1845 53.8863414469495
1846 53.2174218973699
1847 53.1940286672304
1848 52.1262706754667
1849 52.7487538831798
1850 52.209600177681
1851 51.8788428341018
1852 52.0239861858246
1853 51.753178969861
1854 51.9669300485197
1855 51.8827254482009
1856 52.3264234862122
1857 52.4036972581169
1858 52.5199448940628
1859 51.9035104241093
1860 51.9701613546432
1861 51.9546776408429
1862 51.2650857615732
1863 51.6837307887304
1864 51.6015502622662
1865 51.8271347920896
1866 52.6625369426283
};
\addplot [semithick, crimson2143940]
table {%
0 36.4203301904553
1 35.6055282957851
2 34.7886841998414
3 35.3898667297123
4 35.6222633376691
5 35.451787509998
6 35.3800440037139
7 35.6411072732787
8 36.2436087859867
9 35.3527171661116
10 34.0269255801886
11 33.6702021323408
12 33.793170686367
13 33.5722851897054
14 33.4053841527086
15 33.3334499223128
16 33.364649702845
17 32.6369615271342
18 31.8896152422066
19 31.8203222743849
20 31.957813007844
21 32.4952791981748
22 32.7662926917642
23 32.8646836930156
24 33.4805444258586
25 33.9918907412296
26 34.0581665366636
27 34.5405892327722
28 34.2611928701999
29 34.6647649470288
30 34.9593889841522
31 34.8541669094576
32 34.4717662803019
33 34.5068798786886
34 34.8838506950308
35 34.9797289209886
36 35.6683682051322
37 35.9016389216034
38 37.5653914597369
39 37.2543820956612
40 37.6229987088811
41 37.6853301866443
42 38.1241161697962
43 37.90594587561
44 37.8938850827797
45 37.6092767649944
46 37.8189919066777
47 37.4446992760833
48 36.2412579619632
49 36.2368828387041
50 36.158500605904
51 35.8786365192285
52 35.2952601354856
53 34.9583260819332
54 34.4865140794926
55 33.9473014245967
56 33.1714239671662
57 32.8118892044159
58 32.5587960079124
59 33.1877230897747
60 32.3218083915332
61 32.3374700415436
62 32.611562046087
63 32.417650061205
64 31.9039701616002
65 31.5741807995776
66 31.1503576925206
67 31.4951498606757
68 31.7378030622588
69 31.5728015910617
70 31.5985583180938
71 31.9270938985406
72 31.9629656594467
73 32.2878015403635
74 32.9360036898846
75 33.6196235451242
76 34.3958630150481
77 33.9897989135532
78 34.0296444302209
79 34.2879606360402
80 34.4203288917681
81 33.8616454478872
82 33.3635208715591
83 33.2966021077053
84 32.5286448903985
85 32.2967777547793
86 32.240350909284
87 33.3454877949647
88 33.5574646279268
89 32.8321295428727
90 33.822029247722
91 34.2070837883743
92 34.3363089496099
93 34.6027264293577
94 35.249326943305
95 35.8257490678084
96 35.3299084238741
97 34.4913209830792
98 34.5807784122286
99 34.1991766260378
100 33.3382744403591
101 33.4128439493687
102 33.7578693046608
103 33.3972100308291
104 32.8952004784355
105 32.708472001013
106 32.9664208075763
107 33.4344676197495
108 33.1977620436214
109 33.9589300716831
110 34.6809652103082
111 35.0198880849224
112 35.696562745367
113 36.3237341646362
114 36.5426373951222
115 36.6871570767109
116 37.1054712928457
117 36.7271463666385
118 37.0722295762306
119 36.577736985528
120 36.1657068200797
121 35.9712453223061
122 35.3877470723196
123 35.2427907306967
124 35.7229638835533
125 35.4546527405218
126 35.1823182146707
127 35.0557855061419
128 34.5543279857027
129 34.5243693155119
130 34.6422691528405
131 34.3212995652777
132 34.0804565757499
133 33.5968762545213
134 33.7551885737272
135 33.416430048788
136 33.5772468698105
137 33.8784139379117
138 34.4876745254557
139 34.7209593470657
140 35.4497170733532
141 35.6748975211674
142 35.7694765884101
143 36.6897196503022
144 36.34164016568
145 36.9955477760818
146 36.9985490961506
147 37.293959648166
148 36.4096526449403
149 37.1401552103565
150 36.0859936202006
151 36.3388854327651
152 36.4433927589144
153 36.0430139430497
154 35.62406270134
155 35.2901771038726
156 34.9933230715992
157 34.8636034356322
158 35.1591584839328
159 34.7993195432663
160 35.6734569154305
161 35.7612330972464
162 36.1049299904264
163 36.0099110193373
164 36.521879651835
165 37.42653431165
166 37.2225849589541
167 36.779657686297
168 36.9114689217795
169 37.1000530128032
170 36.3848819871755
171 36.446273779202
172 36.1629403526608
173 36.0249211810775
174 36.4247799277939
175 35.8407423346319
176 36.08214222251
177 35.8705869496939
178 35.6270449574248
179 35.1415863746482
180 35.1541480026178
181 34.9960314105618
182 34.9288696671414
183 35.6263360808614
184 35.5323801919776
185 35.6929140444356
186 35.440993096394
187 36.5267732150037
188 37.0154864989468
189 37.327255173165
190 38.054710813862
191 38.1985389413814
192 37.8755839821896
193 38.7132463108856
194 38.1590984968739
195 37.9807983671009
196 38.2603030954223
197 37.7384121262481
198 37.3164545211637
199 37.353139442975
200 36.8068449797167
201 36.4236833614073
202 36.7809024081814
203 36.1477041081356
204 36.4221957548111
205 36.8301244610945
206 37.3300138072129
207 37.5118542767984
208 37.6947419142485
209 37.7680115972633
210 37.7875567055278
211 37.8394388429499
212 37.9527780698249
213 37.1947641044396
214 37.1336877538512
215 36.4916604501506
216 36.355835179865
217 36.343724895314
218 36.490481336542
219 36.3243786242193
220 36.545738821792
221 36.3937297741492
222 36.4585712321069
223 36.7808561992998
224 37.0113559510585
225 36.944954095859
226 36.4618394693613
227 37.0717915078556
228 36.9041945244906
229 36.77048815401
230 36.6945668525607
231 37.3309040513217
232 37.5560024355003
233 37.3841509529389
234 37.0567053634334
235 37.6236334653867
236 37.6486732161409
237 36.7002884559375
238 36.8877932541815
239 36.7154311131652
240 36.9120676053006
241 36.7944011902953
242 36.3287351167998
243 36.2992810464564
244 36.6630827507813
245 36.2524543145505
246 36.8179634719627
247 37.1967258302708
248 37.0688202377673
249 38.3151330004975
250 38.0647061379478
251 38.101763430833
252 38.844483992699
253 39.5717617202421
254 39.3468971616283
255 39.1151990117463
256 38.890566531782
257 39.14142120289
258 39.3514647405174
259 38.5650165554739
260 38.3766458329197
261 37.9210311558969
262 37.1039606083249
263 36.4025477334291
264 36.3685625001849
265 36.1283702936584
266 35.7869533557508
267 35.6600375367922
268 36.4125870476168
269 36.5141434348465
270 37.2286198394715
271 37.4342670762295
272 37.9161600147542
273 39.3311642623259
274 40.0074980825263
275 40.1735758169077
276 40.5832396328398
277 40.8876648724062
278 39.9623829297371
279 40.191814244616
280 39.9154265682538
281 39.9698528525964
282 39.8651756949929
283 38.5331514937632
284 37.6557446939954
285 38.0187181321967
286 37.9092580866977
287 37.1810402909498
288 37.4282473271246
289 36.58465558798
290 36.4812933100334
291 36.5303006693621
292 36.7668788028087
293 37.1845809253356
294 38.1788950836801
295 38.4621731066267
296 38.0818264647906
297 39.0501997002873
298 39.1468323592929
299 39.8618948106985
300 40.3437600647336
301 40.911808355373
302 41.584129381241
303 41.3672550780178
304 41.1341799789881
305 41.9747093806771
306 42.9091504070013
307 42.2210603656021
308 42.2356057480264
309 42.5731398371574
310 42.1885500774374
311 41.891651563627
312 40.995342825203
313 40.7402612736802
314 40.5318306049003
315 39.5888280890926
316 39.2733233895357
317 39.378409367907
318 40.1864016545639
319 40.0150448964906
320 40.2848207179892
321 39.9829560031124
322 40.0943051032437
323 39.6998174198571
324 39.2425400392677
325 38.5755800838501
326 38.1284777185346
327 37.3599065750179
328 36.9159307580898
329 36.1936784848614
330 35.4972667195067
331 35.4139743327023
332 35.5196280443513
333 36.0415212233791
334 36.1737983227977
335 36.7205195467435
336 37.18698855617
337 37.7301075095368
338 37.0843981339044
339 37.3205733922704
340 37.8184298098179
341 38.3455658640542
342 38.4011625551893
343 39.3081095451658
344 39.538084623669
345 39.5495921056381
346 39.5584395977231
347 41.0786669730211
348 41.1576104112047
349 41.6483683877503
350 42.2291862524177
351 42.4804597508619
352 42.7478442196612
353 42.7278514861116
354 42.3291674653631
355 42.0348929344832
356 41.7466302956162
357 41.0536703143415
358 41.1466930791136
359 40.8708953343749
360 39.5825526980879
361 39.2751413516289
362 39.444131669247
363 38.5444229301326
364 38.7264643914465
365 39.0748024684791
366 39.3347599498344
367 38.7668558788794
368 38.6070610692754
369 38.8288899538417
370 39.0765825738204
371 38.7029645309067
372 38.5894213055878
373 38.6239281009448
374 39.089101796389
375 39.2723632309938
376 38.9584696519947
377 38.5010627330433
378 38.891243609303
379 38.8366356396482
380 38.9815164327089
381 39.4289297657089
382 38.5392930379122
383 38.5616766471184
384 38.7926323379375
385 38.9973135779175
386 39.3389098603068
387 39.8951982156787
388 39.9080099649307
389 40.0067674334758
390 40.6716009664906
391 40.318495308067
392 40.6482362434731
393 40.484428190796
394 39.8866336387378
395 39.2887523459816
396 39.8782390168616
397 40.4665284280639
398 40.3297921090894
399 39.846791548949
400 39.4100957520482
401 39.4684612258819
402 39.5394844589397
403 40.9500567930091
404 41.0604265600656
405 41.7268666453465
406 41.8045395923307
407 40.8729585950853
408 41.3706443765843
409 41.9636691667589
410 41.9700972905392
411 41.5976800144076
412 41.3745539154717
413 40.1704872898047
414 39.8283097088837
415 40.1519714803023
416 39.2094966112127
417 39.3706304528472
418 38.8637543948829
419 38.2073466176117
420 38.1637353638268
421 38.4526407104637
422 38.6108340687685
423 39.0782287050767
424 40.3572088970905
425 40.7505535081132
426 41.1529955638693
427 41.5757189149725
428 42.0547837719857
429 42.4598701517007
430 42.6802196367756
431 43.1815602789457
432 43.2960432185323
433 43.1069667812241
434 42.2237633101964
435 41.119310420168
436 40.3886471309906
437 40.2653772920746
438 39.9532910738653
439 39.3824898868684
440 38.8897907629567
441 37.9811553749947
442 38.3565477729131
443 37.8325517601679
444 37.6687194291721
445 37.551330171958
446 37.9966159503466
447 38.1415810162718
448 37.9136723505683
449 38.4696865821224
450 39.2137682389481
451 40.2758210876629
452 40.2674932428555
453 40.5531862859696
454 41.7242035863005
455 41.5327066318281
456 41.2266310971008
457 41.3757791419826
458 42.022841708293
459 42.0235834303532
460 41.6932164545169
461 41.3588584818509
462 40.7589888453425
463 40.9434234586462
464 40.5315847795665
465 40.9320856807807
466 41.4492536199016
467 40.6898594852027
468 40.2382720100333
469 40.8623971615738
470 40.5477642118309
471 40.687187102327
472 40.7948143282056
473 40.7474825012919
474 39.6742809057226
475 39.1907011332088
476 39.6071371566755
477 39.808348254073
478 39.748679894521
479 38.8151462933126
480 38.6532261209324
481 38.323991960537
482 38.8350967459142
483 39.5154749012815
484 40.5704076092331
485 41.1559270278328
486 40.8139120483598
487 41.2303596016765
488 41.1981444244903
489 41.716713597603
490 42.5776375801726
491 42.6122816168749
492 42.5039916202607
493 42.2015612359822
494 42.3723511723818
495 42.5592227445807
496 42.3924884221639
497 41.9408576709595
498 43.1773017497965
499 42.9503432044503
500 42.5830727337507
501 42.7708570672811
502 42.1820991900058
503 42.0141170280238
504 41.2358388067789
505 41.4895402752884
506 41.2228495920971
507 41.2863689848824
508 40.0384614563375
509 40.206931237569
510 39.905433823885
511 39.3170500564644
512 40.5106032873686
513 40.29624307835
514 40.3510734111651
515 39.822103626626
516 40.0005952868299
517 40.1928601567278
518 40.9076526726298
519 40.9915463868737
520 41.9658937303841
521 42.84193893801
522 41.8827087472817
523 42.4803697459318
524 42.7148522708235
525 42.7267697111453
526 43.019092054112
527 42.683228707944
528 42.1515692579924
529 41.8493194473416
530 42.6264975873742
531 42.8762070446582
532 42.8588539010204
533 42.2113899084567
534 42.3432712828242
535 43.2867607505099
536 43.47207652115
537 44.356723312815
538 44.7848455167943
539 44.7706951495966
540 43.7538634170689
541 43.765107867893
542 43.4460938710373
543 43.7099656477733
544 42.9174113897306
545 42.3016500147947
546 42.4444021922172
547 42.1984453210148
548 42.2048358522643
549 42.8828644201678
550 42.1642419425535
551 41.2619279534068
552 41.246095672081
553 40.9311480287067
554 41.2427949906747
555 40.0183359189541
556 39.7966031107016
557 39.5767897161709
558 39.7088745793513
559 39.0344485007252
560 38.7580609104709
561 39.8407975523734
562 39.7736474561824
563 40.6699799877905
564 41.0986205361663
565 43.367035461913
566 42.9909910784946
567 42.764186761551
568 42.7522824311365
569 43.4097484565488
570 43.8836901411656
571 42.8379032072182
572 43.2629621692952
573 42.0154520710676
574 42.3890017091811
575 41.8498122015967
576 42.3028013072179
577 42.1686876275235
578 40.9979088916252
579 40.0820225723599
580 40.6684191382582
581 40.8225416959628
582 41.3942079166205
583 42.2555602951537
584 41.4131449479434
585 40.4815061325754
586 40.2870299800862
587 40.1483513007414
588 41.6291306795476
589 41.4349373183742
590 41.8623116898648
591 41.9613698180975
592 41.51418582371
593 41.7402826905834
594 42.1614950403271
595 42.7314724460011
596 42.9363151631292
597 43.1598500024661
598 42.1556931236118
599 42.7979825441645
600 42.1047895829092
601 42.1031706934238
602 42.0908493445884
603 41.0714292191609
604 40.7676427573641
605 40.9209289988698
606 41.3659029897886
607 41.6833944473712
608 42.7142302425719
609 42.357120527813
610 42.5448954700172
611 43.3065812015311
612 43.4612001626417
613 44.2725597393139
614 45.0023235531748
615 44.2630617994559
616 43.883739576366
617 44.1652285233133
618 43.2174278324294
619 43.568005177328
620 42.747105338377
621 41.6716347439556
622 42.3155033704202
623 42.6715548447368
624 43.2856864685798
625 43.5790293997045
626 43.0925655696122
627 43.6289589141105
628 44.2287064548128
629 44.9557882953234
630 45.6484028664743
631 45.8810564550522
632 45.3589286461385
633 43.9103509952859
634 42.6416716100805
635 42.9210104796332
636 42.3812240420296
637 41.7126579758989
638 41.0789873626727
639 40.4770688769809
640 40.4934505321594
641 41.0420422493618
642 40.38656008971
643 41.1721395764823
644 41.8103082331376
645 41.3304458957742
646 42.1061771585082
647 41.8413432499941
648 42.4013459705557
649 42.0308127830487
650 41.9035436441944
651 42.2509115114547
652 43.3647210467093
653 42.6977798206786
654 42.9849575189433
655 42.9810278330567
656 42.6946420812914
657 42.7548523540583
658 42.6573968537957
659 42.7439065452053
660 43.518403637538
661 42.3441543188275
662 42.6200216255893
663 43.425447872005
664 43.2072426229924
665 43.2326384390638
666 42.841698169354
667 42.7537155934106
668 42.6723267021381
669 43.6548556470729
670 42.8236274285482
671 44.2854878473278
672 43.3757434275334
673 43.10610989347
674 43.0885670856711
675 43.2824429406866
676 43.3501389157688
677 43.4632174681693
678 43.3782648621747
679 43.3016832549601
680 43.7250742974438
681 42.7231076548324
682 42.8141661809478
683 43.1423258744807
684 42.82044618537
685 43.3079174078493
686 44.0995936604783
687 44.0675396396941
688 45.0502257132735
689 44.9208386559615
690 44.5840609258254
691 45.3391914313672
692 45.2754152179408
693 46.4747109345095
694 46.0195640867993
695 45.8847498169731
696 45.6926613233659
697 46.379302427935
698 45.7325505089437
699 45.0413210234928
700 45.0866237257833
701 44.8462195453087
702 44.9682605464105
703 43.3811038392581
704 43.5535672650801
705 43.9206818727536
706 44.3075629686309
707 44.1591905307262
708 44.0638122978657
709 44.4854589521188
710 45.0790460302129
711 45.0374102589862
712 44.897244042699
713 45.8394861457904
714 45.5789739479439
715 44.505354406261
716 44.6408360404854
717 44.6061298775331
718 43.8423662443589
719 43.3434430699068
720 42.3736082027026
721 42.0175484685976
722 42.9750352986299
723 42.3320774254184
724 42.6492477787142
725 42.8887329222932
726 43.6051459622236
727 43.561852127526
728 44.4104251365245
729 44.8414652544813
730 45.1163076691899
731 45.1672330849945
732 44.0211526158482
733 43.6701131423907
734 43.8810770198734
735 44.6570077248239
736 43.4873745790372
737 44.7270550997335
738 44.2445727955868
739 44.0542987119072
740 43.8240081987987
741 42.9436856427639
742 43.6458515804924
743 44.1623245954559
744 43.4854635809208
745 43.2190886644001
746 43.1177879260553
747 41.562912768267
748 42.3458840919288
749 42.1697040817965
750 42.9772993809346
751 44.3421443953991
752 44.0440268562383
753 44.3818709246054
754 44.8144030727488
755 44.851240103243
756 44.619942224587
757 44.6437806374894
758 43.9393218252261
759 43.6457560400608
760 42.6706959461659
761 42.328131965921
762 43.106456920718
763 42.9714097772442
764 43.0969471794892
765 42.6168482327451
766 43.8464078214789
767 43.9426741192903
768 45.2433517542059
769 45.542155547577
770 45.4572962693723
771 44.787645294077
772 44.6069745848538
773 43.9250342673093
774 44.1683954932686
775 44.0028860571265
776 43.0348240502378
777 43.2423578051625
778 42.1043756635166
779 42.070226394924
780 41.9934166480439
781 41.6539936284402
782 40.9195879400223
783 40.5350464588538
784 40.4867975064779
785 40.9201002379131
786 40.8773014875737
787 40.1991743978441
788 39.4634695506299
789 39.6556362939929
790 41.3822090441172
791 43.0657052054288
792 43.3590633951054
793 44.2568449336948
794 44.1714548081582
795 43.9066991998416
796 44.039520545972
797 44.4341858604423
798 45.4307033665907
799 45.2680878946445
800 44.0244693441637
801 42.7220099658422
802 43.7288930926411
803 43.2501941182422
804 42.6831098762081
805 43.0426371592003
806 43.619974807181
807 43.5713488772039
808 43.8569376712211
809 43.4518012229552
810 43.1092526462465
811 43.8634977713192
812 42.262278107938
813 43.0068867997483
814 42.9134861256495
815 42.9039228225267
816 42.8658173024704
817 43.4710021052658
818 43.2190493864528
819 43.7191663542409
820 44.2297721937482
821 43.7431369606913
822 43.7119063169373
823 43.4662028360809
824 44.1184925214768
825 44.3693451456537
826 44.1498061210495
827 43.4192522444619
828 43.3005570559683
829 43.6781680195407
830 43.8099594148133
831 43.9878315017928
832 45.1140985772094
833 45.7818999224881
834 46.2588159987291
835 46.049527512465
836 46.4898425815684
837 47.2860175779402
838 47.95830714303
839 47.8656447442907
840 48.165115733432
841 48.5426567818306
842 47.8727623456309
843 46.7398725218565
844 45.9638203693862
845 45.4830673778588
846 44.4938925315958
847 43.9506933270551
848 42.4835155326785
849 41.7088776150712
850 40.7079350726848
851 41.0080806518592
852 40.4208190291614
853 40.4931940128179
854 40.7422150611882
855 41.7600611040222
856 41.7019772033315
857 42.4113576503654
858 42.9171945709167
859 44.0178107418289
860 44.5031757593953
861 43.8345144487681
862 45.5281009165008
863 46.5491974808612
864 47.2917332706088
865 46.4884263888348
866 46.3468783134257
867 45.6262121788835
868 46.1646363478288
869 45.5137106008711
870 44.9902903448549
871 44.6060537267933
872 43.6459369803711
873 43.0239013991672
874 42.0894109729983
875 41.8772869480651
876 42.1203153965956
877 42.7551466471084
878 42.4995313917451
879 42.5659892601116
880 44.1742781426504
881 44.569382134517
882 45.6965712691206
883 45.4740838955675
884 45.2018874906476
885 45.3006302660723
886 45.3972328570844
887 44.5900344173
888 44.4250362772975
889 44.5591476614974
890 43.5247204674976
891 44.0183077362199
892 42.6245447418437
893 43.21762552878
894 44.0808136896932
895 44.2156511841667
896 44.1297586082846
897 44.0020412571947
898 43.4753924218329
899 43.5227941357021
900 44.2534383260656
901 43.5676929454998
902 43.9998485181225
903 43.6624195073544
904 43.0108903367284
905 42.8627078371069
906 43.1552429569446
907 43.7804381908064
908 44.1438495999146
909 44.3136381809716
910 43.19096929874
911 43.6202530368845
912 43.4232626442099
913 42.7505673971781
914 42.6045238740528
915 42.8779199254077
916 43.5093073781357
917 43.0541829475819
918 43.0135215804829
919 42.2206823688406
920 42.5968745251475
921 42.3095783327312
922 41.725685114956
923 42.4051865352047
924 43.2839010457891
925 44.5244596438701
926 44.1097906585787
927 44.1974878949112
928 44.7591851714648
929 46.3790704328542
930 47.0628566349533
931 47.6197560480232
932 48.610860893051
933 48.3818762169802
934 47.7053959144938
935 46.5933443743692
936 47.6498753539464
937 48.1543516892918
938 48.0716308168346
939 47.5751729346065
940 46.6245017827374
941 46.0603883015013
942 45.7785241594317
943 44.9791190127201
944 44.8210940424827
945 45.0736689500636
946 43.4287447683154
947 43.0532384673127
948 43.0439524460447
949 41.5699734289923
950 41.9008235942327
951 41.289116716322
952 42.0309158397405
953 43.1329664527891
954 44.2028330964312
955 43.3526158719674
956 42.9376553819097
957 42.9895433150877
958 43.8233533815123
959 44.9849864484471
960 44.6395523742457
961 45.1361925216498
962 44.3724470200074
963 45.1972799741293
964 44.9162744967692
965 45.6066987266635
966 46.4315139754156
967 45.9378253947465
968 45.9017102977046
969 47.2637502721001
970 47.8895007668736
971 48.9955342861356
972 50.0243385924691
973 48.2348212126865
974 47.9046246914337
975 48.0935163590751
976 48.1994519855997
977 48.6334831929358
978 47.1252132276062
979 46.2179240403598
980 46.9024449065415
981 45.8653012017356
982 44.7323662807728
983 45.7379327065469
984 46.2875463355854
985 45.7136248483232
986 45.2920511428305
987 45.378043954155
988 45.4164514627128
989 44.8272329277889
990 44.12375302017
991 44.5555026067815
992 44.5272361607493
993 43.7178964045307
994 44.5925770534098
995 44.9322045408969
996 45.5367813184455
997 45.9532219836974
998 46.508716610445
999 46.6654998894184
1000 46.8106008101425
1001 46.6105524748491
1002 46.2791517672419
1003 46.5941838511473
1004 45.2747520433471
1005 45.6836995608296
1006 44.5809287130985
1007 44.4189728346798
1008 43.9933040868392
1009 44.3102242132743
1010 44.1724906179264
1011 43.6217501960205
1012 44.6333730636897
1013 44.7688262440307
1014 44.8142730183748
1015 44.64169632515
1016 45.9932729925008
1017 46.1975480855902
1018 45.9558511650327
1019 45.4448252995617
1020 45.0794541778744
1021 46.293465134376
1022 45.25759729721
1023 45.4903050112141
1024 45.8458510582245
1025 45.9216354733321
1026 45.3312164260374
1027 44.6349307568158
1028 45.7471636396853
1029 46.2416930097485
1030 45.6255287237448
1031 45.5180469616332
1032 46.9059311179922
1033 47.221561372372
1034 46.0114650302435
1035 45.1808721910724
1036 46.437889872055
1037 46.8459545833843
1038 46.4798892617293
1039 45.8319395097242
1040 46.76072815821
1041 45.6679308804543
1042 44.7716707065632
1043 44.5242325884581
1044 45.0743851985726
1045 46.0746526972423
1046 45.4480399201514
1047 44.4419333205261
1048 44.4367888601999
1049 44.765178501075
1050 43.920627105232
1051 44.5002627705498
1052 45.3715173478377
1053 45.0876163200952
1054 45.270457550405
1055 45.2916891537161
1056 45.3737208494392
1057 46.1334108551202
1058 45.6833845315104
1059 46.1430751195523
1060 46.7105171191003
1061 46.3604232208456
1062 45.6713485619359
1063 45.2664437830829
1064 45.705180869581
1065 44.3314041722819
1066 44.1882016381014
1067 44.3962507020686
1068 44.3457524660764
1069 43.5879477813896
1070 43.9440518109046
1071 43.9060955434438
1072 44.660441161205
1073 45.2206129675866
1074 44.8455410046139
1075 45.7822703062202
1076 45.2403992045351
1077 44.8886323218206
1078 45.8628781317171
1079 46.2486515660055
1080 45.7674820212089
1081 47.6765093222483
1082 47.639938907213
1083 48.8200892088662
1084 48.7145412444234
1085 49.0162160166763
1086 49.0524436033745
1087 48.9364151545778
1088 48.5116131475058
1089 47.9779255805723
1090 47.8894809482366
1091 45.7594606170446
1092 45.2980816154219
1093 44.3363403165377
1094 44.3064817133718
1095 44.4076621024804
1096 44.455443934221
1097 45.0761194930828
1098 45.2323915864335
1099 45.9467874536654
1100 46.6952684231662
1101 47.4156666027781
1102 46.4016198392192
1103 46.0401685521971
1104 47.1139585577135
1105 45.887766585098
1106 46.2249973601315
1107 45.3395126681163
1108 45.4007783542921
1109 44.6361716631692
1110 44.4922119089826
1111 44.1631249173537
1112 44.4969138318981
1113 45.2528806080537
1114 43.7501603162706
1115 44.161166726027
1116 43.5153610140962
1117 43.2489338766003
1118 43.0953616438669
1119 43.215814603064
1120 43.6415284038724
1121 43.7183928024089
1122 43.5011350276866
1123 42.6539036932241
1124 43.6489107118908
1125 44.3174012577945
1126 44.4535241919392
1127 44.744516867155
1128 44.390849670431
1129 44.8543600158357
1130 45.5697798720343
1131 45.457790307336
1132 46.3925622345804
1133 46.6422030223276
1134 46.2989140753647
1135 45.9954218839745
1136 45.7830070791975
1137 46.3397772331458
1138 45.9042781175124
1139 45.1870346789092
1140 44.2577591181966
1141 44.3436677744396
1142 43.7211587484419
1143 43.3657842919118
1144 44.0123162027783
1145 44.3090657767643
1146 45.1721507601345
1147 45.4077341254468
1148 45.8225307591824
1149 46.6188892679895
1150 46.1848697618352
1151 46.8807179822779
1152 47.2830570497273
1153 46.5273312467598
1154 45.2883503925692
1155 44.4420098583271
1156 44.1731274384213
1157 43.4793939661302
1158 43.6035227609658
1159 43.0270805145089
1160 42.9619295594173
1161 41.8681849769762
1162 41.7920878477088
1163 43.2473117828894
1164 44.4227509665324
1165 44.4851086086003
1166 43.5423443950598
1167 44.3087430765729
1168 45.0383168579724
1169 45.4607413978341
1170 45.8134766616247
1171 47.3056102859219
1172 47.5764095530701
1173 48.1351149645172
1174 46.9267375310151
1175 46.9335861085145
1176 47.9490146455028
1177 47.7712582219476
1178 47.4107143768719
1179 47.1154077833765
1180 46.7293209132248
1181 45.534269010993
1182 44.3582450099727
1183 43.9610700298331
1184 45.0893383139213
1185 46.0406828820154
1186 46.02881134016
1187 46.1556457704288
1188 45.5240519732018
1189 45.2680222664432
1190 45.3457234602019
1191 44.9022985699086
1192 46.1026022329137
1193 45.1922760937065
1194 45.8452216572103
1195 45.1957483339502
1196 44.8954602342698
1197 43.9006484643959
1198 44.4846853198379
1199 45.3840147999489
1200 45.5800035240227
1201 45.5725602850974
1202 45.5129906108442
1203 45.6903751537287
1204 45.1324044689345
1205 45.6226104436623
1206 46.1508308562041
1207 47.6037753709783
1208 48.1668379048678
1209 47.6799557707267
1210 47.2947596078598
1211 47.7772338356105
1212 47.0978079267925
1213 48.2249486479301
1214 47.570948076795
1215 46.9573498027093
1216 46.3853293987601
1217 45.4988163003195
1218 44.2456031255938
1219 44.0030216288677
1220 43.660419078169
1221 43.4632346889443
1222 43.447055853115
1223 41.7930154983302
1224 41.5591425312589
1225 41.9825956208922
1226 42.758005748625
1227 43.3768415181714
1228 43.9943285080767
1229 44.8332966951942
1230 45.0927405868019
1231 45.5988027472909
1232 46.1964089185356
1233 46.3428701871301
1234 45.7132400136341
1235 44.8973546758049
1236 44.0017276368719
1237 43.4162046883934
1238 43.1804417122149
1239 43.146195770445
1240 43.9124297275944
1241 43.4456034426871
1242 43.1981296374755
1243 43.0585416144725
1244 43.7483129640238
1245 45.0736336738716
1246 45.4397337309833
1247 45.3328700512432
1248 45.4952575663731
1249 44.3074759029602
1250 43.5010366164826
1251 45.1246300376343
1252 45.3292830021789
1253 45.758296498764
1254 46.0130635503699
1255 45.3452516646883
1256 45.5133348033644
1257 46.756519842638
1258 46.2393396512009
1259 47.2176251581248
1260 47.5805716245974
1261 46.5868473063636
1262 45.5685589593228
1263 47.6792107683588
1264 47.8745983960799
1265 47.5384202698552
1266 47.2314909943897
1267 45.9227543093825
1268 46.7563705910221
1269 46.2967905686226
1270 45.8556073402512
1271 45.866083779385
1272 46.5152963580275
1273 45.4544630312558
1274 45.6351995761038
1275 45.3192776864244
1276 44.8389907017119
1277 45.6163284444441
1278 45.8402027599427
1279 46.7338240083574
1280 47.3351332762213
1281 46.9286073100913
1282 46.8880135456736
1283 45.9655085122319
1284 45.9299887457917
1285 47.8582275407652
1286 48.0421393490863
1287 47.3616768408095
1288 46.1532254160792
1289 45.8701908282123
1290 46.2493386264707
1291 45.7631592604093
1292 46.1246297127479
1293 46.4950818431368
1294 45.9020880712109
1295 44.7122897826889
1296 44.9487924667755
1297 44.9240315107407
1298 44.3533661834193
1299 44.3584663499053
1300 44.3060243599782
1301 44.8744145311376
1302 45.0980853831558
1303 44.1767218169025
1304 44.9023633794098
1305 44.5110210327967
1306 45.1582792929407
1307 45.5254631346899
1308 47.406316421101
1309 47.6616022779666
1310 47.9459440785339
1311 48.6778548189707
1312 47.5514853846739
1313 49.1622408960312
1314 48.3284144864848
1315 47.792549537567
1316 47.5439536712936
1317 47.4078114667272
1318 46.4601505710075
1319 45.6660587417308
1320 44.9833909046008
1321 45.0161804697354
1322 45.4882421552112
1323 44.3458821427632
1324 44.0326349233233
1325 44.9149873614505
1326 43.3784806568702
1327 42.7964413494455
1328 42.8951591660368
1329 43.1153240255284
1330 43.3032543794525
1331 41.8948456797406
1332 41.7733291198347
1333 41.1515936521746
1334 41.8376309028432
1335 41.4461980993132
1336 43.0533815651966
1337 43.4468858009058
1338 43.9055923431617
1339 43.758385600658
1340 43.1351642244078
1341 44.0168979447467
1342 43.6171480001749
1343 43.9912322761207
1344 43.6774780302964
1345 44.2158712023292
1346 44.0289093632027
1347 43.8522445774101
1348 43.8564577524728
1349 44.4971810633588
1350 44.7638388306227
1351 44.5554384874319
1352 44.9669334659463
1353 45.5151349779492
1354 45.520477986254
1355 45.0278169993862
1356 44.3982231975565
1357 44.8222934906857
1358 43.8978458070077
1359 43.2663665620252
1360 43.3645315238153
1361 43.2306329961604
1362 42.7443948510494
1363 42.7470417187611
1364 43.309612634113
1365 44.3095549073416
1366 45.8864245099197
1367 46.3033824909937
1368 46.3248272921217
1369 45.95917301912
1370 46.9342722002935
1371 46.7779701606906
1372 49.1067056687437
1373 49.8672381594896
1374 50.6059232759308
1375 50.38746747223
1376 49.1130191869452
1377 48.498776601199
1378 49.4103322831044
1379 50.8574642196037
1380 50.1865477177865
1381 51.2193103348316
1382 50.1378788129518
1383 49.2004978332197
1384 48.3138250692065
1385 48.8057405285342
1386 49.6736347538471
1387 49.5273856754194
1388 49.0497439880799
1389 49.423777729264
1390 49.6257480881901
1391 48.6905595716695
1392 48.1695114465321
1393 48.1391208122558
1394 48.3216838234566
1395 47.6541217898098
1396 48.0993980081359
1397 48.876608150996
1398 49.1349703514438
1399 48.0827679857286
1400 48.7942542548226
1401 48.7715778222035
1402 48.781291473497
1403 49.3943387548412
1404 48.8492604653916
1405 48.5583446384946
1406 47.872627904294
1407 47.9855958130122
1408 47.5778007379549
1409 47.5701748428783
1410 47.470305007747
1411 46.9507899041821
1412 47.5271266095355
1413 47.2153073978132
1414 48.4511155386916
1415 48.1626620493156
1416 47.6084652334039
1417 47.2495477577171
1418 48.3054101383568
1419 48.6895882140935
1420 49.103523652538
1421 50.4709950326244
1422 50.1190641295631
1423 50.0841818857725
1424 48.7312344302301
1425 48.7163872108271
1426 48.9554196077741
1427 48.9127815560128
1428 48.1177972451371
1429 46.8336089336591
1430 45.7592291613239
1431 45.6395676376592
1432 44.92727554813
1433 44.9822433524209
1434 45.6288840540239
1435 46.8002995056468
1436 46.9367996730351
1437 47.0844632335602
1438 47.4493675886584
1439 49.3014174588405
1440 49.6957352559175
1441 50.2734778061595
1442 50.4528058431196
1443 50.3877188139411
1444 49.7001951221476
1445 48.5473074707821
1446 48.1989339298742
1447 48.4236380317736
1448 48.0913902087816
1449 47.0329300368937
1450 46.1227011128842
1451 44.3474610085279
1452 44.3611407078806
1453 44.3650221804843
1454 44.9061293498069
1455 45.1881048127833
1456 45.7632742926261
1457 45.7867053578018
1458 46.8805370508929
1459 46.7977315509345
1460 47.9245401502965
1461 48.4045686907651
1462 48.7968723627202
1463 49.4753037565248
1464 49.9814724270696
1465 50.624373297277
1466 50.4590154219254
1467 50.1425353097019
1468 48.4789772894581
1469 48.6079810525028
1470 48.9883050812192
1471 49.5828149734745
1472 49.2450372090599
1473 48.3927924726351
1474 47.5401611972882
1475 47.0000138431004
1476 46.795491617392
1477 46.9395519424949
1478 47.733835475107
1479 46.837454142395
1480 45.4881770871487
1481 45.5427056706924
1482 45.744310807323
1483 46.1407155155209
1484 46.333215111366
1485 46.0576270899751
1486 45.9955975654572
1487 45.7544466936654
1488 45.2218349809944
1489 46.6769349255587
1490 47.0915757793706
1491 47.01441717229
1492 46.9366892563923
1493 46.6959495592935
1494 46.292600636605
1495 47.0578551425072
1496 47.1065425381094
1497 47.2368531570111
1498 47.919444730509
1499 47.4926957641038
1500 46.9401357593297
1501 46.6262615511634
1502 46.6996361620626
1503 46.3760659363548
1504 47.521946735
1505 46.8617068950298
1506 46.5572086767306
1507 47.0345884524283
1508 46.4443334053894
1509 46.1094808653826
1510 45.9283781753121
1511 46.3662516833848
1512 46.2636640758379
1513 46.0978142076313
1514 46.0745780401414
1515 46.6076551963507
1516 46.8515354167048
1517 46.7972810795332
1518 47.1504507134362
1519 46.4630604772738
1520 46.198863283315
1521 45.7404638293467
1522 46.8249274922352
1523 47.376127685454
1524 46.921031152941
1525 46.6154627579289
1526 46.3081245073678
1527 45.4536593461599
1528 45.132826990804
1529 45.8668848089124
1530 47.0163227630398
1531 46.8384354953049
1532 46.2736842228495
1533 46.1612595655453
1534 45.8272623459031
1535 45.1006338505364
1536 45.2200004094102
1537 45.0172751404847
1538 45.3245931843802
1539 45.197597951033
1540 44.4575782079778
1541 45.3905304656855
1542 44.6851010369714
1543 44.8196691329833
1544 44.8460131548302
1545 45.5559330162896
1546 45.6073857450643
1547 44.8395747346942
1548 44.4672391559206
1549 44.156874501183
1550 43.8984787953038
1551 43.2426664998998
1552 43.190370895304
1553 43.4670793036408
1554 43.9593068565317
1555 43.6154745723164
1556 44.0941454080771
1557 44.5745476290307
1558 44.986989125945
1559 45.4371951957051
1560 45.7986332826788
1561 46.5702960901586
1562 46.9058009294974
1563 46.7102351526372
1564 46.7070712487514
1565 47.9109188958116
1566 47.281702184521
1567 47.5619769030697
1568 47.5766870085217
1569 46.8902240086078
1570 46.8811700071666
1571 46.8069060218185
1572 46.7376590500474
1573 46.406482237022
1574 45.5130614532232
1575 44.9055189948552
1576 45.0185446406831
1577 45.2951479487643
1578 45.8410076971453
1579 46.8136901939188
1580 47.2994173026765
1581 46.2870095200764
1582 46.8496381477309
1583 46.8310170805528
1584 47.5797922614122
1585 46.7882764448397
1586 46.7004729993104
1587 47.1575174702467
1588 45.9406869011083
1589 45.0936766456296
1590 44.5727350964714
1591 44.7538883424462
1592 44.2117773321863
1593 44.5315250118101
1594 43.8451837883219
1595 45.0001430738281
1596 45.2477020733188
1597 44.8396983491892
1598 45.2364038390386
1599 45.4531938374564
1600 45.7992436674063
1601 46.9586256282875
1602 47.6581951525789
1603 47.3212151562241
1604 47.9823846934947
1605 47.4414524966806
1606 47.0060120215005
1607 46.6344846591284
1608 46.8994944892048
1609 47.3162732944896
1610 47.4053653322252
1611 46.2642046604225
1612 45.8653245941987
1613 45.4253408178378
1614 44.3315179309025
1615 44.1552792205714
1616 44.7065639963854
1617 45.4248047591988
1618 44.6308256204387
1619 44.388314468601
1620 44.3365805819497
1621 44.3471346512005
1622 44.1602571717088
1623 44.6272814963016
1624 46.1723139649007
1625 46.372719446047
1626 47.4317465264815
1627 47.5879419146769
1628 48.3343154281841
1629 49.5827017741482
1630 49.395375490501
1631 50.9843437595357
1632 50.7334493515613
1633 51.3718288295268
1634 50.9100062818123
1635 51.4185288449679
1636 51.3791603862257
1637 51.9414657667696
1638 53.0802510526389
1639 52.3172588678522
1640 53.6624871198705
1641 52.1293938072441
1642 52.7574662239412
1643 52.8067201186984
1644 52.8942131346162
1645 52.5023284347847
1646 52.0061972230018
1647 52.0758765447369
1648 51.4827338949337
1649 51.99632473186
1650 50.7521948185314
1651 50.2578832386534
1652 50.4296437875161
1653 50.2948991971606
1654 50.473350439709
1655 50.7642995878321
1656 50.9584426848301
1657 49.7011885147461
1658 49.7360643518908
1659 49.4221766864837
1660 49.771160210651
1661 50.8123037427905
1662 50.4128327499159
1663 50.835775211509
1664 51.3282509232036
1665 50.8008262175686
1666 50.3060474291116
1667 51.247884052341
1668 50.7553594816306
1669 51.0640300205647
1670 51.4350465728201
1671 50.9575277927663
1672 51.1370122653308
1673 51.6056460148905
1674 50.3053248882402
1675 50.5980188181461
1676 50.7628528994977
1677 50.403122306418
1678 49.617333814897
1679 48.0736287805021
1680 47.9992936930135
1681 48.6605235217413
1682 48.3618900258613
1683 48.1971777079005
1684 48.5090605712709
1685 48.6609498619083
1686 49.0734865410524
1687 49.0768724275994
1688 49.5198574955986
1689 49.6111019873021
1690 49.2245778114504
1691 49.0419599277475
1692 49.3133632155072
1693 48.0930328450131
1694 47.9917005513948
1695 47.8310882134604
1696 46.838202543086
1697 46.7043587282306
1698 46.5066943106015
1699 46.623928886938
1700 46.7658903583471
1701 47.8648408319346
1702 48.6096064093199
1703 49.4827488575754
1704 49.0274619822817
1705 49.2965317893029
1706 49.6140898166632
1707 49.1014268986732
1708 49.1773604264573
1709 49.9077627674381
1710 49.4934135281905
1711 47.9719068635257
1712 46.9794940410312
1713 46.4102559416921
1714 46.7719270558803
1715 46.3311373957259
1716 46.3647929843089
1717 46.7559674833788
1718 46.9899006631554
1719 47.2496978826432
1720 47.2431052364145
1721 47.7599228445983
1722 47.6944696325852
1723 47.7108767482221
1724 48.0304696694842
1725 47.805315505189
1726 47.6323989229464
1727 47.4899231847718
1728 47.5731517970517
1729 47.5036776237864
1730 47.9923069121892
1731 47.1212228185251
1732 47.2686612482196
1733 47.4786641484329
1734 48.4677851453956
1735 47.8046823186572
1736 48.5581043059541
1737 48.2556975895442
1738 48.8896931678891
1739 48.1603999811472
1740 47.6561817316596
1741 48.7034096794962
1742 49.228219881743
1743 47.9779149246785
1744 47.0047476487946
1745 46.8650200613101
1746 45.4089218352701
1747 45.0261521736904
1748 44.0705173127354
1749 44.8629280258731
1750 43.7804214689973
1751 42.8227960336907
1752 42.5334305896634
1753 43.5083023756009
1754 43.199073898057
1755 44.0671039247321
1756 44.8466157461018
1757 46.3978333793452
1758 47.1352289225973
1759 47.0920995289239
1760 47.5197208483382
1761 47.8585638626026
1762 47.1450954802459
1763 47.6134555543687
1764 48.2371468055836
1765 48.5364467369322
1766 48.7522986449152
1767 48.3935104054867
1768 48.0682623944603
1769 47.479591409153
1770 49.0028030614179
1771 48.5233202348708
1772 49.0298534525869
1773 49.1084184254027
1774 48.901993159931
1775 48.9807690376501
1776 48.3538637370576
1777 47.6961401456783
1778 47.4509193554746
1779 47.3250670251255
1780 46.0911444522682
1781 46.0371383249473
1782 45.9680454261808
1783 44.9930095076372
1784 44.2656140383025
1785 43.70097714537
1786 43.4937589225154
1787 42.7087751974045
1788 42.8929954862679
1789 43.8670341050799
1790 43.5901507435642
1791 43.8376477656344
1792 43.9291194583793
1793 43.7886212007008
1794 44.5780366515006
1795 44.4251084346489
1796 45.5987526415904
1797 46.4700177242459
1798 46.1023354963036
1799 45.0291557982325
1800 45.188889493828
1801 45.777080016354
1802 45.6863448428557
1803 47.6587019172139
1804 47.8609725812446
1805 47.7804385451917
1806 48.0294110339132
1807 49.5333404144511
1808 49.6925168530768
1809 50.243754113145
1810 50.9727921108358
1811 49.9655579155963
1812 50.382560915076
1813 48.5252893702997
1814 48.1219282765309
1815 48.9263861374583
1816 48.1957942143379
1817 46.8125034068718
1818 47.4839372111706
1819 47.416581980734
1820 47.6605267691941
1821 48.4531519889133
1822 47.6909478343471
1823 47.7660170175432
1824 48.8584946008179
1825 48.6380448695119
1826 48.2246808303882
1827 47.9751651492218
1828 47.9804759962501
1829 48.1120797082661
1830 47.2654454751231
1831 47.03398489379
1832 46.6891831703999
1833 46.6039715534957
1834 45.1414132738178
1835 45.143226132769
1836 45.3209189829481
1837 45.7931735816478
1838 45.5072532855203
1839 45.2633114577268
1840 44.8482463175784
1841 44.3568421435861
1842 44.7754282218815
1843 45.0469162982384
1844 45.0028980620957
1845 44.5240111348334
1846 43.9486589646564
1847 44.71665441137
1848 44.6951707868564
1849 44.4808760313121
1850 44.7446695562646
1851 45.5051738466739
1852 45.6699224908513
1853 46.9100715928561
1854 46.8224839409019
1855 47.7783997371626
1856 48.2047577103784
1857 47.0519780138022
1858 46.6349976599249
1859 46.5131906046032
1860 46.957884221772
1861 46.291688257709
1862 46.6184778693681
1863 45.5913576396496
1864 46.6527743779494
1865 46.4155443955211
1866 46.7666728452328
};
\end{groupplot}
\node at (7,7.5) [anchor=north] {
    \begin{tikzpicture}[scale=0.75] % Nested TikZ environment
 % AdaBelief
 \draw[red,  ultra thick] (0,0) -- ++(0.8,0);
 \node[anchor=west] at (1,0) {{AdaBelief}}; % Increased spacing
 
 % Adam
 \draw[steelblue31119180,  ultra thick] (3.75,0) -- ++(0.8,0);
 \node[anchor=west] at (4.75,0) {{Adam}}; % Increased spacing
 
 % AdaHessian
 \draw[darkorange25512714,  ultra thick] (6.7,0) -- ++(0.8,0);
 \node[anchor=west] at (7.7,0) {{AdaHessian}}; % Increased spacing
 
 % Apollo
 \draw[forestgreen4416044,  ultra thick] (11,0) -- ++(0.8,0);
 \node[anchor=west] at (11.9,0) {{Apollo}}; % Increased spacing
    \end{tikzpicture}
};
\end{tikzpicture}
 \\ % Replace with the correct path to your .tex file
    \end{tabular}
    \caption{The cosine similarity (in degrees), y-axis, between the calculated batch Hessian diagonal and the corresponding optimizer approximations on a \emph{small} batch (128 samples).
    Optimizer updates are denoted on the x-axis.
    Note that these results represent only the Hessian diagonals for the network's \emph{weights}. For the corresponding analysis on biases, please refer to Figure \ref{fig:cosine-bias-small-batch}.
}
    \label{fig:cosine-small-batch}
\end{figure}

\begin{figure}[h!]
    \centering
    \begin{tabular}{cc}
        % This file was created with tikzplotlib v0.10.1.
\begin{tikzpicture}[scale=0.75]

    \definecolor{crimson2143940}{RGB}{214,39,40}
    \definecolor{darkgrey176}{RGB}{176,176,176}
    \definecolor{darkorange25512714}{RGB}{255,127,14}
    \definecolor{forestgreen4416044}{RGB}{44,160,44}
    \definecolor{lightgrey204}{RGB}{204,204,204}
    \definecolor{steelblue31119180}{RGB}{31,119,180}
    
    \begin{groupplot}[group style={group size=2 by 1}]
    \nextgroupplot[
    tick align=outside,
    tick pos=left,
    title={Log Loss (128 samples a batch)},
    x grid style={darkgrey176},
    xmin=-93.3, xmax=1959.3,
    xtick style={color=black},
    y grid style={darkgrey176},
    ymin=-3.69641950578708, ymax=1.20755818918329,
    ytick style={color=black}
    ]
    \addplot [semithick, steelblue31119180]
    table {%
    0 0.717462601981405
    1 0.672144369606316
    2 0.6153235837203
    3 0.550390758059008
    4 0.473096323957108
    5 0.400516283117529
    6 0.32178640017522
    7 0.242391884141269
    8 0.148363483930406
    9 0.0688789064754759
    10 0.00013602046910312
    11 -0.067967275956495
    12 -0.114669009180522
    13 -0.171434811690106
    14 -0.227823001785543
    15 -0.268111286569034
    16 -0.322135649705526
    17 -0.38203180093271
    18 -0.440122180670676
    19 -0.502687685643759
    20 -0.545104387773451
    21 -0.608960983775058
    22 -0.665726016936424
    23 -0.701473691822039
    24 -0.715871499382943
    25 -0.784345011087026
    26 -0.822988410422631
    27 -0.845270723969619
    28 -0.839799410030097
    29 -0.823714753697421
    30 -0.88262626615586
    31 -0.901893618195118
    32 -0.931444755520871
    33 -0.983616169089228
    34 -0.997072092761652
    35 -1.0108639230393
    36 -1.05111816441171
    37 -1.12815312743806
    38 -1.18658464689126
    39 -1.23710564166031
    40 -1.2142323055198
    41 -1.22511054116521
    42 -1.30377460394287
    43 -1.26922432928346
    44 -1.30895003513601
    45 -1.27595947577465
    46 -1.28365350290917
    47 -1.27241393267962
    48 -1.25067283982539
    49 -1.29396768937313
    50 -1.35560501660413
    51 -1.34989562701098
    52 -1.29180831661723
    53 -1.31554585256467
    54 -1.33958853957836
    55 -1.41611188289212
    56 -1.3856507784954
    57 -1.38352823735767
    58 -1.3856934229164
    59 -1.40280120399513
    60 -1.35486340491906
    61 -1.4188093010644
    62 -1.45979925655117
    63 -1.45454404094927
    64 -1.45768364969968
    65 -1.41242820528983
    66 -1.49378936032302
    67 -1.44254907667497
    68 -1.48438472352355
    69 -1.45326199988961
    70 -1.44008792143597
    71 -1.40088860583619
    72 -1.41079136495982
    73 -1.40366117716628
    74 -1.41996116484892
    75 -1.41549566178035
    76 -1.38847213936019
    77 -1.45320089782448
    78 -1.42442452695854
    79 -1.42882566010519
    80 -1.49494031815925
    81 -1.52119935420513
    82 -1.5193464701642
    83 -1.56555910229581
    84 -1.60443361385515
    85 -1.69737345557196
    86 -1.61035493299632
    87 -1.64231351175809
    88 -1.67721948240419
    89 -1.66889807254477
    90 -1.71035562727616
    91 -1.73033156628283
    92 -1.76397407080267
    93 -1.79929654005934
    94 -1.76133824756614
    95 -1.78149061910035
    96 -1.87459740454331
    97 -1.79323379606829
    98 -1.7785860934588
    99 -1.82693345563669
    100 -1.74377951555889
    101 -1.69993563423634
    102 -1.67067458787582
    103 -1.63147941294272
    104 -1.64293025456992
    105 -1.64911147276995
    106 -1.70522929086747
    107 -1.73629697844652
    108 -1.77343396511034
    109 -1.7636741403825
    110 -1.79166808309218
    111 -1.84500592351905
    112 -1.86490155600418
    113 -1.95821394569828
    114 -1.9494561533518
    115 -1.89169461787959
    116 -1.78549833572825
    117 -1.79883415890266
    118 -1.82582565683219
    119 -1.80514490379906
    120 -1.81464535490015
    121 -1.86655647525885
    122 -1.82226753804591
    123 -1.77416378653908
    124 -1.78931028517937
    125 -1.80931004081486
    126 -1.88240887531305
    127 -1.81204404687646
    128 -1.78285998081131
    129 -1.84405836650416
    130 -1.877182456835
    131 -1.84731990251017
    132 -1.90784073181228
    133 -1.90129251423123
    134 -1.91529937445976
    135 -1.8884705741455
    136 -1.90251191204322
    137 -1.96269103636823
    138 -1.97463702069303
    139 -1.90609576234506
    140 -1.90027869482016
    141 -1.87644767285072
    142 -1.82207176942857
    143 -1.85397664921214
    144 -1.84959795923585
    145 -1.90212304711983
    146 -1.91670934864371
    147 -1.94486802522374
    148 -1.93275472524437
    149 -1.98384674143751
    150 -1.98290817765741
    151 -1.94959480985198
    152 -1.95978654352049
    153 -1.8959668630094
    154 -1.95019776256021
    155 -1.96359806122124
    156 -1.97678878513158
    157 -1.91270558963632
    158 -1.83933807211702
    159 -1.80080006237255
    160 -1.82760678302291
    161 -1.8436844287977
    162 -1.93755454056367
    163 -2.02135896659973
    164 -1.96237681978008
    165 -1.89032273064294
    166 -1.82516011153422
    167 -1.93946595502576
    168 -1.99457512055537
    169 -2.04431019342348
    170 -2.00700928148177
    171 -2.08514024371817
    172 -2.0815241762017
    173 -2.03147699563588
    174 -2.03058209944474
    175 -2.16284792054282
    176 -2.24875513733504
    177 -2.164063042519
    178 -2.20673085590777
    179 -2.17936959131991
    180 -2.19460299163313
    181 -2.11110109080158
    182 -2.06914131075385
    183 -2.08140288011745
    184 -2.11177182739788
    185 -2.07262755494412
    186 -2.10650634023817
    187 -2.18498132112814
    188 -2.19420891733141
    189 -2.20548402527292
    190 -2.17004341986432
    191 -2.1532434839371
    192 -2.07723477097482
    193 -2.06876661070503
    194 -2.01122577118015
    195 -1.94366107405959
    196 -1.8998074210842
    197 -1.84867265159676
    198 -1.84754857945543
    199 -1.87464217700389
    200 -1.950610655787
    201 -1.99253343805002
    202 -1.96493317401769
    203 -1.97804777061868
    204 -1.97613700269261
    205 -2.04570939439489
    206 -1.98599574390535
    207 -1.97378870320149
    208 -1.94936527023352
    209 -1.95189718526224
    210 -1.84318718739822
    211 -1.77593321712209
    212 -1.86462635160025
    213 -1.90278613516688
    214 -1.88976240587769
    215 -1.82737899341704
    216 -1.91362581579684
    217 -1.97189658777399
    218 -1.96397540053905
    219 -1.96861193800754
    220 -2.03324736456638
    221 -2.05739445216344
    222 -2.00517561436612
    223 -1.94954043892844
    224 -1.95817567191295
    225 -2.02621818623904
    226 -1.98370116307505
    227 -1.99751491680335
    228 -2.08765063356935
    229 -2.01798896571408
    230 -2.03001800630319
    231 -2.10528969345503
    232 -2.13449997877141
    233 -2.16107207938256
    234 -2.20002767168363
    235 -2.23143303995584
    236 -2.16940641950201
    237 -2.00757957233107
    238 -1.94053138062845
    239 -2.0266885261459
    240 -1.95272853326013
    241 -1.9282496838247
    242 -1.9117669955466
    243 -1.91021128508378
    244 -1.90971060560095
    245 -1.90258008121166
    246 -1.97594954577333
    247 -2.05753632391813
    248 -2.10845332566593
    249 -2.12690757045816
    250 -2.16809101616905
    251 -2.18822928316456
    252 -2.24863464953105
    253 -2.29885931634982
    254 -2.12953133342096
    255 -2.10386159940854
    256 -2.05323307349626
    257 -2.04140226206259
    258 -2.05970783141493
    259 -1.96577818014869
    260 -2.00826033815142
    261 -2.00686343780974
    262 -1.93719255762395
    263 -1.88670924491269
    264 -2.04403086760781
    265 -2.02338434962243
    266 -2.05825722975453
    267 -2.14512955109668
    268 -2.02709849311054
    269 -2.0475645902176
    270 -2.04323587028249
    271 -1.97887055992046
    272 -1.98431743237492
    273 -1.99246468283245
    274 -1.98731330310307
    275 -2.00809604938695
    276 -2.01005773696859
    277 -2.03523540347308
    278 -2.13835358654313
    279 -2.17258279462538
    280 -2.12614687539156
    281 -2.14327117780866
    282 -2.14611294913015
    283 -2.22874266418571
    284 -2.24488007992025
    285 -2.21982321134344
    286 -2.12086007160145
    287 -1.98828580236707
    288 -1.93221053675546
    289 -1.95971961990821
    290 -2.05472617781923
    291 -2.07960243649966
    292 -2.18230057829374
    293 -2.08283485987837
    294 -1.96065667777079
    295 -1.9858413332908
    296 -2.09887428501087
    297 -2.2121506023928
    298 -2.26636647672801
    299 -2.17962514795625
    300 -2.11939679845476
    301 -2.16341364544124
    302 -2.07903712250699
    303 -2.05065034134335
    304 -2.20829185727226
    305 -2.1746756128254
    306 -2.06486109685684
    307 -2.02293634081629
    308 -2.00773402002048
    309 -2.04636629372334
    310 -1.98188022429473
    311 -1.99319635994681
    312 -2.01424205587892
    313 -2.04460560570726
    314 -2.05540055324427
    315 -2.10841676238094
    316 -2.22336233339395
    317 -2.271677162108
    318 -2.28851121523208
    319 -2.31403930335668
    320 -2.41930335675949
    321 -2.14303924169379
    322 -2.18594291823564
    323 -2.17822743054825
    324 -2.13796567417376
    325 -2.07285831309948
    326 -2.08824102525004
    327 -2.01420067749261
    328 -2.02583554397421
    329 -2.01239746775683
    330 -1.9981870578079
    331 -2.20223254219246
    332 -2.16021483001984
    333 -2.25072761021063
    334 -2.24635689694403
    335 -2.30360389009237
    336 -2.27633570879844
    337 -2.30978750152126
    338 -2.31583434584724
    339 -2.38541056758857
    340 -2.43107501805798
    341 -2.49739724515282
    342 -2.57227413256087
    343 -2.50260747528618
    344 -2.60671666327349
    345 -2.51668503242535
    346 -2.41099534972598
    347 -2.48674707409607
    348 -2.23553148208446
    349 -2.16575845255994
    350 -2.14817452066522
    351 -2.11252139666531
    352 -2.1064110953562
    353 -2.12235817804681
    354 -1.98291205805472
    355 -2.05526556439115
    356 -2.08665749730747
    357 -2.03308045576471
    358 -2.16934099411272
    359 -2.17116808989664
    360 -2.09677612642575
    361 -2.00594253375857
    362 -2.021098049559
    363 -1.96097643239889
    364 -2.01214435318419
    365 -2.02421480512331
    366 -1.98275701894633
    367 -1.99780461656808
    368 -1.99250825603273
    369 -1.99097416653181
    370 -2.09812332434276
    371 -2.11961839430491
    372 -2.07224590991288
    373 -2.04758276234633
    374 -2.09675450908126
    375 -1.99308028216897
    376 -2.08983966427695
    377 -2.10941188446225
    378 -2.15128594206367
    379 -2.09615582855558
    380 -2.02546312272474
    381 -2.03612859926092
    382 -2.04497941566486
    383 -2.09280588054458
    384 -2.08456318811955
    385 -2.08641146277762
    386 -2.10064116998534
    387 -2.11540580117319
    388 -2.14917423459034
    389 -2.21670556025543
    390 -2.31556047486726
    391 -2.47668826524678
    392 -2.46656440111536
    393 -2.46096129045277
    394 -2.36789653995656
    395 -2.4592966309327
    396 -2.36628909210357
    397 -2.29207765287599
    398 -2.24478003610898
    399 -2.27457827740307
    400 -2.25241454718552
    401 -2.14338510574015
    402 -2.15826933974894
    403 -2.2036909038998
    404 -2.27510520434424
    405 -2.32510152671753
    406 -2.3466477620473
    407 -2.40232071621091
    408 -2.36638237534682
    409 -2.30233256882517
    410 -2.26686187126917
    411 -2.33691202120672
    412 -2.34214233187959
    413 -2.35701485133987
    414 -2.35478949148291
    415 -2.19649520803283
    416 -2.1822246066643
    417 -2.18391927845725
    418 -2.27359745287704
    419 -2.29909780087016
    420 -2.35128184670564
    421 -2.25749746069269
    422 -2.25811233782288
    423 -2.24887048869336
    424 -2.25135685826128
    425 -2.35445735605024
    426 -2.41393052593372
    427 -2.48005875035825
    428 -2.4575671877757
    429 -2.39784557456295
    430 -2.22254006188831
    431 -2.28867306709219
    432 -2.29540317491861
    433 -2.27554472718693
    434 -2.24947159011881
    435 -2.22506261328921
    436 -2.19614769651042
    437 -2.18955574203959
    438 -2.11286768566198
    439 -2.20548325450374
    440 -2.3720267042551
    441 -2.28331185311649
    442 -2.26984855357653
    443 -2.31868050003581
    444 -2.34975937893057
    445 -2.41265350037133
    446 -2.37807606305775
    447 -2.35621125273501
    448 -2.48177900449604
    449 -2.3535873794638
    450 -2.34263723743207
    451 -2.35044903983124
    452 -2.34784147393082
    453 -2.31356845255007
    454 -2.24246007236225
    455 -2.18157361312224
    456 -2.24778907726437
    457 -2.28240000310206
    458 -2.25920957495461
    459 -2.30276669503301
    460 -2.28139704488135
    461 -2.40267869478368
    462 -2.4381836278614
    463 -2.42186957074927
    464 -2.49365927852735
    465 -2.60020253857558
    466 -2.56509208389033
    467 -2.39819985979954
    468 -2.39908903124183
    469 -2.44267640135506
    470 -2.47761254593948
    471 -2.4909238768553
    472 -2.47639510368253
    473 -2.58965549100995
    474 -2.62623929255237
    475 -2.61178249302651
    476 -2.68629370687938
    477 -2.79682451206888
    478 -2.75817472465827
    479 -2.7459047980041
    480 -2.77091292994642
    481 -2.67587487414291
    482 -2.70744242218582
    483 -2.59369965117991
    484 -2.56764295968825
    485 -2.57589362899667
    486 -2.54548592942335
    487 -2.63428859177827
    488 -2.67878596868426
    489 -2.60448385881313
    490 -2.56772571318164
    491 -2.58847714723684
    492 -2.52415531326113
    493 -2.54269195471329
    494 -2.56879862705383
    495 -2.6125159612152
    496 -2.55440283005867
    497 -2.50750185783878
    498 -2.53913945889561
    499 -2.57617895240122
    500 -2.58336488450698
    501 -2.61557328538419
    502 -2.69868596673028
    503 -2.76320185589635
    504 -2.78848698685329
    505 -2.7035039175958
    506 -2.79045171116036
    507 -2.50823685882365
    508 -2.49892992271542
    509 -2.56688377896076
    510 -2.57469306969022
    511 -2.58060051287166
    512 -2.58992587782455
    513 -2.63013993230356
    514 -2.64591795971998
    515 -2.63554688772091
    516 -2.61894491250945
    517 -3.01927459049318
    518 -2.81728819012562
    519 -2.76182584760777
    520 -2.78995196304412
    521 -2.75534395544563
    522 -2.5655118235996
    523 -2.54638598077836
    524 -2.42554570544737
    525 -2.43057915714345
    526 -2.39934882832661
    527 -2.34219369660415
    528 -2.44592292699187
    529 -2.53128305014884
    530 -2.57282480849559
    531 -2.61130281768991
    532 -2.71723648755848
    533 -2.68078125977939
    534 -2.82114938285973
    535 -2.88488500599104
    536 -2.89890222887927
    537 -2.97806645364436
    538 -2.88203135375201
    539 -2.73802600079773
    540 -2.64768090733817
    541 -2.69159001153699
    542 -2.65699035888308
    543 -2.61437934425707
    544 -2.61450851339221
    545 -2.52747189815735
    546 -2.576905787753
    547 -2.52794310100732
    548 -2.59576194742897
    549 -2.70156085852037
    550 -2.68125851640006
    551 -2.54564131286215
    552 -2.59411378873526
    553 -2.5896977257537
    554 -2.52809286486909
    555 -2.51113791713785
    556 -2.52774006556127
    557 -2.55470469457507
    558 -2.49134341494238
    559 -2.44397270136216
    560 -2.47269733625489
    561 -2.52451296124965
    562 -2.48490534370207
    563 -2.42012715585161
    564 -2.5154540281044
    565 -2.54520315865652
    566 -2.48370285194434
    567 -2.4404611324429
    568 -2.52630685449061
    569 -2.41037920097532
    570 -2.44176169520114
    571 -2.41746348390017
    572 -2.43370276897259
    573 -2.53648458901914
    574 -2.53150362841169
    575 -2.44923319813998
    576 -2.44144854902483
    577 -2.48496604385502
    578 -2.35262115220489
    579 -2.47447617923739
    580 -2.47590598165677
    581 -2.43076472363071
    582 -2.50706685331458
    583 -2.49501079924645
    584 -2.36938127464989
    585 -2.47594850894134
    586 -2.49362557850024
    587 -2.48104040810621
    588 -2.59369549442216
    589 -2.58531089442398
    590 -2.5147058043958
    591 -2.62094758101479
    592 -2.54753484637498
    593 -2.55367698708359
    594 -2.60507676554291
    595 -2.57951167443809
    596 -2.49076150499633
    597 -2.49189466904307
    598 -2.52231388847759
    599 -2.51874255141472
    600 -2.55749080615295
    601 -2.44652045332665
    602 -2.48219527102013
    603 -2.41651039339665
    604 -2.38390968919906
    605 -2.40647441072326
    606 -2.47743103841446
    607 -2.50079424154578
    608 -2.50401696551702
    609 -2.45409959959319
    610 -2.45841628511898
    611 -2.57596547553169
    612 -2.48481758131862
    613 -2.57935266044587
    614 -2.63699811365139
    615 -2.67957924138478
    616 -2.65859837663252
    617 -2.6292229374758
    618 -2.58529504396309
    619 -2.64962934908756
    620 -2.51816828268665
    621 -2.52513807988536
    622 -2.54671239236981
    623 -2.57510103044717
    624 -2.55935622463891
    625 -2.58410573824637
    626 -2.54427161617162
    627 -2.53737265413108
    628 -2.45128673059804
    629 -2.48960486440571
    630 -2.608646365405
    631 -2.53830315771879
    632 -2.55171110338864
    633 -2.55457935675379
    634 -2.6626733264653
    635 -2.62256737800791
    636 -2.72901548047309
    637 -2.74754277759343
    638 -2.9167627821735
    639 -2.75827127005579
    640 -2.78673091690048
    641 -2.79304742656199
    642 -2.90253668809238
    643 -2.86357089994902
    644 -2.71343312490809
    645 -2.74064392338376
    646 -2.69084737404723
    647 -2.69504177245127
    648 -2.56117782987246
    649 -2.65104606715934
    650 -2.65213613568752
    651 -2.7366430097894
    652 -2.68442328171379
    653 -2.49635017659042
    654 -2.51363971005873
    655 -2.3922882387043
    656 -2.42569633791189
    657 -2.41259847338016
    658 -2.46811378103205
    659 -2.45102849350709
    660 -2.39156171042528
    661 -2.35492398834283
    662 -2.39150189123301
    663 -2.43842259821049
    664 -2.46796306119775
    665 -2.59912958657575
    666 -2.48792108013135
    667 -2.49416036725114
    668 -2.4370664983337
    669 -2.45658480999772
    670 -2.5391787682377
    671 -2.58162849707322
    672 -2.57212223065162
    673 -2.57621975976572
    674 -2.61346333237689
    675 -2.57823104109347
    676 -2.62028076231706
    677 -2.52738572783986
    678 -2.55565335960298
    679 -2.39858322274396
    680 -2.40424480946262
    681 -2.34953375184783
    682 -2.24901598982614
    683 -2.27112415456721
    684 -2.26920369916299
    685 -2.26715505060067
    686 -2.17289741125675
    687 -2.26525874672286
    688 -2.33878040573636
    689 -2.47194036981706
    690 -2.36946782081999
    691 -2.42844753360771
    692 -2.54087674319797
    693 -2.67011968761028
    694 -2.52646847325447
    695 -2.5185649564909
    696 -2.71524255467321
    697 -2.63568696278664
    698 -2.51419454001401
    699 -2.5340085032893
    700 -2.59919305381225
    701 -2.57398370766394
    702 -2.53211721136949
    703 -2.48455907058035
    704 -2.67117639343871
    705 -2.55246775711593
    706 -2.56617485994667
    707 -2.59481270506424
    708 -2.48862799836434
    709 -2.45924534772526
    710 -2.41017847502793
    711 -2.37124668346
    712 -2.39479315950353
    713 -2.40602371531373
    714 -2.28916876658394
    715 -2.34457191406785
    716 -2.36591554208034
    717 -2.35055946849839
    718 -2.28043050738007
    719 -2.19883813675332
    720 -2.19849114295585
    721 -2.26644146803562
    722 -2.3112680403381
    723 -2.27321941571791
    724 -2.2885151015151
    725 -2.33661669967092
    726 -2.33214890479501
    727 -2.35258682527427
    728 -2.55772619605799
    729 -2.48451637215736
    730 -2.55881603760901
    731 -2.50937514057808
    732 -2.40638300461265
    733 -2.46960799678622
    734 -2.56837955161879
    735 -2.52592186491094
    736 -2.49490234141557
    737 -2.51361289318117
    738 -2.57993621394214
    739 -2.85971139054502
    740 -2.82765217557415
    741 -2.85236001527637
    742 -2.90009815975667
    743 -2.90326352666684
    744 -2.6920905325041
    745 -2.71603081754528
    746 -2.76694076172853
    747 -2.64712375974403
    748 -2.62996821644621
    749 -2.53760711793286
    750 -2.50945633664988
    751 -2.35415401332009
    752 -2.41078301530606
    753 -2.37850330873862
    754 -2.47194490614334
    755 -2.48161862302164
    756 -2.44817204016742
    757 -2.57439399703946
    758 -2.53654102952936
    759 -2.57553591758265
    760 -2.59633805757644
    761 -2.79567893441283
    762 -2.70434545460632
    763 -2.7016227712323
    764 -2.70855108470199
    765 -2.67148516885694
    766 -2.67703884399344
    767 -2.64101368653195
    768 -2.70359760154146
    769 -2.74071498082615
    770 -2.66439158708847
    771 -2.64741321139847
    772 -2.65995757041145
    773 -2.52975593912971
    774 -2.57883475356737
    775 -2.64743367766141
    776 -2.64370629462006
    777 -2.68959525109357
    778 -2.74024817852715
    779 -2.7500404075028
    780 -2.84313875225766
    781 -2.72649366129941
    782 -2.80806811382955
    783 -2.95872156736554
    784 -2.83406339767979
    785 -2.8339729214101
    786 -2.86772164415392
    787 -2.65295321819963
    788 -2.55867889782951
    789 -2.43938483883181
    790 -2.42675607428763
    791 -2.42129578680715
    792 -2.39424100584496
    793 -2.4315139791645
    794 -2.4486123990305
    795 -2.42579638449575
    796 -2.30471578356417
    797 -2.39237438187382
    798 -2.37535922371447
    799 -2.47109702371367
    800 -2.4375272877075
    801 -2.42992841377957
    802 -2.34433681559423
    803 -2.31698361796544
    804 -2.36050597123251
    805 -2.34474856271662
    806 -2.42011686331254
    807 -2.40924242964082
    808 -2.41641576107722
    809 -2.36744534420175
    810 -2.44047468535506
    811 -2.58738019855193
    812 -2.73211950066046
    813 -2.75518631405297
    814 -2.64075375619284
    815 -2.70343992205976
    816 -2.78422783747092
    817 -2.91771942567569
    818 -3.01626910543758
    819 -3.14479454633903
    820 -3.10518206155398
    821 -3.02780851841998
    822 -2.96306062644533
    823 -2.86838005721536
    824 -3.01745103164427
    825 -2.93749289005667
    826 -2.87335509831889
    827 -2.77778777630209
    828 -2.8107448674829
    829 -2.736536499284
    830 -2.79046346777183
    831 -2.85489306335609
    832 -2.92973384263614
    833 -3.07078832172723
    834 -2.89750713304038
    835 -2.85603735755235
    836 -2.70026823808795
    837 -2.65061825674031
    838 -2.49001105482526
    839 -2.57478897054074
    840 -2.53381111705275
    841 -2.42290323029713
    842 -2.3364615357576
    843 -2.28266933327814
    844 -2.32140496312241
    845 -2.40398057318538
    846 -2.46071825458555
    847 -2.45522324644675
    848 -2.50005785571954
    849 -2.41003338161519
    850 -2.42426386554939
    851 -2.50831801743516
    852 -2.49194295765798
    853 -2.50877195532513
    854 -2.47498012699528
    855 -2.40539829606988
    856 -2.45151981906345
    857 -2.45107864505087
    858 -2.44167519396217
    859 -2.43257010197269
    860 -2.45250764181235
    861 -2.44575741475809
    862 -2.50470719588587
    863 -2.56786044688705
    864 -2.61648668460799
    865 -2.51160900927073
    866 -2.46498259679549
    867 -2.54086420071856
    868 -2.55363285638233
    869 -2.61670000913989
    870 -2.59335725256216
    871 -2.56350887332827
    872 -2.63046423252146
    873 -2.5358492394036
    874 -2.59006830687163
    875 -2.73471415086988
    876 -2.66909777703899
    877 -2.70493938884552
    878 -2.68787254619147
    879 -2.59828821272706
    880 -2.58538045754021
    881 -2.64058532451533
    882 -2.59085520153189
    883 -2.57502788358614
    884 -2.56102122204122
    885 -2.50712674031957
    886 -2.66939774357324
    887 -2.54520039996372
    888 -2.58208890230989
    889 -2.64207877495753
    890 -2.59396883553871
    891 -2.57793088899267
    892 -2.58486513117533
    893 -2.67827492390076
    894 -2.67667986297651
    895 -2.66736220858942
    896 -2.64164905717405
    897 -2.56465982585943
    898 -2.52814406080198
    899 -2.55715305172944
    900 -2.61754873475817
    901 -2.66879127307516
    902 -2.52853900358981
    903 -2.43102278450811
    904 -2.46285341859637
    905 -2.50345671540911
    906 -2.50368223449987
    907 -2.46703135449892
    908 -2.51377232592586
    909 -2.52211441229494
    910 -2.42098824108387
    911 -2.37663000226993
    912 -2.49613497795383
    913 -2.53696554748128
    914 -2.51995513744804
    915 -2.53991041209473
    916 -2.48701260872077
    917 -2.70843286162106
    918 -2.76259588178539
    919 -2.72807825081724
    920 -2.75683820115972
    921 -2.79658929798429
    922 -2.70296864502292
    923 -2.8005194077124
    924 -2.75340240518088
    925 -2.66067167359931
    926 -2.60064530571058
    927 -2.38897396728147
    928 -2.39256674032776
    929 -2.43783715892734
    930 -2.51239509591636
    931 -2.41767987701193
    932 -2.44125873641676
    933 -2.38810982604253
    934 -2.35482841972215
    935 -2.42518785791011
    936 -2.36703760832781
    937 -2.58454371297258
    938 -2.58935888041672
    939 -2.59495962335237
    940 -2.47815277271711
    941 -2.53788311044773
    942 -2.50892507057307
    943 -2.55234106501837
    944 -2.65364054380876
    945 -2.55101757779327
    946 -2.78983401754755
    947 -2.72778891382855
    948 -2.71481083569
    949 -2.65797806700126
    950 -2.71240791550487
    951 -2.73499072835675
    952 -2.78371890282081
    953 -2.755159058847
    954 -2.74014832443166
    955 -2.83912625952931
    956 -2.84854384302988
    957 -2.83457704445516
    958 -2.80465030851453
    959 -2.82999623983754
    960 -2.81877212848356
    961 -2.85278851116808
    962 -2.96146467095371
    963 -3.04043365884935
    964 -2.90965830994947
    965 -2.88490839385747
    966 -2.82527035017665
    967 -2.79884711190944
    968 -2.83662369662483
    969 -2.90546469563579
    970 -3.07187187488005
    971 -3.08335435517031
    972 -3.06582326990322
    973 -2.93798836625726
    974 -2.9981032117979
    975 -3.08947496648861
    976 -3.04030360511039
    977 -3.19605014734661
    978 -3.14451430308635
    979 -3.0309931390465
    980 -2.98537093054954
    981 -2.9233993695325
    982 -2.88738453769563
    983 -2.85667025490915
    984 -2.94214545914326
    985 -2.94055607951046
    986 -2.96575103552127
    987 -2.86696628952935
    988 -2.8693028063492
    989 -2.80332656925329
    990 -2.75728964660015
    991 -2.80498301415504
    992 -2.80025703492534
    993 -2.88541437456608
    994 -2.90061734714701
    995 -2.92305193214029
    996 -2.80567385099801
    997 -2.85821425494118
    998 -2.80733388945397
    999 -2.96743315093065
    1000 -2.90163192448421
    1001 -2.73359322767172
    1002 -2.6470728328503
    1003 -2.68369972099639
    1004 -2.58496154812852
    1005 -2.52707913442121
    1006 -2.59422709650675
    1007 -2.51119902829661
    1008 -2.48917850357678
    1009 -2.43250460566188
    1010 -2.43818678065064
    1011 -2.54764067912813
    1012 -2.4793341014857
    1013 -2.44240407379804
    1014 -2.40855165029257
    1015 -2.42022401314186
    1016 -2.44670147422899
    1017 -2.51769558316107
    1018 -2.53177739083727
    1019 -2.55605657684001
    1020 -2.66560888270654
    1021 -2.64768968990898
    1022 -2.76502930507653
    1023 -2.80718462539741
    1024 -2.9476070456617
    1025 -2.97639506200469
    1026 -3.0020754631485
    1027 -3.01072700474689
    1028 -3.16990348860057
    1029 -3.05637420461594
    1030 -3.01722365085276
    1031 -2.98574474050464
    1032 -3.0435951261704
    1033 -3.02816314315811
    1034 -2.9880394054483
    1035 -2.83903384621779
    1036 -2.77128554728331
    1037 -2.77366930029409
    1038 -2.60702796899502
    1039 -2.56916644700815
    1040 -2.51703395925205
    1041 -2.55788990339256
    1042 -2.55114304074374
    1043 -2.50085735150334
    1044 -2.51711976611253
    1045 -2.66113535022641
    1046 -2.59070886882849
    1047 -2.54085126128066
    1048 -2.57125156012685
    1049 -2.65762932397464
    1050 -2.64273843085023
    1051 -2.62838416484603
    1052 -2.61893665226294
    1053 -2.6683375294387
    1054 -2.63111462327034
    1055 -2.5902351731645
    1056 -2.74536778299728
    1057 -2.82246465688096
    1058 -2.90417727074278
    1059 -2.7521792770796
    1060 -2.67204186718454
    1061 -2.60923809868353
    1062 -2.66704958532913
    1063 -2.62613671722406
    1064 -2.63260401547381
    1065 -2.61311280725362
    1066 -2.56391415845387
    1067 -2.57882117832878
    1068 -2.61147503615686
    1069 -2.67808636527088
    1070 -2.74629351781383
    1071 -2.82473607032072
    1072 -2.67332041502242
    1073 -2.64562268799891
    1074 -2.69243506163203
    1075 -2.74002944503133
    1076 -2.74141510706572
    1077 -2.62968415211288
    1078 -2.61493181900578
    1079 -2.73518542855495
    1080 -2.82396099590044
    1081 -2.82971363076652
    1082 -3.04865811700028
    1083 -3.06015224212027
    1084 -2.95637957827853
    1085 -2.86124061793413
    1086 -2.79309714197752
    1087 -2.86229692003251
    1088 -2.7517541287744
    1089 -2.59792272070685
    1090 -2.48624850479619
    1091 -2.41435247675624
    1092 -2.40069052322797
    1093 -2.42817688197098
    1094 -2.32320619185366
    1095 -2.37064636657678
    1096 -2.37744603945473
    1097 -2.34670911150291
    1098 -2.42655845572852
    1099 -2.50106382149759
    1100 -2.63032144640211
    1101 -2.7289723986491
    1102 -2.74485065850322
    1103 -2.51189511845196
    1104 -2.53557316213102
    1105 -2.53979317335848
    1106 -2.61959644093466
    1107 -2.66891908023279
    1108 -2.67355378879863
    1109 -2.7098509126416
    1110 -2.71239438564965
    1111 -2.62327151924783
    1112 -2.45259134780114
    1113 -2.62340028003693
    1114 -2.70326978783207
    1115 -2.64441090379482
    1116 -2.58520736930717
    1117 -2.63633436542451
    1118 -2.63819051072896
    1119 -2.58980921359374
    1120 -2.56674700325721
    1121 -2.61307531163409
    1122 -2.75363685256433
    1123 -2.82961877975477
    1124 -2.82794947934234
    1125 -2.87984595327852
    1126 -2.79911618436259
    1127 -2.77290076249166
    1128 -2.76827923279524
    1129 -2.82013160338028
    1130 -2.8681391841196
    1131 -2.83343304334183
    1132 -2.76781793856905
    1133 -2.65812591004075
    1134 -2.76223934769111
    1135 -2.76202983376813
    1136 -2.85113607506243
    1137 -2.8140214230246
    1138 -2.72837664182227
    1139 -2.62980489342958
    1140 -2.61803136258615
    1141 -2.68401621916556
    1142 -2.6431304173294
    1143 -2.77966776244041
    1144 -2.64728502212816
    1145 -2.6291564975136
    1146 -2.67197341807239
    1147 -2.69788042884571
    1148 -2.63525758974563
    1149 -2.66588569218226
    1150 -2.60145301758845
    1151 -2.56746371145411
    1152 -2.68466213675136
    1153 -2.62159016443735
    1154 -2.71801119584198
    1155 -2.7107956907649
    1156 -2.73901014697169
    1157 -2.69785210297103
    1158 -2.7981273118589
    1159 -2.89086533834541
    1160 -2.98306880395073
    1161 -2.89399559941487
    1162 -2.92817862083149
    1163 -2.89230872496605
    1164 -2.89530946918759
    1165 -2.7795872980292
    1166 -2.74803843240819
    1167 -2.78185187264243
    1168 -2.82634434586507
    1169 -2.73189325397315
    1170 -2.6326941683233
    1171 -2.66233906116699
    1172 -2.49912848160131
    1173 -2.59264456873994
    1174 -2.56173253341327
    1175 -2.70038928999406
    1176 -2.70633631230235
    1177 -2.68399015165941
    1178 -2.60212968396961
    1179 -2.68496399992774
    1180 -2.58494360588239
    1181 -2.52496604599857
    1182 -2.55693773164871
    1183 -2.49366495167986
    1184 -2.51881535969884
    1185 -2.4979397346397
    1186 -2.50824038911254
    1187 -2.52940352017323
    1188 -2.50450843807955
    1189 -2.45286669581875
    1190 -2.61392917135297
    1191 -2.76379094517932
    1192 -2.91629313268678
    1193 -2.88701812125658
    1194 -2.93434426394023
    1195 -2.94036795137316
    1196 -2.79875448602559
    1197 -2.79395905922194
    1198 -2.98039051094416
    1199 -3.01327539764021
    1200 -2.97823899999801
    1201 -2.95360485872059
    1202 -2.96777719447897
    1203 -3.04041056993036
    1204 -2.92874235691938
    1205 -2.88992768530583
    1206 -3.06999261882351
    1207 -2.9400045064219
    1208 -2.96332541491715
    1209 -2.96868834711271
    1210 -2.86850817406653
    1211 -2.8407278141613
    1212 -2.72086620322319
    1213 -2.64912999352765
    1214 -2.67166679941014
    1215 -2.6383748472328
    1216 -2.57472782073479
    1217 -2.60457828871276
    1218 -2.5888674140511
    1219 -2.55945096684553
    1220 -2.54476671747465
    1221 -2.39572695805426
    1222 -2.45847938212347
    1223 -2.46166874811806
    1224 -2.47999766450169
    1225 -2.57362204518582
    1226 -2.52619479746731
    1227 -2.4883349685469
    1228 -2.4742567172519
    1229 -2.51458115792504
    1230 -2.59661765311179
    1231 -2.79962201644786
    1232 -2.77501215929371
    1233 -2.84239386864752
    1234 -2.77399667461918
    1235 -2.48981412492525
    1236 -2.56254531669847
    1237 -2.62397969885568
    1238 -2.547008852069
    1239 -2.5220101959223
    1240 -2.54063211523275
    1241 -2.53122497300257
    1242 -2.507104661354
    1243 -2.48044593145759
    1244 -2.52546161297148
    1245 -2.75695883465771
    1246 -2.78072776572855
    1247 -2.74031960462311
    1248 -2.79124765601589
    1249 -2.74431896191731
    1250 -2.69183990669676
    1251 -2.73379182429133
    1252 -2.74618513856627
    1253 -2.74962829821049
    1254 -2.78426268082957
    1255 -2.7445517440679
    1256 -2.67890489003629
    1257 -2.73053290152683
    1258 -2.73386588279674
    1259 -2.85296210285181
    1260 -2.94331593710428
    1261 -2.87501742169501
    1262 -2.89130874290622
    1263 -2.99643005147575
    1264 -2.98311839018252
    1265 -2.97211445724564
    1266 -2.98879197187217
    1267 -3.06144018219158
    1268 -3.10308600990461
    1269 -3.04121870705781
    1270 -2.81683703963344
    1271 -2.81770390132139
    1272 -2.79655875999224
    1273 -2.67319229912409
    1274 -2.7274171899917
    1275 -2.68652288884695
    1276 -2.71997534647206
    1277 -2.68336365312111
    1278 -2.65041121163099
    1279 -2.65297549973597
    1280 -2.76030663475762
    1281 -2.81453047716367
    1282 -2.90841801378288
    1283 -2.92365876163787
    1284 -2.8708864331214
    1285 -3.03290650585324
    1286 -3.08063524925201
    1287 -3.05472170658965
    1288 -2.81722590163723
    1289 -2.73896727358732
    1290 -2.84763482609088
    1291 -2.80417564894505
    1292 -2.80532145964825
    1293 -2.91333376767347
    1294 -2.92352981011854
    1295 -2.83248735206395
    1296 -2.75909793078255
    1297 -2.80276525440458
    1298 -3.0353575432645
    1299 -3.20970140430543
    1300 -3.1835218820438
    1301 -3.20100599309887
    1302 -3.20780802643468
    1303 -3.08451477972655
    1304 -3.09945323222289
    1305 -3.10079278857392
    1306 -3.18624581005965
    1307 -3.2260298590921
    1308 -3.25788406631128
    1309 -3.04401084038452
    1310 -3.04637626148259
    1311 -2.98850705283953
    1312 -2.83135185345979
    1313 -2.94290130826869
    1314 -2.9526885134491
    1315 -2.74274806331756
    1316 -2.76521568233588
    1317 -2.65975320308505
    1318 -2.68350609787843
    1319 -2.77356066221344
    1320 -2.73344186798665
    1321 -2.74692896004743
    1322 -2.75224317070265
    1323 -2.67249372379345
    1324 -2.64735695171321
    1325 -2.77566356823083
    1326 -2.62632125740324
    1327 -2.73298262269866
    1328 -2.66815305403126
    1329 -2.57870959901086
    1330 -2.62374311544043
    1331 -2.67501019366546
    1332 -2.66252341876836
    1333 -2.71936568045856
    1334 -2.74550652086966
    1335 -2.88497074416739
    1336 -3.01401726954053
    1337 -2.73485661145325
    1338 -2.59511011942902
    1339 -2.5523780787381
    1340 -2.56317029147747
    1341 -2.44628414009964
    1342 -2.52833479432369
    1343 -2.53535293747593
    1344 -2.47525266269325
    1345 -2.41596021646396
    1346 -2.3482176138635
    1347 -2.48129309769314
    1348 -2.67422020084306
    1349 -2.72064799397625
    1350 -2.58605003616214
    1351 -2.63024843254098
    1352 -2.61338265608734
    1353 -2.54817701394706
    1354 -2.57379600267577
    1355 -2.58104349041483
    1356 -2.65613373566728
    1357 -2.56188180781764
    1358 -2.49136830266399
    1359 -2.61994591427745
    1360 -2.71167699591497
    1361 -2.74793303613551
    1362 -2.76783647060903
    1363 -2.84835293957212
    1364 -2.85047009384315
    1365 -2.94543654716761
    1366 -3.00798892492883
    1367 -3.18435409186478
    1368 -3.38234625751792
    1369 -3.29999629423464
    1370 -3.36412930188217
    1371 -3.40767634753868
    1372 -3.38928038327201
    1373 -3.37051108861817
    1374 -3.47001295243366
    1375 -3.26884770121592
    1376 -3.2405314430362
    1377 -3.17546178805843
    1378 -3.05977028267574
    1379 -2.89821593352429
    1380 -2.77574792546821
    1381 -2.81208197304306
    1382 -2.67769601921971
    1383 -2.66603413036516
    1384 -2.46330675546374
    1385 -2.57577298607114
    1386 -2.53723617078843
    1387 -2.57586628058768
    1388 -2.64480864645112
    1389 -2.41978413492286
    1390 -2.39469852996863
    1391 -2.30714332925347
    1392 -2.35583148658161
    1393 -2.35782259904581
    1394 -2.48771385271872
    1395 -2.46045107952546
    1396 -2.47795065262385
    1397 -2.42537493689424
    1398 -2.35807219828439
    1399 -2.64487086905769
    1400 -2.71165163858592
    1401 -2.75803150129169
    1402 -2.71701821180949
    1403 -2.7367873743699
    1404 -2.73489016518135
    1405 -2.75521631438254
    1406 -2.74087170550647
    1407 -2.80801787703625
    1408 -2.88103872131017
    1409 -2.88035334824173
    1410 -2.95555671314125
    1411 -2.97803964041982
    1412 -3.10996299370876
    1413 -3.10907457027031
    1414 -3.22174319836252
    1415 -3.12273470997443
    1416 -3.04354116693893
    1417 -2.90130906747807
    1418 -2.86052830218175
    1419 -2.95620081645711
    1420 -2.99139160724755
    1421 -3.09071304490102
    1422 -3.01492181479173
    1423 -3.04946797985275
    1424 -2.80270719868426
    1425 -2.79595955742508
    1426 -2.91176985265289
    1427 -3.0157296021145
    1428 -3.0850384727368
    1429 -2.98464905557775
    1430 -2.94985835412429
    1431 -2.95661672206338
    1432 -2.97957995445523
    1433 -2.91678173977108
    1434 -3.09902108990611
    1435 -3.22536516989624
    1436 -3.25197410583451
    1437 -3.32399440349792
    1438 -3.2437032836651
    1439 -3.0867311945932
    1440 -3.09926713313419
    1441 -3.08730369901821
    1442 -3.08603930719131
    1443 -3.16681237155364
    1444 -3.11521116368484
    1445 -2.96282965842649
    1446 -2.93223369284892
    1447 -2.94874763853131
    1448 -2.94402205346387
    1449 -3.18608337538853
    1450 -2.92040599035602
    1451 -2.93150953307912
    1452 -2.96237390127714
    1453 -2.88142716998328
    1454 -2.99964598214028
    1455 -3.01589661826377
    1456 -2.97528538635892
    1457 -2.97627984877543
    1458 -2.96680963738834
    1459 -2.97915196610668
    1460 -3.27475776800604
    1461 -3.11404853922979
    1462 -3.18525498488462
    1463 -3.26002341671934
    1464 -3.06199924277309
    1465 -3.0672132303052
    1466 -3.13104039980972
    1467 -3.1385909196981
    1468 -3.1289697711929
    1469 -2.89338252830834
    1470 -2.8547341611751
    1471 -3.00268575807689
    1472 -2.97423199183956
    1473 -2.95037302743224
    1474 -3.11315927361234
    1475 -3.04807850659679
    1476 -2.98967100365977
    1477 -2.87409212708813
    1478 -2.94823425026242
    1479 -2.97167601086274
    1480 -2.905231144953
    1481 -2.86505377524234
    1482 -2.76914750777253
    1483 -2.75566329495724
    1484 -2.72664814855606
    1485 -2.85586144654762
    1486 -2.87529159405421
    1487 -2.99400409308115
    1488 -2.91865142832579
    1489 -3.00967112382337
    1490 -2.90525893981055
    1491 -2.81159635361013
    1492 -2.94669409186632
    1493 -3.01129927287269
    1494 -2.98044579604758
    1495 -2.93116121218586
    1496 -2.94096599675316
    1497 -2.81364365431484
    1498 -2.88149723424763
    1499 -2.96281044368825
    1500 -3.0812697302787
    1501 -3.22559838589248
    1502 -3.17062683252373
    1503 -2.98885910940914
    1504 -3.02778893303319
    1505 -3.01555327564085
    1506 -2.79786664787041
    1507 -2.87788670133521
    1508 -2.83724432964859
    1509 -2.77423594718683
    1510 -2.81668236377153
    1511 -2.78031970583599
    1512 -2.8187518836909
    1513 -2.76394675949596
    1514 -2.73938291812956
    1515 -2.72784639179733
    1516 -2.8085383768679
    1517 -2.82676187178943
    1518 -2.79350506080198
    1519 -2.80937386774943
    1520 -2.85090523692407
    1521 -2.85548617653478
    1522 -2.81314520462668
    1523 -2.98600064690928
    1524 -3.0305904649996
    1525 -3.02981536662531
    1526 -3.11033490612625
    1527 -3.11990806316122
    1528 -3.18696272155847
    1529 -3.19486474973047
    1530 -2.8023511404605
    1531 -2.83405367940714
    1532 -2.89476778117633
    1533 -2.9114846062952
    1534 -2.85236093355993
    1535 -2.89077200438388
    1536 -2.83920918365044
    1537 -2.68196482245267
    1538 -2.59767232812731
    1539 -2.66463341590667
    1540 -2.91786701661442
    1541 -2.90144350426376
    1542 -2.77721006275694
    1543 -2.69918903932999
    1544 -2.63193706074906
    1545 -2.42893390971534
    1546 -2.44167822934225
    1547 -2.50512235343538
    1548 -2.56357620350066
    1549 -2.49349043805442
    1550 -2.47809423629034
    1551 -2.48662476922718
    1552 -2.53427329386135
    1553 -2.58076264302555
    1554 -2.68504876271809
    1555 -2.8733755761161
    1556 -2.88813786515225
    1557 -2.97320882026792
    1558 -3.0436826363514
    1559 -3.0431545729402
    1560 -3.03409506671659
    1561 -3.02196282575957
    1562 -2.9306461103357
    1563 -2.79347461228511
    1564 -2.72523907282989
    1565 -2.82533351092125
    1566 -2.81906694709315
    1567 -2.83794434512454
    1568 -2.81864834406327
    1569 -2.71836664252588
    1570 -2.73193856789214
    1571 -2.7404700681295
    1572 -2.7509343380423
    1573 -2.87896062739813
    1574 -2.87105337713281
    1575 -2.81007561851609
    1576 -2.91805096719362
    1577 -2.82478797201531
    1578 -2.79665779207226
    1579 -2.89774189000887
    1580 -3.00480219466933
    1581 -2.99602739920647
    1582 -3.12896135241814
    1583 -3.14283220138451
    1584 -3.18852410903134
    1585 -3.10996768138124
    1586 -3.08102036680437
    1587 -3.21671147612845
    1588 -3.20901140966818
    1589 -3.0816834566551
    1590 -3.04642522041027
    1591 -3.08440867674956
    1592 -3.02695356564262
    1593 -2.99872432191832
    1594 -3.06212585909789
    1595 -3.21233764391244
    1596 -3.187371508404
    1597 -3.08839618174685
    1598 -3.03053389535741
    1599 -3.16755864595072
    1600 -2.93810404593613
    1601 -2.93883754558468
    1602 -3.00887705374044
    1603 -2.84155985582974
    1604 -2.83968172035646
    1605 -2.79054963202404
    1606 -2.71436606136069
    1607 -2.77570870784737
    1608 -2.79779706540797
    1609 -2.84969041924832
    1610 -3.04057493475293
    1611 -2.83775968012192
    1612 -2.8182274440628
    1613 -3.02140355790037
    1614 -3.04415296864319
    1615 -3.04509574940285
    1616 -3.1902034161768
    1617 -3.17343628594341
    1618 -3.06660148871179
    1619 -3.08342007093023
    1620 -3.13071662863076
    1621 -3.38163893082583
    1622 -3.3331613216911
    1623 -3.22760711014397
    1624 -3.09498548897625
    1625 -3.12206158917994
    1626 -3.05320239923577
    1627 -2.91104292119334
    1628 -2.84396328259795
    1629 -2.72710363756114
    1630 -2.7029220426607
    1631 -2.69439350817075
    1632 -2.73356840765857
    1633 -2.81736961899017
    1634 -2.79129751888137
    1635 -2.77471380251828
    1636 -2.76952986218506
    1637 -2.88392197538233
    1638 -3.07723409543458
    1639 -3.28159440072706
    1640 -3.25523133046127
    1641 -3.27441659350369
    1642 -3.1460153310544
    1643 -3.0832422477895
    1644 -3.28584987439301
    1645 -3.1899953266355
    1646 -3.04788814927833
    1647 -3.04607414705725
    1648 -3.12782678801704
    1649 -3.11806636634198
    1650 -3.11170579086162
    1651 -2.97663830551294
    1652 -2.98984880745481
    1653 -3.05520383368883
    1654 -2.96929805659236
    1655 -2.95430760976135
    1656 -3.20074557781798
    1657 -3.05650213554465
    1658 -3.04915872598736
    1659 -3.00194229269549
    1660 -3.06781064259711
    1661 -3.14621006888982
    1662 -3.17539377174324
    1663 -3.05111558084829
    1664 -3.13003337314149
    1665 -3.29945406516699
    1666 -3.25855157566304
    1667 -3.45413157851807
    1668 -3.40470557206773
    1669 -3.3709765267521
    1670 -3.28178463830028
    1671 -3.05095350969888
    1672 -3.04092947270551
    1673 -3.02216784452902
    1674 -2.98945367324809
    1675 -2.90256565456473
    1676 -2.87393075423097
    1677 -2.85247243623603
    1678 -2.87322331703962
    1679 -2.80134614329284
    1680 -2.81078847142551
    1681 -3.01060568418491
    1682 -2.81142550451106
    1683 -2.93530402624844
    1684 -2.89764713359522
    1685 -2.80403153169879
    1686 -2.70006350458076
    1687 -2.70559084629894
    1688 -2.64598029996901
    1689 -2.6094964704369
    1690 -2.61132419151077
    1691 -2.63166252396408
    1692 -2.83150742751771
    1693 -2.68182053677694
    1694 -2.74098572719709
    1695 -2.83913594599885
    1696 -2.95527664171244
    1697 -2.97727302272179
    1698 -2.9437219815426
    1699 -3.15798934986859
    1700 -3.16320321888622
    1701 -3.1156444810545
    1702 -3.10332455728261
    1703 -3.16455449222028
    1704 -3.12821940392066
    1705 -3.16484906700025
    1706 -3.18309071797003
    1707 -3.19845168999717
    1708 -3.36272418909169
    1709 -3.34498015722978
    1710 -3.31915580988683
    1711 -3.27650146188086
    1712 -3.19280732339071
    1713 -3.27574296113133
    1714 -3.32335519746733
    1715 -3.47351142874298
    1716 -3.37069975902793
    1717 -3.35886004899354
    1718 -3.27789745773359
    1719 -3.25159617417776
    1720 -3.23467885332595
    1721 -3.26522998400008
    1722 -3.21238890310142
    1723 -3.13221907742175
    1724 -3.05367006767559
    1725 -2.85076520826999
    1726 -2.89542715765065
    1727 -2.90879503662534
    1728 -2.8214404844913
    1729 -2.76862654612616
    1730 -2.76632072814553
    1731 -2.69779451060629
    1732 -2.78387364747134
    1733 -2.69434060954836
    1734 -2.74447878936109
    1735 -2.90624779786753
    1736 -2.94494698454943
    1737 -2.89682955204366
    1738 -3.00112992811604
    1739 -3.02011273398179
    1740 -3.03379114617578
    1741 -3.15278226883287
    1742 -3.09437009918975
    1743 -3.16412487414441
    1744 -3.05196487087127
    1745 -2.96916467456759
    1746 -2.85838912884275
    1747 -2.84684990932427
    1748 -2.82398031892934
    1749 -2.87555420018675
    1750 -2.82349920222043
    1751 -2.69355219400355
    1752 -2.67750945039349
    1753 -2.6813946343514
    1754 -2.59951124251267
    1755 -2.61228745895423
    1756 -2.65854021448234
    1757 -2.58944191219209
    1758 -2.60411518288926
    1759 -2.60476909535698
    1760 -2.59509684176439
    1761 -2.73874922628915
    1762 -2.81460346233094
    1763 -2.85465691296873
    1764 -3.01318692964031
    1765 -3.04280176788417
    1766 -3.05955473839474
    1767 -3.0166035855836
    1768 -2.97375594101412
    1769 -2.83022592997558
    1770 -2.93527419397014
    1771 -2.87844289943263
    1772 -2.87143428091469
    1773 -2.86405968420593
    1774 -2.86183238351201
    1775 -2.81622887138794
    1776 -2.59654014131665
    1777 -2.6455122260236
    1778 -2.69892036837436
    1779 -2.70658248802512
    1780 -2.71824539610376
    1781 -2.6939978257312
    1782 -2.69149805031346
    1783 -2.70897267305043
    1784 -2.66324195664973
    1785 -2.67781537756442
    1786 -2.96361374570763
    1787 -3.07163958146611
    1788 -3.03440854905673
    1789 -3.17628495152261
    1790 -3.10612448743948
    1791 -3.20768062730518
    1792 -3.18128649526665
    1793 -3.27004810640533
    1794 -3.30695719182431
    1795 -3.24367726749187
    1796 -3.14906660060896
    1797 -3.11256158180011
    1798 -2.99978315606803
    1799 -2.92479725743672
    1800 -2.9040382156316
    1801 -2.92925130082912
    1802 -2.86079882308813
    1803 -2.80962230663743
    1804 -2.84319090634012
    1805 -2.85034670073827
    1806 -2.85940134074402
    1807 -2.90197983292647
    1808 -2.83655984100569
    1809 -2.8428243518715
    1810 -2.83851557391089
    1811 -2.73669754443144
    1812 -2.76911697372786
    1813 -2.80390688929959
    1814 -2.6413987006827
    1815 -2.67008352691948
    1816 -2.69410341041948
    1817 -2.64110087457014
    1818 -2.80180030870593
    1819 -2.81415545043064
    1820 -2.89774038361698
    1821 -2.92106554876892
    1822 -2.60411855974487
    1823 -2.62026465998348
    1824 -2.78425836303352
    1825 -2.84095904620381
    1826 -2.85925993865495
    1827 -2.88212991072717
    1828 -2.86510151437145
    1829 -2.92665704251877
    1830 -2.83352628508533
    1831 -2.86389089637921
    1832 -3.10008311172357
    1833 -2.9518807905165
    1834 -2.84626479498877
    1835 -2.68944776918723
    1836 -2.57233538376074
    1837 -2.57108508566336
    1838 -2.53854635103743
    1839 -2.5129154193482
    1840 -2.54903793002213
    1841 -2.52065023404774
    1842 -2.65161117475686
    1843 -2.77158694526195
    1844 -2.79428405980432
    1845 -2.71398747279955
    1846 -2.73621646393499
    1847 -2.6608537171612
    1848 -2.67629419848046
    1849 -2.70070203781626
    1850 -2.69651470341679
    1851 -2.72930332038489
    1852 -2.57070754347525
    1853 -2.54922037462662
    1854 -2.60487185073341
    1855 -2.82339412107009
    1856 -2.87188755277799
    1857 -2.97566277621591
    1858 -2.98334280230156
    1859 -2.93506637111387
    1860 -2.94459064462741
    1861 -2.93344338209847
    1862 -3.07788502285921
    1863 -3.14006248490372
    1864 -2.9530710611169
    1865 -2.8668559525064
    1866 -2.92851985919674
    };
    \addplot [semithick, darkorange25512714]
    table {%
    0 0.984650112139183
    1 0.984053797777505
    2 0.870979366267666
    3 0.851741519354561
    4 0.839848501826247
    5 0.83779370068915
    6 0.834563643055556
    7 0.826631417348051
    8 0.815958600934919
    9 0.809245711415308
    10 0.798842717139343
    11 0.784129685470995
    12 0.765955671704291
    13 0.741492748530195
    14 0.714750284542091
    15 0.683289190263756
    16 0.640519671971582
    17 0.591454655284376
    18 0.534487166846564
    19 0.466783876048746
    20 0.402204932907515
    21 0.337517581274881
    22 0.27266883710482
    23 0.212860511120098
    24 0.146816292144976
    25 0.0730068553625185
    26 0.0229986917438484
    27 -0.0240054050085826
    28 -0.0525545143062451
    29 -0.0802609724082796
    30 -0.122871732501848
    31 -0.140404794319239
    32 -0.162649935063223
    33 -0.212621085738432
    34 -0.225719492078163
    35 -0.253288748159694
    36 -0.277707028205204
    37 -0.300882972554305
    38 -0.342730809865893
    39 -0.348899452836056
    40 -0.350182960822915
    41 -0.368702140627063
    42 -0.409807646860757
    43 -0.378391649197046
    44 -0.415456912107317
    45 -0.376084075448505
    46 -0.382997314690083
    47 -0.391982514231272
    48 -0.385119531749247
    49 -0.401020464938874
    50 -0.436370689311293
    51 -0.439219615660039
    52 -0.45899702006781
    53 -0.49398830216262
    54 -0.495655097945725
    55 -0.558036300686401
    56 -0.54775433812834
    57 -0.574115029218028
    58 -0.570787460457441
    59 -0.603636290861489
    60 -0.591511699792491
    61 -0.629651065652658
    62 -0.645276883559028
    63 -0.642230745552733
    64 -0.637899668497908
    65 -0.588920106391284
    66 -0.654886886482909
    67 -0.623835131048688
    68 -0.663985613949198
    69 -0.653744760816818
    70 -0.622334438561258
    71 -0.604459161349805
    72 -0.58832304728057
    73 -0.595002340160247
    74 -0.632937515593229
    75 -0.662920894858452
    76 -0.658378884588106
    77 -0.667192641587908
    78 -0.656560644408337
    79 -0.668764851009575
    80 -0.725889534973115
    81 -0.778418078551261
    82 -0.790635774778175
    83 -0.835155316349101
    84 -0.827552295647966
    85 -0.874365294304363
    86 -0.80937638579279
    87 -0.853733247532734
    88 -0.872629732109906
    89 -0.857870846179953
    90 -0.865012690431528
    91 -0.859582409957087
    92 -0.86423240134707
    93 -0.861565353937104
    94 -0.855980598548321
    95 -0.862108954685586
    96 -0.888082081798127
    97 -0.846196754441107
    98 -0.81374729393718
    99 -0.851205374972672
    100 -0.84606870902777
    101 -0.838461497253623
    102 -0.843817602147816
    103 -0.846808083811461
    104 -0.880531985744934
    105 -0.875835704645355
    106 -0.949274085117619
    107 -0.977035082374749
    108 -1.03754814484008
    109 -1.02873545149525
    110 -1.05322513450836
    111 -1.08206485143533
    112 -1.09456120260224
    113 -1.11227552765405
    114 -1.09252193294527
    115 -1.08742249877964
    116 -1.02118438890499
    117 -1.04246854706696
    118 -1.05790563531019
    119 -1.04292164794133
    120 -1.00096214504819
    121 -1.0303503054336
    122 -1.00343395083694
    123 -1.00860815810992
    124 -1.03340320639123
    125 -1.06067501851781
    126 -1.14147724002331
    127 -1.11174581802225
    128 -1.05639997090745
    129 -1.08575761359326
    130 -1.10462820701365
    131 -1.08355966762512
    132 -1.11675313162572
    133 -1.12176295673682
    134 -1.1015013272014
    135 -1.08610313375991
    136 -1.08169488727669
    137 -1.07937751122961
    138 -1.10872892875909
    139 -1.12342229316353
    140 -1.17224294063559
    141 -1.16348726547891
    142 -1.14500651916597
    143 -1.15759148815788
    144 -1.16809663978336
    145 -1.22304095644902
    146 -1.21188315175044
    147 -1.24985660514708
    148 -1.25711879640552
    149 -1.30454321412471
    150 -1.294177456403
    151 -1.31779333696848
    152 -1.33616819168056
    153 -1.31037201099524
    154 -1.32992631645679
    155 -1.28896226976465
    156 -1.30554739293466
    157 -1.28045137405578
    158 -1.2486949312773
    159 -1.21983905046167
    160 -1.25933240332603
    161 -1.22536499268555
    162 -1.25467298046576
    163 -1.26839092603117
    164 -1.25228877847055
    165 -1.25156903754036
    166 -1.20104413132292
    167 -1.26715669072225
    168 -1.28055231024654
    169 -1.28849162217254
    170 -1.2763655405767
    171 -1.31591285141735
    172 -1.3261525103902
    173 -1.30736793398875
    174 -1.30582264889325
    175 -1.36643050770326
    176 -1.43116549886429
    177 -1.37300905575505
    178 -1.41876644610413
    179 -1.37757733544953
    180 -1.35282073578058
    181 -1.35563462384209
    182 -1.33177204463126
    183 -1.3282015811124
    184 -1.36583764189098
    185 -1.34668302384735
    186 -1.37189719015701
    187 -1.43889628121026
    188 -1.46444387786487
    189 -1.46560575333015
    190 -1.44084403377553
    191 -1.41233518717136
    192 -1.38634126511484
    193 -1.44413581935764
    194 -1.39282138536526
    195 -1.35277223657003
    196 -1.30028792204645
    197 -1.27012833655202
    198 -1.27314134506561
    199 -1.32879319454786
    200 -1.38871164999304
    201 -1.4601055920244
    202 -1.44732850374948
    203 -1.41315099949936
    204 -1.43028948705475
    205 -1.47581633087331
    206 -1.47726389705816
    207 -1.44675498079427
    208 -1.42009529905093
    209 -1.39631035806278
    210 -1.37152968933753
    211 -1.32301967188017
    212 -1.35597083326275
    213 -1.41031140896341
    214 -1.39998278540724
    215 -1.37322908874841
    216 -1.43105392257063
    217 -1.50296773163223
    218 -1.52323054090249
    219 -1.51643799848109
    220 -1.5331611669109
    221 -1.51298702768482
    222 -1.5322111432931
    223 -1.4799966923791
    224 -1.52146813679735
    225 -1.56296402847169
    226 -1.49760479409424
    227 -1.49417324199934
    228 -1.50630526447304
    229 -1.49678137133631
    230 -1.46183226921501
    231 -1.53293735585443
    232 -1.52319059000211
    233 -1.53350781258892
    234 -1.50089239495238
    235 -1.50839807940559
    236 -1.4691208369371
    237 -1.40080274780078
    238 -1.37823259219675
    239 -1.42761517595022
    240 -1.34111407799294
    241 -1.34242883276136
    242 -1.31723327007299
    243 -1.31854216669608
    244 -1.31407750572374
    245 -1.29852488292311
    246 -1.37665319954532
    247 -1.40289800584017
    248 -1.4451183485348
    249 -1.45441122909484
    250 -1.58252195922905
    251 -1.54998190515917
    252 -1.59705545418203
    253 -1.635217245564
    254 -1.60786897497779
    255 -1.63723149889452
    256 -1.60921715747273
    257 -1.61136959278647
    258 -1.61055401734805
    259 -1.53233128334026
    260 -1.54179611815121
    261 -1.52632705505378
    262 -1.5149461838907
    263 -1.49793510688858
    264 -1.58498617304874
    265 -1.55333879502485
    266 -1.58095035636382
    267 -1.60199874325213
    268 -1.55961100468509
    269 -1.5506183800549
    270 -1.54033496418079
    271 -1.53385969745867
    272 -1.51567813853023
    273 -1.52581840257075
    274 -1.50150118326694
    275 -1.51704464357251
    276 -1.4992561055487
    277 -1.53317883288826
    278 -1.5425639365515
    279 -1.60604820344846
    280 -1.61288835865046
    281 -1.63697093962332
    282 -1.6810538890321
    283 -1.7146212747728
    284 -1.74520597822072
    285 -1.73047679168336
    286 -1.65646481768924
    287 -1.55848555829004
    288 -1.52950736292042
    289 -1.55872005892754
    290 -1.59239497578307
    291 -1.59832574696025
    292 -1.62499624136021
    293 -1.5704266729436
    294 -1.511928933941
    295 -1.53413684231665
    296 -1.64930215030444
    297 -1.70714511673036
    298 -1.7266660785081
    299 -1.64798594912204
    300 -1.62983470294504
    301 -1.61872374391588
    302 -1.61684222944466
    303 -1.54841268390598
    304 -1.61219948369178
    305 -1.58722321543252
    306 -1.53821257019601
    307 -1.58528934164498
    308 -1.63487751464336
    309 -1.65086024986242
    310 -1.5729393428588
    311 -1.54175974265482
    312 -1.50242775385634
    313 -1.56880555615046
    314 -1.59057875121698
    315 -1.61317752519818
    316 -1.66962582745897
    317 -1.64186106665734
    318 -1.622012389535
    319 -1.68716621671745
    320 -1.81039209630784
    321 -1.79307002048497
    322 -1.84227464545275
    323 -1.78997408394438
    324 -1.75679232901681
    325 -1.72161149470853
    326 -1.7074034393564
    327 -1.67809853630838
    328 -1.68565265056919
    329 -1.62522491288081
    330 -1.58278612715313
    331 -1.65761907355667
    332 -1.6166028294997
    333 -1.68587515134785
    334 -1.67566285424593
    335 -1.72226009847385
    336 -1.71344030029256
    337 -1.69670793116649
    338 -1.69196762505899
    339 -1.73391657538068
    340 -1.75295073997245
    341 -1.77830325037511
    342 -1.83363794025708
    343 -1.84721082391844
    344 -1.92275668372916
    345 -1.87260410744597
    346 -1.87955749740886
    347 -1.95500593289403
    348 -1.79717281099182
    349 -1.74127824903935
    350 -1.72345842194101
    351 -1.71037703126527
    352 -1.71640088415718
    353 -1.73473767633097
    354 -1.63646091719163
    355 -1.64919656746477
    356 -1.62182980552061
    357 -1.62557612570337
    358 -1.72665000268766
    359 -1.74292089874372
    360 -1.67268783292016
    361 -1.60397237023229
    362 -1.61791521023985
    363 -1.54659956488939
    364 -1.54763368142977
    365 -1.58489386964169
    366 -1.6009555138733
    367 -1.56640379046701
    368 -1.56740533557098
    369 -1.57129994352615
    370 -1.66820677464268
    371 -1.72472486410102
    372 -1.70626036227175
    373 -1.70990906670834
    374 -1.77469245665436
    375 -1.73423463691198
    376 -1.76547143786352
    377 -1.78007254676458
    378 -1.78849087384423
    379 -1.72739192931725
    380 -1.70908411555946
    381 -1.66504642667628
    382 -1.59948138714315
    383 -1.61741746711208
    384 -1.58275048578073
    385 -1.5407622475667
    386 -1.51259187684828
    387 -1.5311575766095
    388 -1.53472855701268
    389 -1.58200768898556
    390 -1.58953830490258
    391 -1.66688221216628
    392 -1.70783981022872
    393 -1.70489956619045
    394 -1.69810093320986
    395 -1.73147701556854
    396 -1.71499397279615
    397 -1.65804594865972
    398 -1.63983726866126
    399 -1.66717757251972
    400 -1.68576258802428
    401 -1.65600887253408
    402 -1.66104200785218
    403 -1.68736658248249
    404 -1.74842994861149
    405 -1.78873424144554
    406 -1.7093977507564
    407 -1.75944988480566
    408 -1.73637367723933
    409 -1.68712707683302
    410 -1.66510342755253
    411 -1.70473875223421
    412 -1.75154719529007
    413 -1.73582114179154
    414 -1.65758841828882
    415 -1.64775033380921
    416 -1.7292576028622
    417 -1.73046162573797
    418 -1.82445950242996
    419 -1.90355828352414
    420 -1.92959287646233
    421 -1.86747068393416
    422 -1.79722298562105
    423 -1.83397830888251
    424 -1.83679036308414
    425 -1.85659008624413
    426 -1.90111725751723
    427 -1.87716387202079
    428 -1.85759077301394
    429 -1.79341431378776
    430 -1.68556439253371
    431 -1.70656027574559
    432 -1.73424905406948
    433 -1.76281536268961
    434 -1.80075888730337
    435 -1.75134299486631
    436 -1.73370214961167
    437 -1.78230742513696
    438 -1.75564322727126
    439 -1.79994598181842
    440 -1.9366231721563
    441 -1.94509300158288
    442 -1.90250010651426
    443 -1.89227485701815
    444 -1.89180764073253
    445 -1.92929408165098
    446 -1.9028207280632
    447 -1.90347739577159
    448 -1.96814590838359
    449 -1.92652993952942
    450 -1.8533167466678
    451 -1.84015629768236
    452 -1.89167489688103
    453 -1.83016217066298
    454 -1.78834781406541
    455 -1.77347103675291
    456 -1.75833247342286
    457 -1.79889098135482
    458 -1.7811042379615
    459 -1.76192736958711
    460 -1.75091790073403
    461 -1.76312146926225
    462 -1.78872397738424
    463 -1.81314356640262
    464 -1.82980990997119
    465 -1.88792528038926
    466 -1.94404728582643
    467 -1.82648229385622
    468 -1.81357861945167
    469 -1.87142207991988
    470 -1.96624610930527
    471 -2.02850134406032
    472 -2.01897610077542
    473 -2.07284390062708
    474 -2.1441747966409
    475 -2.10176205830941
    476 -2.15881882524741
    477 -2.29406988642185
    478 -2.28997238677368
    479 -2.27123450849538
    480 -2.24265683622981
    481 -2.12097513412923
    482 -2.09196970498017
    483 -2.08334583399385
    484 -2.05004104164994
    485 -2.09990766505758
    486 -2.03132690354552
    487 -1.99593511386129
    488 -2.00245400883759
    489 -1.97729486921318
    490 -1.98348079377337
    491 -2.00682219731997
    492 -1.93774692422504
    493 -1.88061698847151
    494 -1.86081118553533
    495 -1.86218844157996
    496 -1.87993713716434
    497 -1.89605363050239
    498 -1.88700008001691
    499 -1.91441486289031
    500 -1.85682908954949
    501 -1.83412936813873
    502 -1.91448871541545
    503 -1.97671711839316
    504 -2.02188465978813
    505 -1.97815273341976
    506 -1.97912868009057
    507 -1.84707181238678
    508 -1.88607653067413
    509 -1.95729332598474
    510 -2.03619703254032
    511 -2.10209201685892
    512 -2.08659636420696
    513 -2.07668340074131
    514 -2.07374647235412
    515 -2.0233300358582
    516 -2.04906452772891
    517 -2.26312896290231
    518 -2.18720419931532
    519 -2.1190550722254
    520 -2.12100344514399
    521 -2.10080526579919
    522 -2.08294027869591
    523 -2.05136669820184
    524 -2.01107735742857
    525 -2.05452309078722
    526 -2.00896613934571
    527 -1.93017461835557
    528 -1.97316140362851
    529 -1.96724974359889
    530 -1.92627242736848
    531 -1.95558412621896
    532 -1.98539598555125
    533 -1.96847505785009
    534 -2.02757563655684
    535 -2.0347921128448
    536 -2.0877342395571
    537 -2.1287722881458
    538 -2.11552165906283
    539 -2.15868593390061
    540 -2.18006926814608
    541 -2.21067735230669
    542 -2.16794809241311
    543 -2.25227371570905
    544 -2.26030812089951
    545 -2.18358089479687
    546 -2.16263329675059
    547 -2.17079299293129
    548 -2.19237459145475
    549 -2.17057042196974
    550 -2.11198341469002
    551 -1.99181239098593
    552 -1.94722303167475
    553 -1.89583945922493
    554 -1.87485507637924
    555 -1.89747063094149
    556 -1.87595554895352
    557 -1.86604229657732
    558 -1.85785107184066
    559 -1.80151987593684
    560 -1.82565097140559
    561 -1.86336398774891
    562 -1.91874095074991
    563 -1.94800614427672
    564 -1.98931457499221
    565 -1.9149963900075
    566 -1.92881462773504
    567 -1.82534683971489
    568 -1.83107110469797
    569 -1.87454543234503
    570 -1.92360532526355
    571 -1.9652749384437
    572 -1.99983502788401
    573 -1.96373871389543
    574 -1.97481260917651
    575 -2.01070462621801
    576 -2.00842963948643
    577 -2.06722017888962
    578 -1.9406089772059
    579 -1.95858279864781
    580 -1.89432250800649
    581 -1.84850038307219
    582 -1.86557922362519
    583 -1.90209197760313
    584 -1.84739454382805
    585 -1.87180805281924
    586 -1.88626985140351
    587 -1.91292591718314
    588 -2.0630517810061
    589 -2.03978347422945
    590 -2.06816702122585
    591 -2.11333957668766
    592 -2.0300775324319
    593 -2.04274226883406
    594 -2.00699854004065
    595 -2.07148600427419
    596 -1.98230841789933
    597 -1.98991239424799
    598 -1.97453253894747
    599 -1.95730657188246
    600 -1.93545203649219
    601 -1.89671019255538
    602 -1.94585013419436
    603 -1.91782778745475
    604 -1.90857173084491
    605 -1.88177393983656
    606 -1.90841755315096
    607 -1.96930952664414
    608 -1.96582054736378
    609 -1.93352411475562
    610 -1.96777599323095
    611 -2.01273449503172
    612 -1.95628832332291
    613 -2.00157462400566
    614 -2.05601808316765
    615 -2.0756524944674
    616 -2.06640791204032
    617 -2.00563723785743
    618 -1.93688923542945
    619 -1.98718798689861
    620 -1.8350231653867
    621 -1.83419308270154
    622 -1.82354248963973
    623 -1.81633809983316
    624 -1.80459729659084
    625 -1.83487330591013
    626 -1.85249013733494
    627 -1.86988890852304
    628 -1.89344701100597
    629 -1.92202143260536
    630 -2.13877759601915
    631 -2.09639404225132
    632 -2.1559071047659
    633 -2.16298555860467
    634 -2.19346142908026
    635 -2.16477248663194
    636 -2.21430840770845
    637 -2.21759752506758
    638 -2.27679819799481
    639 -2.22102763914596
    640 -2.19125274891154
    641 -2.14780329912233
    642 -2.16073501812559
    643 -2.14348271820116
    644 -2.13132947191019
    645 -2.162518476065
    646 -2.14997763935046
    647 -2.18331065647836
    648 -2.06030496794769
    649 -2.05837168260458
    650 -2.02126869499082
    651 -2.07709070441654
    652 -2.05329965631575
    653 -1.91899362651223
    654 -1.87248605212467
    655 -1.81247550913541
    656 -1.80468234260122
    657 -1.77992638767794
    658 -1.81718316817755
    659 -1.85387376804438
    660 -1.86661298659525
    661 -1.88002884325744
    662 -1.88216323985099
    663 -1.99315624242428
    664 -2.02945436442459
    665 -2.10079524916119
    666 -2.06950610383673
    667 -2.03109194719523
    668 -2.04166733499079
    669 -2.01965669089586
    670 -2.04830432266519
    671 -2.09122315872181
    672 -2.06887756926605
    673 -1.97577006059544
    674 -1.94959167953088
    675 -1.94264989191931
    676 -1.98279628040414
    677 -1.95097118941904
    678 -1.90272171814062
    679 -1.8506924548753
    680 -1.83975076418811
    681 -1.74451841991261
    682 -1.73023503429884
    683 -1.75793794267212
    684 -1.80229778001154
    685 -1.78899913791049
    686 -1.70392869094771
    687 -1.76859693663379
    688 -1.85915350198342
    689 -1.94111449834686
    690 -1.90126945579501
    691 -2.01963295842438
    692 -2.06419528541524
    693 -2.17056682441268
    694 -2.02311708651152
    695 -1.98734471750778
    696 -2.12447689669161
    697 -2.02491334396607
    698 -1.98532660173891
    699 -1.96831724587456
    700 -1.9369730679843
    701 -1.88468587957755
    702 -1.87542177898735
    703 -1.82454800071725
    704 -2.00067512148937
    705 -1.94475665259853
    706 -1.9187595885841
    707 -2.02043039040957
    708 -1.96103243471979
    709 -1.94643954040583
    710 -1.93033384606721
    711 -1.93548213159511
    712 -1.96321398706701
    713 -1.98934399339061
    714 -1.87817980948697
    715 -1.89420872017079
    716 -1.87590713376414
    717 -1.83346957046209
    718 -1.80492691112727
    719 -1.77628754087933
    720 -1.81794871908629
    721 -1.81421813555004
    722 -1.82197990074892
    723 -1.80411936488531
    724 -1.79606495881072
    725 -1.81822300272098
    726 -1.85075043827278
    727 -1.9036394202978
    728 -1.96148383198694
    729 -1.8863256034391
    730 -1.912065657702
    731 -1.91152761553317
    732 -1.88376931866666
    733 -1.8703925269824
    734 -1.95254058033234
    735 -1.90297528675642
    736 -1.90498618345884
    737 -1.89428076844717
    738 -1.93451171512013
    739 -2.06696024881353
    740 -2.0841646337254
    741 -2.14573252126863
    742 -2.12558792615285
    743 -2.20273193622639
    744 -2.12646514437578
    745 -2.19483890069047
    746 -2.14908973717355
    747 -2.01123959519185
    748 -1.95687515232006
    749 -1.90906069851291
    750 -1.82612456034375
    751 -1.6984834988193
    752 -1.71722032776518
    753 -1.68711133836885
    754 -1.73628812684978
    755 -1.78654072578862
    756 -1.80946879450175
    757 -1.91677203894027
    758 -1.96494474076251
    759 -1.96378560605641
    760 -2.04605826348805
    761 -2.17180295825732
    762 -2.15877162681058
    763 -2.20344041491076
    764 -2.18973351404021
    765 -2.14955599665844
    766 -2.18607299325567
    767 -2.07385961192119
    768 -2.08999861768995
    769 -2.15618795140528
    770 -2.12808470179855
    771 -2.14107957841703
    772 -2.2008889099157
    773 -2.03248892756739
    774 -2.09524096088554
    775 -2.10620569924837
    776 -2.07512144822832
    777 -2.18755290127736
    778 -2.18272415299611
    779 -2.16324828655733
    780 -2.20202973847815
    781 -2.14530563117808
    782 -2.14974532092199
    783 -2.20918925795745
    784 -2.10451853921662
    785 -2.0941960464274
    786 -2.07011891250716
    787 -2.06144397572097
    788 -2.073252391729
    789 -2.03750229355968
    790 -1.98734800088373
    791 -1.99365921393512
    792 -1.90171654406767
    793 -2.00953643319982
    794 -2.00707516854092
    795 -2.01124071982787
    796 -1.96647697239197
    797 -1.87740296180196
    798 -1.81955441007702
    799 -1.8382135338597
    800 -1.88882787209919
    801 -1.9156873588204
    802 -1.9793737270519
    803 -1.91288844490517
    804 -1.97916380135658
    805 -1.96948995177392
    806 -2.01478794735818
    807 -2.10874189671327
    808 -2.1822612953424
    809 -2.15714251914859
    810 -2.12417345253643
    811 -2.17656522688337
    812 -2.18647326390366
    813 -2.19789874904197
    814 -2.07215673667013
    815 -2.13680531039752
    816 -2.20794272372552
    817 -2.24731445681012
    818 -2.190740668862
    819 -2.25276669849572
    820 -2.27354080133713
    821 -2.19613982268345
    822 -2.14116209564335
    823 -2.18057254791433
    824 -2.28134575213079
    825 -2.12767260073643
    826 -2.06490516892911
    827 -2.02465451401392
    828 -2.10097156578361
    829 -2.11060295217913
    830 -2.08357839077679
    831 -2.10350844400261
    832 -2.12286431870697
    833 -2.16995783553197
    834 -2.1340174003704
    835 -2.26777529574148
    836 -2.24993456899611
    837 -2.24658286859031
    838 -2.10263068358521
    839 -2.11056619953345
    840 -2.1001208122636
    841 -2.07100077264937
    842 -2.087795607549
    843 -2.07467390499104
    844 -2.08849116427573
    845 -2.10093875849598
    846 -2.14923434956328
    847 -2.11730376985164
    848 -2.25892038445106
    849 -2.1080844285133
    850 -2.16310267197343
    851 -2.25142038638579
    852 -2.21531278477218
    853 -2.1130476695796
    854 -2.00938282667232
    855 -1.95988148061957
    856 -1.89885129645604
    857 -1.89634522354322
    858 -1.84981565964616
    859 -1.80615398510801
    860 -1.80289788447232
    861 -1.75545144920118
    862 -1.70896312758573
    863 -1.7741307454251
    864 -1.83232381069752
    865 -1.6934857564079
    866 -1.66780358574362
    867 -1.68135363970204
    868 -1.65569735579017
    869 -1.74533365738013
    870 -1.75109368326354
    871 -1.76055982160287
    872 -1.86200412025163
    873 -1.78046229977007
    874 -1.75893379100858
    875 -1.90292826591251
    876 -1.96648984804019
    877 -2.03814272356402
    878 -2.0191576668409
    879 -2.02358054448845
    880 -1.96574479465385
    881 -1.94170424302248
    882 -1.95233328580313
    883 -1.99282549140348
    884 -2.07161543264345
    885 -2.04481642445892
    886 -2.06568188449093
    887 -1.95181853994856
    888 -1.98878537140597
    889 -2.0642995558108
    890 -2.02571182215167
    891 -2.10304475684604
    892 -2.0768914402603
    893 -2.1097244467937
    894 -2.07593233744567
    895 -2.06034783093917
    896 -2.04618367237759
    897 -1.99215064257829
    898 -2.00218540065346
    899 -1.95415173197709
    900 -2.02681182600604
    901 -2.03063438749266
    902 -1.99568555595962
    903 -1.9306961820476
    904 -1.95179893991792
    905 -2.02051747603065
    906 -2.08590326054596
    907 -2.16527679489107
    908 -2.13576808439002
    909 -2.11398055338361
    910 -2.07110665653901
    911 -2.05012550621552
    912 -2.02384847277175
    913 -2.04759400931296
    914 -2.06437701934793
    915 -2.03045102663446
    916 -1.87603081695538
    917 -1.94070816309312
    918 -1.99260036185798
    919 -2.01554842110843
    920 -2.03896505628862
    921 -2.01283890512976
    922 -2.00692877753528
    923 -2.01066378344395
    924 -1.97137268911862
    925 -1.9995044131604
    926 -2.06888718277469
    927 -1.97945723128184
    928 -1.98450866542372
    929 -2.01583898312069
    930 -1.98886894391651
    931 -1.93245879216259
    932 -1.99809811477137
    933 -2.02929512825398
    934 -2.03453829863059
    935 -2.03530544381277
    936 -2.03370197647659
    937 -2.13485486523686
    938 -2.11930937102036
    939 -2.11659135537264
    940 -2.1168099946002
    941 -2.19771833892398
    942 -2.16100977732796
    943 -2.1488325279643
    944 -2.10792337162651
    945 -1.99634880521756
    946 -2.08254050203818
    947 -2.04927215939996
    948 -2.06908576958255
    949 -2.07688920780863
    950 -2.01957455042179
    951 -2.03801913101752
    952 -1.98274940547014
    953 -1.95512528781991
    954 -2.05875066334648
    955 -2.19969021021348
    956 -2.16943311259474
    957 -2.13935077795442
    958 -2.14624132841365
    959 -2.0326197302821
    960 -2.05037643577594
    961 -2.0403357225063
    962 -2.12369343205775
    963 -2.18015482576934
    964 -2.09753235131423
    965 -2.08582081525846
    966 -2.0449529324288
    967 -2.01800314418354
    968 -2.03475729385869
    969 -2.15238438606994
    970 -2.2793163779255
    971 -2.30313780684631
    972 -2.27878683868092
    973 -2.23234607853457
    974 -2.19134482838369
    975 -2.21832614786631
    976 -2.261215525841
    977 -2.3807997840712
    978 -2.29589577269642
    979 -2.24903511733204
    980 -2.14925538682272
    981 -2.10562461480359
    982 -2.11579319603918
    983 -2.09657862512731
    984 -2.17798063052025
    985 -2.1308997914093
    986 -2.10012667518389
    987 -2.04027739951953
    988 -2.06002490673839
    989 -2.04499576433139
    990 -2.08266908494393
    991 -2.08759711053616
    992 -2.07136075208485
    993 -2.12042693829845
    994 -2.10915393733551
    995 -2.11427235649833
    996 -2.1673664681495
    997 -2.20674112539978
    998 -2.22863828136935
    999 -2.25751529105938
    1000 -2.23526583051227
    1001 -2.18172856308894
    1002 -2.15154260124508
    1003 -2.20399369741664
    1004 -2.19798669140006
    1005 -2.19540033110646
    1006 -2.13575350541222
    1007 -2.14298766917753
    1008 -2.12219712323628
    1009 -2.15318458405029
    1010 -2.16010437335791
    1011 -2.25423061233617
    1012 -2.25270088040722
    1013 -2.18878588358492
    1014 -2.16204437070511
    1015 -2.15092281813702
    1016 -2.14567972699945
    1017 -2.03561066400906
    1018 -2.04767009998156
    1019 -2.0212713437193
    1020 -2.08573564463
    1021 -2.06241559842327
    1022 -2.11870774708644
    1023 -2.15871627819718
    1024 -2.16386389153045
    1025 -2.13280035510254
    1026 -2.15728896448688
    1027 -2.23068024361579
    1028 -2.2842277356919
    1029 -2.27317623834604
    1030 -2.22583883956454
    1031 -2.24713824459509
    1032 -2.24118105508755
    1033 -2.23946854220756
    1034 -2.24804534599808
    1035 -2.2699693063624
    1036 -2.26942548403862
    1037 -2.24023566874597
    1038 -2.17124519600626
    1039 -2.11132287104688
    1040 -2.09704043616953
    1041 -2.09772896053928
    1042 -2.06690759516639
    1043 -1.93933028899103
    1044 -1.95949108855505
    1045 -1.93463217346055
    1046 -1.91207596581177
    1047 -1.9806424501682
    1048 -1.96714934441897
    1049 -2.04275565062395
    1050 -2.09404967593316
    1051 -2.12791688791314
    1052 -2.16634337783451
    1053 -2.29358540100447
    1054 -2.26057997410235
    1055 -2.31961128894234
    1056 -2.38527524987061
    1057 -2.42531879267706
    1058 -2.46318998946104
    1059 -2.29693798225171
    1060 -2.14834230162379
    1061 -2.06679989192274
    1062 -2.08028661269731
    1063 -2.05581836902473
    1064 -2.10214304710647
    1065 -2.10571247242041
    1066 -2.11165883458473
    1067 -2.05000410185917
    1068 -2.0645955600414
    1069 -2.1281442943918
    1070 -2.15939982045474
    1071 -2.17095813295873
    1072 -2.04625068693978
    1073 -2.03251854059768
    1074 -1.97762820182295
    1075 -2.00120968080984
    1076 -1.96094131407886
    1077 -1.94789989823877
    1078 -1.93885770921476
    1079 -2.00196598986719
    1080 -2.0766746808619
    1081 -2.08968582456713
    1082 -2.25722611654907
    1083 -2.30091792968402
    1084 -2.33691608479417
    1085 -2.31885163749517
    1086 -2.33762782968074
    1087 -2.36558245267484
    1088 -2.3756024360302
    1089 -2.28446571910153
    1090 -2.1509778362963
    1091 -2.1115061683627
    1092 -2.06451808544631
    1093 -2.08544609835376
    1094 -1.98554611932387
    1095 -1.98449421407814
    1096 -1.97530124784345
    1097 -1.9481152711789
    1098 -1.93684218704119
    1099 -1.99774652488853
    1100 -2.12979624628431
    1101 -2.21491697365835
    1102 -2.22737973644617
    1103 -2.14856373806321
    1104 -2.19979256578528
    1105 -2.23287835213002
    1106 -2.27734905698493
    1107 -2.26014222232809
    1108 -2.27466921837462
    1109 -2.22464343788047
    1110 -2.21590500878799
    1111 -2.05792142838241
    1112 -1.99825491062952
    1113 -2.07675993922369
    1114 -2.11982130296369
    1115 -2.01838154452073
    1116 -1.9469564604859
    1117 -2.00686206864253
    1118 -2.01873749316693
    1119 -1.96562332512035
    1120 -1.96492466382809
    1121 -2.08301781800816
    1122 -2.1491824693483
    1123 -2.11863454382562
    1124 -2.09043423977103
    1125 -2.20550274365258
    1126 -2.21128784314735
    1127 -2.18803698161302
    1128 -2.09593758628241
    1129 -2.21678437845644
    1130 -2.18713208940382
    1131 -2.12067857060455
    1132 -2.060477381318
    1133 -2.04723527724559
    1134 -2.05892589213197
    1135 -2.01501210951145
    1136 -2.10945586075653
    1137 -2.14994188912876
    1138 -2.17923966410539
    1139 -2.09453137180228
    1140 -2.14067433104307
    1141 -2.19704367125208
    1142 -2.19687462967011
    1143 -2.23357195513982
    1144 -2.2478198054488
    1145 -2.26306519151851
    1146 -2.19355692236355
    1147 -2.17341114163838
    1148 -2.15393156821655
    1149 -2.17685737997356
    1150 -2.08477440163841
    1151 -2.09109918500826
    1152 -2.1093093092879
    1153 -2.07814022396812
    1154 -2.09625800281537
    1155 -2.12878428596423
    1156 -2.18749310932768
    1157 -2.18515172619725
    1158 -2.25499072656769
    1159 -2.25418438902629
    1160 -2.33109176822926
    1161 -2.24295111406129
    1162 -2.28080235063282
    1163 -2.28123391942318
    1164 -2.33995291247834
    1165 -2.21778942250977
    1166 -2.19481269808836
    1167 -2.17829805237813
    1168 -2.21027326581973
    1169 -2.19951098547187
    1170 -2.20760038922398
    1171 -2.31789331638992
    1172 -2.15199720079808
    1173 -2.18539261900275
    1174 -2.08853577291675
    1175 -2.17779242690211
    1176 -2.15920233826263
    1177 -2.17291640645479
    1178 -2.12333470522348
    1179 -2.12283444233314
    1180 -2.03073081378865
    1181 -1.93549179360036
    1182 -1.9759043261657
    1183 -1.93719575277449
    1184 -1.98630308901996
    1185 -1.97975025639597
    1186 -2.01079028460868
    1187 -2.01208594908295
    1188 -1.90752151336267
    1189 -1.87108012881164
    1190 -1.95942051333766
    1191 -2.0377154188059
    1192 -2.16140744603323
    1193 -2.09989492585219
    1194 -2.07824279149772
    1195 -2.01486846627778
    1196 -1.96846360756292
    1197 -1.95688622548826
    1198 -2.09807966748208
    1199 -2.16638575299279
    1200 -2.11911859124808
    1201 -2.12410469400938
    1202 -2.10426050070368
    1203 -2.23294705839145
    1204 -2.17143708549042
    1205 -2.155360010629
    1206 -2.21743310555081
    1207 -2.16116445473062
    1208 -2.18016765065045
    1209 -2.15451528459814
    1210 -2.20597786222948
    1211 -2.18754680653515
    1212 -2.08336044350276
    1213 -2.00512011699622
    1214 -2.07455288827413
    1215 -2.16538558997722
    1216 -2.13152664895772
    1217 -2.13636343439616
    1218 -2.11279836896757
    1219 -2.11759363407904
    1220 -2.08968030857782
    1221 -2.08533411656555
    1222 -2.21541925132189
    1223 -2.2066788872612
    1224 -2.2375522283668
    1225 -2.27565606322769
    1226 -2.25011987501871
    1227 -2.2317249664124
    1228 -2.25019669092756
    1229 -2.29732061461107
    1230 -2.30828999114595
    1231 -2.33822761987752
    1232 -2.31747438064101
    1233 -2.38453417410232
    1234 -2.34016261024588
    1235 -2.16710029589889
    1236 -2.23851515690377
    1237 -2.3464741328881
    1238 -2.30153577123695
    1239 -2.283911347425
    1240 -2.30872246516561
    1241 -2.31839302989086
    1242 -2.29116701891846
    1243 -2.26423784473548
    1244 -2.29514714404724
    1245 -2.43893266685467
    1246 -2.41144528357809
    1247 -2.36450254042965
    1248 -2.39722748049027
    1249 -2.36349426129802
    1250 -2.31678665936087
    1251 -2.3596587925
    1252 -2.3966616549796
    1253 -2.45804980869007
    1254 -2.44744467681494
    1255 -2.34083739128077
    1256 -2.36486340317792
    1257 -2.40529506373024
    1258 -2.28593844859623
    1259 -2.39646571247916
    1260 -2.3638799307197
    1261 -2.27593189306081
    1262 -2.27849266263789
    1263 -2.27741082515841
    1264 -2.26321185602311
    1265 -2.30609722085851
    1266 -2.27759317686255
    1267 -2.24441730124913
    1268 -2.33569389015233
    1269 -2.2714230170415
    1270 -2.31506663683698
    1271 -2.37022660133057
    1272 -2.39870809706026
    1273 -2.38802189892881
    1274 -2.45490376164864
    1275 -2.46833097112438
    1276 -2.38879823678286
    1277 -2.43479076123074
    1278 -2.48965511321046
    1279 -2.51454775387387
    1280 -2.46591336727266
    1281 -2.45103340267253
    1282 -2.4364123813032
    1283 -2.39883425643783
    1284 -2.2938251937393
    1285 -2.3517483491982
    1286 -2.41291655059338
    1287 -2.37484926977844
    1288 -2.23453741161236
    1289 -2.16108953189709
    1290 -2.26802921888622
    1291 -2.27916391361842
    1292 -2.26644804513351
    1293 -2.29270767005756
    1294 -2.40043470951452
    1295 -2.30413434243291
    1296 -2.28162957839015
    1297 -2.26799667224457
    1298 -2.41593004126259
    1299 -2.52621722044645
    1300 -2.4710327948988
    1301 -2.39556355473017
    1302 -2.40312871174176
    1303 -2.31745820491714
    1304 -2.28038891215918
    1305 -2.37184609333548
    1306 -2.30985494241337
    1307 -2.25942375936877
    1308 -2.2312464709653
    1309 -2.13193745207416
    1310 -1.98267316297668
    1311 -2.01061255427366
    1312 -1.93398065964872
    1313 -2.02252203457297
    1314 -2.04791392531766
    1315 -1.98797876935804
    1316 -2.0666998831159
    1317 -2.05000457354551
    1318 -2.02313823404477
    1319 -2.05379617690013
    1320 -2.13741245173347
    1321 -2.1470976946593
    1322 -2.0683434658668
    1323 -1.97789475999997
    1324 -1.92184892058557
    1325 -1.91138325221038
    1326 -1.89431727251112
    1327 -1.96045974095803
    1328 -1.98439181932184
    1329 -1.91362769934396
    1330 -1.92159452475619
    1331 -1.87634728952797
    1332 -1.94750691213893
    1333 -1.97261970087484
    1334 -2.0155076875468
    1335 -2.08543120949753
    1336 -2.09117071965015
    1337 -2.00793501165051
    1338 -1.93757017308631
    1339 -1.94879076503154
    1340 -2.00088573597246
    1341 -2.01438640223138
    1342 -2.08177232983584
    1343 -2.13310388169715
    1344 -2.10707294801093
    1345 -2.1136251083051
    1346 -2.05704052233638
    1347 -2.13707499253172
    1348 -2.23036377234062
    1349 -2.30333368708411
    1350 -2.2346438564951
    1351 -2.21697680725458
    1352 -2.19854449734627
    1353 -2.19734930038243
    1354 -2.13668216983275
    1355 -2.12801655920756
    1356 -2.14899648963332
    1357 -2.191756321607
    1358 -2.12727381834354
    1359 -2.15970808785765
    1360 -2.22229511393094
    1361 -2.25191671644186
    1362 -2.23011251071542
    1363 -2.24222700256689
    1364 -2.31955777843214
    1365 -2.27769280442745
    1366 -2.36485495449788
    1367 -2.29431904297542
    1368 -2.37303122832112
    1369 -2.33539745370479
    1370 -2.28517291065818
    1371 -2.27534655475736
    1372 -2.33440833834777
    1373 -2.31154141093018
    1374 -2.31912983906259
    1375 -2.25847911301179
    1376 -2.18157751785302
    1377 -2.21369303369969
    1378 -2.17371455696485
    1379 -2.19421330626112
    1380 -2.17039512303691
    1381 -2.20706347316624
    1382 -2.18081227735304
    1383 -2.18095408952395
    1384 -2.05485213476921
    1385 -2.10414001387099
    1386 -2.09925025355587
    1387 -2.07539353065678
    1388 -2.11574722189878
    1389 -1.98771113050457
    1390 -2.02588827649861
    1391 -1.98061677477225
    1392 -1.93514069746142
    1393 -1.94946862065816
    1394 -2.02136053854462
    1395 -2.03971006923603
    1396 -2.0570173604302
    1397 -2.05478893769039
    1398 -2.08069695350794
    1399 -2.20620631113151
    1400 -2.20856125522129
    1401 -2.2689959328506
    1402 -2.12781945365293
    1403 -2.12118095324834
    1404 -2.10375261145799
    1405 -2.11908400799646
    1406 -2.07407260792615
    1407 -2.09326643564913
    1408 -2.05533647781501
    1409 -2.04891823330192
    1410 -2.06422652752785
    1411 -2.06022605923659
    1412 -2.27350638496776
    1413 -2.24053116747062
    1414 -2.33061515453597
    1415 -2.28715242725341
    1416 -2.27199121646714
    1417 -2.1682724583404
    1418 -2.11358689990505
    1419 -2.19505413987505
    1420 -2.23220390272593
    1421 -2.24127280696135
    1422 -2.12080578805717
    1423 -2.17876096974472
    1424 -2.15096385658887
    1425 -2.12608265233379
    1426 -2.21064800614149
    1427 -2.26315380218328
    1428 -2.28979260403392
    1429 -2.26213033408463
    1430 -2.23203271438912
    1431 -2.24532917319263
    1432 -2.34772237727404
    1433 -2.28335011676056
    1434 -2.32450361160025
    1435 -2.40102185672732
    1436 -2.44750316232279
    1437 -2.52311340123259
    1438 -2.61675395942871
    1439 -2.52968994330124
    1440 -2.5326809190427
    1441 -2.5589771265089
    1442 -2.52593440301376
    1443 -2.62104411636711
    1444 -2.55743127698438
    1445 -2.53249221465194
    1446 -2.55811268188917
    1447 -2.53586491553933
    1448 -2.49136853662905
    1449 -2.55072937376224
    1450 -2.47276787932135
    1451 -2.45304277241015
    1452 -2.35970950296229
    1453 -2.3346954069225
    1454 -2.37065939835363
    1455 -2.39065431817127
    1456 -2.34107481975903
    1457 -2.3892620734014
    1458 -2.39026493999767
    1459 -2.38988902256724
    1460 -2.47523638549212
    1461 -2.4441896011888
    1462 -2.54509740313308
    1463 -2.54555503271957
    1464 -2.47277628820486
    1465 -2.43728591403817
    1466 -2.41709900470216
    1467 -2.36051193582255
    1468 -2.31878375373065
    1469 -2.29095442580877
    1470 -2.25407834933389
    1471 -2.31275835570514
    1472 -2.34662538666745
    1473 -2.35638487284129
    1474 -2.39408973666713
    1475 -2.33465881056771
    1476 -2.35787771161926
    1477 -2.32784144933152
    1478 -2.39689753055823
    1479 -2.35133694710502
    1480 -2.29046852297208
    1481 -2.2029064049196
    1482 -2.114618199825
    1483 -2.07870944149976
    1484 -2.09907378728304
    1485 -2.172024679096
    1486 -2.15627448609554
    1487 -2.16192730676568
    1488 -2.08080295592172
    1489 -2.14447544128523
    1490 -2.15832286739367
    1491 -2.18432644592054
    1492 -2.2954426955762
    1493 -2.30532512879475
    1494 -2.25505281991357
    1495 -2.22240841910033
    1496 -2.18862927453702
    1497 -2.17498593078997
    1498 -2.25038831100451
    1499 -2.19775001854126
    1500 -2.23680777149152
    1501 -2.2022394498838
    1502 -2.15366753524348
    1503 -2.14027409356379
    1504 -2.21394081488522
    1505 -2.25343081180805
    1506 -2.16822930780135
    1507 -2.23701640994125
    1508 -2.24051646450855
    1509 -2.28407336153574
    1510 -2.30188392444897
    1511 -2.28173016713042
    1512 -2.34965027444867
    1513 -2.1862831485315
    1514 -2.18387301089299
    1515 -2.19092881001381
    1516 -2.20255017348673
    1517 -2.19417107238637
    1518 -2.13562366603785
    1519 -2.18020828858052
    1520 -2.166092391822
    1521 -2.21643680319154
    1522 -2.21935473244381
    1523 -2.39251659403993
    1524 -2.35903258809229
    1525 -2.29975517208723
    1526 -2.33023722949293
    1527 -2.32192610103666
    1528 -2.40656671389349
    1529 -2.34257605584515
    1530 -2.18874011836222
    1531 -2.24069036682889
    1532 -2.22714140070616
    1533 -2.22689976816349
    1534 -2.2140976113436
    1535 -2.26248941708615
    1536 -2.11839596839299
    1537 -2.01738021825849
    1538 -1.87622986652693
    1539 -1.89716250376974
    1540 -1.97744124564423
    1541 -1.95947694667987
    1542 -1.89460097976719
    1543 -1.8846044584946
    1544 -1.84303534195711
    1545 -1.81218500748608
    1546 -1.98221757690813
    1547 -2.07529033786066
    1548 -2.24980168386828
    1549 -2.26451001182349
    1550 -2.30228158750182
    1551 -2.2933755893784
    1552 -2.31217145353239
    1553 -2.28500720252054
    1554 -2.38432044090663
    1555 -2.33238834349328
    1556 -2.26988105599008
    1557 -2.21219490937485
    1558 -2.22516363775401
    1559 -2.19332077447588
    1560 -2.20042826399176
    1561 -2.16309713757604
    1562 -2.21518089176923
    1563 -2.25138066712008
    1564 -2.1682584660078
    1565 -2.25395149835712
    1566 -2.32049155779483
    1567 -2.41835452008772
    1568 -2.42206879621784
    1569 -2.37011907301173
    1570 -2.37912491010808
    1571 -2.45775058291148
    1572 -2.42908543817329
    1573 -2.47712957935761
    1574 -2.42674674453798
    1575 -2.4218164349914
    1576 -2.46591607358826
    1577 -2.31578775995315
    1578 -2.30539034174924
    1579 -2.35056866883219
    1580 -2.38342746454522
    1581 -2.38326920235652
    1582 -2.45717007018814
    1583 -2.46280998628788
    1584 -2.46402544736833
    1585 -2.4256224386335
    1586 -2.42017185724482
    1587 -2.58399102201966
    1588 -2.49098425981143
    1589 -2.39029670119937
    1590 -2.3129500751205
    1591 -2.31729507462544
    1592 -2.25955309421307
    1593 -2.24834631995953
    1594 -2.34188981765463
    1595 -2.39028377699327
    1596 -2.32997517557463
    1597 -2.22082747782561
    1598 -2.27824555996024
    1599 -2.38048215800972
    1600 -2.29123819904349
    1601 -2.24604619183223
    1602 -2.25133286593241
    1603 -2.22103185208958
    1604 -2.24371233726701
    1605 -2.17121843163337
    1606 -2.11558343379689
    1607 -2.2188947985678
    1608 -2.1961265568821
    1609 -2.19615671916847
    1610 -2.30976029455708
    1611 -2.31387471036583
    1612 -2.36464750252753
    1613 -2.38785753260322
    1614 -2.3583519755357
    1615 -2.41719235022873
    1616 -2.48430310396208
    1617 -2.51168341446364
    1618 -2.48203714737967
    1619 -2.49540151472337
    1620 -2.46869248913763
    1621 -2.48647387305348
    1622 -2.45172861700283
    1623 -2.43167553342345
    1624 -2.44692642301314
    1625 -2.40769282336474
    1626 -2.43307381899042
    1627 -2.27969304276529
    1628 -2.0935672931036
    1629 -2.0714134042238
    1630 -2.12631013237842
    1631 -2.08737983578156
    1632 -2.06176950481522
    1633 -2.12332878249404
    1634 -2.09293216844917
    1635 -2.13374105267112
    1636 -2.07765605822393
    1637 -2.13390283523565
    1638 -2.3236848148051
    1639 -2.31133607721667
    1640 -2.31872578224338
    1641 -2.31333976964815
    1642 -2.33232115794842
    1643 -2.27880521641671
    1644 -2.32489884495726
    1645 -2.30113208263989
    1646 -2.35481379496091
    1647 -2.42177551063058
    1648 -2.5328604761014
    1649 -2.5673009657783
    1650 -2.37662147966609
    1651 -2.41178553134614
    1652 -2.40324335902337
    1653 -2.42113458366562
    1654 -2.36461838079562
    1655 -2.35907766549577
    1656 -2.44341868668824
    1657 -2.33860701907412
    1658 -2.33562728112817
    1659 -2.33655524361893
    1660 -2.46499999194016
    1661 -2.43452545009048
    1662 -2.41300567286364
    1663 -2.34423333439138
    1664 -2.40731336974833
    1665 -2.46702363927897
    1666 -2.42389126194891
    1667 -2.49478151141727
    1668 -2.50847394895386
    1669 -2.50069150590071
    1670 -2.55306650811458
    1671 -2.43488254553429
    1672 -2.50043594905402
    1673 -2.54310006361132
    1674 -2.50402569372173
    1675 -2.44591346398971
    1676 -2.45241185627228
    1677 -2.43849995462515
    1678 -2.38957726257136
    1679 -2.26786376705937
    1680 -2.18937941209719
    1681 -2.31332239545539
    1682 -2.18072769570739
    1683 -2.22546427891892
    1684 -2.27205357595496
    1685 -2.22821960366111
    1686 -2.17538741957731
    1687 -2.17817160970368
    1688 -2.15736847761376
    1689 -2.22209052947252
    1690 -2.26304295055765
    1691 -2.30467333756941
    1692 -2.46797054632406
    1693 -2.3533381944633
    1694 -2.30700465728745
    1695 -2.26374436085433
    1696 -2.34724339784352
    1697 -2.34493805613969
    1698 -2.36880101537374
    1699 -2.45527752116807
    1700 -2.45330491948016
    1701 -2.39711160042959
    1702 -2.41352576559479
    1703 -2.42985951089183
    1704 -2.3931080588394
    1705 -2.43309032614365
    1706 -2.34061050592222
    1707 -2.37032466154886
    1708 -2.36842835229964
    1709 -2.36320970316895
    1710 -2.35810298280818
    1711 -2.39479719459598
    1712 -2.36634077917282
    1713 -2.41868339560836
    1714 -2.45684367546726
    1715 -2.6108195848886
    1716 -2.62941232501618
    1717 -2.61128828212944
    1718 -2.51706459810847
    1719 -2.43708355308699
    1720 -2.44462857013803
    1721 -2.44801863463883
    1722 -2.37993899994045
    1723 -2.39780538185849
    1724 -2.36453840048091
    1725 -2.17694884973803
    1726 -2.19900729900296
    1727 -2.24862721261871
    1728 -2.29417791091728
    1729 -2.38617416753862
    1730 -2.36456266857277
    1731 -2.27057875873732
    1732 -2.32479039102949
    1733 -2.2075731251447
    1734 -2.2775341258998
    1735 -2.38675276976528
    1736 -2.37625879175555
    1737 -2.3038869194081
    1738 -2.30474068082634
    1739 -2.21578824029717
    1740 -2.22631588947263
    1741 -2.28692631679705
    1742 -2.22943812647616
    1743 -2.20344807653649
    1744 -2.17040159884277
    1745 -2.20601506077226
    1746 -2.1830947101237
    1747 -2.17803777276754
    1748 -2.16462782634595
    1749 -2.19871333123248
    1750 -2.14594011106883
    1751 -2.1250887217856
    1752 -2.15333830278107
    1753 -2.17791568426628
    1754 -2.1355254937702
    1755 -2.11080420562462
    1756 -2.12855777316455
    1757 -2.09126058845181
    1758 -2.06597559309661
    1759 -2.10442619380587
    1760 -2.12696129867313
    1761 -2.14465822163969
    1762 -2.20232075194216
    1763 -2.29126834328735
    1764 -2.34577823720353
    1765 -2.4310234537716
    1766 -2.49944391101619
    1767 -2.60761757510787
    1768 -2.62627231866681
    1769 -2.466915081424
    1770 -2.55555988563901
    1771 -2.53043575418408
    1772 -2.52133412159275
    1773 -2.46752837827904
    1774 -2.41021221816689
    1775 -2.35313783944636
    1776 -2.12961926745948
    1777 -2.03081814534325
    1778 -2.080546068209
    1779 -2.1515551668742
    1780 -2.14185208399825
    1781 -2.05388298114803
    1782 -2.0036453667246
    1783 -1.9945968995588
    1784 -1.95297539114535
    1785 -1.96071900608551
    1786 -2.13660719400417
    1787 -2.23402923641509
    1788 -2.19881902891719
    1789 -2.25783009371821
    1790 -2.19704582333777
    1791 -2.34607701644333
    1792 -2.35747405223228
    1793 -2.45177199664657
    1794 -2.58433869372368
    1795 -2.49429149188511
    1796 -2.43107798330768
    1797 -2.34186338142126
    1798 -2.27678223864356
    1799 -2.22677713661421
    1800 -2.22086121288832
    1801 -2.22931614573163
    1802 -2.20880279046195
    1803 -2.2067649074299
    1804 -2.1718447972497
    1805 -2.22251776202791
    1806 -2.21990387731912
    1807 -2.2662481627168
    1808 -2.35781827737212
    1809 -2.36497757638619
    1810 -2.38859759690969
    1811 -2.30291288285057
    1812 -2.30475245954917
    1813 -2.28397286775389
    1814 -2.22003890561864
    1815 -2.19260926448576
    1816 -2.21501508411144
    1817 -2.22892436796712
    1818 -2.23990550156194
    1819 -2.20311652117569
    1820 -2.24916625494117
    1821 -2.22799043059727
    1822 -2.2303973835608
    1823 -2.25741721195318
    1824 -2.34909711347834
    1825 -2.43561488083094
    1826 -2.42331223993576
    1827 -2.42287729212043
    1828 -2.37709143246495
    1829 -2.35520271525383
    1830 -2.31995216452366
    1831 -2.40894643833913
    1832 -2.30956856479557
    1833 -2.21118224390933
    1834 -2.18421454802579
    1835 -2.11438390440007
    1836 -2.10287813311743
    1837 -2.13280330063475
    1838 -2.07059345413408
    1839 -2.08553181702541
    1840 -2.09341905925164
    1841 -2.05084908094652
    1842 -2.16053672590012
    1843 -2.23887089474136
    1844 -2.29834328148429
    1845 -2.32489543940429
    1846 -2.2415957397434
    1847 -2.20946105810918
    1848 -2.28245431095276
    1849 -2.31917182368267
    1850 -2.34987183271427
    1851 -2.38707851117798
    1852 -2.29462929171037
    1853 -2.27266668290864
    1854 -2.21351857463002
    1855 -2.25565860279505
    1856 -2.34936596051052
    1857 -2.36301467390688
    1858 -2.38956808231988
    1859 -2.34681621538485
    1860 -2.33516642250028
    1861 -2.36216390511506
    1862 -2.4594773713123
    1863 -2.50879044503261
    1864 -2.47140713744927
    1865 -2.41529253388295
    1866 -2.42015894335031
    };
    \addplot [semithick, forestgreen4416044]
    table {%
    0 0.833775720213007
    1 0.832608203782676
    2 0.83072441000407
    3 0.830162244303165
    4 0.829430960268624
    5 0.827927699122431
    6 0.826938029833826
    7 0.825515344062167
    8 0.824242101742764
    9 0.822702309846222
    10 0.82077212867119
    11 0.818460792883992
    12 0.816594587835618
    13 0.813178379781245
    14 0.809069573720248
    15 0.804222918204433
    16 0.797992294379999
    17 0.789637312750278
    18 0.782714106816226
    19 0.772174307439306
    20 0.759712565976037
    21 0.745204885865576
    22 0.72544111978143
    23 0.699948681963001
    24 0.671512433289457
    25 0.635389650363218
    26 0.594966618180423
    27 0.547062425223611
    28 0.489515338011306
    29 0.430915914416555
    30 0.359990935014442
    31 0.289563354020862
    32 0.228825416052972
    33 0.160270502452082
    34 0.124373644848618
    35 0.0578455603193085
    36 0.0646288478971436
    37 0.0265128915076435
    38 0.00252123070342227
    39 0.0395152290562357
    40 0.097985977166923
    41 0.110797224064251
    42 0.0944617542700736
    43 0.118491783009782
    44 0.105395640654149
    45 0.155822583046502
    46 0.108129112120938
    47 0.116822552991751
    48 0.109619903765999
    49 0.0412958755223066
    50 -0.0458521909561264
    51 -0.071475008798204
    52 -0.0971830719796135
    53 -0.15196826371715
    54 -0.205210315753906
    55 -0.31506361188139
    56 -0.342435919928275
    57 -0.409618599117378
    58 -0.44950039031014
    59 -0.484967827291009
    60 -0.494475284651043
    61 -0.57304195483393
    62 -0.607406177663385
    63 -0.59987511135361
    64 -0.624492745878302
    65 -0.591813408437465
    66 -0.655547378413628
    67 -0.640862989607321
    68 -0.662040474212683
    69 -0.686820592716976
    70 -0.689529412551264
    71 -0.669044845577666
    72 -0.679645583571328
    73 -0.73355010937904
    74 -0.767259025189086
    75 -0.790373738954781
    76 -0.786133886746427
    77 -0.836371975156059
    78 -0.858902023928095
    79 -0.878318785543412
    80 -0.934566565311068
    81 -0.996057783710425
    82 -1.01481954144247
    83 -1.04294888635836
    84 -1.06582114541084
    85 -1.15523347380334
    86 -1.13784848746642
    87 -1.18388191928905
    88 -1.21513531017933
    89 -1.21096980250537
    90 -1.2474382856686
    91 -1.26146612086234
    92 -1.25820504997994
    93 -1.27725089318014
    94 -1.28298386076697
    95 -1.30082028216501
    96 -1.33567778572343
    97 -1.25554882886832
    98 -1.24664708910664
    99 -1.25397246619273
    100 -1.22856122684387
    101 -1.20307964524619
    102 -1.19100486368817
    103 -1.19264473077689
    104 -1.18995113429526
    105 -1.19252730826152
    106 -1.26386041752552
    107 -1.30155963696831
    108 -1.31500351589284
    109 -1.31126279332909
    110 -1.3429411623251
    111 -1.3927412833054
    112 -1.457659304961
    113 -1.47628035923034
    114 -1.46914735958578
    115 -1.42702786900945
    116 -1.33283105950156
    117 -1.36839438275212
    118 -1.4173216589693
    119 -1.44240452053983
    120 -1.42933693537821
    121 -1.45093838327538
    122 -1.43058767350346
    123 -1.44481647762992
    124 -1.47173696126584
    125 -1.49777349009657
    126 -1.5877007557405
    127 -1.50853306925534
    128 -1.44809797805872
    129 -1.46889754619995
    130 -1.51029610871871
    131 -1.46504485083715
    132 -1.47098451415899
    133 -1.43587816932073
    134 -1.43061280105217
    135 -1.38284762544858
    136 -1.3850585849268
    137 -1.4174198767453
    138 -1.41281944841534
    139 -1.36355480819492
    140 -1.36925041659385
    141 -1.37122574353262
    142 -1.35630065992041
    143 -1.39392897148458
    144 -1.39260966157505
    145 -1.47449707951482
    146 -1.47660525766363
    147 -1.48539064558132
    148 -1.5145637733443
    149 -1.56862824423035
    150 -1.51907897187966
    151 -1.54350858254781
    152 -1.56921182001091
    153 -1.52123452556369
    154 -1.55496012533597
    155 -1.55979993650181
    156 -1.57228180829735
    157 -1.56340524862459
    158 -1.50317471720477
    159 -1.49609810027906
    160 -1.53613495406291
    161 -1.49168807711598
    162 -1.53427658215959
    163 -1.60525115916238
    164 -1.59357444963114
    165 -1.54658488919862
    166 -1.46931959584839
    167 -1.54734855592269
    168 -1.56820224115886
    169 -1.61498207173811
    170 -1.64252424489173
    171 -1.72892359715601
    172 -1.70424323611068
    173 -1.6813148727631
    174 -1.64703381031002
    175 -1.73076673377683
    176 -1.83143826613032
    177 -1.78533834937899
    178 -1.85940017398886
    179 -1.78460675227399
    180 -1.74996454184566
    181 -1.70055613720239
    182 -1.7181128682958
    183 -1.70965937389269
    184 -1.76282565955159
    185 -1.72360381759008
    186 -1.75215767934214
    187 -1.79289609901016
    188 -1.83100487820191
    189 -1.86562831573663
    190 -1.85193934568638
    191 -1.82132988694751
    192 -1.74323601606104
    193 -1.75884767726421
    194 -1.68017655673163
    195 -1.64692685776191
    196 -1.61470373763096
    197 -1.56410026115517
    198 -1.55471498922272
    199 -1.57297720446455
    200 -1.63144399245178
    201 -1.69934561902291
    202 -1.67417107359893
    203 -1.67289877437829
    204 -1.71632680056794
    205 -1.77250725206531
    206 -1.76605766771899
    207 -1.77761997800809
    208 -1.71646872499638
    209 -1.72004964930455
    210 -1.68755278169177
    211 -1.6436446522863
    212 -1.72071592321048
    213 -1.75927773162401
    214 -1.75718892992726
    215 -1.74415265049505
    216 -1.78627581722302
    217 -1.86118025151205
    218 -1.89672203916014
    219 -1.92140330723339
    220 -1.90174439329743
    221 -1.8625375516937
    222 -1.8725484293873
    223 -1.85719031704347
    224 -1.85134632378419
    225 -1.84506338008412
    226 -1.82924797287623
    227 -1.83380492572322
    228 -1.91170186889296
    229 -1.86955294858288
    230 -1.89641531144452
    231 -2.00722420662454
    232 -1.98983947836494
    233 -1.99272087917627
    234 -1.99306246709038
    235 -2.04982374010337
    236 -1.97754530235352
    237 -1.88746687269482
    238 -1.85335361123632
    239 -1.92572714891137
    240 -1.81832934102498
    241 -1.80309922613379
    242 -1.79577954394232
    243 -1.77498349847443
    244 -1.77435551901992
    245 -1.71434092389757
    246 -1.77063616811048
    247 -1.7896417146203
    248 -1.79227211412633
    249 -1.7794422386507
    250 -1.84925836980013
    251 -1.82520311031068
    252 -1.87738265622039
    253 -1.91665944134589
    254 -1.86093255084017
    255 -1.8850595999688
    256 -1.82845013995684
    257 -1.81825584634743
    258 -1.827333411149
    259 -1.77593083664031
    260 -1.81884593468396
    261 -1.7836820793225
    262 -1.73193699456902
    263 -1.68350192311677
    264 -1.78097214588191
    265 -1.7534511527629
    266 -1.80539935587921
    267 -1.87771841968148
    268 -1.82271563991814
    269 -1.80678853313645
    270 -1.81699405883607
    271 -1.83071437777616
    272 -1.81976700210798
    273 -1.85288308861158
    274 -1.84020083532679
    275 -1.8777333171879
    276 -1.87680659657695
    277 -1.89622871656589
    278 -1.95622718743703
    279 -2.02221705065815
    280 -1.97742518447088
    281 -1.97179336603945
    282 -1.98683635375681
    283 -2.03932741826195
    284 -2.05344120961831
    285 -2.03094622554687
    286 -1.95138344252012
    287 -1.83675805069526
    288 -1.7845196862589
    289 -1.80029258340817
    290 -1.85959728941482
    291 -1.91869923513534
    292 -2.01501620871765
    293 -1.97625033453039
    294 -1.86810911079586
    295 -1.88224716862861
    296 -1.9945718763321
    297 -2.09476402087387
    298 -2.1562686546643
    299 -2.0617743741378
    300 -2.01516794426077
    301 -1.98346829584672
    302 -1.9284859331298
    303 -1.85822811456809
    304 -1.98406470984814
    305 -1.97885152740171
    306 -1.92117435237848
    307 -1.94149940832228
    308 -1.95049896138892
    309 -2.01972645426072
    310 -1.90520887505543
    311 -1.90346987770704
    312 -1.90355030993357
    313 -1.92835306293239
    314 -1.9374107664335
    315 -1.96603953044701
    316 -2.03588056288027
    317 -1.98337045136014
    318 -1.99408710782532
    319 -2.01670059580328
    320 -2.20804643252368
    321 -2.03136650287788
    322 -2.07127352887523
    323 -2.02468802652748
    324 -2.00575390210446
    325 -1.94983901032944
    326 -1.95995198788449
    327 -1.91368239697895
    328 -1.90617854879906
    329 -1.8414760550836
    330 -1.80214672154544
    331 -1.95632832201608
    332 -1.89591984311533
    333 -2.00260320699258
    334 -1.98073525156806
    335 -2.01609462636857
    336 -1.99453214708603
    337 -2.03649698639912
    338 -2.01729983292992
    339 -2.09412750082692
    340 -2.110233584039
    341 -2.14489954134698
    342 -2.23071764445105
    343 -2.18756664322277
    344 -2.25425526351205
    345 -2.21343896046496
    346 -2.15343045101543
    347 -2.19004295515134
    348 -2.04589825206008
    349 -2.01199592284175
    350 -2.01002979926762
    351 -2.01050142487393
    352 -2.00583972443089
    353 -2.01733728422217
    354 -1.93152494534533
    355 -1.96765230994544
    356 -1.99083568857557
    357 -1.95988030915461
    358 -2.05516039626608
    359 -2.04993236761273
    360 -1.96739296772795
    361 -1.86741775001349
    362 -1.88791150248608
    363 -1.83958218509006
    364 -1.83179946468022
    365 -1.83786131906348
    366 -1.81677316324988
    367 -1.84395985173394
    368 -1.89017061077631
    369 -1.91581207375686
    370 -2.02314684305278
    371 -2.0511646514683
    372 -1.97871845003528
    373 -1.96360508966916
    374 -2.03819766198352
    375 -2.00345398583527
    376 -2.08422536680277
    377 -2.08262139172916
    378 -2.05803095372847
    379 -1.9719339611116
    380 -1.8988831038729
    381 -1.91368802017551
    382 -1.91647250232937
    383 -1.97343568619332
    384 -1.95148537592877
    385 -1.97074626122011
    386 -1.95282378204067
    387 -1.96484207994706
    388 -2.0029458219376
    389 -2.04094030558805
    390 -2.1213133684896
    391 -2.24300665901846
    392 -2.26306283526384
    393 -2.240257530693
    394 -2.22325767444492
    395 -2.22825627968413
    396 -2.15974982430116
    397 -2.14457307435822
    398 -2.09451070754717
    399 -2.15021880045188
    400 -2.1390347495682
    401 -2.02363231551083
    402 -2.03686672595699
    403 -2.06308749078558
    404 -2.10411117739065
    405 -2.15408373521245
    406 -2.12264479688618
    407 -2.12818273766022
    408 -2.09672379560767
    409 -2.09420273671032
    410 -2.05452813134213
    411 -2.15168396807109
    412 -2.16028369028046
    413 -2.20967464065723
    414 -2.18411639531921
    415 -2.11206727267049
    416 -2.1491299751263
    417 -2.11365726123438
    418 -2.2269642793132
    419 -2.25892360823487
    420 -2.3602811585788
    421 -2.28984140042903
    422 -2.29788295418799
    423 -2.2407335963142
    424 -2.24414138689443
    425 -2.27519720090119
    426 -2.2878790431417
    427 -2.3406546792234
    428 -2.31822743500271
    429 -2.27350117792598
    430 -2.14792867538912
    431 -2.14797231041296
    432 -2.10077340471354
    433 -2.15125856662371
    434 -2.13448116729782
    435 -2.11306180413603
    436 -2.08928968711056
    437 -2.06859979352956
    438 -2.01034621267328
    439 -2.0056860645433
    440 -2.09583418448592
    441 -2.10097763467123
    442 -2.08912468642196
    443 -2.07665100620809
    444 -2.13660018552362
    445 -2.18987548547873
    446 -2.21368445130165
    447 -2.23375721282687
    448 -2.305641554282
    449 -2.26548650008599
    450 -2.20824987548568
    451 -2.25513180623176
    452 -2.31755262749867
    453 -2.28663589634248
    454 -2.19646766210783
    455 -2.13312753511108
    456 -2.17984640248058
    457 -2.2312185857421
    458 -2.19649753315537
    459 -2.19847781907013
    460 -2.1908154480245
    461 -2.18012948302224
    462 -2.19494404244216
    463 -2.18050246757359
    464 -2.21475453308509
    465 -2.31503766768938
    466 -2.32313155497547
    467 -2.21885496967798
    468 -2.22840298339324
    469 -2.27931426331532
    470 -2.37990276044743
    471 -2.43235638407767
    472 -2.41500645600599
    473 -2.53215993168247
    474 -2.56226156456848
    475 -2.51434138633982
    476 -2.55810050334431
    477 -2.67970021797388
    478 -2.69809011283925
    479 -2.69330499812383
    480 -2.65730404073971
    481 -2.53595411307751
    482 -2.50563157569129
    483 -2.44178198580006
    484 -2.42943983258317
    485 -2.41792184824128
    486 -2.36053266808266
    487 -2.37529908790536
    488 -2.38968877745492
    489 -2.33181055564612
    490 -2.30461213189982
    491 -2.33173031455335
    492 -2.34053659788538
    493 -2.34354284382372
    494 -2.34586205257563
    495 -2.41242527789141
    496 -2.39906366858517
    497 -2.36500118105117
    498 -2.3759115966261
    499 -2.43436073391346
    500 -2.429379505669
    501 -2.47706566271858
    502 -2.52909331730217
    503 -2.54953513983929
    504 -2.60166468601411
    505 -2.55561760265083
    506 -2.60761683722617
    507 -2.41487890261178
    508 -2.38509587963671
    509 -2.4650540833736
    510 -2.5185868114165
    511 -2.51322274502702
    512 -2.48792841796937
    513 -2.50995522545082
    514 -2.49702742087965
    515 -2.47399703066765
    516 -2.47244463918067
    517 -2.71742401301955
    518 -2.63429219146218
    519 -2.59753274331114
    520 -2.56038328987677
    521 -2.51141299267107
    522 -2.48633065256731
    523 -2.45438134253454
    524 -2.40522453498413
    525 -2.4308026931245
    526 -2.39962768212427
    527 -2.3021979314748
    528 -2.3750201547239
    529 -2.37832849229574
    530 -2.43475616465491
    531 -2.4709849610441
    532 -2.49025925318954
    533 -2.48954565378114
    534 -2.58077127532194
    535 -2.53248300235199
    536 -2.56913469474347
    537 -2.68349912830337
    538 -2.66982991429648
    539 -2.5712653863265
    540 -2.47592453589502
    541 -2.52182305823617
    542 -2.4491899651511
    543 -2.48603764054121
    544 -2.50107727843754
    545 -2.47608293434463
    546 -2.488928789148
    547 -2.44959676348947
    548 -2.48722242744385
    549 -2.59347884858996
    550 -2.57095582172997
    551 -2.43672185180401
    552 -2.45760263481812
    553 -2.36949703632403
    554 -2.32829100098857
    555 -2.35805769730045
    556 -2.37861799688786
    557 -2.41888731024495
    558 -2.37963251497495
    559 -2.30516312903842
    560 -2.37265184731322
    561 -2.44623565198185
    562 -2.44230967308187
    563 -2.43360225104228
    564 -2.48512913397123
    565 -2.43333453141169
    566 -2.41414793424235
    567 -2.30942287490818
    568 -2.33668055506634
    569 -2.3641704006623
    570 -2.37656909402513
    571 -2.34958385773308
    572 -2.39477137682166
    573 -2.45564611084223
    574 -2.44651657938272
    575 -2.42348176952933
    576 -2.40726879272266
    577 -2.46839425273635
    578 -2.3337746638085
    579 -2.36298307101696
    580 -2.37403852700499
    581 -2.34529817782556
    582 -2.36083695120572
    583 -2.34367629056097
    584 -2.25151454454964
    585 -2.32658758349929
    586 -2.31616810564529
    587 -2.27533923190373
    588 -2.43336638488553
    589 -2.4180157869781
    590 -2.33902284599859
    591 -2.4215466550914
    592 -2.40373325859813
    593 -2.3836614183573
    594 -2.40094459796057
    595 -2.2921756169482
    596 -2.25467626770801
    597 -2.2924852971815
    598 -2.26904412836875
    599 -2.26688107359244
    600 -2.3087645436146
    601 -2.27945378985866
    602 -2.26558296858185
    603 -2.32212503949889
    604 -2.24343072532387
    605 -2.34473662197087
    606 -2.36784281199676
    607 -2.4423782524853
    608 -2.4367239988729
    609 -2.34022234464127
    610 -2.34989859868727
    611 -2.36166522652864
    612 -2.34067613774228
    613 -2.32064393323142
    614 -2.42586112263219
    615 -2.45296201835096
    616 -2.44555115554515
    617 -2.35407920147476
    618 -2.30189364999556
    619 -2.38804784329828
    620 -2.21547525459961
    621 -2.24717554355801
    622 -2.28135224746231
    623 -2.31173392914329
    624 -2.29057006184929
    625 -2.27147066161343
    626 -2.23980354012819
    627 -2.31815273743987
    628 -2.36217733688778
    629 -2.38537959655093
    630 -2.61896018085511
    631 -2.53364481994233
    632 -2.5104983740417
    633 -2.50734196320782
    634 -2.5916539345568
    635 -2.58905011392206
    636 -2.67347209591126
    637 -2.61256829203353
    638 -2.63982532929601
    639 -2.5410496544489
    640 -2.55213502362676
    641 -2.55920429801154
    642 -2.61421846591757
    643 -2.56869724446676
    644 -2.47472528379395
    645 -2.48025345014103
    646 -2.46773726791198
    647 -2.44978455513345
    648 -2.32628034527101
    649 -2.38179581673714
    650 -2.3310987826696
    651 -2.30746556503774
    652 -2.24848844157921
    653 -2.18073548538907
    654 -2.21575177962309
    655 -2.14892574424268
    656 -2.14880036110078
    657 -2.1439512495525
    658 -2.20108550986662
    659 -2.18861732126437
    660 -2.218449408332
    661 -2.29808645975828
    662 -2.34433232167723
    663 -2.3637843697156
    664 -2.36569176392637
    665 -2.47414409210134
    666 -2.4094969186778
    667 -2.41330036933919
    668 -2.37923834631022
    669 -2.40176231115755
    670 -2.41168824356346
    671 -2.41641220030322
    672 -2.39899496863799
    673 -2.38788978135958
    674 -2.34686695629077
    675 -2.36356024829181
    676 -2.44164182251367
    677 -2.36877727706296
    678 -2.3478048162017
    679 -2.23404875423013
    680 -2.22126043174292
    681 -2.15229231814653
    682 -2.10423995826074
    683 -2.12804918447478
    684 -2.18046902171966
    685 -2.165568250487
    686 -2.10002120263721
    687 -2.16423378328063
    688 -2.25532938816003
    689 -2.39288731942541
    690 -2.29007089715972
    691 -2.39056360021375
    692 -2.48790948497736
    693 -2.61411672449148
    694 -2.50419909116236
    695 -2.47698241975016
    696 -2.59060750875659
    697 -2.58315531793167
    698 -2.50808445054196
    699 -2.48171563443078
    700 -2.50536977314127
    701 -2.41989979774777
    702 -2.43584213965885
    703 -2.38773036928796
    704 -2.53825808527214
    705 -2.45863176868587
    706 -2.42399119118741
    707 -2.46561384050981
    708 -2.43736182274186
    709 -2.45379283663898
    710 -2.50081939738924
    711 -2.49941651510123
    712 -2.47610722580263
    713 -2.46047046163172
    714 -2.36392235547664
    715 -2.40523631213378
    716 -2.40778555153247
    717 -2.38896406151735
    718 -2.25540528335694
    719 -2.16603061466041
    720 -2.1680475885277
    721 -2.21536302926189
    722 -2.25607367841412
    723 -2.20464332389262
    724 -2.20721512609969
    725 -2.23781715428559
    726 -2.24587190656814
    727 -2.28082835819356
    728 -2.43404270553206
    729 -2.39760369697846
    730 -2.4641244797632
    731 -2.40448078127666
    732 -2.33802652345114
    733 -2.39633954673023
    734 -2.46361845560966
    735 -2.37850384529129
    736 -2.38820948327763
    737 -2.39785844730546
    738 -2.44072341893453
    739 -2.55902489015537
    740 -2.5031090629335
    741 -2.59432055777921
    742 -2.61213387205321
    743 -2.65473173187488
    744 -2.49216213259899
    745 -2.61050568261812
    746 -2.60929270087861
    747 -2.41488722199341
    748 -2.43202344812732
    749 -2.43285416269206
    750 -2.41227567497516
    751 -2.24106245020232
    752 -2.26350139480131
    753 -2.24704206600077
    754 -2.36209953237899
    755 -2.34930267177283
    756 -2.38668070830486
    757 -2.53769617647132
    758 -2.51981929429681
    759 -2.5544940153324
    760 -2.54053782063282
    761 -2.69482557975783
    762 -2.67907902509273
    763 -2.67175900793284
    764 -2.63645093165623
    765 -2.62774567475861
    766 -2.57434488258189
    767 -2.53675124143968
    768 -2.59579266858892
    769 -2.6332957969849
    770 -2.58244103373911
    771 -2.62186807054516
    772 -2.59853362858635
    773 -2.50152457711877
    774 -2.58097724005112
    775 -2.58586093053609
    776 -2.60741710610059
    777 -2.6539202874162
    778 -2.63862626451608
    779 -2.59188960553532
    780 -2.65726392686383
    781 -2.57194106830224
    782 -2.61245582999286
    783 -2.63954675842359
    784 -2.59423935404266
    785 -2.58960008915717
    786 -2.58984418611264
    787 -2.53140265312855
    788 -2.50012585299578
    789 -2.4530485012166
    790 -2.38068499849817
    791 -2.38329995867851
    792 -2.37861507867626
    793 -2.48518556391267
    794 -2.44737447891425
    795 -2.4445329931753
    796 -2.34283703908413
    797 -2.33391165649832
    798 -2.30475804095757
    799 -2.29968874690281
    800 -2.40073227448374
    801 -2.36221231886298
    802 -2.30760308880722
    803 -2.27180126421661
    804 -2.29378272005979
    805 -2.26648970228916
    806 -2.2916807078763
    807 -2.30548958327826
    808 -2.31080826922185
    809 -2.31298034650038
    810 -2.31880125336005
    811 -2.41385220305951
    812 -2.51094400251081
    813 -2.52695202236152
    814 -2.37294837006307
    815 -2.43989427469184
    816 -2.54584383728631
    817 -2.62870445322215
    818 -2.66793111782228
    819 -2.74689614146042
    820 -2.70547091879641
    821 -2.63352581862462
    822 -2.5362047000101
    823 -2.49745894373974
    824 -2.69334642170833
    825 -2.61909853179161
    826 -2.54082838956723
    827 -2.47035824477228
    828 -2.49265813754444
    829 -2.43864065562266
    830 -2.475872418322
    831 -2.57407879763815
    832 -2.67348602289378
    833 -2.77186986256304
    834 -2.65205611171421
    835 -2.6718683091072
    836 -2.69357129421663
    837 -2.72183464148343
    838 -2.55695176971629
    839 -2.64273441065461
    840 -2.62978063548771
    841 -2.5192289754488
    842 -2.4498848655286
    843 -2.38593887407783
    844 -2.44369523525195
    845 -2.51605596426263
    846 -2.54182066264927
    847 -2.53513682146985
    848 -2.70056303626046
    849 -2.57600856095722
    850 -2.60296048617163
    851 -2.66370373446503
    852 -2.58630604956323
    853 -2.63037180648049
    854 -2.5122253067763
    855 -2.43907830519864
    856 -2.35612458466289
    857 -2.36714829473276
    858 -2.29448863689265
    859 -2.33837776844336
    860 -2.27446940110305
    861 -2.25374880298916
    862 -2.32117631818473
    863 -2.34965864249851
    864 -2.40582725686313
    865 -2.31055719187853
    866 -2.37147647449624
    867 -2.36520249759534
    868 -2.34957776079071
    869 -2.38934177452129
    870 -2.42974192999172
    871 -2.42600547852628
    872 -2.48938713154423
    873 -2.41758012089899
    874 -2.44386153705973
    875 -2.57542449499693
    876 -2.46945453375816
    877 -2.52182894446118
    878 -2.52001374324142
    879 -2.4474217778844
    880 -2.42756883883132
    881 -2.46006517768168
    882 -2.43579870286354
    883 -2.43631791425612
    884 -2.42376424816821
    885 -2.3560825405212
    886 -2.49278321692047
    887 -2.34729627186063
    888 -2.36437660843708
    889 -2.43691578820911
    890 -2.45065225348654
    891 -2.50203267045969
    892 -2.47662468715135
    893 -2.47248180741857
    894 -2.47927067945701
    895 -2.49459614101724
    896 -2.46652112991923
    897 -2.50691406800032
    898 -2.50058557713489
    899 -2.48689783457278
    900 -2.51671255020118
    901 -2.5161118929807
    902 -2.37883942193557
    903 -2.29318921710991
    904 -2.34517160307473
    905 -2.38862365162981
    906 -2.411670085472
    907 -2.35493535457839
    908 -2.39059459594635
    909 -2.42091543739404
    910 -2.33188531760154
    911 -2.30464464741197
    912 -2.37489934481316
    913 -2.44684868959863
    914 -2.46848247421472
    915 -2.47238697231566
    916 -2.38102051626067
    917 -2.47264906502745
    918 -2.50443874952627
    919 -2.42646434991635
    920 -2.50206529523734
    921 -2.54336129163943
    922 -2.55129028163416
    923 -2.58389259853785
    924 -2.50977381983108
    925 -2.44594009906547
    926 -2.47391808839851
    927 -2.40488378423907
    928 -2.38080266435311
    929 -2.41953390411788
    930 -2.39209583195681
    931 -2.32987032346416
    932 -2.35941943060889
    933 -2.37550622912558
    934 -2.41797931377709
    935 -2.46973817334157
    936 -2.38981740952232
    937 -2.55348207920982
    938 -2.57615404858089
    939 -2.60662757248671
    940 -2.57752169389953
    941 -2.57827157401457
    942 -2.54788921071498
    943 -2.53194078650666
    944 -2.48545142441652
    945 -2.43017263733074
    946 -2.57443780194522
    947 -2.51132843143133
    948 -2.52490812634646
    949 -2.50917417503998
    950 -2.56836959937561
    951 -2.58973882542615
    952 -2.60927858684174
    953 -2.59848547008823
    954 -2.67912655973529
    955 -2.77306280775558
    956 -2.78180761139
    957 -2.7884236553907
    958 -2.73912862856532
    959 -2.63628423467998
    960 -2.53537182886735
    961 -2.49689230405042
    962 -2.54718068368595
    963 -2.59761230603238
    964 -2.51548622718703
    965 -2.51649435315103
    966 -2.43612580101644
    967 -2.38258483222979
    968 -2.43115055418079
    969 -2.55385183630125
    970 -2.67435728474524
    971 -2.77599689666853
    972 -2.76252270664313
    973 -2.71714200212742
    974 -2.71361886236414
    975 -2.70233260255633
    976 -2.74988228452355
    977 -2.86956945289082
    978 -2.78178556421201
    979 -2.73335709347238
    980 -2.70916446180462
    981 -2.67248001689313
    982 -2.69641725094706
    983 -2.70237334887742
    984 -2.73892915376894
    985 -2.6892037983343
    986 -2.7147954575707
    987 -2.68249481794868
    988 -2.74516987831581
    989 -2.70442951559358
    990 -2.69750145626346
    991 -2.67205221661941
    992 -2.6000823318352
    993 -2.64977649003625
    994 -2.68195467614336
    995 -2.74121981990783
    996 -2.80113159251526
    997 -2.78393610967177
    998 -2.72849891191586
    999 -2.76333585900766
    1000 -2.72765813130933
    1001 -2.70073539818033
    1002 -2.71274924326079
    1003 -2.70591875276723
    1004 -2.68621279950471
    1005 -2.67170967488452
    1006 -2.57441173632842
    1007 -2.59911465811536
    1008 -2.60176333189276
    1009 -2.64613117777515
    1010 -2.63514317559926
    1011 -2.69415090415892
    1012 -2.64046555171267
    1013 -2.61180389740319
    1014 -2.59082023836619
    1015 -2.60825669159012
    1016 -2.67197433166659
    1017 -2.61073671852037
    1018 -2.58741145327262
    1019 -2.55600074477926
    1020 -2.67811378696095
    1021 -2.64966499375318
    1022 -2.70339835913892
    1023 -2.70736055338482
    1024 -2.79521246542516
    1025 -2.7997596244862
    1026 -2.78189297735401
    1027 -2.81766480219451
    1028 -2.91996645205022
    1029 -2.86845846634164
    1030 -2.80528524350984
    1031 -2.79914510507104
    1032 -2.86296151213051
    1033 -2.90731051819909
    1034 -2.86572172864294
    1035 -2.85607163310774
    1036 -2.80731597461601
    1037 -2.87198443199271
    1038 -2.73186176206608
    1039 -2.6366331683895
    1040 -2.6648019887742
    1041 -2.68950480603669
    1042 -2.61987599218637
    1043 -2.46621959146195
    1044 -2.49721765512744
    1045 -2.48881768360241
    1046 -2.46035148986646
    1047 -2.43850065665124
    1048 -2.38042812610743
    1049 -2.46026539809445
    1050 -2.38969587176496
    1051 -2.39591636359073
    1052 -2.44406276176771
    1053 -2.55953131114747
    1054 -2.52075177657077
    1055 -2.55686912474967
    1056 -2.63559579910874
    1057 -2.66378677292916
    1058 -2.82115850881101
    1059 -2.61591605918456
    1060 -2.56490252942376
    1061 -2.51024136729448
    1062 -2.55576861340546
    1063 -2.52224086215802
    1064 -2.51035743082039
    1065 -2.458350312453
    1066 -2.48005889716632
    1067 -2.51121290572187
    1068 -2.56759915885692
    1069 -2.69045191016797
    1070 -2.70837762063671
    1071 -2.74826771478974
    1072 -2.62968470883034
    1073 -2.59964301141456
    1074 -2.60366770939086
    1075 -2.62430794142505
    1076 -2.5866042563267
    1077 -2.47713902304366
    1078 -2.4298771539792
    1079 -2.50388991964145
    1080 -2.52221186935966
    1081 -2.56062563267906
    1082 -2.68243449896106
    1083 -2.801782365383
    1084 -2.65450005815681
    1085 -2.62462810390153
    1086 -2.59434669992332
    1087 -2.64343249050069
    1088 -2.58237292526951
    1089 -2.48176883841136
    1090 -2.46197388586332
    1091 -2.31639613606249
    1092 -2.29984291120694
    1093 -2.29650005802345
    1094 -2.24938035454954
    1095 -2.28824678878259
    1096 -2.31878275040765
    1097 -2.35969137122141
    1098 -2.42628486627399
    1099 -2.55035049594142
    1100 -2.62510417721499
    1101 -2.81202508130647
    1102 -2.75258339950647
    1103 -2.59970790643751
    1104 -2.73066088549292
    1105 -2.78603675790711
    1106 -2.85559728994436
    1107 -2.77896904557561
    1108 -2.74232616623654
    1109 -2.71631861716234
    1110 -2.7192144344895
    1111 -2.62464306728408
    1112 -2.56874805600821
    1113 -2.72907721625325
    1114 -2.77953804115769
    1115 -2.66714815717178
    1116 -2.59803814661023
    1117 -2.67252366607662
    1118 -2.70288958490508
    1119 -2.67148002130234
    1120 -2.71820826551025
    1121 -2.77442638774436
    1122 -2.89344863444608
    1123 -2.78629870076646
    1124 -2.73071851591772
    1125 -2.80825608842735
    1126 -2.75472257888004
    1127 -2.70209248384729
    1128 -2.68335544020283
    1129 -2.75945199927962
    1130 -2.73330045695392
    1131 -2.65505005044677
    1132 -2.6536105702505
    1133 -2.64614805397581
    1134 -2.68299687156967
    1135 -2.67348846013566
    1136 -2.78691339150225
    1137 -2.84006618537265
    1138 -2.83574598670232
    1139 -2.63604001684933
    1140 -2.6529238958568
    1141 -2.77242594030344
    1142 -2.58113134215409
    1143 -2.66297944033656
    1144 -2.69067921211766
    1145 -2.67343394614924
    1146 -2.66490038747253
    1147 -2.62441105786266
    1148 -2.49999810462914
    1149 -2.60100851206813
    1150 -2.4652111745937
    1151 -2.46131081000991
    1152 -2.60246872191283
    1153 -2.57635362225631
    1154 -2.58855122009027
    1155 -2.65977993626752
    1156 -2.68499418415891
    1157 -2.70189686354216
    1158 -2.92748356497095
    1159 -2.94584257661733
    1160 -3.15650396995771
    1161 -3.05603183036748
    1162 -3.0400418486013
    1163 -2.97992323464133
    1164 -3.02409978748131
    1165 -2.84026119105496
    1166 -2.78916551432594
    1167 -2.7787973859259
    1168 -2.73009217003342
    1169 -2.66956728421522
    1170 -2.65541163876293
    1171 -2.6526640316319
    1172 -2.61864067275217
    1173 -2.69742647071439
    1174 -2.56040425582052
    1175 -2.65983199634266
    1176 -2.6124350515348
    1177 -2.57456180321011
    1178 -2.60951112935067
    1179 -2.63198576999597
    1180 -2.53107463168792
    1181 -2.4793128744179
    1182 -2.42119394544464
    1183 -2.36678549441429
    1184 -2.48595601915919
    1185 -2.48068167519747
    1186 -2.5204797416935
    1187 -2.57720545088293
    1188 -2.42129427456806
    1189 -2.44851119673833
    1190 -2.5652917581452
    1191 -2.7265090339585
    1192 -2.89802752738085
    1193 -2.85945442958559
    1194 -2.79010306531332
    1195 -2.74908750763415
    1196 -2.6516163319523
    1197 -2.63920383552507
    1198 -2.74530151558388
    1199 -2.66457703776563
    1200 -2.54871944260416
    1201 -2.49704867664543
    1202 -2.47719785624424
    1203 -2.56519954907844
    1204 -2.56623687701393
    1205 -2.53105053591825
    1206 -2.6482423815848
    1207 -2.61002081337764
    1208 -2.65835252482024
    1209 -2.78640248123378
    1210 -2.80025038947506
    1211 -2.79977172684001
    1212 -2.68685659012997
    1213 -2.55287272876045
    1214 -2.5584087570639
    1215 -2.6120937478711
    1216 -2.58447150675564
    1217 -2.53445961720371
    1218 -2.58372785059281
    1219 -2.45843148106319
    1220 -2.51422186274957
    1221 -2.45597108125582
    1222 -2.55532670540747
    1223 -2.65388082249296
    1224 -2.67404168958275
    1225 -2.69856019729269
    1226 -2.56704722110458
    1227 -2.55346861879442
    1228 -2.47946237590886
    1229 -2.6136404409423
    1230 -2.61864585175554
    1231 -2.75138847070716
    1232 -2.77858853880071
    1233 -2.77614119420309
    1234 -2.7739296435562
    1235 -2.6107750619626
    1236 -2.76920867964499
    1237 -2.90852766983535
    1238 -2.83620948926309
    1239 -2.74253513753478
    1240 -2.77468415376224
    1241 -2.73958289436486
    1242 -2.7045331366011
    1243 -2.68492477311914
    1244 -2.65769724987185
    1245 -2.84104926798681
    1246 -2.86260532387528
    1247 -2.77407768431696
    1248 -2.89378849576502
    1249 -2.8748129599581
    1250 -2.79072053370454
    1251 -2.78474539649454
    1252 -2.78468230154213
    1253 -2.79380692268952
    1254 -2.77995686647109
    1255 -2.6930023463758
    1256 -2.63195173127298
    1257 -2.64426088033391
    1258 -2.56956163282492
    1259 -2.62689787362552
    1260 -2.69327993446726
    1261 -2.67397736463161
    1262 -2.57538320134219
    1263 -2.56945673833359
    1264 -2.6371006874977
    1265 -2.66728183402076
    1266 -2.65184135205166
    1267 -2.73204878233999
    1268 -2.76825788839547
    1269 -2.83677059940944
    1270 -2.75065333806284
    1271 -2.78657814878663
    1272 -2.90483978748953
    1273 -2.93492549374489
    1274 -2.94991233078379
    1275 -2.88777690268028
    1276 -2.83711879862694
    1277 -2.80114412393693
    1278 -2.84543573070891
    1279 -2.79242428551888
    1280 -2.83988067391628
    1281 -2.86171349705596
    1282 -2.8967911899721
    1283 -2.81142147679185
    1284 -2.76122035880451
    1285 -2.85226128131208
    1286 -2.9749324893669
    1287 -2.93015221512577
    1288 -2.65229939229745
    1289 -2.57610381184713
    1290 -2.64383736929358
    1291 -2.61595314929584
    1292 -2.60182340644341
    1293 -2.7174281560209
    1294 -2.72274137873185
    1295 -2.70770134940604
    1296 -2.6959806017661
    1297 -2.73030760484526
    1298 -3.05566717366394
    1299 -3.23880691038005
    1300 -3.23870971333341
    1301 -3.27179342691502
    1302 -3.30750687408891
    1303 -3.20982082860955
    1304 -3.29673867085022
    1305 -3.35607592969296
    1306 -3.40755498343696
    1307 -3.41456718525977
    1308 -3.38489116954144
    1309 -3.14612404374566
    1310 -3.00377225957389
    1311 -2.99353637945153
    1312 -2.77045333834511
    1313 -2.79138726706909
    1314 -2.76277566322314
    1315 -2.59588405600917
    1316 -2.60861764986455
    1317 -2.54498125064857
    1318 -2.56705062067031
    1319 -2.62157480719596
    1320 -2.64404579631934
    1321 -2.67160381507806
    1322 -2.72374739872558
    1323 -2.61443512603641
    1324 -2.50182101953322
    1325 -2.59220267907618
    1326 -2.53676804052676
    1327 -2.59226221434509
    1328 -2.55784350907272
    1329 -2.49611400832343
    1330 -2.51590342486521
    1331 -2.46478007814308
    1332 -2.48847200038377
    1333 -2.58240544195425
    1334 -2.66637721477526
    1335 -2.66759561928363
    1336 -2.65220261672815
    1337 -2.51451796562195
    1338 -2.40312422993031
    1339 -2.34328346503328
    1340 -2.39577371621739
    1341 -2.35790092093036
    1342 -2.42067115404778
    1343 -2.35778725359831
    1344 -2.35718794612856
    1345 -2.3521576853642
    1346 -2.30862164114622
    1347 -2.36693276436427
    1348 -2.47752182466787
    1349 -2.57243450351647
    1350 -2.45304537050425
    1351 -2.52167470804826
    1352 -2.51492379575898
    1353 -2.5783264818357
    1354 -2.52961877217714
    1355 -2.54438792821495
    1356 -2.61827384679999
    1357 -2.69197444835466
    1358 -2.68520937166394
    1359 -2.80280931348755
    1360 -2.88891277491835
    1361 -2.82430157702372
    1362 -2.81948859264165
    1363 -2.83344695374807
    1364 -2.88684211130548
    1365 -2.83514269757495
    1366 -2.86134459199966
    1367 -2.85845845306392
    1368 -2.93557464043618
    1369 -2.82880391961096
    1370 -2.86271393563353
    1371 -2.91675355652513
    1372 -2.94574515866701
    1373 -2.93097358778341
    1374 -3.0204981596079
    1375 -2.90210697986453
    1376 -2.84560567975662
    1377 -2.81840548550007
    1378 -2.73137675002163
    1379 -2.70709845412431
    1380 -2.65615173648004
    1381 -2.66070290147824
    1382 -2.60285851299434
    1383 -2.6017104631438
    1384 -2.40921588394071
    1385 -2.53758895941865
    1386 -2.57711530843683
    1387 -2.62328587195806
    1388 -2.69508195249122
    1389 -2.53207406408903
    1390 -2.49381075327119
    1391 -2.4423504076008
    1392 -2.40372543542897
    1393 -2.44080423222686
    1394 -2.58653585135364
    1395 -2.61729023399221
    1396 -2.55986853043298
    1397 -2.56720625113825
    1398 -2.56184644060394
    1399 -2.72289052187742
    1400 -2.8108951439169
    1401 -2.83620520485061
    1402 -2.76834392160184
    1403 -2.72626213520364
    1404 -2.73791342866466
    1405 -2.74852771124622
    1406 -2.68371835035954
    1407 -2.68692622977049
    1408 -2.67047226570658
    1409 -2.60876556698536
    1410 -2.61826257577301
    1411 -2.62591994963398
    1412 -2.75065966924633
    1413 -2.69330078317894
    1414 -2.70333287983764
    1415 -2.65302324812376
    1416 -2.67576089243969
    1417 -2.56247303899583
    1418 -2.45763525990916
    1419 -2.61668501241504
    1420 -2.65987649475954
    1421 -2.69658967862981
    1422 -2.63228295763398
    1423 -2.73831543413785
    1424 -2.61175302501184
    1425 -2.60113439632452
    1426 -2.70774689019735
    1427 -2.75315191640103
    1428 -2.82313671875183
    1429 -2.77973401162996
    1430 -2.67399522505742
    1431 -2.68590959997442
    1432 -2.74932389500838
    1433 -2.65095982331162
    1434 -2.80626028809537
    1435 -2.85119310501456
    1436 -2.82804189457294
    1437 -2.82039283805962
    1438 -2.90284829393947
    1439 -2.85004871171108
    1440 -2.96129266229634
    1441 -3.012517876593
    1442 -3.00480945941171
    1443 -3.12069171379637
    1444 -3.06842204288906
    1445 -3.00625124948975
    1446 -2.97403241875308
    1447 -3.00589316422301
    1448 -3.02004797484935
    1449 -3.06697276553832
    1450 -2.95295133138185
    1451 -2.89501971276042
    1452 -2.75389132681362
    1453 -2.73480633104
    1454 -2.76904713591442
    1455 -2.71767955466411
    1456 -2.72943899085166
    1457 -2.71623183252046
    1458 -2.64124248510487
    1459 -2.67554619076508
    1460 -2.73082970786777
    1461 -2.70159318989689
    1462 -2.88068716355147
    1463 -2.88283760463158
    1464 -2.76765119235205
    1465 -2.79757456911797
    1466 -2.78087017829132
    1467 -2.84304484978537
    1468 -2.83840689135228
    1469 -2.70678967965345
    1470 -2.7209357869932
    1471 -2.76307893782073
    1472 -2.78610009243887
    1473 -2.81226788039249
    1474 -2.95454007993625
    1475 -2.91123708228916
    1476 -2.98215364386059
    1477 -2.88619358637049
    1478 -2.9485376916547
    1479 -2.97588270913988
    1480 -2.91158448955788
    1481 -2.91246514039722
    1482 -2.83436732635907
    1483 -2.82852342479857
    1484 -2.83501690825062
    1485 -3.01409121419887
    1486 -2.98429213488936
    1487 -3.11481827891395
    1488 -3.11765830453787
    1489 -3.01781502655877
    1490 -2.97572767966138
    1491 -2.9522615739944
    1492 -3.01966697887778
    1493 -3.00386482415034
    1494 -2.88200128368033
    1495 -2.80619330601455
    1496 -2.79675268170392
    1497 -2.7088529600705
    1498 -2.74079830424497
    1499 -2.94135137792679
    1500 -3.01373284472174
    1501 -3.03716045467518
    1502 -3.01312478380667
    1503 -2.91576165571175
    1504 -3.01422320433945
    1505 -2.96732572586352
    1506 -2.88783802249123
    1507 -2.96538612362122
    1508 -2.88214811160426
    1509 -2.80706262171174
    1510 -2.76063804088724
    1511 -2.72337921096167
    1512 -2.77851630357804
    1513 -2.74330940819488
    1514 -2.66074440870002
    1515 -2.70646744877367
    1516 -2.69296327084634
    1517 -2.69521090687791
    1518 -2.69236982317
    1519 -2.68676273829464
    1520 -2.72157903284051
    1521 -2.76191086037313
    1522 -2.76522256789518
    1523 -2.86866096682207
    1524 -3.01836853517385
    1525 -3.05902751563219
    1526 -3.1521447438894
    1527 -3.13260188078053
    1528 -3.16569048405932
    1529 -3.27597490196296
    1530 -2.9147758121425
    1531 -2.90736245099504
    1532 -2.92930729465446
    1533 -2.91877145310684
    1534 -2.85057089840603
    1535 -2.77552913751616
    1536 -2.73610093163614
    1537 -2.57923433826065
    1538 -2.48899137798933
    1539 -2.50168411359165
    1540 -2.72967354346746
    1541 -2.7503684573387
    1542 -2.63628190205051
    1543 -2.57288080926132
    1544 -2.54180075704436
    1545 -2.49543288029491
    1546 -2.49138715052544
    1547 -2.61481849781398
    1548 -2.76476731484388
    1549 -2.73967550138183
    1550 -2.74872359866352
    1551 -2.75692600358357
    1552 -2.80866610213221
    1553 -2.92909623136156
    1554 -3.0470264510318
    1555 -3.18092569357523
    1556 -3.1824696453739
    1557 -3.25992238166158
    1558 -3.29407547519451
    1559 -3.26139811387056
    1560 -3.21325620605869
    1561 -3.13664781650836
    1562 -3.15525096187777
    1563 -3.10810107681114
    1564 -2.86407269154835
    1565 -2.86513952325849
    1566 -2.90218326289607
    1567 -2.86659657147877
    1568 -2.8258988429026
    1569 -2.79298751643543
    1570 -2.79926486712966
    1571 -2.83854677103971
    1572 -2.83258144393989
    1573 -2.81149260869966
    1574 -2.90695916201959
    1575 -2.86675358794143
    1576 -2.92191130116027
    1577 -2.8034026107577
    1578 -2.79472764870802
    1579 -2.79146057018184
    1580 -2.84900129215924
    1581 -2.83745649959243
    1582 -2.85806909432413
    1583 -2.88941489446042
    1584 -2.92488908638881
    1585 -2.90088314576493
    1586 -2.92482328768156
    1587 -3.11834775216245
    1588 -3.01021201178685
    1589 -2.93918671160934
    1590 -2.84497020937802
    1591 -2.83505542282896
    1592 -2.81503058969141
    1593 -2.83063859119178
    1594 -2.81324025331335
    1595 -2.92887374026447
    1596 -2.87959241698578
    1597 -2.77354032194666
    1598 -2.85701074132934
    1599 -2.89668464031951
    1600 -2.77629926950742
    1601 -2.79130659658009
    1602 -2.82443920232858
    1603 -2.78729553430979
    1604 -2.83867664750624
    1605 -2.74222697610852
    1606 -2.71932487163133
    1607 -2.76007847099605
    1608 -2.78672468717908
    1609 -2.79941630571408
    1610 -2.95550479628692
    1611 -2.8857218910016
    1612 -2.86859939879475
    1613 -2.89765108159986
    1614 -2.89526592456354
    1615 -2.94979477542748
    1616 -2.97242774902054
    1617 -3.01164310138272
    1618 -2.99344381074774
    1619 -3.07760191650698
    1620 -3.05856318169772
    1621 -3.13139131043254
    1622 -3.12011630641287
    1623 -3.10924954189587
    1624 -3.08233919291424
    1625 -2.96937192028852
    1626 -3.04185686678964
    1627 -2.85790500321021
    1628 -2.69260678472355
    1629 -2.66128744432417
    1630 -2.64839477767875
    1631 -2.64323467483924
    1632 -2.65178557031992
    1633 -2.66434585495406
    1634 -2.63979841053378
    1635 -2.71481984832858
    1636 -2.65773121455076
    1637 -2.78825415233891
    1638 -2.98230527680522
    1639 -3.0023290473704
    1640 -3.10319027858955
    1641 -2.98171698917216
    1642 -2.9820389521947
    1643 -3.00156554607249
    1644 -3.07839468444699
    1645 -3.03196116056442
    1646 -2.96098818589882
    1647 -3.03071117727485
    1648 -3.06729562099594
    1649 -3.03815666648628
    1650 -2.91460031877126
    1651 -3.01547478679069
    1652 -2.94124955993201
    1653 -2.93079438757168
    1654 -2.91287198863551
    1655 -2.93402844180862
    1656 -3.08545729156682
    1657 -2.93613870532497
    1658 -2.90785076275262
    1659 -2.90675717804849
    1660 -2.95429259497905
    1661 -2.90949775642499
    1662 -2.98160797465921
    1663 -2.93414170623355
    1664 -2.98928376707805
    1665 -3.11694110112134
    1666 -3.08891384100015
    1667 -3.19360333420456
    1668 -3.22649171951115
    1669 -3.33813961040101
    1670 -3.40017523753676
    1671 -3.26321618494353
    1672 -3.33132860093687
    1673 -3.29823257408748
    1674 -3.215531452959
    1675 -3.05500408422631
    1676 -3.08905846473365
    1677 -3.14056539533522
    1678 -3.13168373722042
    1679 -2.9292400055394
    1680 -2.9314479399724
    1681 -3.00632904781529
    1682 -2.81572413295862
    1683 -2.86496390953611
    1684 -2.91014459480249
    1685 -2.91300847602345
    1686 -2.78974633990598
    1687 -2.71813308557252
    1688 -2.61566036188304
    1689 -2.70286678421886
    1690 -2.71673178343447
    1691 -2.76095893436752
    1692 -2.90970609887729
    1693 -2.80018949175524
    1694 -2.78947823070244
    1695 -2.74732310686008
    1696 -2.83945871420531
    1697 -2.88384526230995
    1698 -2.99581659905519
    1699 -3.08604291566898
    1700 -3.04678145868217
    1701 -3.1131642287081
    1702 -3.1175609364599
    1703 -3.17356640966323
    1704 -3.16512553171275
    1705 -2.99414605196228
    1706 -2.94095454367822
    1707 -2.96184046306318
    1708 -2.94476477086943
    1709 -2.93980720325232
    1710 -2.93660137578406
    1711 -2.88758473562025
    1712 -2.89220099924434
    1713 -2.94741180067453
    1714 -2.95809417358988
    1715 -3.30365023257622
    1716 -3.25827590671839
    1717 -3.16832102758282
    1718 -3.08510246922244
    1719 -3.02861795169692
    1720 -2.9740000014227
    1721 -3.04469773274499
    1722 -2.91788122894867
    1723 -2.78889638998243
    1724 -2.73233109163573
    1725 -2.49296591151733
    1726 -2.53820176584137
    1727 -2.56758044991554
    1728 -2.56259748467697
    1729 -2.56813705584748
    1730 -2.60106730989067
    1731 -2.53935642052099
    1732 -2.60875569791096
    1733 -2.58658310621828
    1734 -2.5820047103026
    1735 -2.73664693130824
    1736 -2.68682400959085
    1737 -2.64742639622036
    1738 -2.73640657407021
    1739 -2.69508939627667
    1740 -2.7005456073095
    1741 -2.70605110878156
    1742 -2.7070436873337
    1743 -2.77937622451941
    1744 -2.81128052303433
    1745 -2.78785358332374
    1746 -2.74639151249484
    1747 -2.79243814468773
    1748 -2.72203507846732
    1749 -2.7759255032567
    1750 -2.79460399295649
    1751 -2.81931202115961
    1752 -2.75710453078615
    1753 -2.72815225226452
    1754 -2.68350044778892
    1755 -2.76332366408516
    1756 -2.77718607781594
    1757 -2.72057718821415
    1758 -2.70483411600608
    1759 -2.69136834830211
    1760 -2.64548761292049
    1761 -2.65983091272853
    1762 -2.72625889154701
    1763 -2.8560209717898
    1764 -2.83986081805862
    1765 -2.9041518131936
    1766 -3.00264619687722
    1767 -3.06091333649222
    1768 -3.06952501147261
    1769 -2.92986603993272
    1770 -3.00455421572105
    1771 -2.83078838420424
    1772 -2.82360313087872
    1773 -2.74437893206168
    1774 -2.81928262683708
    1775 -2.76391114995631
    1776 -2.57610991132764
    1777 -2.52193682761538
    1778 -2.59438153598158
    1779 -2.61956859713331
    1780 -2.6308582254647
    1781 -2.55763576605104
    1782 -2.56598794222644
    1783 -2.5698073126977
    1784 -2.44612179758425
    1785 -2.45249727237435
    1786 -2.61677463400383
    1787 -2.70822468656975
    1788 -2.57535895593945
    1789 -2.67820805628146
    1790 -2.62510622161539
    1791 -2.8377167560442
    1792 -2.80952269360372
    1793 -2.85770028818036
    1794 -3.01111834911122
    1795 -2.9551558839049
    1796 -2.91094409820665
    1797 -2.83534141800451
    1798 -2.93192257669328
    1799 -2.87162624327139
    1800 -2.82779859293259
    1801 -2.87122317276884
    1802 -2.85406980554995
    1803 -2.84160804150885
    1804 -2.89525302730907
    1805 -2.97926061975003
    1806 -3.06117875503146
    1807 -3.10127233286319
    1808 -3.02452808417181
    1809 -2.97905443613443
    1810 -3.06064002222826
    1811 -2.97920201255473
    1812 -2.94995163996573
    1813 -2.99299744395478
    1814 -2.89434748190745
    1815 -2.7653860158527
    1816 -2.71504653840185
    1817 -2.72657036187888
    1818 -2.83849291840196
    1819 -2.84025101056726
    1820 -2.86005400056445
    1821 -2.86811161102494
    1822 -2.87469313423372
    1823 -2.83977164847462
    1824 -2.84529963233368
    1825 -3.02226910344644
    1826 -3.08710249149646
    1827 -3.00736421206637
    1828 -2.94833753604247
    1829 -2.99447926774032
    1830 -2.92932286444747
    1831 -2.98834350502132
    1832 -2.8982757076591
    1833 -2.84037267298074
    1834 -2.85413687473424
    1835 -2.72264903103787
    1836 -2.66546367776024
    1837 -2.72821555977481
    1838 -2.72931351201282
    1839 -2.72661698647933
    1840 -2.75872221971244
    1841 -2.64040107103412
    1842 -2.68361930677356
    1843 -2.78501782801448
    1844 -2.79654522210638
    1845 -2.77774660033276
    1846 -2.77287253128284
    1847 -2.78701583588537
    1848 -2.82008653069599
    1849 -2.87061964067627
    1850 -2.8747824888909
    1851 -3.00340911251042
    1852 -2.99179654814071
    1853 -2.9952152251944
    1854 -3.00785398053839
    1855 -3.19050679280185
    1856 -3.16262503974406
    1857 -3.13476718701089
    1858 -3.13110606447743
    1859 -3.00718894455809
    1860 -3.03045500484482
    1861 -3.00996219095372
    1862 -3.13139165175601
    1863 -3.13735402280042
    1864 -3.01489971852349
    1865 -3.02278130450341
    1866 -3.14977773250474
    };
    \addplot [semithick, crimson2143940]
    table {%
    0 0.695548115568414
    1 0.642051490552746
    2 0.578513525994387
    3 0.504302297603001
    4 0.417333747102418
    5 0.341222987402709
    6 0.253348180612645
    7 0.172072925125814
    8 0.0760141827308862
    9 -0.00777887034101162
    10 -0.0689615114205824
    11 -0.137015840285357
    12 -0.187991182795206
    13 -0.237047977088253
    14 -0.27767818304662
    15 -0.329127230632846
    16 -0.389499279581935
    17 -0.464694262582558
    18 -0.525555173427225
    19 -0.575274646574062
    20 -0.613283073520245
    21 -0.660855907568266
    22 -0.718060431266753
    23 -0.762458489467728
    24 -0.791086633917275
    25 -0.851465049784355
    26 -0.872347611779618
    27 -0.892753220546638
    28 -0.90243368624192
    29 -0.90094841164124
    30 -0.969322671851685
    31 -0.999590458721554
    32 -1.00578725632263
    33 -1.052523578544
    34 -1.07919809437145
    35 -1.10979699859793
    36 -1.15239490150821
    37 -1.22305327907597
    38 -1.2719417996038
    39 -1.32384248120208
    40 -1.30105773117622
    41 -1.31783829881092
    42 -1.40062371942357
    43 -1.35776780619802
    44 -1.37180914143033
    45 -1.33300100271475
    46 -1.34851458138014
    47 -1.34092973482816
    48 -1.31015620145198
    49 -1.33808612460042
    50 -1.37559598619935
    51 -1.36141187488104
    52 -1.31588704145306
    53 -1.35056690860698
    54 -1.37476754806317
    55 -1.43160455047536
    56 -1.40470002148982
    57 -1.40936666170162
    58 -1.42697480214445
    59 -1.44155360298348
    60 -1.40171270435388
    61 -1.46261545985368
    62 -1.48827477378935
    63 -1.47166268104822
    64 -1.49140529691852
    65 -1.46704763072797
    66 -1.55058138016748
    67 -1.48953928372195
    68 -1.51366652479027
    69 -1.48702933652698
    70 -1.47932878719
    71 -1.44189074384614
    72 -1.4562339788713
    73 -1.45538473139991
    74 -1.4693369820491
    75 -1.45717153647731
    76 -1.42943419343024
    77 -1.49145122182286
    78 -1.47353024090294
    79 -1.46757808801417
    80 -1.53009193227799
    81 -1.55750736436111
    82 -1.57130356169611
    83 -1.62149019529542
    84 -1.65224051527032
    85 -1.71759301215267
    86 -1.62870573867459
    87 -1.66415270234194
    88 -1.6854947291016
    89 -1.67944715072271
    90 -1.72796558529739
    91 -1.73157831031892
    92 -1.7505878782882
    93 -1.76752315286951
    94 -1.74262868421433
    95 -1.78305803682501
    96 -1.86021637996952
    97 -1.77629615947593
    98 -1.74560233804818
    99 -1.78783207501745
    100 -1.711911770744
    101 -1.67337989354364
    102 -1.64835271250321
    103 -1.615619079815
    104 -1.61787656643381
    105 -1.62444266862178
    106 -1.6761179754739
    107 -1.71314557716235
    108 -1.76112173274203
    109 -1.75499657930876
    110 -1.78050372608903
    111 -1.84367941414789
    112 -1.85107849092296
    113 -1.93641940390408
    114 -1.94795768151355
    115 -1.89612811425431
    116 -1.80169328117779
    117 -1.78314181565371
    118 -1.8128683254718
    119 -1.80910312830362
    120 -1.82613070223045
    121 -1.88057276115113
    122 -1.84906086966879
    123 -1.81208934125687
    124 -1.81733657498447
    125 -1.83235415048583
    126 -1.90296700804977
    127 -1.8470283731789
    128 -1.83044151431526
    129 -1.88980088288554
    130 -1.91275337520296
    131 -1.87317008349252
    132 -1.93132759980641
    133 -1.92354213373225
    134 -1.94120298913936
    135 -1.91262470522504
    136 -1.92821698229915
    137 -2.00010037911201
    138 -1.9910483655681
    139 -1.90260376921284
    140 -1.9013004299364
    141 -1.90828772452005
    142 -1.86278298182681
    143 -1.89879079176457
    144 -1.88048302026816
    145 -1.93323020177055
    146 -1.95458498188726
    147 -1.97867354695522
    148 -1.97866201971347
    149 -2.04906029522179
    150 -2.05375084891673
    151 -1.99146508892449
    152 -1.99574790533463
    153 -1.92757412414518
    154 -1.99507858727794
    155 -2.01251382241505
    156 -2.02214556266282
    157 -1.94157057602293
    158 -1.86736941614391
    159 -1.82320251011304
    160 -1.83976334081724
    161 -1.85672931821069
    162 -1.95699777604167
    163 -2.03824967481304
    164 -1.95925564577882
    165 -1.87679799145047
    166 -1.80940650392507
    167 -1.92933304068039
    168 -1.95193000931194
    169 -2.00853952276938
    170 -1.98303928064673
    171 -2.05645971620919
    172 -2.04653359519597
    173 -2.0019555975704
    174 -2.00218524616851
    175 -2.1414536389701
    176 -2.21917110569701
    177 -2.13841670262875
    178 -2.22230067987304
    179 -2.20114082045498
    180 -2.19839106876787
    181 -2.13835347590539
    182 -2.09868932352012
    183 -2.10269536103162
    184 -2.14411818029077
    185 -2.09257680279407
    186 -2.1305365855751
    187 -2.20983457400188
    188 -2.18551941806064
    189 -2.18215349893422
    190 -2.15763840419797
    191 -2.11557294359532
    192 -2.05194289738352
    193 -2.02744497416999
    194 -1.95649885086359
    195 -1.91625679430192
    196 -1.88881944614899
    197 -1.82114551625009
    198 -1.82784976111968
    199 -1.85470733051502
    200 -1.92575516378559
    201 -1.95990613987895
    202 -1.92891007147136
    203 -1.95831354149162
    204 -1.98455254111856
    205 -2.00609088800071
    206 -1.96006269342143
    207 -1.95086873186038
    208 -1.93829662949097
    209 -1.94582228076828
    210 -1.83442984298495
    211 -1.79104921478124
    212 -1.87667530831188
    213 -1.920807510052
    214 -1.90667830250411
    215 -1.85020452321921
    216 -1.90273962521792
    217 -1.97703600594498
    218 -1.97825050501946
    219 -1.98803905434307
    220 -2.04535594044801
    221 -2.04377292640508
    222 -1.99311971867052
    223 -1.94350601352545
    224 -1.95224130894502
    225 -2.04360119808014
    226 -2.01317821493865
    227 -2.0353149737614
    228 -2.1059717366973
    229 -2.01340683166764
    230 -2.0426153058743
    231 -2.12895857426678
    232 -2.15117665932558
    233 -2.17508188289549
    234 -2.18938207667627
    235 -2.23097698914462
    236 -2.18379798821214
    237 -1.9925900145227
    238 -1.93346140335695
    239 -2.03783144951071
    240 -1.95638991733543
    241 -1.9374992196881
    242 -1.9319409733295
    243 -1.91130172905674
    244 -1.92376824420661
    245 -1.89911319728374
    246 -1.9611073917744
    247 -2.05675246461366
    248 -2.10619359737096
    249 -2.12061857964957
    250 -2.15091713946986
    251 -2.17090863707217
    252 -2.21670315006305
    253 -2.28337655976376
    254 -2.12726679681558
    255 -2.11691879377923
    256 -2.06255733458139
    257 -2.06370146448306
    258 -2.08135375726896
    259 -2.00879488104919
    260 -2.05254540526685
    261 -2.01947440268689
    262 -1.94411512360241
    263 -1.87577260864221
    264 -2.03466326841842
    265 -1.99825682312693
    266 -2.03619536003885
    267 -2.11256961945245
    268 -1.99988482163353
    269 -2.004266452166
    270 -1.99988570788125
    271 -1.96339271857386
    272 -1.96391260608198
    273 -1.9811640004711
    274 -1.95744886425137
    275 -2.00121900705489
    276 -1.99523057366155
    277 -2.02949005388128
    278 -2.11916529743547
    279 -2.16376532633459
    280 -2.15546804079369
    281 -2.16550299127198
    282 -2.14143441085835
    283 -2.23354282984787
    284 -2.25101443016811
    285 -2.20446466822486
    286 -2.13395269504307
    287 -2.02355956233149
    288 -1.97561524094426
    289 -1.98834669626685
    290 -2.06158582013131
    291 -2.08172347260485
    292 -2.21947302203339
    293 -2.11253894007658
    294 -1.99156010116319
    295 -1.99823553031245
    296 -2.10197568043506
    297 -2.18652676705714
    298 -2.24521316375709
    299 -2.15207100001198
    300 -2.09053895304138
    301 -2.14493472673805
    302 -2.05354675175904
    303 -2.01622434553713
    304 -2.16656845973635
    305 -2.17069427135634
    306 -2.06445043441593
    307 -2.02848475871462
    308 -2.00151763067149
    309 -2.05111703809182
    310 -1.96759682574413
    311 -1.97968035151109
    312 -2.01802506669589
    313 -2.07094288525186
    314 -2.06825607809941
    315 -2.10903460311227
    316 -2.2280998254398
    317 -2.26033469722306
    318 -2.27641255471126
    319 -2.29501692377334
    320 -2.44502901926724
    321 -2.14666559490324
    322 -2.17827972450271
    323 -2.16515002998389
    324 -2.13739729610279
    325 -2.09212094762539
    326 -2.08773187158116
    327 -2.04639220891362
    328 -2.06077103346662
    329 -2.05300541778787
    330 -2.02274148033025
    331 -2.24676392929264
    332 -2.21582938182971
    333 -2.31779846070146
    334 -2.31916258956631
    335 -2.38632625513477
    336 -2.36522494351753
    337 -2.35977905900696
    338 -2.37293740674709
    339 -2.42863952527576
    340 -2.52349513181123
    341 -2.57550001521552
    342 -2.65161591208747
    343 -2.56576751509874
    344 -2.68839338342306
    345 -2.57912169605096
    346 -2.5170956741709
    347 -2.61209373264113
    348 -2.3230192603
    349 -2.23537056089377
    350 -2.18388938212069
    351 -2.17483625949386
    352 -2.1805723748012
    353 -2.19755376966637
    354 -2.02656097311493
    355 -2.08903592709122
    356 -2.1132759427565
    357 -2.04562669934862
    358 -2.19876530057901
    359 -2.18494809606913
    360 -2.1230570068107
    361 -2.00746595072035
    362 -2.01036451912782
    363 -1.94092346113653
    364 -2.00398539017882
    365 -2.02139320125018
    366 -1.97707526402506
    367 -1.98620494778012
    368 -1.9838922159988
    369 -2.01146914737087
    370 -2.10761505930608
    371 -2.13587476584383
    372 -2.06947256300875
    373 -2.03933797437897
    374 -2.0985832095842
    375 -2.01654557990451
    376 -2.08369596444336
    377 -2.11046824621842
    378 -2.09966140859196
    379 -2.01137897061208
    380 -1.94541193030032
    381 -1.96345338649902
    382 -1.98235582929213
    383 -2.03362892827436
    384 -1.98542433941188
    385 -1.95220654328175
    386 -1.9625757837837
    387 -1.97051046835455
    388 -2.00243356188379
    389 -2.08063131415805
    390 -2.17378676034671
    391 -2.28716508643145
    392 -2.22828782456078
    393 -2.2192117663037
    394 -2.137543161606
    395 -2.18935693477572
    396 -2.09504729060739
    397 -2.08162093574215
    398 -2.08819494457562
    399 -2.1409613648293
    400 -2.0903724414868
    401 -2.02018978964034
    402 -2.06998262490086
    403 -2.10841927188403
    404 -2.23425564353044
    405 -2.32212906243417
    406 -2.3613051207734
    407 -2.36994185866787
    408 -2.35433083878276
    409 -2.3137119752084
    410 -2.32454995795051
    411 -2.36516190029171
    412 -2.37681955420338
    413 -2.39936508763542
    414 -2.37204285677421
    415 -2.25571062090304
    416 -2.26734233849621
    417 -2.28211398175152
    418 -2.35382760786636
    419 -2.34265535339046
    420 -2.40523542873955
    421 -2.32475622925939
    422 -2.34445881548395
    423 -2.33426948188754
    424 -2.33938498080021
    425 -2.40842787848666
    426 -2.47223858438655
    427 -2.52068967576422
    428 -2.50261739018123
    429 -2.44624150105355
    430 -2.27388353879807
    431 -2.30028621583887
    432 -2.28111026376986
    433 -2.25731266964089
    434 -2.21886737213854
    435 -2.15793139071333
    436 -2.14179108528118
    437 -2.13812964604509
    438 -2.06851427587426
    439 -2.15870656766401
    440 -2.29447163149481
    441 -2.24276227164892
    442 -2.23891935181042
    443 -2.2955756606749
    444 -2.35492679112557
    445 -2.45617343075663
    446 -2.39141029023243
    447 -2.3564988428396
    448 -2.45881303349791
    449 -2.29678500955237
    450 -2.27603251972852
    451 -2.30054516463689
    452 -2.30577395411214
    453 -2.27709393135887
    454 -2.23370491242425
    455 -2.16460927675941
    456 -2.1936375215419
    457 -2.24024089442888
    458 -2.21428048948865
    459 -2.26314624218109
    460 -2.22093846761973
    461 -2.32585514031858
    462 -2.34652266878435
    463 -2.30375933800092
    464 -2.3392304958521
    465 -2.44384204944883
    466 -2.4704816319235
    467 -2.31432998731992
    468 -2.3160824756038
    469 -2.37601936877725
    470 -2.41555930317091
    471 -2.41892098131144
    472 -2.41040686886956
    473 -2.54875676209362
    474 -2.54102454403189
    475 -2.46446109115492
    476 -2.51754163880929
    477 -2.67085847177874
    478 -2.67430541382668
    479 -2.66630754712825
    480 -2.70113513512488
    481 -2.64827667934011
    482 -2.68972375495868
    483 -2.62490944345092
    484 -2.6434912774563
    485 -2.75137041896373
    486 -2.70408162021278
    487 -2.75624237931337
    488 -2.78440520871568
    489 -2.63719142365943
    490 -2.60379013396827
    491 -2.61987767289933
    492 -2.53767942540536
    493 -2.5279516765393
    494 -2.51990134750131
    495 -2.53986347716762
    496 -2.51252384741321
    497 -2.4653282513771
    498 -2.4899591099983
    499 -2.63293631587701
    500 -2.64918701556931
    501 -2.63361948589521
    502 -2.73353580966615
    503 -2.75080472453422
    504 -2.82638737108161
    505 -2.73873383861296
    506 -2.80408839182594
    507 -2.55313074584366
    508 -2.53452606480815
    509 -2.55787487116639
    510 -2.57054883474698
    511 -2.60103067365766
    512 -2.61047159683932
    513 -2.67334540436572
    514 -2.67656355049373
    515 -2.61364213408805
    516 -2.57409474744469
    517 -2.89401011584199
    518 -2.81763503538092
    519 -2.75475397637935
    520 -2.77705272053887
    521 -2.71761471039866
    522 -2.52036754599633
    523 -2.48526907444321
    524 -2.35493587863346
    525 -2.40815493144064
    526 -2.39205419528609
    527 -2.34735450936119
    528 -2.36963358182757
    529 -2.44540334036994
    530 -2.484688542182
    531 -2.5202941869515
    532 -2.58005100675414
    533 -2.56202016017165
    534 -2.73353827450557
    535 -2.80605229214704
    536 -2.83057270377513
    537 -2.87941280106177
    538 -2.84346012638618
    539 -2.71353656783229
    540 -2.64877622503008
    541 -2.72087795630576
    542 -2.71244033570784
    543 -2.70650458394029
    544 -2.67336995713863
    545 -2.56180509554887
    546 -2.6022220972498
    547 -2.55867924914976
    548 -2.60411551528617
    549 -2.72759741330967
    550 -2.63509922269471
    551 -2.52209544143878
    552 -2.60181193366396
    553 -2.54106735222215
    554 -2.46329613672693
    555 -2.41794932321197
    556 -2.43144677535499
    557 -2.46140723455189
    558 -2.40972011051153
    559 -2.34467263863345
    560 -2.40181680677904
    561 -2.39800685742861
    562 -2.3484655396463
    563 -2.32643395258573
    564 -2.44277700091304
    565 -2.48209554396092
    566 -2.41991669103703
    567 -2.32896659117889
    568 -2.40637088023511
    569 -2.3480251969486
    570 -2.38568385379965
    571 -2.3852611208495
    572 -2.41029980836885
    573 -2.50907155381269
    574 -2.49274060518582
    575 -2.42563396344374
    576 -2.42883788734977
    577 -2.5237187354247
    578 -2.41787882337063
    579 -2.51084761901175
    580 -2.51397772923794
    581 -2.46859147284248
    582 -2.55129299268628
    583 -2.55585223524822
    584 -2.47927423119767
    585 -2.6187514944587
    586 -2.63531034233626
    587 -2.55953311982349
    588 -2.61359651421854
    589 -2.58937044817434
    590 -2.48383709358596
    591 -2.61281298072189
    592 -2.50881319995181
    593 -2.48126369356787
    594 -2.44412592303743
    595 -2.37298706037785
    596 -2.34032888364022
    597 -2.36834109636538
    598 -2.41832633238377
    599 -2.41597232935967
    600 -2.46995458693309
    601 -2.37437877423476
    602 -2.43297496216796
    603 -2.40487958960238
    604 -2.37613944800138
    605 -2.44566748529882
    606 -2.46827145919615
    607 -2.51508789753956
    608 -2.51475303328387
    609 -2.4587659482586
    610 -2.46862287949352
    611 -2.54309610010522
    612 -2.44820914446013
    613 -2.49209917727258
    614 -2.57203325000671
    615 -2.61332661784291
    616 -2.61245634292128
    617 -2.60893595479624
    618 -2.56561867968061
    619 -2.63943059512697
    620 -2.50294923412311
    621 -2.54713971933177
    622 -2.59742392302675
    623 -2.64638210471459
    624 -2.65759184024901
    625 -2.66793126544752
    626 -2.63269501822501
    627 -2.64185902380557
    628 -2.54945502481399
    629 -2.59477519457792
    630 -2.76525450686026
    631 -2.65781080385955
    632 -2.58935003968317
    633 -2.6051942314979
    634 -2.66560075005096
    635 -2.59995169493644
    636 -2.64831640535913
    637 -2.65302998337639
    638 -2.81977040394588
    639 -2.68473812013117
    640 -2.67193363306734
    641 -2.69836034552172
    642 -2.86965140517771
    643 -2.73796307819725
    644 -2.58956862945883
    645 -2.61853404285531
    646 -2.60561442891285
    647 -2.56642198977991
    648 -2.44058972253182
    649 -2.51971379167246
    650 -2.53030915036576
    651 -2.57620805172153
    652 -2.50787216911423
    653 -2.45137446284002
    654 -2.48467657997551
    655 -2.33838411987992
    656 -2.3765869727887
    657 -2.36213184277553
    658 -2.39986434260858
    659 -2.38113115953702
    660 -2.35643669328837
    661 -2.36630619245366
    662 -2.41245402114364
    663 -2.4092080882179
    664 -2.46127068197646
    665 -2.65574749656675
    666 -2.54329925417045
    667 -2.57966498565512
    668 -2.59026998673633
    669 -2.6064221529059
    670 -2.66711410178787
    671 -2.67379291917033
    672 -2.695443474463
    673 -2.72624607351563
    674 -2.73026383312597
    675 -2.74056288722874
    676 -2.7270094638618
    677 -2.56332216674412
    678 -2.53999443288669
    679 -2.40458144202097
    680 -2.40712951705752
    681 -2.39094257519438
    682 -2.31491769984268
    683 -2.33148260121363
    684 -2.33250033838954
    685 -2.30663806453015
    686 -2.25314918018756
    687 -2.39557938261403
    688 -2.49020529559308
    689 -2.64895835433765
    690 -2.56318770950127
    691 -2.58836150707402
    692 -2.67912217341429
    693 -2.79992455922088
    694 -2.71955523643308
    695 -2.66959359950406
    696 -2.91306166932924
    697 -2.81425706713029
    698 -2.59143723572052
    699 -2.59236096983376
    700 -2.63098028300507
    701 -2.57303274345548
    702 -2.49708116863646
    703 -2.41401680062053
    704 -2.53153153592765
    705 -2.47791534257362
    706 -2.46783858688149
    707 -2.4869948241815
    708 -2.47436169805593
    709 -2.4720286156471
    710 -2.42758619249265
    711 -2.41194823894218
    712 -2.42971874120548
    713 -2.46284454395321
    714 -2.3490076686519
    715 -2.37092391074142
    716 -2.37249786600699
    717 -2.34180045336653
    718 -2.27263849697463
    719 -2.16203158194875
    720 -2.13988929173855
    721 -2.17348785120566
    722 -2.22144707376207
    723 -2.20025490975961
    724 -2.21120347982342
    725 -2.26532319291653
    726 -2.29034225188539
    727 -2.31626322034109
    728 -2.50172124472682
    729 -2.43500165530686
    730 -2.49145061450323
    731 -2.4759920991805
    732 -2.42150564107211
    733 -2.46918767264861
    734 -2.49728714549278
    735 -2.42543666337001
    736 -2.35655688743515
    737 -2.37251832090205
    738 -2.41387532036917
    739 -2.64136766471357
    740 -2.63784527086668
    741 -2.68357374740052
    742 -2.61561101743294
    743 -2.65525825975861
    744 -2.56678122211397
    745 -2.61989693862506
    746 -2.71700497041186
    747 -2.61001567378505
    748 -2.59152498832945
    749 -2.4848663295488
    750 -2.51595701632711
    751 -2.34545047285615
    752 -2.46586592240749
    753 -2.41365847919138
    754 -2.45905555225944
    755 -2.48575699581597
    756 -2.43947202642944
    757 -2.55432213245595
    758 -2.52539478271901
    759 -2.61810048165466
    760 -2.59515637239813
    761 -2.77870249062349
    762 -2.65743469486148
    763 -2.6706655707967
    764 -2.66215785184941
    765 -2.61647704752582
    766 -2.66268057025487
    767 -2.65088998655168
    768 -2.73678227913288
    769 -2.75542979873437
    770 -2.67861780153649
    771 -2.71305307676143
    772 -2.73136209882319
    773 -2.53080425118237
    774 -2.65964489461937
    775 -2.73568368598376
    776 -2.72866745522983
    777 -2.77608129249776
    778 -2.76533838848113
    779 -2.74282324807279
    780 -2.79207747669198
    781 -2.6296515220364
    782 -2.71546543651751
    783 -2.89655916598814
    784 -2.78396202646625
    785 -2.72966541948253
    786 -2.71451138042406
    787 -2.58097781828334
    788 -2.50736326790881
    789 -2.40115675274549
    790 -2.37744653328424
    791 -2.40229847774162
    792 -2.36510066142659
    793 -2.43539568039965
    794 -2.41270916463424
    795 -2.43448327972326
    796 -2.33006294746856
    797 -2.37320176486668
    798 -2.3832152700213
    799 -2.4455633077406
    800 -2.45270156924539
    801 -2.37316454539954
    802 -2.32258068946177
    803 -2.29212203016558
    804 -2.33593681762898
    805 -2.28785527430229
    806 -2.35558366755037
    807 -2.32493896202528
    808 -2.29388885593879
    809 -2.2626432240015
    810 -2.32184853282615
    811 -2.50469001263132
    812 -2.6322271633132
    813 -2.65637681908919
    814 -2.54472314908732
    815 -2.62868829984221
    816 -2.70147934212235
    817 -2.83430397919926
    818 -2.93138700911482
    819 -3.08714019900593
    820 -3.11594965551322
    821 -3.02499444903637
    822 -2.89326621000717
    823 -2.81701331169318
    824 -2.95180991701585
    825 -2.85972561281339
    826 -2.78138402066987
    827 -2.70118030571264
    828 -2.70566169143842
    829 -2.69098171630501
    830 -2.66744521070216
    831 -2.71536700629706
    832 -2.79167320162313
    833 -2.90361988542964
    834 -2.79394014326251
    835 -2.81163095519924
    836 -2.68843419857055
    837 -2.63589760738172
    838 -2.50645615582161
    839 -2.55868551518374
    840 -2.49108781791858
    841 -2.3973529576332
    842 -2.31858030042295
    843 -2.272980811345
    844 -2.33062696807283
    845 -2.37103986909342
    846 -2.41688486092344
    847 -2.46510892279451
    848 -2.55780303361405
    849 -2.46921710228443
    850 -2.53894141834591
    851 -2.64271501657857
    852 -2.67956066996208
    853 -2.67919132263819
    854 -2.62142664102228
    855 -2.52641476518395
    856 -2.58472148762709
    857 -2.56220035256549
    858 -2.46552895187947
    859 -2.48458416317812
    860 -2.50241995713054
    861 -2.45632694122238
    862 -2.45281931827443
    863 -2.51773432182966
    864 -2.50149352440529
    865 -2.43838445350601
    866 -2.43496647520923
    867 -2.4704148871191
    868 -2.46187345311594
    869 -2.4669124514347
    870 -2.45604886472087
    871 -2.46718118515791
    872 -2.56140900059114
    873 -2.47281143050693
    874 -2.5605467626082
    875 -2.66715820397295
    876 -2.47886233117275
    877 -2.50518711028191
    878 -2.58856472475929
    879 -2.50057673359406
    880 -2.47345361879314
    881 -2.49176679527191
    882 -2.44935743571609
    883 -2.43457697847883
    884 -2.45431419607922
    885 -2.39106957176999
    886 -2.60543870827296
    887 -2.45576322801008
    888 -2.47360063568846
    889 -2.57416547752868
    890 -2.55261422573769
    891 -2.58563225281377
    892 -2.55117231140806
    893 -2.64072276112982
    894 -2.62383556005154
    895 -2.63365031603495
    896 -2.59252449042806
    897 -2.57892129421398
    898 -2.53367340920096
    899 -2.47965289510435
    900 -2.51687840202162
    901 -2.54160976571691
    902 -2.47183254104551
    903 -2.38757339984245
    904 -2.41066856085821
    905 -2.4960040430686
    906 -2.48653745358505
    907 -2.46209478442389
    908 -2.47408185628059
    909 -2.54744083615198
    910 -2.45150016434557
    911 -2.41753807819126
    912 -2.4792666832082
    913 -2.47142418381506
    914 -2.4667535255464
    915 -2.40870395089038
    916 -2.34998872283342
    917 -2.49731852952932
    918 -2.56046648372944
    919 -2.51288967234825
    920 -2.53614992536547
    921 -2.53206738151175
    922 -2.46830986861727
    923 -2.5306357438672
    924 -2.47035726228321
    925 -2.48464379805076
    926 -2.50906755002699
    927 -2.3817848366406
    928 -2.38043592013234
    929 -2.42358702693946
    930 -2.50205216606934
    931 -2.444149583523
    932 -2.47031715288111
    933 -2.43052860911596
    934 -2.40949934387875
    935 -2.43136907383221
    936 -2.31125447861057
    937 -2.46509694490357
    938 -2.48602792224183
    939 -2.50939139209225
    940 -2.41051256619459
    941 -2.47240492276865
    942 -2.41563069184077
    943 -2.49684863693521
    944 -2.57720126897612
    945 -2.49861816090707
    946 -2.73864771694186
    947 -2.70125778777551
    948 -2.68626404518494
    949 -2.63263388155912
    950 -2.67243318808637
    951 -2.67814864474825
    952 -2.79969173213947
    953 -2.72814612327719
    954 -2.74195225749062
    955 -2.83687912174711
    956 -2.86937265494029
    957 -2.84454500178645
    958 -2.77654069146832
    959 -2.76723967303678
    960 -2.74908530400219
    961 -2.76446932480883
    962 -2.86208001886154
    963 -2.90083793817677
    964 -2.71379755010873
    965 -2.65775839722083
    966 -2.58141566064431
    967 -2.50937866114443
    968 -2.59790109461387
    969 -2.6689543041211
    970 -2.80070943225715
    971 -2.82451138715278
    972 -2.78761724612145
    973 -2.76081880637455
    974 -2.8157562187232
    975 -2.97517155084604
    976 -3.01065062938476
    977 -3.20467267022612
    978 -3.14048381975604
    979 -2.9743564784829
    980 -2.87209257269687
    981 -2.76948783791905
    982 -2.76626657360081
    983 -2.74868694559745
    984 -2.88744514044206
    985 -2.84042083497592
    986 -2.84352900559577
    987 -2.81012486498373
    988 -2.78291242128219
    989 -2.73974462701503
    990 -2.76548794433346
    991 -2.83778017344435
    992 -2.80900447381432
    993 -2.88844006318232
    994 -2.87993929324629
    995 -2.94742842341307
    996 -2.93433114699699
    997 -2.88409370215078
    998 -2.90845729220618
    999 -3.07245855433731
    1000 -2.97766327707391
    1001 -2.83844031961824
    1002 -2.67610104449158
    1003 -2.65606941180322
    1004 -2.58543921975335
    1005 -2.49646564286529
    1006 -2.47547945316039
    1007 -2.47951320030302
    1008 -2.42318967056828
    1009 -2.38410833369066
    1010 -2.43385677194812
    1011 -2.48741355706548
    1012 -2.53697935269568
    1013 -2.54166877916011
    1014 -2.49116803514523
    1015 -2.52383272301468
    1016 -2.50399419872605
    1017 -2.50971223204078
    1018 -2.52828511056579
    1019 -2.53294050377977
    1020 -2.58591872694417
    1021 -2.59133581422906
    1022 -2.65810181110169
    1023 -2.64455789503062
    1024 -2.75798484261536
    1025 -2.76877427291654
    1026 -2.85673050604938
    1027 -2.8412140251612
    1028 -2.95313455929925
    1029 -2.84464564800556
    1030 -2.80399723203989
    1031 -2.82499114611929
    1032 -2.85016634818348
    1033 -2.90908506187284
    1034 -2.8672070161915
    1035 -2.84544279070747
    1036 -2.83927700130961
    1037 -2.91001605941075
    1038 -2.71697519834612
    1039 -2.77927964514322
    1040 -2.72991744934756
    1041 -2.7448618490062
    1042 -2.6960084467962
    1043 -2.60619686816952
    1044 -2.66037182475078
    1045 -2.71250174138092
    1046 -2.60578785732711
    1047 -2.54463127015058
    1048 -2.53934106483116
    1049 -2.60375428014267
    1050 -2.62537468898202
    1051 -2.6352425355302
    1052 -2.66745921228708
    1053 -2.66952459996645
    1054 -2.57865176728984
    1055 -2.54076804772361
    1056 -2.66755519240666
    1057 -2.7597410980942
    1058 -2.92679344954252
    1059 -2.75628872805778
    1060 -2.66561262637464
    1061 -2.60328052522175
    1062 -2.67321740669048
    1063 -2.66470116242885
    1064 -2.68227380543292
    1065 -2.67757941775468
    1066 -2.63597737030305
    1067 -2.63213422199444
    1068 -2.62550745243953
    1069 -2.67038534993519
    1070 -2.76075086362765
    1071 -2.81087309011057
    1072 -2.66812509470982
    1073 -2.61494948014507
    1074 -2.63162710368227
    1075 -2.64932266507328
    1076 -2.67733716261242
    1077 -2.5550777357213
    1078 -2.51419325802041
    1079 -2.63333624556313
    1080 -2.59781344259066
    1081 -2.63420496311548
    1082 -2.7942382657507
    1083 -2.93709958271439
    1084 -2.87924044050598
    1085 -2.85146906921592
    1086 -2.74230602643575
    1087 -2.78503835178544
    1088 -2.74643299533201
    1089 -2.5752078030617
    1090 -2.58842442839886
    1091 -2.48639178179089
    1092 -2.46065923324066
    1093 -2.41558142975592
    1094 -2.2893788149616
    1095 -2.28181112667165
    1096 -2.26691757412053
    1097 -2.2900951123793
    1098 -2.34530215122236
    1099 -2.45274434395226
    1100 -2.50824727813082
    1101 -2.60770554360548
    1102 -2.61210212693246
    1103 -2.4500544190157
    1104 -2.5454527687563
    1105 -2.58662315756547
    1106 -2.66761288310718
    1107 -2.66906945136682
    1108 -2.67678416285371
    1109 -2.65958373828206
    1110 -2.62699859215028
    1111 -2.5731157952761
    1112 -2.42367650192695
    1113 -2.58471633071439
    1114 -2.69981133280017
    1115 -2.62617025223914
    1116 -2.63040350353174
    1117 -2.6938502936385
    1118 -2.70209816636561
    1119 -2.6776083947489
    1120 -2.66249695518112
    1121 -2.66775877619554
    1122 -2.83415667286402
    1123 -2.88335165398151
    1124 -2.78185067547234
    1125 -2.89455823653178
    1126 -2.76544215852754
    1127 -2.75830947186893
    1128 -2.75260841701164
    1129 -2.83912736004258
    1130 -2.91514038253622
    1131 -2.78827943504137
    1132 -2.66591982228071
    1133 -2.61636605313557
    1134 -2.73833328486411
    1135 -2.71207253205931
    1136 -2.82521592769516
    1137 -2.81110459106645
    1138 -2.7286475829357
    1139 -2.6023313856554
    1140 -2.57374514896757
    1141 -2.71002574107919
    1142 -2.75754433589129
    1143 -2.84836175781005
    1144 -2.76119103523125
    1145 -2.69587145683472
    1146 -2.7683011978541
    1147 -2.76225092473307
    1148 -2.76073684770626
    1149 -2.80609316489123
    1150 -2.72456756911059
    1151 -2.68813114104837
    1152 -2.75774495164866
    1153 -2.71264740585739
    1154 -2.67579670443092
    1155 -2.72441160090029
    1156 -2.6983130996804
    1157 -2.69566899570245
    1158 -2.71281870523834
    1159 -2.76547218109107
    1160 -2.87996483527108
    1161 -2.81278751532392
    1162 -2.85884826559381
    1163 -2.86794180847405
    1164 -3.03047110759252
    1165 -2.93166376305612
    1166 -2.91661653812711
    1167 -2.94037679475309
    1168 -3.05153067992972
    1169 -2.93206955948586
    1170 -2.77918863401629
    1171 -2.76246526432322
    1172 -2.62082444243463
    1173 -2.66841267719528
    1174 -2.62013852974216
    1175 -2.69480389688608
    1176 -2.6991659379353
    1177 -2.63135745990458
    1178 -2.54898591055703
    1179 -2.64359491770636
    1180 -2.62951651029879
    1181 -2.5600997938378
    1182 -2.53124594982629
    1183 -2.4352483462517
    1184 -2.4491928508868
    1185 -2.47268783064415
    1186 -2.50710665874787
    1187 -2.55720512777073
    1188 -2.49290468440893
    1189 -2.43197893247375
    1190 -2.53237802563193
    1191 -2.70261713362216
    1192 -2.85715851161701
    1193 -2.93036365672485
    1194 -2.9482057197914
    1195 -2.92869134334491
    1196 -2.74862796189037
    1197 -2.75865954046737
    1198 -2.92388074887238
    1199 -2.98459953983069
    1200 -2.95807442000059
    1201 -2.94104545596214
    1202 -2.94677277486381
    1203 -2.95991884682283
    1204 -2.82612412260351
    1205 -2.72354938209745
    1206 -2.86583632602023
    1207 -2.71579404367402
    1208 -2.76831577274704
    1209 -2.77029479578941
    1210 -2.65349251444458
    1211 -2.65733288362418
    1212 -2.58690232061422
    1213 -2.50747372245831
    1214 -2.58482553179872
    1215 -2.62567081220633
    1216 -2.59810091759526
    1217 -2.64424035701439
    1218 -2.63368525558826
    1219 -2.55367097023432
    1220 -2.60394462496515
    1221 -2.46789123452366
    1222 -2.50199570624474
    1223 -2.50096061751329
    1224 -2.53381533263493
    1225 -2.61024994597705
    1226 -2.50343334886425
    1227 -2.46077640011437
    1228 -2.41826848096429
    1229 -2.43501893217542
    1230 -2.49070836111235
    1231 -2.63028361790811
    1232 -2.6712749002926
    1233 -2.78722354139647
    1234 -2.71387315795046
    1235 -2.4476538529759
    1236 -2.54567440377533
    1237 -2.64664664491662
    1238 -2.5210552609447
    1239 -2.52161560137217
    1240 -2.53556313488894
    1241 -2.51238759700925
    1242 -2.47041786560136
    1243 -2.44879269738737
    1244 -2.47027130043003
    1245 -2.69299483003865
    1246 -2.71221900517479
    1247 -2.64504730923919
    1248 -2.75229897018019
    1249 -2.79083794643916
    1250 -2.7371299189445
    1251 -2.75490101761954
    1252 -2.78586777066771
    1253 -2.75934711961558
    1254 -2.71818298757646
    1255 -2.64259773361901
    1256 -2.65441021777239
    1257 -2.67703649559431
    1258 -2.6673375078041
    1259 -2.69893001967463
    1260 -2.78386168476091
    1261 -2.77048866711828
    1262 -2.68888579408921
    1263 -2.76697431740699
    1264 -2.81186956398221
    1265 -2.78814797893026
    1266 -2.77585653383809
    1267 -2.84264741572469
    1268 -2.93047289295381
    1269 -2.93345108284365
    1270 -2.8212544599178
    1271 -2.85317605470357
    1272 -2.89097071953883
    1273 -2.77342913719252
    1274 -2.83798819210647
    1275 -2.89719948401873
    1276 -2.91061993352948
    1277 -2.90135029134616
    1278 -2.83754798013159
    1279 -2.81661081971634
    1280 -2.79858645753983
    1281 -2.81880122533638
    1282 -2.94168340754568
    1283 -2.9357075665739
    1284 -2.8559637549293
    1285 -2.94855465827563
    1286 -2.94330775879135
    1287 -2.9373118801022
    1288 -2.74676197798611
    1289 -2.59096959330064
    1290 -2.6900411134189
    1291 -2.67779274706673
    1292 -2.67785858525167
    1293 -2.80008769542147
    1294 -2.85472313829308
    1295 -2.79631494739046
    1296 -2.70206738227835
    1297 -2.73271685355248
    1298 -2.93833625622898
    1299 -3.26571226412017
    1300 -3.22425298265153
    1301 -3.19983604121664
    1302 -3.21158439215258
    1303 -3.16094470205132
    1304 -3.16224396235856
    1305 -3.11543399240561
    1306 -3.24328394962751
    1307 -3.23172390042932
    1308 -3.26264904781483
    1309 -2.94933330746234
    1310 -2.92268039766103
    1311 -2.94402028822769
    1312 -2.72215752769392
    1313 -2.75827632907541
    1314 -2.76596868127761
    1315 -2.55481177070029
    1316 -2.57222470013866
    1317 -2.43934961525045
    1318 -2.45013128378282
    1319 -2.54639135021723
    1320 -2.47424482168549
    1321 -2.42968666106301
    1322 -2.46185782035077
    1323 -2.36526927762759
    1324 -2.32146061496423
    1325 -2.47256782262801
    1326 -2.42851856597532
    1327 -2.54600052612085
    1328 -2.49731536347173
    1329 -2.43597552205456
    1330 -2.5353837219759
    1331 -2.57969181520028
    1332 -2.57866307683208
    1333 -2.62585917943747
    1334 -2.65437126574648
    1335 -2.7261937332663
    1336 -2.71603939338156
    1337 -2.59408227148127
    1338 -2.47214063203423
    1339 -2.43229207658083
    1340 -2.43066494094845
    1341 -2.36917309269038
    1342 -2.44696805833139
    1343 -2.47046554014906
    1344 -2.41909565299352
    1345 -2.37672709367456
    1346 -2.33677445499587
    1347 -2.42520992905075
    1348 -2.56192564500746
    1349 -2.64567894338534
    1350 -2.48868037554122
    1351 -2.49168669878043
    1352 -2.49524649931009
    1353 -2.48689726570711
    1354 -2.52615068674319
    1355 -2.55072894883192
    1356 -2.6094420849594
    1357 -2.47891953282107
    1358 -2.46395109031325
    1359 -2.56764084978465
    1360 -2.71345488683967
    1361 -2.75221651683718
    1362 -2.74512608039695
    1363 -2.80156635120473
    1364 -2.8107360001666
    1365 -2.89313217037382
    1366 -2.98223758915172
    1367 -3.1355061585361
    1368 -3.27283049318861
    1369 -3.10172352409643
    1370 -3.13162516991265
    1371 -3.18152068425893
    1372 -3.24091290679154
    1373 -3.21617621890587
    1374 -3.31525362697377
    1375 -3.1515470292429
    1376 -3.10491274252066
    1377 -3.10092146288702
    1378 -3.01685363689207
    1379 -2.99274524909069
    1380 -2.89987345347861
    1381 -2.87856823161128
    1382 -2.75082850198595
    1383 -2.75207186393115
    1384 -2.52552593129825
    1385 -2.63327251655651
    1386 -2.5939623214147
    1387 -2.64186925839027
    1388 -2.72145847377954
    1389 -2.5149793675052
    1390 -2.45850292207431
    1391 -2.42056368278614
    1392 -2.42347557387252
    1393 -2.35387228949211
    1394 -2.47673912065718
    1395 -2.4352055365929
    1396 -2.4699101751839
    1397 -2.48987318016817
    1398 -2.4346428614887
    1399 -2.66439505606546
    1400 -2.76744630545671
    1401 -2.82621038862393
    1402 -2.82584638374372
    1403 -2.92034369716542
    1404 -2.94966412239301
    1405 -2.98535145658318
    1406 -2.90011283613985
    1407 -2.85862111172232
    1408 -2.89446131701427
    1409 -2.88353624113583
    1410 -2.88756311389182
    1411 -2.8259601980546
    1412 -2.9124836842067
    1413 -2.91237771833706
    1414 -2.96999923779976
    1415 -2.91301363403831
    1416 -2.83441258448663
    1417 -2.71768536594465
    1418 -2.69278896876747
    1419 -2.77261023974804
    1420 -2.85003381974273
    1421 -2.95680162613015
    1422 -2.89721239965021
    1423 -2.93877925782011
    1424 -2.69796222078587
    1425 -2.68030085160355
    1426 -2.83978228943862
    1427 -2.92970603303304
    1428 -2.9563762188023
    1429 -2.90334152321549
    1430 -2.88304948512929
    1431 -2.87635518083041
    1432 -2.84996402575401
    1433 -2.78328688680409
    1434 -2.9933475174643
    1435 -3.10520290913235
    1436 -3.13404574207357
    1437 -3.19290176795059
    1438 -3.20135941007395
    1439 -3.15271485368147
    1440 -3.15318917066743
    1441 -3.13690128090698
    1442 -3.17924901616394
    1443 -3.30148116711696
    1444 -3.23792802494592
    1445 -3.09152827451315
    1446 -3.02905524810805
    1447 -3.06701267996336
    1448 -2.98034567647264
    1449 -3.07234892464298
    1450 -2.85840097642343
    1451 -2.8260280159409
    1452 -2.80063357380783
    1453 -2.71891742287975
    1454 -2.80813921604314
    1455 -2.81512303143809
    1456 -2.82442497409077
    1457 -2.84698562975759
    1458 -2.85067198057945
    1459 -2.89440030020976
    1460 -3.11002245778411
    1461 -3.09492483900823
    1462 -3.24694485106584
    1463 -3.36459682213265
    1464 -3.21836509949989
    1465 -3.20267849204975
    1466 -3.1685497451988
    1467 -3.12151388887595
    1468 -3.13589410543783
    1469 -2.84845707239699
    1470 -2.85254637472229
    1471 -2.94229728956126
    1472 -2.8613677679347
    1473 -2.81085014810341
    1474 -2.91116231174339
    1475 -2.68548838443208
    1476 -2.68634701676011
    1477 -2.60027601888231
    1478 -2.63132028955953
    1479 -2.69719171831808
    1480 -2.6702916443018
    1481 -2.62596911177837
    1482 -2.57714084531619
    1483 -2.62317512668578
    1484 -2.61604797409052
    1485 -2.90637204433964
    1486 -2.88903050478799
    1487 -2.99532149380588
    1488 -2.92549628894919
    1489 -2.97815200200631
    1490 -2.94748759912636
    1491 -2.8833054026542
    1492 -3.03279885999181
    1493 -3.00853204289427
    1494 -2.89826856708354
    1495 -2.81950099860223
    1496 -2.85953393159307
    1497 -2.79718556175593
    1498 -2.90652813228093
    1499 -2.98628314033206
    1500 -3.01451628291195
    1501 -3.08005178144431
    1502 -3.05421625776747
    1503 -2.85776366812195
    1504 -2.90832133408656
    1505 -2.92976054923162
    1506 -2.62763786852413
    1507 -2.66602140515735
    1508 -2.61158089566705
    1509 -2.53544085976733
    1510 -2.54138573748443
    1511 -2.53139047711762
    1512 -2.56149904371582
    1513 -2.5551960524984
    1514 -2.52396175871161
    1515 -2.49507400363034
    1516 -2.6437077800787
    1517 -2.63133541293221
    1518 -2.5390503661147
    1519 -2.62853891086405
    1520 -2.64963684424484
    1521 -2.65603027574059
    1522 -2.6191706849208
    1523 -2.77915424599089
    1524 -2.85313620274656
    1525 -2.87238927490314
    1526 -2.94525525902726
    1527 -2.96830354702239
    1528 -3.11323926092891
    1529 -3.05260219475986
    1530 -2.71329115727338
    1531 -2.77349721752029
    1532 -2.82640265411176
    1533 -2.83630864408059
    1534 -2.74493749000271
    1535 -2.77039128144532
    1536 -2.74029608294872
    1537 -2.61649124950697
    1538 -2.55655827524375
    1539 -2.61839737222308
    1540 -2.88983987995059
    1541 -2.83002252370008
    1542 -2.67410076240872
    1543 -2.607738167329
    1544 -2.62104046097155
    1545 -2.44914138931081
    1546 -2.51483247180538
    1547 -2.54246251960412
    1548 -2.64461424198093
    1549 -2.59510116157165
    1550 -2.54208261217823
    1551 -2.57737290576182
    1552 -2.67254448828198
    1553 -2.70294179245718
    1554 -2.78203788456283
    1555 -3.00537589584468
    1556 -2.93363298468754
    1557 -3.06444812931637
    1558 -3.09021164291169
    1559 -3.04019150493644
    1560 -3.08273966449095
    1561 -3.07536115125888
    1562 -3.0541063310287
    1563 -2.99844463893442
    1564 -2.90062520399438
    1565 -2.96218228672677
    1566 -3.02463899039758
    1567 -3.07955239273375
    1568 -3.02049720875147
    1569 -2.97716777533878
    1570 -3.00811276850443
    1571 -3.02374323680793
    1572 -2.9843964662737
    1573 -2.97508547249588
    1574 -3.00452424748602
    1575 -2.93171316737808
    1576 -2.99043599124801
    1577 -2.82813461921338
    1578 -2.80610614078832
    1579 -2.80223511729826
    1580 -2.82036126278121
    1581 -2.7333571894692
    1582 -2.77561168238587
    1583 -2.86311781037139
    1584 -2.81134629619385
    1585 -2.8408296515201
    1586 -2.78924230471696
    1587 -2.9222278064769
    1588 -2.91555777952511
    1589 -2.895797154286
    1590 -2.85115857716495
    1591 -2.94079679201994
    1592 -2.86346311545549
    1593 -2.86261264472946
    1594 -2.93735270006716
    1595 -2.93662263049427
    1596 -2.87539287941647
    1597 -2.77076223933438
    1598 -2.75784969276989
    1599 -2.85790920634367
    1600 -2.71104064652938
    1601 -2.73109819379947
    1602 -2.84756194370112
    1603 -2.67060318640236
    1604 -2.70028544734757
    1605 -2.67452422487759
    1606 -2.65778685997667
    1607 -2.7136481961077
    1608 -2.75318990414225
    1609 -2.74764222464042
    1610 -2.94781133935012
    1611 -2.73088226703769
    1612 -2.70760134290587
    1613 -2.84789531450756
    1614 -2.84616557834473
    1615 -2.87318094109648
    1616 -2.97977554625565
    1617 -3.05551703430421
    1618 -3.00763995087006
    1619 -3.06291261327238
    1620 -3.05691175245117
    1621 -3.33117915739105
    1622 -3.36423677068036
    1623 -3.33063960312318
    1624 -3.22136657435205
    1625 -3.17322408634608
    1626 -3.10271503775924
    1627 -3.00697974006532
    1628 -2.77833120327699
    1629 -2.7278698618373
    1630 -2.70513250605713
    1631 -2.69845068467796
    1632 -2.68715435557475
    1633 -2.73124899753763
    1634 -2.66397516192302
    1635 -2.76543715041609
    1636 -2.75564409197755
    1637 -2.76545802338413
    1638 -3.06966724300436
    1639 -3.15311409545459
    1640 -3.12096881487105
    1641 -3.15801801874154
    1642 -3.16108708568968
    1643 -3.13890414395263
    1644 -3.43048774237733
    1645 -3.14692819683608
    1646 -3.11285968414487
    1647 -3.13292276443289
    1648 -3.13602342701993
    1649 -3.07367588767791
    1650 -3.15816871691249
    1651 -3.10120279749708
    1652 -3.00423596184985
    1653 -3.04216604206027
    1654 -2.93428619894785
    1655 -2.98367148744813
    1656 -3.11175983315805
    1657 -3.00910494931497
    1658 -2.98815144888719
    1659 -2.98834809778198
    1660 -3.02678244564863
    1661 -3.03615817261738
    1662 -3.11580029000193
    1663 -3.0561777512642
    1664 -3.17571292113534
    1665 -3.30941733238997
    1666 -3.2335875906582
    1667 -3.39474803914525
    1668 -3.42423269857159
    1669 -3.36179593792939
    1670 -3.25842388321211
    1671 -3.01955587587288
    1672 -3.04531431291709
    1673 -3.06915124152031
    1674 -3.04534768295898
    1675 -3.04107903585781
    1676 -3.05327821194678
    1677 -3.00622600760639
    1678 -2.97800183741017
    1679 -2.90904902615458
    1680 -2.85303531206946
    1681 -3.01730025393264
    1682 -2.86950100574537
    1683 -2.93253325241215
    1684 -2.89864042045828
    1685 -2.736999944654
    1686 -2.70727823552241
    1687 -2.72159181030502
    1688 -2.66294603339628
    1689 -2.6447444312782
    1690 -2.69362772885832
    1691 -2.75177291496583
    1692 -2.87711723421887
    1693 -2.71770283384174
    1694 -2.73022264855009
    1695 -2.8431162971769
    1696 -2.90022483741026
    1697 -2.93866698347583
    1698 -2.93689538962367
    1699 -3.08826646706293
    1700 -3.14726749387421
    1701 -3.13853000632022
    1702 -3.09087758802218
    1703 -3.00267684600289
    1704 -2.96113359313077
    1705 -3.02724735856442
    1706 -3.02051117955834
    1707 -3.03187127312988
    1708 -3.17326752769269
    1709 -3.13681230356816
    1710 -3.07558464709729
    1711 -2.95178432984469
    1712 -2.88289313709502
    1713 -3.01473854778032
    1714 -3.08488344210625
    1715 -3.12390240145569
    1716 -3.07177401252602
    1717 -3.03973901969467
    1718 -2.9958580249091
    1719 -2.9955567011348
    1720 -3.00088685344878
    1721 -3.1287205301389
    1722 -3.12466833879448
    1723 -3.09661300021796
    1724 -2.99190459347805
    1725 -2.8033593030875
    1726 -2.83437980732909
    1727 -2.83959387490668
    1728 -2.72005626955809
    1729 -2.78194241203077
    1730 -2.80431561087585
    1731 -2.7371085575258
    1732 -2.84402913637992
    1733 -2.78500580483803
    1734 -2.82912781773015
    1735 -2.98584423835538
    1736 -3.04058349851493
    1737 -3.04877192851071
    1738 -3.26509629814546
    1739 -3.1946953410226
    1740 -3.05604225272388
    1741 -3.0569790939813
    1742 -2.98222700601909
    1743 -3.08352026120898
    1744 -3.12018743424239
    1745 -3.08256042436473
    1746 -2.94089729261384
    1747 -2.93724121646449
    1748 -2.82989688678164
    1749 -2.86322509594054
    1750 -2.89274037239168
    1751 -2.81738529467132
    1752 -2.76504886216439
    1753 -2.73464835039382
    1754 -2.61042681825478
    1755 -2.65224076037767
    1756 -2.67128436922459
    1757 -2.5455421108144
    1758 -2.55588110657489
    1759 -2.53473615660008
    1760 -2.55705880565136
    1761 -2.6723816730003
    1762 -2.78238141649293
    1763 -2.81899474835986
    1764 -2.83331653070095
    1765 -2.82360421130427
    1766 -2.8272498618278
    1767 -2.8909261842808
    1768 -2.89960923431547
    1769 -2.90997817968909
    1770 -2.98991967338157
    1771 -2.82170694927001
    1772 -2.75526288799429
    1773 -2.68203808716929
    1774 -2.75831193683647
    1775 -2.73314501744134
    1776 -2.51739717110818
    1777 -2.54532995753084
    1778 -2.52256291600208
    1779 -2.40007040591977
    1780 -2.39988547895896
    1781 -2.35274039800073
    1782 -2.36817236221802
    1783 -2.39896007792008
    1784 -2.32671576308856
    1785 -2.30578901042444
    1786 -2.53183379234515
    1787 -2.59609928386298
    1788 -2.62062614327026
    1789 -2.80191388152838
    1790 -2.75624190434369
    1791 -2.97132651092917
    1792 -2.93016124773196
    1793 -3.03860039477584
    1794 -3.13418264583787
    1795 -3.1760279877069
    1796 -3.06631366285677
    1797 -3.00387467542603
    1798 -2.97535075808198
    1799 -2.94291941100442
    1800 -2.81352770454791
    1801 -2.84629641768116
    1802 -2.85902868486035
    1803 -2.82169177383325
    1804 -2.89455728877044
    1805 -2.88614954045793
    1806 -2.98662137806089
    1807 -3.05232273755478
    1808 -3.01002214906412
    1809 -3.02169164703012
    1810 -3.10957358646506
    1811 -2.94742650484458
    1812 -2.9033411529939
    1813 -2.98385599974029
    1814 -2.85546695543144
    1815 -2.80458676056152
    1816 -2.78285780637852
    1817 -2.69082539965208
    1818 -2.74400088823941
    1819 -2.6578768913135
    1820 -2.73258915167481
    1821 -2.76979992980497
    1822 -2.77555918347852
    1823 -2.73558811638424
    1824 -2.83019482719053
    1825 -2.91351467129001
    1826 -2.89304406892137
    1827 -3.00585326237253
    1828 -3.0103307141421
    1829 -3.15753345963409
    1830 -3.07362056359822
    1831 -3.0331502161649
    1832 -2.98171039536287
    1833 -2.87564411086364
    1834 -2.86342606609695
    1835 -2.76591700885634
    1836 -2.73043782861125
    1837 -2.64961045331801
    1838 -2.61945990695643
    1839 -2.56419441663405
    1840 -2.56524863543889
    1841 -2.62638846453089
    1842 -2.74548836035224
    1843 -2.86815473674274
    1844 -2.84443693647991
    1845 -2.77653261001895
    1846 -2.75527045373831
    1847 -2.67381699964421
    1848 -2.73682036464507
    1849 -2.80759536813773
    1850 -2.84219320676652
    1851 -2.76255206191617
    1852 -2.61752972222024
    1853 -2.63470750340679
    1854 -2.6525505081325
    1855 -2.81822777627871
    1856 -2.83244471223087
    1857 -2.97262550505061
    1858 -2.95011877595234
    1859 -2.87937042565385
    1860 -2.9001851158144
    1861 -2.99745509232936
    1862 -3.07509600738678
    1863 -3.08054059678885
    1864 -2.80705241582239
    1865 -2.73694717009363
    1866 -2.77116154977125
    };
    
    \nextgroupplot[
        tick align=outside,
        tick pos=left,
        title={Log Loss (1028 samples a batch)},
        x grid style={darkgrey176},
        xmin=-8.8, xmax=184.8,
        xtick style={color=black},
        y grid style={darkgrey176},
            ymin=-2.94658160309521, ymax=1.29509444507854,
        ytick style={color=black}
        ]
        \addplot [semithick, steelblue31119180]
        table {%
        0 0.838301971653526
        1 0.828826725330321
        2 0.815978390386228
        3 0.785747337251929
        4 0.755846750842255
        5 0.698784997139127
        6 0.632010282092578
        7 0.541953789033851
        8 0.430269069544001
        9 0.263650951677177
        10 0.149241052856639
        11 -0.00356623888656975
        12 -0.0780660515932956
        13 -0.0853267872143466
        14 -0.267464421169385
        15 -0.224078499665351
        16 -0.423078410141685
        17 -0.30174917853234
        18 -0.505489732337354
        19 -0.658265033266905
        20 -0.505454555736983
        21 -0.682620453634407
        22 -0.601068105228986
        23 -0.818361588796143
        24 -0.850691460676616
        25 -0.76170849685772
        26 -0.8493239660811
        27 -0.886963501970419
        28 -0.91894403285419
        29 -0.886909239077543
        30 -0.865292345145298
        31 -0.778011434081545
        32 -1.14522027611315
        33 -0.871278643108495
        34 -1.1030825704814
        35 -1.16937512731408
        36 -1.07930265017243
        37 -1.12755580059146
        38 -1.14534599275437
        39 -1.14307737567512
        40 -1.2792586065639
        41 -1.31620774670978
        42 -1.26935175963833
        43 -1.39381042923176
        44 -1.21239730633511
        45 -1.29294642616557
        46 -1.36490554082022
        47 -1.46091125717116
        48 -1.32768682999585
        49 -1.48387406375919
        50 -1.40804267213633
        51 -1.48772383257584
        52 -1.49729314214235
        53 -1.46719565475068
        54 -1.38882754567649
        55 -1.43854920803999
        56 -1.56876086307212
        57 -1.46648132740561
        58 -1.46953841718264
        59 -1.54801389820414
        60 -1.77722430367705
        61 -1.58137142313585
        62 -1.62040946429544
        63 -1.61292562886413
        64 -1.61088980257516
        65 -1.83373298415071
        66 -1.67423674106425
        67 -1.68796133234405
        68 -1.80568025904985
        69 -1.72251801766633
        70 -1.65023661546889
        71 -1.70199051152652
        72 -1.6979305372363
        73 -1.62505273363886
        74 -1.79804686032845
        75 -1.91428918513292
        76 -1.98511616857952
        77 -1.89318148637848
        78 -1.77712570581522
        79 -1.70419361731784
        80 -1.90405949753535
        81 -1.95795789983374
        82 -1.83086160239516
        83 -1.71371202178374
        84 -2.01592675496469
        85 -1.97949608023499
        86 -1.70105970530472
        87 -1.878124507302
        88 -1.6830575002457
        89 -1.81131833012752
        90 -1.65465568683679
        91 -1.83891465482148
        92 -1.8472194333371
        93 -1.94888710413068
        94 -1.89116078142969
        95 -1.85773850341972
        96 -1.9664610677215
        97 -2.07538913774137
        98 -2.03235347633709
        99 -1.9318119474718
        100 -1.75166974501057
        101 -1.9070194444687
        102 -2.02117662370117
        103 -2.01666359937692
        104 -1.93205860712737
        105 -2.06561158867344
        106 -1.94793165589378
        107 -2.16396553763972
        108 -1.94451835122211
        109 -2.12255686723087
        110 -2.18342371646949
        111 -1.93951927048057
        112 -2.00797507165128
        113 -1.76978659879353
        114 -2.10914150960454
        115 -1.92406212650451
        116 -2.11050395309009
        117 -1.92670861771024
        118 -2.06637386312503
        119 -2.06624456134797
        120 -2.23072638815543
        121 -2.04697124310435
        122 -2.39043267169239
        123 -2.26592935615116
        124 -2.19879073051326
        125 -2.09777700338317
        126 -2.07762644040053
        127 -2.18667223845921
        128 -2.24923603297821
        129 -2.24994949060864
        130 -2.22779574305121
        131 -2.1803601589912
        132 -2.09168012525091
        133 -2.38485839265079
        134 -2.06653789689705
        135 -2.31203745940285
        136 -2.17037608754393
        137 -2.16089464499012
        138 -2.00743892560966
        139 -2.42401239888717
        140 -2.13281363364295
        141 -2.29882952138129
        142 -2.07579894921067
        143 -2.20806790698777
        144 -2.15021493629832
        145 -2.17361491457642
        146 -2.50359354885127
        147 -2.36155885982031
        148 -2.12165872126424
        149 -2.06955224312419
        150 -2.31860616231087
        151 -2.24976282744051
        152 -2.3577138542489
        153 -2.22825498623054
        154 -2.37767876054436
        155 -2.49067333508597
        156 -2.38652524456728
        157 -2.61789258731183
        158 -2.3659926398749
        159 -2.42297517040029
        160 -2.45560672608561
        161 -2.33874798747423
        162 -2.39752175555233
        163 -2.51311509201039
        164 -2.75377814636004
        165 -2.08267907712805
        166 -2.23736456319391
        167 -2.21580724819213
        168 -2.38300163884612
        169 -2.18218641329117
        170 -2.62372530344811
        171 -2.5428673307921
        172 -2.51083136216565
        173 -2.43129980467861
        174 -2.20277731127388
        175 -2.20474924826717
        176 -2.50429047407976
        };
        \addplot [semithick, darkorange25512714]
        table {%
        0 0.838301971653526
        1 0.823417641522596
        2 0.790649079626582
        3 0.747370378157745
        4 0.638016891970126
        5 0.558940222137023
        6 0.434309837950668
        7 0.317859675696422
        8 0.132423906433381
        9 0.000674497054567166
        10 -0.135062189179858
        11 -0.194041145858225
        12 -0.273991020789286
        13 -0.26933245694654
        14 -0.377668392299269
        15 -0.434613229281255
        16 -0.46688958642956
        17 -0.416858132215799
        18 -0.504390554496559
        19 -0.622616214516718
        20 -0.500861807797118
        21 -0.53351210245159
        22 -0.501039258077078
        23 -0.726363768323301
        24 -0.694087587852413
        25 -0.587032449958303
        26 -0.633656615418573
        27 -0.643249936710923
        28 -0.642433285969507
        29 -0.647715769588679
        30 -0.663581924834384
        31 -0.592196759515961
        32 -0.910388163629229
        33 -0.615902576316635
        34 -0.761669367034761
        35 -0.879463996257615
        36 -0.73581641133663
        37 -0.806915033942406
        38 -0.814077027371366
        39 -0.760311304900962
        40 -0.882465998634889
        41 -0.917556701065104
        42 -0.799239074182258
        43 -0.988204443600516
        44 -0.796896715394257
        45 -0.844579731613066
        46 -0.815379685349701
        47 -1.0054673467197
        48 -0.937398055560639
        49 -1.00536146051355
        50 -0.90695246006693
        51 -1.06033406749284
        52 -1.03793712339025
        53 -0.948274523158072
        54 -1.02170228696724
        55 -1.00391357391424
        56 -1.02899265622032
        57 -0.976853193566005
        58 -0.994718090629825
        59 -1.09080761515189
        60 -1.1795575943652
        61 -1.1155020729513
        62 -1.06210368731853
        63 -1.10960089487771
        64 -1.06209394657279
        65 -1.16626185671354
        66 -1.2439138884131
        67 -1.24733156460904
        68 -1.2466944735805
        69 -1.18717910376739
        70 -1.09994262497257
        71 -1.23663127045088
        72 -1.22268852980956
        73 -1.20961693067013
        74 -1.23600824080552
        75 -1.30689070935174
        76 -1.40724676710495
        77 -1.39303506823826
        78 -1.20680700435882
        79 -1.10963379914185
        80 -1.35063152622978
        81 -1.37059863643814
        82 -1.3013141465987
        83 -1.1162759125728
        84 -1.43713998054059
        85 -1.3771890121265
        86 -1.30111115909049
        87 -1.36058987720628
        88 -1.19651598134263
        89 -1.37917957175908
        90 -1.16219692301195
        91 -1.37773834365207
        92 -1.40064999176328
        93 -1.48865163468806
        94 -1.44406434430249
        95 -1.44915817096934
        96 -1.60506593114875
        97 -1.58126609545293
        98 -1.54623109007554
        99 -1.40160817922844
        100 -1.39742441513023
        101 -1.49588823113947
        102 -1.54100128709089
        103 -1.50769161799728
        104 -1.40296981639268
        105 -1.63633645246737
        106 -1.56517276998221
        107 -1.59302873180477
        108 -1.4730083443208
        109 -1.5007263598552
        110 -1.62253241834373
        111 -1.66292833705623
        112 -1.70516292974169
        113 -1.46528258301008
        114 -1.70348722105357
        115 -1.52044481183309
        116 -1.62262006356469
        117 -1.67467198345994
        118 -1.72721648147739
        119 -1.64328053226761
        120 -1.67889628318669
        121 -1.53397995628424
        122 -1.7733521147723
        123 -1.85613502070523
        124 -1.69983114943515
        125 -1.76690505910853
        126 -1.86380529058446
        127 -1.7015216507222
        128 -1.83785079797769
        129 -1.75353158816046
        130 -1.76603262012087
        131 -1.79315355006736
        132 -1.62956006123702
        133 -1.92164046414383
        134 -1.5868619127524
        135 -1.77167865666564
        136 -1.83625624380128
        137 -1.86860794498359
        138 -1.72774451076418
        139 -2.05713715069061
        140 -1.84276059923439
        141 -1.88198491397197
        142 -1.69985047801493
        143 -1.82009115201904
        144 -1.78571381314739
        145 -1.76559450844051
        146 -2.07997210413706
        147 -2.04124819739963
        148 -1.7023187141257
        149 -1.69787884986124
        150 -1.85541241915964
        151 -1.88543445176717
        152 -1.89460431914448
        153 -1.88472903082802
        154 -2.12630463237398
        155 -2.06600635571633
        156 -2.1425767307286
        157 -2.15902840753483
        158 -1.87802265440274
        159 -1.91024763050562
        160 -1.91849905586651
        161 -2.16215274021225
        162 -1.91175787256986
        163 -1.98106899575842
        164 -2.12574080040245
        165 -1.78977362741452
        166 -1.97409224367907
        167 -1.83992081885613
        168 -1.8346818670318
        169 -1.77310233468893
        170 -1.93540542209598
        171 -2.08265396231298
        172 -1.92524992281458
        173 -1.99940705031109
        174 -1.96242862959346
        175 -1.9445483506315
        176 -2.10093605150128
        };
        \addplot [semithick, forestgreen4416044]
        table {%
        0 0.838301971653526
        1 0.83396888117636
        2 0.834409080809651
        3 0.83319184739959
        4 0.830556982141101
        5 0.831014472287955
        6 0.828395619334816
        7 0.826544781609576
        8 0.824495466604375
        9 0.819257476477885
        10 0.812173140737928
        11 0.804347103778118
        12 0.796317630496424
        13 0.787012606015008
        14 0.767547394155057
        15 0.736109522062694
        16 0.706455212947811
        17 0.648976487536636
        18 0.5834208836647
        19 0.42741895983021
        20 0.312317119836173
        21 0.148423984234228
        22 0.0676464422593537
        23 0.161406887438871
        24 1.10229098834337
        25 0.953422239614833
        26 0.54371206297338
        27 0.169982122218525
        28 0.349991566609427
        29 0.38014777288539
        30 0.397417578162826
        31 0.368102840517935
        32 0.297817328205269
        33 0.289018290311179
        34 0.184407568682565
        35 -0.00230237444045099
        36 -0.115070705053878
        37 -0.190737042507982
        38 -0.232269995960904
        39 -0.332124518679413
        40 -0.367091376770882
        41 -0.43395905015171
        42 -0.331757103492593
        43 -0.524237556084739
        44 -0.442555951838553
        45 -0.51712523098642
        46 -0.548236566545454
        47 -0.750541053452753
        48 -0.721920232085667
        49 -0.825531824510311
        50 -0.854973996094766
        51 -0.90351027719986
        52 -0.847138688917674
        53 -0.872503277648529
        54 -0.944689145654497
        55 -0.964459602631114
        56 -1.14916445127075
        57 -1.03225928352818
        58 -1.08884847046896
        59 -1.18281394144548
        60 -1.24561629529977
        61 -1.20620406835903
        62 -1.27232119118489
        63 -1.26838226986739
        64 -1.24377752739632
        65 -1.39595715163454
        66 -1.41455047295502
        67 -1.37292669039792
        68 -1.4437331804942
        69 -1.40059865704083
        70 -1.29042728223978
        71 -1.41576839722839
        72 -1.41494546708875
        73 -1.39999323918551
        74 -1.39527797211189
        75 -1.56074975120558
        76 -1.62648680631256
        77 -1.6317017898234
        78 -1.46293360564562
        79 -1.35893585447157
        80 -1.68552068047426
        81 -1.56659188656277
        82 -1.57032865700779
        83 -1.35086610574822
        84 -1.63769690597782
        85 -1.62484168904415
        86 -1.40500026852293
        87 -1.66926297734361
        88 -1.48550181129545
        89 -1.58565578409307
        90 -1.43793927178626
        91 -1.54704204976023
        92 -1.5836163653457
        93 -1.69662069997903
        94 -1.62265056378951
        95 -1.50310060368885
        96 -1.75485321226592
        97 -1.67233015809976
        98 -1.74805912144906
        99 -1.69710735694846
        100 -1.59269178643048
        101 -1.63773200830763
        102 -1.71850770106186
        103 -1.73378812026789
        104 -1.59003735078267
        105 -1.76870263456416
        106 -1.6132355824405
        107 -1.79901397491004
        108 -1.61052925783314
        109 -1.72513048447272
        110 -1.74425872029772
        111 -1.70151454350994
        112 -1.81142756120196
        113 -1.69484220973372
        114 -1.84525423256129
        115 -1.58851921632333
        116 -1.71199700683962
        117 -1.67242918198459
        118 -1.76806093965012
        119 -1.71611826006901
        120 -1.75015229872519
        121 -1.70784909590993
        122 -1.92880088233146
        123 -2.04387856029964
        124 -1.84878046523635
        125 -1.74834030439312
        126 -1.97603478282307
        127 -1.85483155207395
        128 -1.98624770063113
        129 -1.88754278340373
        130 -1.89282433853556
        131 -2.00304134322691
        132 -1.79171849187436
        133 -2.21720221456344
        134 -1.72636030524164
        135 -1.93804250686442
        136 -1.93716870289065
        137 -2.05471999363372
        138 -1.77154660771481
        139 -2.11509468875537
        140 -1.94027449581705
        141 -1.8901343983418
        142 -1.76235501326187
        143 -1.9648294088654
        144 -1.85052673137103
        145 -1.92004204771351
        146 -2.13408131354107
        147 -2.22255042280991
        148 -1.94401922229221
        149 -1.82363067605062
        150 -2.0988798792966
        151 -2.05650257489021
        152 -2.18035158762612
        153 -2.11569846862589
        154 -2.18985572940842
        155 -2.25160114985356
        156 -2.0845901902374
        157 -2.30097414523092
        158 -2.03413852073372
        159 -2.09347445434099
        160 -2.16211421813286
        161 -2.31024635297678
        162 -2.02921105008354
        163 -2.09180811161852
        164 -2.43535121645423
        165 -2.03384838502119
        166 -2.11149472379765
        167 -2.04637387675913
        168 -2.06076668091836
        169 -1.93769411440958
        170 -2.13159270772468
        171 -2.16348865833296
        172 -2.17356920317072
        173 -2.08851470889734
        174 -2.0900274898105
        175 -2.06439346218724
        176 -2.16423779930582
        };
        \addplot [semithick, crimson2143940]
        table {%
        0 0.838301971653526
        1 0.82828659013972
        2 0.814199782955458
        3 0.776858172502318
        4 0.74118111301677
        5 0.662601479922838
        6 0.579446645729969
        7 0.443432843895724
        8 0.326774074846713
        9 0.157593369531989
        10 0.053809834115423
        11 -0.113876320583394
        12 -0.0395196345882323
        13 -0.134871657836619
        14 -0.271087828951772
        15 -0.217163881955579
        16 -0.376530091118064
        17 -0.42677465224159
        18 -0.528228505288601
        19 -0.684637023292988
        20 -0.511107951214291
        21 -0.729177616435995
        22 -0.6516006179751
        23 -0.877755256383936
        24 -0.868123235534027
        25 -0.768380672239239
        26 -0.912708419964991
        27 -0.940996364438879
        28 -0.992457436474722
        29 -0.96915067155518
        30 -0.872581155330543
        31 -0.832793663482944
        32 -1.20615817087935
        33 -0.948920074618124
        34 -1.09987064866886
        35 -1.17964203703466
        36 -1.17275450198399
        37 -1.11647167326685
        38 -1.18897551410003
        39 -1.16886847571428
        40 -1.28106212288482
        41 -1.35912714375671
        42 -1.28613986770264
        43 -1.43273003690082
        44 -1.20940375963633
        45 -1.297152065519
        46 -1.34454723466862
        47 -1.44100445288984
        48 -1.28787881591876
        49 -1.49870836668367
        50 -1.3718421756848
        51 -1.47443665481509
        52 -1.51130186859
        53 -1.3632592816008
        54 -1.35990868938588
        55 -1.46057941142579
        56 -1.58560434777012
        57 -1.48881011632862
        58 -1.41842811034401
        59 -1.55005458061574
        60 -1.72929692771413
        61 -1.62256518060221
        62 -1.60641470509001
        63 -1.58141974429261
        64 -1.57115782685906
        65 -1.78504938960008
        66 -1.69332920422093
        67 -1.71759095336047
        68 -1.75440817032806
        69 -1.65409466690516
        70 -1.5778279642658
        71 -1.79697093782577
        72 -1.58675578670449
        73 -1.65591423598057
        74 -1.72438666420413
        75 -1.84962778087422
        76 -1.94665148832787
        77 -1.88985604809095
        78 -1.75740538526836
        79 -1.67315263942817
        80 -1.87421107706427
        81 -1.86538090159179
        82 -1.77515745811635
        83 -1.66490014467894
        84 -1.96850767308337
        85 -1.88791173133701
        86 -1.65548583165992
        87 -1.88981738898214
        88 -1.67138484682054
        89 -1.76252457983128
        90 -1.67439923908207
        91 -1.82416672726069
        92 -1.82600020624069
        93 -1.89529582033919
        94 -1.825077269279
        95 -1.89421932377433
        96 -2.04367897827091
        97 -1.99108378794
        98 -2.04183389802328
        99 -1.96473044813896
        100 -1.72057708335021
        101 -1.87431545549686
        102 -1.92280877169936
        103 -1.97179523929723
        104 -1.89896810442486
        105 -2.03335226235701
        106 -1.91244742108712
        107 -2.15896954671626
        108 -1.88020711365349
        109 -2.08711772744068
        110 -2.08049119372454
        111 -1.90771316171883
        112 -2.1007467931887
        113 -1.81481864968884
        114 -2.11926909022663
        115 -1.85158451831774
        116 -2.08576300233178
        117 -1.94302555947689
        118 -2.05876779921007
        119 -1.96147393409577
        120 -2.16281244257221
        121 -1.97484151275822
        122 -2.26077469836162
        123 -2.26223716311808
        124 -2.07961607931054
        125 -2.05133099305611
        126 -1.92776588007278
        127 -2.18336307144783
        128 -2.25556010495947
        129 -2.19722363111293
        130 -2.18657696245065
        131 -2.12164236941065
        132 -2.01609939018014
        133 -2.25310201234102
        134 -1.9729861637571
        135 -2.25598257100139
        136 -2.14408548117483
        137 -2.2318154764088
        138 -1.98599686222144
        139 -2.38073681069418
        140 -2.24731208802783
        141 -2.29332905257166
        142 -1.84249550255063
        143 -2.13127890466322
        144 -2.14652917442026
        145 -1.99812930932465
        146 -2.36104947130187
        147 -2.31997925597113
        148 -1.98968192856493
        149 -2.05301395691675
        150 -2.16795515150485
        151 -2.08675878968719
        152 -2.26039405903871
        153 -2.04752901479327
        154 -2.17482342248182
        155 -2.22504656573924
        156 -2.29386748826804
        157 -2.33804418360706
        158 -2.28815943327488
        159 -2.22028121888135
        160 -2.20729015854848
        161 -2.30960658592502
        162 -2.30644700906634
        163 -2.30461772372366
        164 -2.53497526682172
        165 -2.13271467754794
        166 -2.28921589313079
        167 -2.16522997257852
        168 -2.36346550173867
        169 -2.17629333159422
        170 -2.56231922531695
        171 -2.35044612842499
        172 -2.48166738264954
        173 -2.22725917775218
        174 -2.05612935971406
        175 -2.24388429685796
        176 -2.31058441489999
        };
    \end{groupplot}
    \node at (7,7.5) [anchor=north] {
    \begin{tikzpicture}[scale=0.75] % Nested TikZ environment
 % AdaBelief
 \draw[red,  ultra thick] (0,0) -- ++(0.8,0);
 \node[anchor=west] at (1,0) {{AdaBelief}}; % Increased spacing
 
 % Adam
 \draw[steelblue31119180,  ultra thick] (3.75,0) -- ++(0.8,0);
 \node[anchor=west] at (4.75,0) {{Adam}}; % Increased spacing
 
 % AdaHessian
 \draw[darkorange25512714,  ultra thick] (6.7,0) -- ++(0.8,0);
 \node[anchor=west] at (7.7,0) {{AdaHessian}}; % Increased spacing
 
 % Apollo
 \draw[forestgreen4416044,  ultra thick] (11,0) -- ++(0.8,0);
 \node[anchor=west] at (11.9,0) {{Apollo}}; % Increased spacing
    \end{tikzpicture}
};
    \end{tikzpicture}
     \\ % Replace with the correct path to your .tex file
    \end{tabular}
    \caption{The log loss of the model during training, y-axis, after each update step, x-axis, while training with  \emph{small}- (left) and \emph{big} batches (right) of training data.}
    \label{fig:loss-big-batch}
\end{figure}



\comment{
\section{Optimizer Behavior in Parameter Space}
To gain deeper insight into how different second-order approximations influence the movement of optimizers
through the parameter space, we will define several metrics whose evolution we will plot over the entire training
duration.
\subsection{Step Length}
To measure how the step length in the parameter space develops over time,
we will record $\alpha$, where $\alpha = \lVert \Delta \theta \rVert_2$ and $\Delta \theta = \theta_t - \theta_{t-1}$, $t \in \mathbb{N}$.
\subsection{Distance Traveled}
To measure the exploration ability of the parameter space by each optimizer,
we introduce $\beta$, where $\beta = \lVert \theta_{t} - \theta_{0} \rVert_2$, $t \in \mathbb{N}$.
This metric indicates how far the optimizer has traveled from its initial point to the
current point in time.
\subsection{Step Oscillations}
To measure how much the optimizers oscillate between steps, we measure the angle between two successive update steps. We define this angle $\gamma$ as
\[
    \gamma = \arccos\left(\frac{\langle \Delta \theta_t, \Delta \theta_{t-1} \rangle}{\lVert \Delta \theta_t \rVert \cdot \lVert \Delta \theta_{t-1} \rVert}\right),
\]
where 
\[
    \Delta \theta_{t} =  \theta_{t} -  \theta_{t-1}, \quad \Delta \theta_{t-1} =  \theta_{t-1} -  \theta_{t-2}.
\]
\subsection{Sharpness of Minima}
To measure the sharpness of the minima found by the optimizers during training, we focus on the curvature of these minima. 
Minima with flatter curvature are generally associated with better generalization performance because small changes in the input are less likely to significantly affect the output of the loss function.
To quantify this, we define the metric \( \phi \) as 
\[
    \phi =  \frac{1}{n}\sum_{i=0}^{n} \left| \frac{\partial^2 \mathcal{L}}{\partial \theta_{ii}} \right|.
\]
This metric, \( \phi \), provides a measure of the absolute mean curvature at a given point in the parameter space.
A lower value of \( \phi \) indicates flatter minima, which are typically associated with better generalization.
In contrast, a higher \( \phi \) value corresponds to sharper minima.
\subsection{Results}
}



\comment{\section{The SApollo Optimizer}
As we noted in the previous section, the approximation quality of the batch Hessian in \emph{Apollo} is not competitive with existing first-order methods.
However, an interesting new investigation could involve smoothing the calculated Hessian diagonal in \emph{Apollo}.
As we recall from Apollo's Algorithm \ref{alg:apollo}, $B_t$ represents the Hessian diagonal that it outputs.
Unlike other optimizers, \emph{Apollo} does not smooth these approximations of $B_t$ over multiple steps to reduce variance.
The authors themselves suggest that investigating the moving average of the diagonal $B_t$ might be a promising direction for future work \cite{apollo}.
In the following, we will explore this by applying an \emph{EMA} (Exponential Moving Average) to smooth $B_t$ and investigate the consequences on Hessian approximation quality and the converge of the loss.
We will call this \emph{SApollo} (\textbf{S}moothedApollo) \ref{alg:Sapollo}.

\marginpar{\cite{apollo}}
\label{alg:Sapollo}
\begin{algorithm}
    \caption{SApollo}
    \begin{algorithmic}[1]
    \State \textbf{Initial:} $m_0, d_0, B_0, v_0 \leftarrow 0, 0, 0,0$ \Comment{Initialize $m_0, d_0, B_0$ to zero}
    \State Good default settings are $\beta_1 = 0.9,\boldsymbol{\beta_2 = 0.999}$ and $\epsilon = 10^{-4}$
    \While{$t \in \{0, \ldots, T\}$}
        \For{$\theta \in \{\theta_1, \ldots, \theta_L\}$}
            \State $g_{t+1} \leftarrow \nabla f_t(\theta_t)$ \Comment{Calculate gradient at step $t$}
            \State $m_{t+1} \leftarrow \frac{\beta_1(1-\beta_1^t)}{1-\beta_1^{t+1}}m_t + \frac{1-\beta_1}{1-\beta_1^{t+1}}g_{t+1}$ \Comment{Bias-corrected EMA}
            \State $\alpha \leftarrow \frac{d_t^T(m_{t+1}-m_t)+d_t^TB_td_t}{(\|d_t\|_4+\epsilon)^4}$ \Comment{Calculate coefficient of $B$ update}
            \State $B_{t+1} \leftarrow B_t - \alpha \cdot \text{Diag}(d_t^2)B_{t+1} $ \Comment{Update diagonal Hessian}
            \State $\boldsymbol{v_{t+1} \leftarrow \beta_2 v_t + (1-\beta_2)B_{t+1}}$ \Comment{\textbf{Update EMA of B}}
            \State $\boldsymbol{\hat{v}_{t+1} \leftarrow \frac{v_t}{1-\beta_2^{t+1}}}$ \Comment{\textbf{Bias-corrected EMA}}

            \State $D_{t+1} \leftarrow \text{rectify}(\boldsymbol{\hat{v}_{t+1}}, 0.01)$ \Comment{Handle nonconvexity}
            \State $d_{t+1} \leftarrow D_{t+1}^{-1}m_{t+1}$ \Comment{Calculate update direction}
            \State $\theta_{t+1} \leftarrow \theta_t - \eta_{t+1}d_{t+1}$ \Comment{Update parameters}
        \EndFor
    \EndWhile
    \State \textbf{return} $\theta_T$
    \end{algorithmic}
    \end{algorithm}

\section{Optimizer Behaviour Visualization}

\section{Hessian Diagonal in FFN's}
}
