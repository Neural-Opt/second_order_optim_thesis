%************************************************
\chapter{Theoretical foundations}\label{ch:introduction}
%************************************************

\section{Foundations of Differential Calculus}
\label{sec:foundationcalc}


In this section, we explore the fundamental concept of the derivative.
In mathematics, a derivative measures the rate at which a function changes as its input changes. 
Simply put, it represents the slope of the tangent line to the function's graph at any given point. 
For a function $f(x) $ of a single variable $ x $, the derivative is often written as $ f'(x)$  or $\frac{df}{dx} $,
and it quantifies how much the function changes with a small change in $ x $.

\subsection{Derivatives for Functions of a Single Variable}
\label{sec:derivative}

Let $ f: \mathbb{R} \to \mathbb{R} $ be a single-variable function. The derivative of the function $ f $ at a point $x$ denoted as $ f'(x) $, is defined as
\begin{align}
    f'(x) = \lim_{h \to 0} \frac{f(x + h) - f(x)}{h} \quad h \in \mathbb{R}.
\end{align}
\subsection{Partial Derivatives}
\label{sec:partial_derivative}


Consider a function $ f: \mathbb{R}^n \to \mathbb{R} $.
The partial derivative of $f$ with respect to the variable $ x_i $ at a point $ \mathbf{x} = (x_1, x_2, \ldots, x_n)$ is defined as
\begin{align}
 \frac{\partial f}{\partial x_i}(\mathbf{x}) = \lim_{h \to 0} \frac{f(x_1, \ldots, x_i+h, \ldots, x_n) - f(x_1, \ldots, x_i, \ldots, x_n)}{h}, \quad h \in \mathbb{R}.
\end{align}

Assuming this limit exists, this definition encapsulates how $f$ responds to infinitesimal changes in $x_i$, while keeping its other variables fixed.

\subsection{The Gradient}
\label{sec:gradient}

Let \( f: \mathbb{R}^n \to \mathbb{R} \) be a function  that is differentiable at a point $ \mathbf{x} $.
Than the gradient of $ f $ at $ \mathbf{x} \in \mathbb{R}^n$, that is denoted as $ \nabla f(\mathbf{x}) $, is the vector of all its first partial derivatives
\begin{align}
\nabla f(\mathbf{x}) = \left( \frac{\partial f}{\partial x_1}, \frac{\partial f}{\partial x_2}, \ldots, \frac{\partial f}{\partial x_n} \right) \quad \mathbf{x} \in \mathbb{R}^n.
\end{align}
The gradient vector points in the direction of the greatest rate of increase of the function, and its magnitude represents the rate of change in that direction~\cite{bachman2007}.


\subsection{The Jacobian Matrix}
\label{sec:jaocobian}

Let $ \mathbf{f}: \mathbb{R}^n \to \mathbb{R}^m $ be a vector-valued function.
The Jacobian matrix \( \mathbf{J_f} \) of \( \mathbf{f} \) is an \( m \times n \) matrix that contains all first-order partial derivatives of the component functions \( f_i \) with respect to the input variables \( x_j  \in \mathbb{R} \), where $1 \leq i \leq m$, $1 \leq j \leq n$. It is defined as follows

\begin{align}
\mathbf{J_f} = \begin{bmatrix}
\frac{\partial f_1}{\partial x_1} & \frac{\partial f_1}{\partial x_2} & \cdots & \frac{\partial f_1}{\partial x_n} \\
\frac{\partial f_2}{\partial x_1} & \frac{\partial f_2}{\partial x_2} & \cdots & \frac{\partial f_2}{\partial x_n} \\
\vdots & \vdots & \ddots & \vdots \\
\frac{\partial f_m}{\partial x_1} & \frac{\partial f_m}{\partial x_2} & \cdots & \frac{\partial f_m}{\partial x_n}
\end{bmatrix}.
\end{align}


Each element \( \frac{\partial f_i}{\partial x_j} \) of the Jacobian matrix represents the partial derivative of the \( i \)-th component function \( f_i \) with respect to the \( j \)-th input variable \( x_j \).
Later, we will see that the Jacobian matrix plays a crucial role in the backpropagation algorithm used for optimizing neural networks.
\subsection{The Hessian Matrix}
\label{sec:hessian}

Suppose \( f: \mathbb{R}^n \to \mathbb{R} \) is function that is an at least twice differentiable and takes as input a vector \( \mathbf{x} \in \mathbb{R}^n \) and 
outputs a scalar \( f(\mathbf{x}) \in \mathbb{R} \). Then the Hessian matrix \( \mathbf{H} \) of \( f \) is given by

\begin{align}
    \label{eq:hessian}
    \mathbf{H}_f(\mathbf{x}) = \begin{bmatrix}
    \frac{\partial^2 f}{\partial x_1^2} & \frac{\partial^2 f}{\partial x_1 \partial x_2} & \cdots & \frac{\partial^2 f}{\partial x_1 \partial x_n} \\
    \frac{\partial^2 f}{\partial x_2 \partial x_1} & \frac{\partial^2 f}{\partial x_2^2} & \cdots & \frac{\partial^2 f}{\partial x_2 \partial x_n} \\
    \vdots & \vdots & \ddots & \vdots \\
    \frac{\partial^2 f}{\partial x_n \partial x_1} & \frac{\partial^2 f}{\partial x_n \partial x_2} & \cdots & \frac{\partial^2 f}{\partial x_n^2}
    \end{bmatrix}.
\end{align}
\emph{Schwarz's Theorem} states states the following: Let \( U \subseteq \mathbb{R}^n \) be an open set and \( f \colon U \to \mathbb{R} \) be at least \( k \)-times partially differentiable.
If all \( k \)-th partial derivatives in \( U \) are at least continuous, then \( f \) is \( k \)-times totally differentiable.
In particular, the order of differentiation in all \( l \)-th partial derivatives with \( l \leq k \) is irrelevant. \cite{arens2011mathematik} \\
We can therefore conclude that \( \mathbf{H} \) is in fact a symmetric real-valued $n\times n$ matrix, meaning \( \mathbf{H} = \mathbf{H}^T \).\\
\subsubsection{Eigenvalues of the Hessian}
Because \( \mathbf{H} \) is symmetric, there exist $n$ lineraly independent eigenvectors,
such that \( \mathbf{H} \) can be factorized as
\begin{align}
    \mathbf{H} =  \mathbf{Q} \mathbf{\Lambda} \mathbf{Q}^{-1}
\end{align}
where $\mathbf{Q} $ is an n × n matrix whose $i$th column is the eigenvector $q_i$ of $ \mathbf{H}$, and $\mathbf{\Lambda}$ is a diagonal matrix whose elements are the corresponding eigenvalues~\cite{kashyap2023survey}.\\
The eigenvalues of the Hessian matrix play a crucial role in numerical optimization as they provide unique insight into the curvature of the function at a given point.\\
For example, a given point is a local minimizer if all $\lambda_i > 0$ or a local maximizer if all $\lambda_i < 0$. If there exist eigenvalues of
different sign, than a given point presents a saddle point which are particularly challenging in the context of machine learning as optimizers as SGD often times get stuck at these regions~\cite{bottou2018optimization}. \\\\
\subsubsection{Matrix definiteness}
\label{sec:matrix_definiteness}
Let $ \mathbf{M}$  be an $n \times n$ symmetric real matrix with  $n \in \mathbb{R}$, than the following statements hold:  
\[
M \text{ is positive-definite} \quad \iff \quad \mathbf{x}^{\operatorname{T}} M \mathbf{x} > 0 \text{ for all } \mathbf{x} \in \mathbb{R}^n \setminus \{ \mathbf{0} \}
\]
\[
M \text{ is positive semi-definite} \quad \iff \quad \mathbf{x}^{\operatorname{T}} M \mathbf{x} \geq 0 \text{ for all } \mathbf{x} \in \mathbb{R}^n
\]
\[
M \text{ is negative-definite} \quad \iff \quad \mathbf{x}^{\operatorname{T}} M \mathbf{x} < 0 \text{ for all } \mathbf{x} \in \mathbb{R}^n \setminus \{ \mathbf{0} \}
\]
\[
M \text{ is negative semi-definite} \quad \iff \quad \mathbf{x}^{\operatorname{T}} M \mathbf{x} \leq 0 \text{ for all } \mathbf{x} \in \mathbb{R}^n
\]\\
Additionally we define the positive semi-definite order of matrices by $A \preceq B \Longleftrightarrow x^T (B - A) x \geq 0 \quad \forall x \in \mathbb{R}^n$,
were $ A,B \in \mathbb{R}^{n \times n}$ are symmetric matrices.

%Consider $ M \mathbf{x} = \lambda \mathbf{x} $ for a proof. Multiply both sides by $ \mathbf{x}^T $:
%\[
%\Rightarrow \mathbf{x}^T M \mathbf{x} = \lambda \mathbf{x}^T \mathbf{x}
%\]
%Since $ \mathbf{x}^T \mathbf{x} = \| \mathbf{x} \|^2 > 0 $, the $ \lambda $s determine the sign \cite{strang2022introduction}.
\subsubsection{Singular Matrix }

Let $ \mathbf{M}$  be an $n \times n$ symmetric matrix with  $n \in \mathbb{R}$, then $ \mathbf{M}$ is called \emph{singular} if and only if it's determinant is $0$. Therefore $\mathbf{M}^{-1}$ does not exist, meaning $ \mathbf{M}$ does not have an inverse.\\ 
\subsubsection{Ill-conditioned Matrices}
A matrix $\mathbf{M} \in \mathcal{R}^{n\times m}$ is called \emph{ill-conditioned} when its \emph{condition number} is high.
This implies that small changes in its input (or elements) result in disproportionately large changes in its output.
The condition number of a matrix $M$, is defined as 
\begin{align}
    \kappa(M) = \frac{\sigma_{\max}(M)}{\sigma_{\min}(M)},
    \end{align}
where $\sigma_{\max}$ and $\sigma_{\min}$ are the largest and smallest singular values (square roots of non-negative eigenvalues \cite{SZABO2015320}) of $M$ respectively.
$M$ is called \emph{ill-conditioned} if $\kappa(M) \gg 1$.
\cite{strang2022introduction}.
\section{Introduction to Optimization}
\label{sec:optimization}

Optimization is a crucial tool used in nearly all areas of decision science, engineering, economics, machine learning, and physical sciences \cite{nocedal2006numerical}.
The process of optimization begins by identifying an objective, which is a measurable indicator of the performance of a model or system. 
Common objectives in optimization often involve maximizing or minimizing quantities like profit, cost, or time.
These objectives depend on system characteristics known as variables or parameters.
The goal of optimization is to identify the values of the set of variables or parameters that minimize or maximize a given objective.
The parameters often face some constraints, that are necessary to arrive at a practical solution.
Examples for this might be the non-negativity of the interest rate on a loan or the electron
density in a molecule.\cite{nocedal2006numerical}
The process of building such an objective-(function), choosing its parameters and constraints is called \emph{modelling}.
Constructing an effective model is often the most important step in solving a given problem.
If the model is too simple, it might not capture the core of the problem.
On the other hand, if it's overly complex, it may lead to difficult-to-compute solutions and become prone to issues like \emph{overfitting},
which will be covered in later sections. \\\\
\marginpar{\cite{nocedal2006numerical}}
Mathematically, a model and its constraints may be defined as follows
\[
\min_{\mathbf{x} \in \mathbb{R}^{n}} f(\mathbf{x}) \quad \text{subject to} \quad
c_i(\mathbf{x}) = 0, \, i \in \mathcal{E}, \quad
c_i(\mathbf{x}) \geq 0, \, i \in \mathcal{I}.
\]
In this example model we want to minimize the scalar-valued objective function $f$ given a vector of variables or parameters $\mathbf{x}$,
where $\mathbf{x}$ has to satisfy the scalar-valued constraint functions $c_i$, with sets of indices $\mathcal{E}$ and $\mathcal{I}$.

The set of parameters that serve as optimal solutions to a given model are often denoted as $\mathbf{x^*}$.
Depending on whether the problem involves maximization or minimization, these parameters are also referred to as maximizers or minimizers, respectively.
Once the model has been formulated, an optimization algorithm can be used to
find its solution, usually with the help of a computer~\cite{nocedal2006numerical}. 
The choice of an optimization algorithm heavily relies on the model's characteristics.
Factors like linearity, differentiability, convexity, and dimensionality influence whether methods like simplex (for linear models)
or gradient descent (for differentiable non-linear models) are suitable.\cite{sun2019survey}.\\
In this work, we focus on \emph{unconstrained nonlinear } optimization problems, which involve finding the optimal values of a nonlinear objective function 
without explicit constraints on the variables.
These types of problems are the most frequently encountered in modern machine learning models, particularly in the realm of deep learning \cite{Goodfellow-et-al-2016}.
For these nonlinear optimization problems, we often differentiate between \emph{first-order} and \emph{second-order} optimization techniques.
First-order methods, like (stochastic) gradient descent, use only the gradient information of the loss landscape to guide their search for
a local \emph{minimizer} of the objective function~\cite{kashyap2023survey},
while second-order methods, such as Newton's method, utilize both the gradient and the Hessian matrix to better approximate the curvature,
leading to more efficient optimization but with increased computational cost~\cite{yao2021adahessian}.
In the following sections, we explore various optimization algorithms within first- and second-order techniques,
highlighting their strengths and weaknesses.
We also provide a foundational overview of neural networks, focusing on how these optimization strategies are applied to train them. 


\section{First-order Optimization Algorithms}
\label{sec:firstorder}

First-order methods are optimization techniques that rely on gradient information to guide the search for a function's minimum or maximum.
They are termed "first-order" because they only use first derivatives \ref{sec:gradient} to update parameters.
They are computationally efficient and work well with large-scale problems, especially in machine learning \cite{Goodfellow-et-al-2016}.
\subsection{Gradient Descent}
Gradient descent (GD) is one of the most established first-order optimization algorithms. GD iteratively adjusts model
parameters to minimize a given loss function by following the negative gradient direction \cite{toussain2014gradient}.
Since the gradient is a vector indicating the direction of steepest ascent of the loss function, adjusting the parameters in the direction
of the negative gradient will result in a decrease in the function's value~\cite{toussain2014gradient}.\\
Formely we write:\\
\noindent
Let \( \mathcal{L} : \mathbb{R}^d \to \mathbb{R} \) be a differentiable function, such that \(\textbf{arg min } \mathcal{L}  \neq \emptyset \).  
Let \( \theta_0 \in \mathbb{R}^d \) and \( \gamma > 0 \) be a step size.  
\emph{The Gradient Descent (GD)} algorithm defines a sequence \( (\theta_t)_{t \in \mathbb{N}} \) satisfying:
\begin{align}
\theta_{t+1} = \theta_t - \gamma \nabla \mathcal{L}(x_t).
\end{align}
The step size, denoted as $\gamma$, is often called the \emph{learning rate} and is a crucial hyperparameter when training a model.
An incorrect choice of $\gamma$ can yield significantly different outcomes: if too large, the step size may overshoot the minimum, resulting in oscillation or divergence;
if too small, progress will be slow, requiring many iterations to reach convergence. \cite{wu2020wngrad} 


\subsection{Empirical Risk Minimization (ERM)}
\label{sec:erm}

\marginpar{\cite{bottou2018optimization}}
In machine learning, optimization problems naturally arise due to the formulation of prediction models and the associated loss functions.
These are typically used to evaluate measures like the \emph{expected} and \emph{empirical risk}, which practitioners aim to minimize \cite{bottou2018optimization}.
To formalize this, let's define a family of model functions for some given \( f(\cdot; \cdot) : \mathbb{R}^{d_x} \times \mathbb{R}^d \to \mathbb{R}^{d_y} \) as follows:
\begin{align}
\mathbb{F} := \{ f(\cdot; \theta) : \theta \in \mathbb{R}^d \}.
\end{align}

\marginpar{\cite{bottou2018optimization}}
The set \(\mathbb{F}\) represents a collection of functions parameterized by a parameter vector \( w \) that determine the prediction functions in use.
Now assume a loss function \( l : \mathbb{R}^{d_y} \times \mathbb{R}^{d_y} \to \mathbb{R} \), which, given an input-output pair \((x, y)\) and a model $f \in \mathbb{F}$, yields the loss \( l(f(x; w), y) \).
The most gratifying behavior for such a prediction function is to minimize the expected loss between any input-output pair.
For that, let's assume that losses are measured with respect to a probability distribution $P(x,y)$ with $ P : \mathbb{R}^{d_x} \times \mathbb{R}^{d_y} \to [0, 1] $,
then we could write the \emph{expected risk} as
\begin{align}
R(\theta) = \int_{\mathbb{R}^{d_x} \times \mathbb{R}^{d_y}} l(f(x; \theta), y) dP(x, y) = \mathbb{E}[l(f(x; \theta), y)].
\end{align}
\marginpar{\cite{bottou2018optimization}}
Minimization of $R(\theta)$ would ensure that the expected loss for the resulting model $f(\cdot; w) $ over all possible $(x, y)$ is minimal.\\
In practice, however, this is unfeasible when one lacks complete information about $P$.
Instead, one seeks to solve a problem by estimating \( R \).
In supervised learning, a set of \( n \in \mathbb{N} \) independently drawn input-output
samples \(\{ (x_i, y_i) \}_{i=1}^{n} \subseteq \mathbb{R}^{d_x} \times \mathbb{R}^{d_y}\) is typically available.
In machine learning specifically, we refer to this set of data as the training data.
From these samples, the \emph{empirical risk} function \( R_n : \mathbb{R}^d \to \mathbb{R} \) is defined as:
\begin{align}
R_n(\theta) = \frac{1}{n} \sum_{i=1}^{n} l(f(x_i; \theta), y_i)
\end{align}
where \( l \) is the loss function that measures the discrepancy between the prediction \( f(x_i; \theta) \) and the actual output \( y_i \).
In standard gradient descent, it is actually this \emph{empirical risk} function we strive to minimize, leading to the following gradient update step:
\begin{align}
    \theta_{t+1} = \theta_t - \gamma \nabla R_n(\theta_t).
\end{align}

The gradient descent algorithm continues to iterate through this update step until a convergence criterion is met, such as a maximum number of iterations or an acceptable error tolerance.
\subsection{Stochastic Gradient Descent}
\label{sec:sgd}

\marginpar{\cite{stanfordSGD}}

Since standard gradient descent requires the evaluation of the gradient over the whole set of training data, large training sets can quickly become 
computationally expensive.
Another issue with standard gradient descent optimization methods is that they don’t give an easy way to incorporate new data in an 'online' setting.
 Stochastic Gradient Descent (SGD) addresses both of these issues by following the negative gradient of the objective after seeing only a single or a few training examples.
 Therefore approximating the true gradient of $ R_n(\theta_t)$ by the gradient of a single or a few data points  \(B \subseteq  \{ (x_i, y_i) \}_{i=1}^{n} \subseteq \mathbb{R}^{d_x} \times \mathbb{R}^{d_y}\).
\begin{align}
    \nabla R_b(\theta) = \frac{1}{|B|} \sum_{\{\hat{x}, \hat{y}\} \in B} \nabla l(f(\hat{x}; \theta), \hat{y})
\end{align}
    with $B \in \mathbb{N}$ beeing the sample size or \emph{batch size}.
 Leading to the following update rule \cite{stanfordSGD}
 \begin{align}
    \theta_{t+1} = \theta_t - \gamma  {\nabla} R_b(\theta).
\end{align}
\marginpar{\cite{garrigos2024handbook}}
It can be shown that, this estimate $\nabla R_b(\theta)$ provides an unbiased estimator of the true gradient, satisfying
\begin{align}
\mathbb{E}[{\nabla} R_b(\theta)] = \nabla R_n(\theta).
\end{align}
While this introduces noise into the gradient estimates, repeated updates over many mini-batches allow SGD to approximate the true gradient descent path.
\subsection{Momentum}
\label{sec:momentum}

\marginpar{\cite{Goodfellow-et-al-2016}}

While Stochastic Gradient Descent (SGD) is a very effective algorithm for optimization,
its convergence rates can often be slow. To tackle the problem of slow convergence,
the method of momentum \cite{polyak1964some} was introduced. Momentum has the benefit of
leading to faster convergence even in settings with high curvature or noisy gradients \cite{Goodfellow-et-al-2016}.
The momentum algorithm introduces an additional variable $v$, called the \emph{velocity},
which defines the direction and speed at which the parameters move through parameter space \cite{Goodfellow-et-al-2016}.
The update rule for SGD with momentum is given by
\[    v_t = \beta v_{t-1} + (1 - \beta) \nabla_{\theta} J(\theta_t), \]
\[     \theta_{t+1} = \theta_t - \eta v_t .\]

where
\begin{itemize}
    \item $v_t$ is the velocity vector at iteration $t$,
    \item $\beta \in [0, 1) $ is the momentum term, typically set between 0.9 and 0.99,
    \item $\nabla_{\theta} J(\theta_t)$ is the gradient of the loss function at iteration $t$.
\end{itemize}
\marginpar{\cite{Goodfellow-et-al-2016}}
The hyperparameter $\beta$ determines how quickly the contributions of previous gradients
exponentially decay. Therefore, the larger $\beta$ is, the more previous gradients affect
the next update direction. With this exponentially weighted summation, we can avoid the gradient successively changing sign
and jumping around, because the moving average smooths out the updates by considering the influence of past gradients.
This results in a more stable and consistent direction of the gradient descent~\cite{Goodfellow-et-al-2016}.

\subsection{RMSProp}
\label{sec:rmsprop}

\marginpar{\cite{Goodfellow-et-al-2016}}
RMSProp (\emph{Root Mean Square Propagation}) \cite{hintonrms} is an adaptive learning rate optimization algorithm that adjusts the learning rate
for each parameter based on the magnitude of recent gradients.
It works by maintaining an exponential moving average of the past squared gradients to obtain curvature information. The scalar learning rate
then gets divided by this moving average and the resulting vector is mulplied with the current gradient.
This results in a parameter wise scaling of the learning rate, increasing it in low and
decreasing in high curvature regions.
This helps to stabilize the training process and reduce oscillations \cite{Goodfellow-et-al-2016}. The RMSProp update rules are given by:\marginpar{\cite{Goodfellow-et-al-2016}}

\[v_t = \beta v_{t-1} + (1 - \beta) \nabla_{\theta} J(\theta_t)^2,\]
\[\theta_{t+1} = \theta_t - \frac{\eta}{\sqrt{v_t + \epsilon}} \nabla_{\theta} J(\theta_t),\]

where
\begin{itemize}
    \item $v_t$ is the exponentially weighted moving average of the squared gradients at time step $t$,
    \item $\beta \in [0, 1) $ is the decay rate, typically set to around 0.9,
    \item $\eta$ is the learning rate,
    \item $\nabla_{\theta} J(\theta_t)$ is the gradient of the loss function at iteration $t$,
    \item $\epsilon$ is a small constant (e.g., $10^{-8}$) to prevent division by zero.
\end{itemize}

\subsection{Adam}
\label{sec:adam}

\marginpar{\cite{d2l2023adam}}

Adam (\emph{Adaptive Momentum})~\cite{kingma2017adam} is one of the most well known and popular
optimization algorithms for neural networks today. It builds upon the ideas of Momentum and
adaptive scaling of the learning rate, as it combines both the ideas of Momentum and RMSProp.
Adam calculates an exponential moving average not only from the gradients, but also the squared gradients.
It therefore introduces two hyperparameters $\beta_1 \in [0,1)$ and $\beta_2 \in [0,1)$ which determine the
influence of past (squared) gradients into the moving average
\[m_t = \beta_1 m_{t-1} + (1 - \beta_1) \nabla_{\theta} J(\theta_t), \]
\[v_t = \beta_2 v_{t-1} + (1 - \beta_2) \nabla_{\theta} J(\theta_t)^2, \]
\[\hat{m}_t = \frac{m_t}{1-\beta_1^t}, \]
\[\hat{v}_t = \frac{v_t}{1-\beta_2^t}, \]
\[\theta_{t+1} = \theta_t - \frac{\eta}{\sqrt{\hat{v_t} + \epsilon}} \hat{m_t}, \]

where
\begin{itemize}
    \item $m_t$ is the velocity vector at iteration $t$,
    \item $v_t$ is the exponentially weighted moving average of the squared gradients at time step $t$,
    \item $\beta_1 \in [0, 1) $ is the decay rate, used for the first order moment,
    \item $\beta_2 \in [0, 1) $ is the decay rate,  used for the second order moment,
    \item $\hat{m}_t$ is the bias corrected first order moment at $t$,
    \item $\hat{v}_t$ is the bias corrected second order moment at $t$,
    \item $\eta$ is the learning rate,
    \item $\nabla_{\theta} J(\theta_t)$ is the gradient of the loss function at iteration $t$,
    \item $\epsilon$ is a small constant (e.g., $10^{-8}$) to prevent division by zero.
\end{itemize}
As we can see, the Adam algorithm is very similar to RMSProp, with the exception that it
uses a first-order moment estimate ($m_t$) in addition to the second-order moment estimate
($v_t$). Unlike RMSProp, which directly uses the current gradient
 $\nabla_{\theta} J(\theta_t)$, Adam incorporates bias-corrected estimates to improve the
stability of the training process.
Adam initializes $m_0 = 0$ and $v_0 = 0$. Consequently, $m_t$ and $v_t$ are biased towards
zero, especially when the decay rates $\beta_1$ and $\beta_2$ are close to 1 \cite{d2l2023adam}. This bias
can lead to very large step sizes in the early stages of training.\cite{Goodfellow-et-al-2016} To counteract this,
Adam applies bias correction by dividing $m_t$ and $v_t$ by $(1 - \beta_1^t)$ and
$(1 - \beta_2^t)$, respectively, ensuring that the gradients at earlier steps are
accurately represented \cite{d2l2023adam}.

\subsection{AdaBelief}
\label{sec:adabelief}

\marginpar{\cite{zhuang2020adabeliefoptimizeradaptingstepsizes}}

AdaBelief \cite{zhuang2020adabeliefoptimizeradaptingstepsizes} is a novel optimizer that builds
upon the Adam algorithm. The core concept of AdaBelief is to adapt the step size based on the
"belief" in the current gradient direction. This "belief" is derived from the proximity of the
current gradient estimate to the exponential moving average of past gradients, denoted as $m_t$.
If the current gradient estimate is close to $m_t$, we have a higher confidence in the gradient
estimate and take a larger step in the proposed direction. Conversely, if the current gradient
estimate significantly deviates from $m_t$, we take a much smaller step.

This adaptive mechanism is achieved by using the squared difference between the gradient
and the exponential moving average, rather than the gradient itself, in the calculation of the
second moment estimate $v_t$ \cite{zhuang2020adabelief}



\[m_t = \beta_1 m_{t-1} + (1 - \beta_1) \nabla_{\theta} J(\theta_t),\]
\[s_t = \beta_2 s_{t-1} + (1 - \beta_2) (\nabla_{\theta} J(\theta_t) - m_t)^2 + \epsilon,\]
\[\hat{m}_t = \frac{m_t}{1-\beta_1^t},\]
\[\hat{s}_t = \frac{s_t}{1-\beta_2^t},\]
\[\theta_{t+1} = \theta_t - \frac{\eta}{\sqrt{\hat{s_t} + \epsilon}} \hat{m_t},\]

where
\begin{itemize}
    \item $m_t$ is the velocity vector at iteration $t$,
    \item $s_t$ is the exponentially weighted moving average of the squared difference of the gradient and $m_t$ at time step $t$,
    \item $\beta_1 \in [0, 1) $ is the decay rate, used for the first order moment,
    \item $\beta_2 \in [0, 1) $ is the decay rate,  used for the second order moment,
    \item $\hat{m}_t$ is the bias-corrected first order moment at $t$,
    \item $\hat{s}_t$ is the bias-corrected second order moment at $t$,
    \item $\eta$ is the learning rate,
    \item $\nabla_{\theta} J(\theta_t)$ is the gradient of the loss function at iteration $t$,
    \item $\epsilon$ is a small constant (e.g., $10^{-8}$) to prevent division by zero.
\end{itemize}
Following this approach, AdaBelief achieves fast convergence, training stability, and good generalization results, comparable to Stochastic Gradient Descent (SGD)~\cite{zhuang2020adabelief}.


 \section{Second-order Optimization Algorithms}
 \label{sec:secondorder}
This section examines optimization algorithms utilizing second-order information, particularly the Hessian matrix (Equation \ref{eq:hessian}). 
We begin by examining the classical \emph{Newton method}, which serves as a foundation for understanding the motivation
for approximating the Hessian matrix in subsequent algorithms. Our discussion then progresses to established
second-order optimization methods, including the popular Broyden-Fletcher-Goldfarb-Shanno (\emph{BFGS}\cite{BFGS})
and Davidon-Fletcher-Powell (\emph{DFP}\cite{Goodfellow-et-al-2016}) quasi-Newton algorithms. We then shift our analysis to recent advancements in the field
like \emph{AdaHessian}\cite{yao2021adahessian} and \emph{Apollo}\cite{apollo}, which are algorithms for Hessian diagonal approximation, that are particularly useful in the context of neural network optimization.
Given their central role in this work, we provide a more comprehensive examination of \emph{AdaHessian} and \emph{Apollo} compared to the other algorithms discussed.
 \subsection{The Newton method}
 \label{sec:newton}

\emph{Newton's method} forms the basis of second-order optimization algorithms that aims to find the local minimum or maximum of a differentiable function by iteratively improving an initial estimate. 
The basic idea is to use a Taylor series expansion to approximate the function by a paraboloid at a given point.
The algorithm then identifies the minimum of this paraboloid, which provides a direction vector that can guide subsequent algorithms,
such as line search, towards a local minimum or maximum.In the following we will express this mathematically.\\
Let $f$ be the function we wish to optimize. 
Suppose \( f: \mathbb{R}^n \to \mathbb{R} \) is continuously differentiable and that $\Delta x \in \mathbb{R}^n$ then we have
that
\label{eq:taylor}
\begin{equation}
    f(x+\Delta x ) = f(x) + \nabla f(x)^T\Delta x  + \frac{1}{2}\Delta x ^T\nabla^2f(x)\Delta x 
\end{equation}
This follows from the second-order Taylor expansion with $x \in \mathbb{R}^n$ choosen as the expansion point. \cite{nocedal2006numerical}\\
As previously mentioned, this function approximates a paraboloid. To find the minimizer of this approximation we differentiate with respect to $\Delta x $ and set the gradient to zero.
\begin{align}
    \begin{array}{rcl}
    \frac{d}{d\Delta x } f(x+\Delta x ) & = & \nabla f(x) + \nabla^2 f(x) \Delta x  = 0 \\
    & \iff & \Delta x  = -(\nabla^2 f(x))^{-1} \nabla f(x)
    \end{array}
\end{align}
From this, we can conclude that $\Delta x $ represents the vector offset from the current position. This leads us to the following algorithm, which implements the Newton update rule
\begin{align}
x_k + \Delta x  = x_k - (\nabla^2 f(x))^{-1} \nabla f(x), \quad k \in \mathbb{N}.
\end{align}
Here, $x_k$ denotes the current position, while $x_k + \Delta x $ denotes the next position in the algorithm's path. Therefore, we set $x_k + \Delta x  = x_{k+1} $ and $ \mathbf{H} = (\nabla^2 f(x))^{-1} $, which yields the Newton update step \cite{nocedal2006numerical}
\begin{align}\label{eq:newtonstep}
x_{k+1} = x_k - \mathbf{H}_f(x_k)^{-1} \nabla f(x), \quad k \in \mathbb{N}.
\end{align}
This iterative process ensures that each step moves in the direction that minimizes the function \( f \) based on its local curvature and gradient.\\\\
In practice however the pure Newton method does not necessarily converge~\cite{nocedal2006numerical}. \\
There are several factors that contribute to this behavior, which can be categorized as follows:
\begin{itemize}
    \item \textbf{Non-Convex Function:} The Newton method relies on a second-order Taylor series expansion to approximate the function near the current point as a paraboloid.
     In highly non-covex settings however this approximation might not reflect the function's true behavior in 
    the neighbourhood of the expansion point. This way finding the minimum of the paraboloid might actually lead to a point that \emph{increases} the function value.\cite{kashyap2023survey}
    \item \textbf{Singular Hessian Matrix :} In order for Newton's method to lead to a local minimum,
     the Hessian matrix \( \nabla^2 f(x) \) must be \emph{positive-definite} at each step.
     If the Hessian has mixed eigenvalues or is not positive-definite, the steps may lead away from a minimum.
     \emph{BFGS} (Broyden–Fletcher–Goldfarb–Shanno) is a popular \emph{quasi-newton} optimization algorithm 
     that solves this problem of non positive-definite Hessians, by iteratively approximating the inverse Hessian with a rank-2 update formula,
      to preserve the non singular property of the Hessian (see \ref{sec:bfgs}) .\cite{nocedal2006numerical}
    \item \textbf{Sensitive to Initial Point:} The choice of initial point $x_0$ can greatly affect the convergence and accuracy of Newton's method. A good initial value should be close to the actual minimizer.
     Choosing a poor initial value can lead to divergent or inaccurate results, as the newton method converges and diverges quadratically.\cite{nocedal2006numerical}
   
\end{itemize}

\subsubsection{Proof of Convergence}
\label{sec:newton_conv}
\marginpar{\cite{yao2021adahessian}}

We conclude this section on the Newton method by a proof of its convergence in a strongly convex and strictly smooth setting.
This proof serves as a reference for our subsequent discussion, where we argue that using only the diagonal of the Hessian inverse also yields
a convergent algorithm under strongly convex and strictly smooth conditions. Although  we refer to \cite{yao2021adahessian} for this proof,
it was originally described by \cite{boyd2004convex}.
\\\\
\noindent\textbf{Theorem (Quadratic Convergence of Newton's Method).} 
\textit{Let $f: \mathbb{R}^d \to \mathbb{R}$ be a twice continuously differentiable, strongly convex and strictly smooth function.
Then, the Newton update (see \ref{eq:newtonstep}),
yields a quadratically converging algorithm with the following guarantee:
\[ f(\theta_{t+1}) - f(\theta_t) \leq - \frac{\alpha}{2\beta^{2}} \|\nabla f(\theta_t)\|^2, \quad \theta  \in \mathbb{R}^d\]
where  $\nabla f(\theta_t)$ denotes the gradient at $\theta_t$.}
\vspace{1em}
\noindent\\\textbf{Proof}\\
As $f$ is twice continuously differentiable, strongly convex, and strictly smooth, we can state:
\begin{equation}
\label{eq:convex_smooth}
\exists \alpha, \beta > 0 : \alpha I \preceq \nabla^2 f(\theta) \preceq \beta I, \quad \forall \theta \in \mathbb{R}^d,
\end{equation}
where $I$ is the identity matrix, $\alpha$ is the strong convexity parameter
(satisfying $\exists \alpha > 0 : \alpha I \preceq \nabla^2 f(\theta), \quad \forall \theta \in \mathbb{R}^d$) \cite{convexity_smoothness},
and $\beta$ is the strict smoothness parameter (satisfying $\exists \beta > 0 : \nabla^2 f(\theta) \preceq \beta I, \quad \forall \theta \in \mathbb{R}^d$) \cite{convexity_smoothness}.
While $\preceq$ denotes the positive semidefinite ordering of matrices (see \ref{sec:matrix_definiteness}).\\
Now define $\lambda\theta_t = \left( g_t^T \mathbf{H}_t^{-1} g_t \right)^{1/2}$ and $
\Delta \theta = \mathbf{H}^{-1}g_t$.
Given the $\beta$-smoothness property of $f$, we can infer that 
\begin{align}
    \begin{split}
        f(\theta_t - \eta\Delta \theta_t) &\leq f(\theta_t) + g_t^T ((\theta_t - \eta\Delta \theta_t) - \theta_t) \\
        &\quad + \frac{\beta}{2} \Vert (\theta_t - \eta\Delta \theta_t) - \theta_t \Vert ^2\\
        &= f(\theta_t) - \eta g_t^T \Delta \theta_t + \frac{\eta^2\beta}{2} \Vert \Delta \theta_t \Vert ^2.
    \end{split}
\end{align}
Now $(\lambda \theta_t)^2 = g_t^T \mathbf{H}_t^{-1} g_t  = \Delta \theta_t^T \mathbf{H}_t \Delta \theta_t$, and $g_t^T \Delta \theta_t =(\lambda\theta_t)^2 $. \\
Because of the strong convexity of $f$ (see \ref{eq:convex_smooth}) we get
\begin{align}
    \label{eq:inequality}
    \Delta \theta_t^T (\mathbf{H}_t - \alpha I) \Delta \theta_t &\geq 0 
    \Longleftrightarrow \Delta \theta_t^T \mathbf{H}_t \Delta \theta_t   \geq \alpha \|\Delta \theta_t\|^2,
\end{align}
thus $  \Vert \Delta \theta_t \Vert ^2 \leq  \frac{1}{\alpha} \Delta \theta_t^T \mathbf{H}_t \Delta \theta_t = \frac{1}{\alpha} (\lambda \theta_t)^2 $.
Now putting everything together
\begin{equation}
    f(\theta_t - \eta\Delta \theta_t) \leq f(\theta_t) - \eta (\lambda \theta_t)^2 + \frac{\eta^2\beta}{2\alpha} (\lambda \theta_t)^2.
\end{equation}
Setting the stepsize $\eta = \frac{a}{\beta}$ and expanding, it follows
\begin{equation}
    f(\theta_t - \eta\Delta \theta_t) \leq f(\theta_t) - \frac{1}{2} \eta (\lambda \theta_t)^2 .
\end{equation}
We follow \ref{eq:inequality} and since $\frac{1}{\beta}I \preceq (\nabla^2 f(\theta))^{-1}  $, we get
\begin{equation}
    (\lambda \theta_t)^2 = g_t^T \mathbf{H}_t^{-1} g_t \geq \frac{1}{\beta} \Vert g_t \Vert ^2,
\end{equation}
with which we finally arrive at the claim
\begin{equation}
    f(\theta_t - \eta\Delta \theta_t) - f(\theta_t) \leq -\frac{1}{2 \beta} \eta \Vert g_t\Vert^2 =  -\frac{a}{2 \beta^{2}}  \Vert g_t\Vert^2.
     \hfill \fbox{\phantom{\rule{.2ex}{.2ex}}}
\end{equation}

\subsection{DFP \& BFGS}
\marginpar{\cite{nocedal2006numerical}}
\label{sec:bfgs}
Now that we know why the Hessian is a very useful quantity for optimization, we will take a look at how we can approximate it,
as exact calculation is infeasible for most large-scale problems.
The Davidon-Fletcher-Powell (\emph{DFP}\cite{Goodfellow-et-al-2016})  and Broyden-Fletcher-Goldfarb-Shanno (\emph{BFGS}\cite{BFGS})
algorithms are so-called quasi-Newton algorithms 
that use a positive definite approximation of the Hessian. In the following,
we cover the main ideas of \emph{DFP} and \emph{BFGS}. We start by introducing the quasi-Newton update formula,
\begin{equation}
x_{k+1} = x_k - \mathbf{B}_k^{-1} \nabla f(x_k), \quad k \in \mathbb{N},
\end{equation}
where $\mathbf{B}_k$ is the Hessian approximation at timestep $k$ \cite{nocedal2006numerical}.
From this, we can derive the \emph{secant equation},
\begin{equation}
\mathbf{B}_k s_k = y_k,
\end{equation}
where $y_k = \nabla f(x_{k+1}) - \nabla f(x_k)$ and $s_k = x_{k+1} - x_k$.
We cover the derivation of these in full detail in \ref{sec:apollo}.
By multiplying the above with $s_k^T$, we can conclude, that if $s_k^T y_k > 0$,
known as the \emph{curvature condition}, holds, there exists a $\mathbf{B}_k$ with positive curvature along $s_k$,
meaning $s_k^T\mathbf{B}_k s_k > 0$ (see \ref{sec:matrix_definiteness}).
\begin{equation}
    \mathbf{B}_{k+1} = \min_{B} \|B - \mathbf{B}_k\| \\
    \text{s.t. }   B = B^T, \; Bs_k = y_k 
   \end{equation}
When using the Frobenius norm for this optimization problem, we get a unique solution for $\mathbf{B}_{k+1}$ with,
\begin{equation} \label{eq:DFP}
\mathbf{B}_{k+1} = \left(I - \rho_k y_k s_k^T\right) \mathbf{B}_k \left(I - \rho_k s_k y_k^T\right) + \rho_k y_k y_k^T,
\end{equation}
where $\rho_k = \frac{1}{y_k^T s_k}$. This update formula is usually referred to as the \emph{DFP} updating formula \cite{DFP}.
As we need the inverse $\mathbf{C}_{k+1} := \mathbf{B}_{k+1}^{-1}$ for performing the Newton step (see \ref{eq:newtonstep}), one can employ the \emph{Sherman--Morrison--Woodbury} formula, defined as
\begin{equation}
    (A + uv^T)^{-1} = A^{-1} - \frac{A^{-1}uv^T A^{-1}}{1 + v^T A^{-1}u} \\ A \in \mathbb{R}^{n \times n},\\
    u, v \in \mathbb{R}^n,\\
\end{equation}
    to show by expanding and subsequently rearranging \ref{eq:DFP}, that
\begin{equation}
\mathbf{C}_{k+1} = \mathbf{C}_k - \frac{\mathbf{C}_k y_k y_k^T \mathbf{C}k}{y_k^T \mathbf{C}k y_k} + \frac{s_k s_k^T}{y_k^T s_k},
\end{equation}
which is the update equation that is used in the \emph{DFP} algorithm.
The \emph{BFGS} algorithm works very similarly, with the subtle difference that it imposes the above conditions on the inverses of the Hessian approximations, meaning we have
\begin{equation}
\mathbf{C}{k+1} = \min_{C} \|C - \mathbf{C}k\| \\\text{s.t. } C = C^T, \quad Cy_k = s_k
\end{equation}
Again, \emph{BFGS} uses the Frobenius norm, which leads to the following update formulation:
\begin{equation}
\mathbf{C}_{k+1} = (I - \rho_k s_k y_k^T)  \mathbf{C}_k(I - \rho_k y_k s_k^T) + \rho_k s_k s_k^T,
\end{equation}
with $\rho_k = \frac{1}{y_k^T s_k}$. As $\mathbf{C}_{k+1}$ is already an approximation of the Hessian inverse at 
$x_k$, we can directly use it for step calculation. Regarding the initial choice of $\mathbf{C}_0$, one often selects the identity matrix
or approximates the Hessian inverse using finite differences on the gradient, when computationally feasible.
As there is no universally effective initialization method for $\mathbf{C}_0$ across all optimization problems.


\comment{
\subsection{The Damped Newton's Method and Line search}
\label{sec:damped_newton}
One approach to dampen the effects that a non-convex landscape and or a singular Hessian has on the convergence properties of the newton method is the so-called \emph{damped newton method}.\\ The idea is to introduce a dampening parameter $t \in \mathbb{R}$, to scale the inverse Hessian such that we achive a descent of the function values:
\begin{align}
    x_{k+1} = x_k - t\mathbf{H}_f(x_k)^{-1} \nabla f(x), \quad k \in \mathbb{N},
\end{align}

where $t \in \mathbb{R}$ is determinded with the \emph{backtracking line search} algorithm. This algorithm introduces two parameters $0< \alpha\leq \frac{1}{2}$ and $0 < \beta < 1$. At start $t=1$, while
\begin{align}
    f(x_k -t\mathbf{H}_f(x_k)^{-1} \nabla f(x)) >  f(x_k) -\alpha t\mathbf{H}_f(x_k)^{-1} \nabla f(x),
\end{align}

we shrink $t = \beta t$, until the above condition no longer holds and we perform the Newton step \cite{nocedal2006numerical}. \\\\
As we can see, this approach helps ensure that each step leads to a sufficient decrease in the function's value, therefore mitigating the convergence issues posed by non-convexity or a singular Hessian.
In practice, however, due to the computational burden, line search is often avoided, especially when the optimization problem involves many parameters.
Instead, the step size is often fixed and chosen as a hyperparameter, at the expense of losing the guaranteed decrease provided by line search \cite{linesearch}.
}

\subsection{AdaHessian}
\label{sec:adahessian}
\emph{AdaHessian}, introduced in 2020 by Yao et al. \cite{yao2021adahessian}, is an adaptive second-order optimization algorithm.
While being conceptually very similar to \emph{Adam} (\ref{sec:adam}), \emph{AdaHessian} replaces the square of the gradients
in \emph{Adam's} second moment with the square of a Hessian diagonal approximation.
To estimate the Hessian diagonal, AdaHessian employs two key techniques.
First, it utilizes a Hessian-free method based on the Hessian-vector product \cite{pearlmutter1994fast}.
This approach allows for efficient computation without explicitly forming the full Hessian matrix.
Second, AdaHessian implements a stochastic Hessian diagonal approximation based on \cite{BEKAS20071214},
which leverages the Hutchinson method \cite{hutchinson}, a technique for trace estimation of matrices.
Now let $f$ with $f: \mathbb{R}^n \to \mathbb{R}$ and $x \in \mathbb{R}^n$, be a neural network with subsequent
loss calculation. We start by taking the scalar product of $g=\nabla_\theta f$ with $z$, where
$z\in \mathbb{R}^n$ is a random vector which follows a Rademacher distribution. This results in a scalar.
We then calculate the derivative of this scalar with respect to $\theta$, such that we get
\begin{equation}
 \frac{\partial g^T z}{\partial \theta} = \frac{\partial g^T}{\partial \theta} z + g^T \frac{\partial z}{\partial \theta} = \frac{\partial g^T}{\partial \theta} z = Hz.
\end{equation}
This method is known as a Hessian-free approach,because by calculating this derivative, we obtain a Hessian-vector product without explicitly forming the Hessian matrix.
Following the results from  \cite{BEKAS20071214}, we get
\begin{equation}
    D = \text{diag}(H) = \mathbb{E}[z \odot (Hz)].
\end{equation}
The Hessian diagonal estimation in AdaHessian leverages an unbiased stochastic approximation technique.
Specifically, the expression $z \odot (Hz)$, where $\odot$ denotes the Hadamard (element-wise) product, serves as an unbiased estimator for the diagonal elements of the Hessian matrix.
Note that the authors of \cite{yao2021adahessian} found that a single sampling of $z$ is usually sufficient to lead to a reasonable diagonal approximation.
Next up, we demonstrate that employing this diagonal approximation of the Hessian in the update step yields convergence properties equivalent to those achieved
when utilizing the full Hessian matrix.
\\\\
\marginpar{\cite{yao2021adahessian}}
\noindent\textbf{Theorem (Convergence Rate of Hessian Diagonal Method)} \\
\textit{Let $f: \mathbb{R}^d \to \mathbb{R}$ be a twice continuously differentiable, strongly convex and strictly smooth function. Then, the update rule given by
    \[ \theta_{t+1} = \theta_t - \eta D_t^{-1} g_t, \]
    where $D_t$ is the diagonal of the Hessian $H_t = \nabla^2 f(\theta_t)$ and $g_t = \nabla f(\theta_t)$, yields a converging algorithm with the following guarantee
    \[ f(\theta_{t+1}) - f(\theta_t) \leq - \frac{\alpha}{2\beta^{2}} \|g_t\|^2, \quad \theta \in \mathbb{R}^d. \]
   }\noindent\textbf{Proof}\\ As $f$ is strongly convex and strictly smooth function, we know from \ref{sec:newton_conv}, that $\exists \alpha, \beta > 0 : \alpha I \preceq \nabla^2 f(\theta) \preceq \beta I, \quad \forall \theta \in \mathbb{R}^d$.
    To demonstrate that these bounds also apply to the diagonal matrix $D$, let's consider the standard basis vectors.
For any $e_i$, where all elements are $0$ except for the $i$-th one, which is $1$, we can observe:
\begin{equation}
       \alpha \leq e_i^T H e_i = e_i^T D e_i = D_{i,i} \quad \text{and} \quad \beta \geq e_i^T H e_i = e_i^T D e_i = D_{i,i}
\end{equation}   
This relationship implies that $D_{i,i} \in [\alpha,\beta], \quad \forall i \in \{1, \ldots, d\}$.
Consequently, we can extend the matrix inequality to $D$, such that   
   $\exists \alpha, \beta > 0 : \alpha I \preceq D \preceq \beta I, \quad \forall \theta \in \mathbb{R}^d$.   
Given this result, we can apply the same convergence analysis as in \ref{sec:newton_conv}, thus proving the claim. $ \hfill \fbox{\phantom{\rule{.2ex}{.2ex}}}$
\\







To mitigate the inherent stochastic variance associated with this approximation, AdaHessian employs two key strategies.
First, it maintains an Exponential Moving Average (EMA) of the diagonal estimates $D$.
Second, AdaHessian implements a spatial averaging algorithm. Consider a Convolutional Neural Network (CNN) as an example. In a CNN, for a convolutional kernel with block size $b$ (for instance, $b = 9$ for a $3 \times 3$ kernel), we perform spatial averaging among the kernel's parameters.
This can be mathematically expressed as:
\marginpar{\cite{yao2021adahessian}}
\begin{equation}
    \label{eq:spatial-average}
    D^{(s)}[ib+j] = \frac{1}{b} \sum_{k=1}^b D[ib + k], \quad \text{for } 1 \leq j \leq b, \; 0 \leq i \leq \left\lfloor\frac{d}{b}\right\rfloor - 1,
\end{equation}
where $d$ is the number of model parameters.After applying the spatial averaging, we can define the second momentum of AdaHessian with
\begin{equation}
    \label{eq:hessian-ema}
    \bar{D}_t = \beta_2 \bar{D}_{t-1} + (1 - \beta_2) (D^{(s)})^2.
\end{equation}
Here, $\bar{D}_t$ represents the smoothed estimate of the squared Hessian diagonal at time step $t$, $\beta_2$ is the exponential decay rate for the second moment estimate, and $D_{(s)}$ is the spatially averaged Hessian diagonal estimate.
As mentioned earlier, the rest of AdaHessian functions exactly analogous to Adam (see \ref{sec:adam}), leading to the
algorithm described in \ref{alg:adahessian}. \cite{yao2021adahessian}
\begin{algorithm}
    \caption{AdaHessian}
    \label{alg:adahessian}

    \begin{algorithmic}[1]
    \Require Initial parameter $\theta_0$
    \Require Learning rate $\eta$
    \Require Exponential decay rates $\beta_1, \beta_2$
    \Require Block size $b$
    \Require Hessian power $k$
    \State Initialize $m_0 = 0, v_0 = 0$
    \For{$t = 1, 2, \ldots$} 
        \State $g_t \gets \text{current step gradient}$
        \State $D_t \gets \text{current step estimated diagonal Hessian}$
        \State Compute $D_t^{(s)}$ based on \ref{eq:spatial-average}
        \State Update $\bar{D}_t$ based on \ref{eq:hessian-ema}
        \State $m_t = \beta_1 m_{t-1} + (1 - \beta_1) g_t$
        \State $v_t = \beta_2 v_{t-1} + (1 - \beta_2) (D_t^{(s)})^2$
        \State $\hat{m}_t = \frac{m_t}{1-\beta_1^t}$
        \State $\hat{v}_t = \frac{v_t}{1-\beta_2^t}$
        \State $\theta_t = \theta_{t-1} - \eta \frac{m_t}{v_t}$
    \EndFor
    \end{algorithmic}
    \end{algorithm}
    
\subsection{Apollo}
\label{sec:apollo}
\emph{Apollo} \cite{apollo} is a rather newly proposed quasi-Newton algorithm for non-convex stochastic
optimization. It operates by calculating a non-singular diagonal approximation  of the Hessian matrix,
utilizing the weak secant equation. To elucidate the algorithm's mechanics, we first briefly
revisit the theoretical foundations of the secant equation in general, and subsequently derive
the algorithm from this basis. We recall from \ref{eq:newtonstep} that for a supposed neural network with subsequent
loss calculation $f$ with $f: \mathbb{R}^n \to \mathbb{R}$ and $x \in \mathbb{R}^n$, we consider the Newton update step as follows:
\begin{equation}
    x_{k+t} = x_t - \mathbf{H}_f(x_t)^{-1} \nabla f(x_t), \quad k \in \mathbb{N}.
\end{equation}
For simplicity, we write $\mathbf{H}_t$ instead of $\mathbf{H}_f(x_t)^{-1}$. With this, we can derive the general \emph{quasi-Newton} update formula:
\begin{equation}
    x_{k+1} = x_k - \mathbf{B}^{-1} \nabla f(x_t), \quad t \in \mathbb{N},
\end{equation}
where $\mathbf{B}$ is an approximation of the Hessian matrix at $x_t$.
We can rewrite this as
\begin{align}
    x_{t+1} &= x_t - \mathbf{B}^{-1} \nabla f(x_t), \quad k \in \mathbb{N} \nonumber \\
    &\Longleftrightarrow x_{t+1} - x_t = - \mathbf{B}_t^{-1} \nabla f(x_t) \nonumber \\
    &\Longleftrightarrow \mathbf{B}(x_{t+1} - x_t) = - \nabla f(x_t) \nonumber \\
    &\Longleftrightarrow \mathbf{B}(x_{t+1} - x_t) + \nabla f(x_t) = 0
\end{align}
From \ref{sec:newton} we know that$\nabla_{\Delta x} f(x+\Delta x)$ should satisfy $\nabla_{\Delta x} f(x+\Delta x) = 0$. We can therefore 
conclude that $\mathbf{B}$ has to satisfy
\begin{equation}
    \nabla_{\Delta x} f(x+\Delta x) = \mathbf{B}(x_{t+1} - x_t) + \nabla f(x_t).
\end{equation}
This is equivalent to computing a second-order Taylor expansion (see \ref{eq:taylor}), then taking the gradient with respect to $\Delta x$.
We then proceed by defining
\begin{align}
    y_t &= \nabla_{\Delta x} f(x+\Delta x) -  \nabla f(x_t) \\
    s_t &= x_{t+1} - x_t,
\end{align}
such that the before formula is now
\label{eq:secant-eq}
\begin{equation}
    \mathbf{B}s_t = y_t,
\end{equation}
which is also known as the \emph{strong secant equation}. For the next approximation, we choose the
closest matrix to $\mathbf{B}$ under the condition of the \emph{strong secant equation}.
Therefore, our algorithm for updating
the Hessian approximation $\mathbf{B}$ will now be 
\begin{equation}
    \mathbf{B}_{t+1} = argmin_{\mathbf{B}} \|\mathbf{B} - \mathbf{B}_t\|,\\
    \text{s.t. }  \mathbf{B}_{t+1}\mathbf{s}_t = \mathbf{y}_t.
\end{equation}
This optimization problem forms the foundation for a family of quasi-Newton algorithms
such as BFGS\cite{BFGS}, DFP\cite{DFP}, or SR1\cite{SR1}.\cite{apollo}.
\emph{Apollo} employs a weakened form of the \emph{secant equation}, known as the \emph{weak secant equation}. The rationale behind this choice is as follows:
While we have used a scalar-valued function $f$ in our example, the difference of the gradients $y_t$
is a vector-valued function. For such functions, the \emph{mean value theorem}—which forms the basis of the standard secant equation in \ref{eq:secant-eq}—generally does not hold\cite{apollo}. 
Therefore, we apply the scalar product with $s_t^T$ to weaken the condition. This relaxes the equality requirement to hold only in the direction of $s_t$.
\begin{equation}
    \label{eq:weak-secant-eq}
    \mathbf{B}_{t+1} = argmin_{\mathbf{B}} \|\mathbf{B} - \mathbf{B}_t\|, \\
    \text{s.t. } \mathbf{s}^T_t \mathbf{B}_{t+1}\mathbf{s}_t =\mathbf{s}^T_t \mathbf{y}_t.
\end{equation}
This optimization problem can be solved by an approach first proposed in \cite{Zhu1999TheQR},
where the norm in \ref{eq:weak-secant-eq} is interpreted as the Frobenius norm.
\begin{equation}
    \label{eq:apollo-update}
    \Lambda= \mathbf{B}_{t+1} - \mathbf{B}_t = \frac{s_t^T y_t - s_t^T \mathbf{B}_t s_t}{\|s_t\|_4^4} \text{Diag}(s_t^2)
\end{equation}
here $s_t^2$ is the element-wise square vector of $s_t$, and $\text{Diag}(s_t^2)$ is the diagonal matrix with diagonal elements from vector $s_t^2$, and $\|\cdot\|_4$ is the 4-norm of a vector \cite{apollo}.
To ensure that the Hessian update remains invariant to the chosen stepsize, the step direction \( s_t \) is normalized by \( \eta_t \), leading to
\begin{equation}
    \Lambda' = - \frac{d_t^T y_t + d_t^T \mathbf{B}_t d_t}{\|d_t\|_4^4} \, \text{Diag}(d_t^2),
\end{equation}
where \( d_t = - \frac{s_t}{\eta_t} \). The whole algorithm for \emph{Apollo} is displayed in \ref{alg:apollo}.
Instead of working with gradients \( g_t \) directly, we choose the exponential moving average of them, in the
same fashion as we do in the \emph{Adam} optimizer \ref{sec:adam}. Because \emph{Apollo} uses the newton-step \ref{eq:newtonstep}
for its parameter update, we have to make sure that the approximation $ \mathbf{B}$ is non singular.
For that we define another diagonal matrix $ \mathbf{D_t}$ for which we choose
\begin{equation}
    \mathbf{D_t} =  \text{rectify}(|\mathbf{B_t}|,\sigma) = \max(|\mathbf{B_t}|,\sigma), \\ |\mathbf{B_t}| = \sqrt{\mathbf{B_t}^T\mathbf{B_t}},
\end{equation}
with $\sigma \geq 0$.
This approach achieves two key objectives: First, it ensures that small steps are taken in regions of very high curvature (sharp edges),
as \( |\mathbf{B_t}| \) becomes large in those cases, and that larger steps are taken when \( |\mathbf{B_t}| \) is low, although not excessively large, since saddle points are common in the loss landscape.
Secondly, it guarantees that \( \mathbf{D_t} \) has no zero-valued diagonal elements, ensuring that it remains non-singular.
In practice \emph{Apollo} chooses $\sigma=0.01$, although this can be tuned as a hyperparameter.\cite{apollo}

\marginpar{\cite{apollo}}
\label{alg:apollo}
\begin{algorithm}
    \caption{Apollo}
    \begin{algorithmic}[1]
    \State \textbf{Initial:} $m_0, d_0, B_0 \leftarrow 0, 0, 0$ \Comment{Initialize $m_0, d_0, B_0$ to zero}
    \State Good default settings are $\beta = 0.9$ and $\epsilon = 10^{-4}$
    \While{$t \in \{0, \ldots, T\}$}
        \For{$\theta \in \{\theta_1, \ldots, \theta_L\}$}
            \State $g_{t+1} \leftarrow \nabla f_t(\theta_t)$ \Comment{Calculate gradient at step $t$}
            \State $m_{t+1} \leftarrow \frac{\beta(1-\beta^t)}{1-\beta^{t+1}}m_t + \frac{1-\beta}{1-\beta^{t+1}}g_{t+1}$ \Comment{Bias-corrected EMA}
            \State $\alpha \leftarrow \frac{d_t^T(m_{t+1}-m_t)+d_t^TB_td_t}{(\|d_t\|_4+\epsilon)^4}$ \Comment{Calculate coefficient of $B$ update}
            \State $B_{t+1} \leftarrow B_t - \alpha \cdot \text{Diag}(d_t^2)$ \Comment{Update diagonal Hessian}
            \State $D_{t+1} \leftarrow \text{rectify}(B_{t+1}, 0.01)$ \Comment{Handle nonconvexity}
            \State $d_{t+1} \leftarrow D_{t+1}^{-1}m_{t+1}$ \Comment{Calculate update direction}
            \State $\theta_{t+1} \leftarrow \theta_t - \eta_{t+1}d_{t+1}$ \Comment{Update parameters}
        \EndFor
    \EndWhile
    \State \textbf{return} $\theta_T$
    \end{algorithmic}
    \end{algorithm}

 \section{Introduction to artificial Neural Networks}
Biological neural networks are intricate systems of interconnected neurons that communicate with each other to process information.
Each neuron, a specialized nerve cell, is designed to receive, process, and transmit signals through electrochemical processes.
Around the year 1900, a groundbreaking realization emerged that these tiny physical building blocks of the brain — the nerve cells and their intricate
connections — are responsible for perception, associations, thoughts, consciousness, and the ability to learn.\cite{GrundkursAI}
The idea of creating an artificial version of the brain to replicate its functions and achieve a synthetic form of intelligence has a long history.

The significant leap towards neural network-based artificial intelligence was made in 1943 by McCulloch and Pitts in their article "A Logical Calculus of the Ideas Immanent in Nervous Activity" \cite{McCullochPitts1943}.
They were the first to present a mathematical model of the neuron as a fundamental computational unit of the brain.
This article laid the foundation for constructing artificial neural networks and, consequently, for this crucial subfield of AI.\cite{GrundkursAI}
With the advent of efficient optimization methods and improved computational power, researchers were able to develop and train neural
networks that could genuinely learn and significantly enhance their performance.
These advances in optimization algorithms, such as (stochastic) gradient descent, coupled with accelerating hardware like GPUs,
enabled neural networks to process vast amounts of data. This breakthrough resulted in models that could recognize patterns,
make predictions, and solve complex problems across various domains, effectively demonstrating their learning capabilities. \cite{GrundkursAI}
In this section, we discuss the architecture and some of the underlying theory behind neural networks, including the algorithms used for training.
We will explore how neural networks are structured, the roles of different layers, and how information flows through them.
Additionally, we will delve into the principles and methods behind training algorithms like backpropagation, which adjust the network's weights to improve its predictive accuracy.
\subsection{The artificial Neuron}
\label{sec:neuron}

\marginpar{\cite{GrundkursAI}}
A neuron in an artificial neural network is modeled as a mathematical function that processes input signals and produces an output. The basic structure is as follows:
\[
y_j = f \left( \sum_{i=1}^{n} \theta_i x_i + b \right).
\]
where:
\begin{itemize}
    \item \( y_j \) is the output of the neuron,
    \item \( f \) is a nonlinear activation function applied to the weighted sum of the inputs,
    \item \( \sum_{i=1}^{n} \) is the summation over all input values,
    \item \( \theta_i \) are the weights associated with each input,
    \item \( x_i \) are the input values, and
    \item \( b \) is the bias term.
\end{itemize}
The inputs $x_i$ of a neuron are each weighted with an individual weighting factor $\theta_i$ and summed up.
These factors $\theta_i$ together with the \emph{bias term} $b$,
represent the trainable parameters of a neuron and are responsible for its performance on a given task.
The sum is then fed into a non linear activation function $f$.
This function is modeled on the idea that the activity of real biological neurons depends on a certain activation threshold\cite{GrundkursAI}.
Activation functions that are commonly used in practice are the \textbf{sigmoid-function}  $f(z) = \frac{1}{1 + e^{-z}}$ or \textbf{ReLU} (Rectified Linear Unit) $f(z) = \max(0, z)$\cite{nair2010rectified}.
The \emph{bias term} $b$ helps adjust the output by shifting the activation function horizontally.
This allows the neuron to represent patterns that are not centered around the origin, making it able to better approximate a hyperplane.\cite{GrundkursAI}

\subsection{The Multi-Layer Perceptron (MLP)}
\label{sec:mlp}

A single neuron is limited in its ability to distinguish between linearly separable data points, as its mathematical formulation models a hyperplane. Consequently, it cannot represent the output of an XOR (exclusive OR) gate.
This is because the XOR function is not linearly separable.\cite{Goodfellow-et-al-2016}\\
To overcome this limitation, the Multilayer Perceptron (MLP) was introduced.
An MLP consists of multiple stacked layers of single perceptrons (neurons with an activation function).
Each layer takes an input, processes it through its perceptrons, and outputs a vector that can then be fed into the next perceptron layer.
This structure enables the MLP to handle complex, non-linearly separable data, such as the XOR function.\cite{Goodfellow-et-al-2016}
The output of an MLP layer can be mathematically expressed as:
\[
x^{(l)} = f \left( \theta^{(l)} x^{(l-1)} + b^{(l)} \right),
\]

where:
\begin{itemize}
    \item \( l \) is the current layer,
    \item \( x^{(l)} \) is the output of layer \( l \),
    \item \( f \) is the activation function applied to the layer's output,
    \item \( \theta^{(l)} \) is the weight matrix of layer \( l \),
    \item \( x^{(l-1)} \) is the input vector to layer \( l \) (output of the previous layer),
    \item \( b^{(l)} \) is the bias vector of layer \( l \).
\end{itemize}
As we can see, the formulation follows the same principle as a single neuron but now in a vectorized form. 
\( \mathbf{\theta}^{(l)}  \) is a weight matrix of size \( n \times m \), where \( m \) is the number of inputs to a single neuron of the next layer and \( n \) is the number of neurons in a single layer.
\( \mathbf{x}^{(l-1)} \) is the output vector from the previous layer.
From their matrix-vector product, we get the pre-activations for each layer in vectorized form.
This vector is then fed into a non-linear activation function, whose output is then fed into the next layer.

It is easy to see why the activation function must be non-linear.
If the activation functions were linear, the stacked MLP layers would collapse into a single layer,
rendering the model unable to fit complex, highly non-linear data.\cite{Goodfellow-et-al-2016}
\marginpar{\cite{Goodfellow-et-al-2016}}

The MLP falls into the category of feedforward neural networks, meaning that the flow of information is unidirectional,
from one layer to the next. Each layer passes its information forward to the subsequent layer.
In contrast, there are other network architectures, such as recurrent neural networks (RNNs),
where the flow of information is not restricted to only consecutive layers. In RNNs, information can loop back to the same layer, 
allowing the network to maintain and utilize internal state information over time.
In practice, a Multilayer Perceptron (MLP) typically consists of an input layer that takes the data 
and passes it to one or more hidden layers before reaching the output layer.
In the output layer, the data is usually mapped to a probability distribution using a softmax function:
\begin{align}
\sigma(z_i) = \frac{e^{z_i}}{\sum_{j=1}^{m} e^{z_j}} \quad \text{for } i = 1, 2, \ldots, m ,
\end{align}

which can then be used to get the model's prediction depending on the task that should be performed.
\comment{
\subsection{Neural Networks as Universal Function Approximators}
\label{sec:nn_approx}

Neural Networks are exceptional function approximators, particularly when large datasets from the relevant domain
are available. Almost all practical problems, such as playing a game of Go or mimicking intelligent behavior,
can be represented by mathematical functions.\cite{universal_approx_theorem_2023} Universal approximation theorems are theoretical results 
that demonstrate the types of functions neural networks can learn or approximate in theory.
Recently, Lu et al. (2017)\cite{lu2017expressive} made an interesting discovery regarding width-bounded ReLU networks:

\begin{quote}    
    For any Lebesgue-integrable function \( f : \mathbb{R}^n \to \mathbb{R} \) and any \( \varepsilon > 0 \), there exists a fully-connected ReLU network \( A \) with width \( d_m \leq n + 4 \), such that the function \( F_A \) represented by this network satisfies
    \begin{align}
    \int_{\mathbb{R}^n} |f(x) - F_A(x)| \, dx < \varepsilon.
    \end{align}

    \end{quote}
\marginpar{\cite{lu2017expressive}}

This result applies to fully-connected ReLU networks with arbitrary depth,
meaning an arbitrary number of hidden layers, but with bounds on the width of each layer,
that is, the number of neurons in each layer.
We can therefore conclude that we can always find a neural network that is able to approximate a function $f : \mathbb{R}^n \to \mathbb{R}$, to
arbitrary degree.
However, determining the specific structure of this network (i.e. the values of the parameters) cannot be derived from these theoretical guarantees.
To find such a network, we need to optimize or train it to achieve a good approximation of the desired function.
The algorithm used for this process will be the topic of the next section.
}


\subsection{Training of Neural Networks}
\label{sec:nn_training}

The goal of training is to find a set of parameters such that the neural network minimizes a particular cost function.
The cost or loss function is a measure that determines the amount of error in the predictions \( \hat{y} \) 
made by the neural network on a given dataset. In supervised training,
we have a set of data points $x \in \mathbf{X}$ and the corresponding \emph{ground truth} labels $y \in \mathbf{Y}$.
We formulate our cost function such that it calculates the deviation between a ground truth label \( y \) and the
prediction \( \hat{y} \) of the network given the corresponding data point. One of the most popular loss functions is the Mean Squared Error (MSE):
\begin{align}
\text{MSE} = \frac{1}{m} \sum_{i=1}^{m} (\hat{y_i} - y_i)^2.
\end{align}
The MSE measures the Euclidean distance between the predictions \(\hat{y}\) and the ground truth \(y\). It is straightforward to show that minimizing the MSE is equivalent to maximizing the log likelihood:
\begin{align}
\theta_{\text{ML}} = \arg \max_{\theta} \sum_{i=1}^{m} \log P(y^{(i)} | x^{(i)}; \theta).
\end{align}
where \(\theta_{\text{ML}}\) is the set of parameters that maximizes the probability that the labels \(y^{(i)}\) correspond to the data points \(x^{(i)}\).

Another common loss function, often used for classification tasks, is the Cross-Entropy Loss:
\begin{align}
H(y, \hat{y}) = - \frac{1}{m} \sum_{i=1}^{m} \sum_{c=1}^{C} y_{i,c} \log(\hat{y}_{i,c}).
\end{align}

where $m$ and $C$ are the number of samples and classes respectively.
To understand the intuition behind this definition, consider a target or true probability distribution \(P\) and an approximate distribution \(Q\).
The cross-entropy between \(Q\) and \(P\) quantifies the additional number of bits needed to encode events from \(P\) using the distribution \(Q\) instead of the true distribution \(P\).
By minimizing the cross-entropy between \(Q\) and \(P\), we improve the approximation of the true distribution \(P\) using the model distribution \(Q\).\cite{bishop2006pattern}\\
To train the model, i.e., to tune its parameters such that the loss of the model's output is minimized,
we use gradient-based optimization methods as described in \ref{eq:newtonstep}.
To calculate the gradient of the parameters, we employ the \emph{backpropagation algorithm}.
The backpropagation algorithm was first introduced by Rumelhart, Hinton, and Williams in 1986 \cite{rumelhart1986learning}.
Unless otherwise noted, all information in this section regarding the technique is taken from their work.
It utilizes the chain rule to efficiently backpropagate gradient information from the last layer to the first.\\
Before applying the backpropagation algorithm, we first have to evaluate all the layers and activations in the network.
This process is often called the forward pass.
After computing the loss \(L\), we calculate the gradient of \(L\) with respect to the pre-activations of the model output, which we denote as
$\delta^{M} = \frac{\partial L}{\partial z^{(M)}} = \frac{\partial L}{\partial \hat{y}^{(M)}} \cdot \frac{\partial \hat{y}^{(M)}}{\partial z^{(M)}}$
, where \(z^{(M)} = \theta^{(M)} x^{(M-1)} + b^{(M)}\) and \(\hat{y} = f(z^{(M)})\), with \(M\) being the last layer.
We then loop back to the first layer with \(0 \leq m < M\) and calculate the respective gradients of the current layer using the gradient information from the previous layer. For that, we set
\begin{align}
    \delta^{(m)} &= \frac{\partial L}{\partial z^{(m)}} \\
                 &= \left[ \frac{\partial L}{\partial z^{(m+1)}} \frac{\partial z^{(m+1)}}{\partial \hat{y}^{(m)}} \right] \odot \frac{\partial \hat{y}^{(m)}}{\partial z^{(m)}} \\
                 &= \left[ \delta^{(m+1)} \frac{\partial z^{(m+1)}}{\partial \hat{y}^{(m)}} \right] \odot \frac{\partial \hat{y}^{(m)}}{\partial z^{(m)}} \\
                 &= \left[\delta^{(m+1)} \theta^{(m+1)}  \right] \odot \nabla_{z^{(m)}} f.
\end{align}
Here, $\odot$ denotes the element-wise (Hadamard) product.
To calculate the gradient information for the $i$-th the parameter of a neuron $j$ in layer $m$, we set
\begin{align}
\frac{\partial L}{\partial \theta_{j,i}^{(m)}} &= \frac{\partial L}{\partial z_j^{(m)}} \cdot \frac{\partial z_j^{(m)}}{\partial \theta_{j,i}^{(m)}} = \delta^{(m)}_j \cdot \hat{y}_i^{(m-1)},\\
\frac{\partial L}{\partial b_{j}^{(m)}} &= \frac{\partial L}{\partial z_j^{(m)}} \cdot \frac{\partial z_j^{(m)}}{\partial b_{j}^{(m)}} = \delta^{(m)}_j.
\end{align}
We do this iteratively until we reach the first layer. After that, we employ the optimization step in which we tune
the parameters to follow the negative gradient step as seen in \ref{sec:sgd}.
This step represents the crux of this work, which focuses on the efficient adjustment of network parameters by
incorporating not only gradient information but also second-order information to achieve faster convergence.
Algorithm \ref{alg:backprop} provides an overview of the backpropagation algorithm in pseudocode.


\begin{algorithm}
    \caption{Neural Network Backpropagation \cite{rumelhart1986learning}}
    \label{alg:backprop}
    \begin{algorithmic}[1]
    \Require Training data $\{(x^{(i)}, y^{(i)})\}_{i=1}^{m}$, learning rate $\alpha$
    \State Initialize weights $W$ and biases $b$ randomly
    \Repeat
        \ForAll{training example $(x^{(i)}, y^{(i)})$}
            \State \textbf{Compute Loss:}
            \State Compute loss $L(y, \hat{y}^{(M)})$
            \State \textbf{Backward Pass:}
            \State Compute $\delta^{M} = \frac{\partial L}{\partial z^{(M)}} = \frac{\partial L}{\partial \hat{y}^{(M)}} \odot \frac{\partial \hat{y}^{(M)}}{\partial z^{(M)}}$
            \For{$m = M-1$ \textbf{to} $0$}
                \State Compute $\delta^{(m)} = [\delta^{(m+1)} \theta^{(m+1)}] \nabla_{z^{(m)}} f$
                \State Compute gradients:
                \State $\frac{\partial L}{\partial \theta_{i,j}^{(m)}} = \delta^{(m)}_j \odot \hat{y}_i^{(m-1)}$
                \State $\frac{\partial L}{\partial b_j^{(m)}} = \delta_j^{(m)}$
            \EndFor
        \EndFor
        \For{$l = 0$ \textbf{to} $M$}
            \State Update weights: $\theta^{(l+1)} \leftarrow \theta^{(l)} - \alpha \frac{\partial L}{\partial \theta^{(l)}}$
            \State Update biases: $b^{(l+1)} \leftarrow b^{(l)} - \alpha \frac{\partial L}{\partial b^{(l)}}$
        \EndFor
    \Until{convergence}
    \end{algorithmic}
    \end{algorithm}
\marginpar{\cite{Goodfellow-et-al-2016}}
In addition to the gradient calculation and optimization steps, there are many other factors that determine the success of training a neural network.
A crucial aspect of training a neural network is finding a good set of values for the hyperparameters,
as most deep learning algorithms come with several hyperparameters that control various aspects of the algorithm's behavior.
Key hyperparameters include the learning rate,
the number of epochs, the batch size, the architecture of the network (number of layers and neurons per layer) and as weighting factors.
Optimizing these hyperparameters is highly non-trivial and are either determined through manual or automatic selection.
The goal is to find a set of hyperparameter such that the generalization error is as small as possible.
To facilitate this optimization, the training data is typically divided into three distinct sets: the \emph{training set},
the \emph{validation set}, and the \emph{test set}. The training set is used to train the model,
while the test set is reserved for evaluating the model's final performance. 
The validation set is specifically used for tuning hyperparameters and monitoring the model's performance to avoid overfitting
(i.e., achieving low training error but high generalization error), as the generalization error
for many hyperparameters, such as the learning rate, often follows a U-shaped curve when plotted as a function
of the hyperparameter, making careful selection of these hyperparameters essential.
As we have already stated, overfitting and underfitting are common pitfalls during model training, resulting in the model either memorizing the training data (overfitting) 
or failing to learn the training data at all (underfitting).
 A common method to tackle this problem is \emph{regularization}, which is often defined as "any modification made to a learning algorithm that
is intended to reduce its generalization error but not its training error" \cite{Goodfellow-et-al-2016}.
In theory, regularization is used to reduce the amount of variance in a model, at the cost of increasing the model's
 bias. This is often referred to as the \emph{bias-variance tradeoff}. While the bias is a measures for the inherent inability of
 a model to capture the true relationship of the data, variance measures the variability of predictions across different 
sets of data. Thus reducing the variance of a model, results in a better generalization ability on unseen data.\cite{IBMRegularization}
\marginpar{\cite{Goodfellow-et-al-2016}}
There are various techniques to facilitate regularization, one of 
the most widely used regularization techniques is the norm penalty on the model's parameters. This introduces a penalty
\(\Omega(\theta)\), where \(\Omega\) is a norm and \(\theta\) represents the network's parameters.
We denote the regularized objective function by \(\tilde{J}\) such that we get:
\begin{align}
\tilde{J}(\theta; X, y) = J(\theta; X, y) + \lambda \Omega(\theta),
\end{align}
where \(\lambda\) is a hyperparameter that controls the strength of the regularization and is often called the \emph{weight decay}.
In practice we only consider the \emph{weights} of the network for regularization instead of  all parameters \(\theta\) \cite{Goodfellow-et-al-2016}.
The most widely used norms for \(\Omega\) are the \emph{L1} norm and the \emph{L2} norm.
The \emph{L2} norm follows the definition of \(\Omega(\theta) = \frac{1}{2} \|w\|_2^2\).
This formulation incentivizes the network to have \emph{weights} that lie closer to the origin
by pushing parameters whose directions, corresponding to eigenvectors of the Hessian of $J$ with small
eigenvalues (i.e., directions with small curvature), closer to zero.
The \emph{L1} norm is defined as  \(\Omega(\theta) = \|\mathbf{w}\|_1 = \sum_i |\theta_i|\).
It can be shown that \emph{L1} regularization encourages sparsity in the solution, meaning that many parameters are driven to an optimal value of zero.
This property of the \emph{L1} regularizer is often used in a mechanism called \emph{feature selection},
where the model learns which features are most important by driving many parameters to zero. 
Other important regularization techniques that do not influence the loss function are \emph{dropout}
and \emph{early stopping}. \emph{Dropout} is based on the idea of training an ensemble of subnetworks
during the training of the larger model. This is achieved by multiplying the output of a unit 
by zero with a given probability, effectively removing the unit from the network.
This way, a large number of subnetworks are trained to predict the correct output, even when there
is "brain damage" present. In practice, this often leads to better generalization abilities of the
entire model. \emph{Early stopping} is another regularization technique that helps to prevent overfitting
by monitoring the model's performance on the validation set during training. We stop the training onnce
the model's performance on the validation step stops improving (or starts worsening), even if the performance
on the training set continues to improve. Although simple, this form of regularization has been proven to be very effective.
\cite{Goodfellow-et-al-2016}

\subsection{Decoupled Weight Decay}
\label{sec:dcp_weightdecay}

\emph{Decoupled Weight Decay} \cite{Loshchilov2017FixingWD} introduces a novel approach that reorders the application of the network parameters $\theta$ within the optimizer.
The authors find that the traditional implementation of weight decay, which is the addition of $\lambda \theta$ to the gradient in the optimizer \cite{Loshchilov2017FixingWD},
is effectively equivalent to $L_2$ regularization
for standard stochastic gradient descent (SGD). Given our \emph{L2} regularized loss $\tilde{J}(\theta; X, y)$,
the gradient of this regularized loss function is
\begin{align}
\nabla_{\theta} \tilde{J}(\theta; X, y) = \nabla_{\theta} J(\theta; X, y) + \lambda \theta.
\end{align}
This yields the optimizer's update step
\begin{align}
\theta_{t+1} = \theta_t - \eta \nabla_{\theta} \tilde{J}(\theta; X, y) = \theta_t - \eta (\nabla_{\theta} J(\theta; X, y) + \lambda \theta_t)
\end{align}
\noindent as described in \cite{DecoupledWeightDecay}.

This equivalence does not hold for adaptive gradient optimizers
such as \emph{Adam}. In adaptive optimizers, the weight decay must be decoupled from the gradient update to
maintain effective regularization and optimization~\cite{Loshchilov2017FixingWD}. In the standard implementation of Adam, \emph{weight decay} is incorporated into the gradient update step as follows:
\begin{align*}
    g_t &= \nabla_{\theta} J(\theta_t) + \lambda \theta_t, \\
    m_t &= \beta_1 m_{t-1} + (1 - \beta_1) g_t, \\
    v_t &= \beta_2 v_{t-1} + (1 - \beta_2) g_t^2.
    \end{align*}
Here, the weight decay term $\lambda \theta_t$ is added directly to the gradient of the loss function $J(\theta_t)$, which means the regularization is coupled with the gradient update.


In \emph{decoupled weight decay}, however, the weight decay is applied simultaneously but separately from the gradient update step. This results in a new update rule:
\begin{align}
\theta_{t+1} = \theta_t - \eta (\frac{ \hat{m_t}}{\sqrt{\hat{v_t} + \epsilon}} + \lambda \theta_t).
\end{align}

In this formulation:
\begin{itemize}
  \item $\theta_t$ represents the parameters at iteration $t$,
  \item $\lambda$ is the weight decay coefficient.
\end{itemize}

The decoupled weight decay ensures that the weight decay term $\lambda \theta_t$ is applied separately from the gradient step,
avoiding interference with the adaptive nature of the gradient updates, where weights with large gradient magnitudes are regularized by a smaller relative
amount than other weights. \cite{Loshchilov2017FixingWD}.


\subsection{Exact Calculation of the Hessian Matrix for MLPs}
\label{sec:hessian_exact}

In this section we will discuss how we can calculate the exact Hessian of a Loss using the backpropagation
algorithm that we covered earlier. We see that the exact values of the Hessian can be computed
using only a few forward and backward propagations. It should be noted that although an efficient 
form of calculation for the Hessian might be tempting, the storage requirements for such 
computations are infeasible for larger neural networks. Therefore, in practice, only approximations
of the Hessian or its diagonal are considered.
Unless otherwise specified, all information of this section are taken from \cite{Bishop1992}.
We will start by considering a feed-forward network with the standard notion introduced in \ref{sec:neuron}
\begin{align}                                                                                              
    z_i = \sum_j \theta_{ij} a_j + b_i  \quad &\quad
    a_i = f(z_i),
\end{align}

with $z_i$ and $a_i$ beeing an output of the previous layer and it's corresponding current activation.
We now want to find the first and second derivatives of an error function \(E\), which we model to consist of a sum of individual errors computed for each training instance
\begin{align}
E = \sum_p E_p,
\end{align}

where \(p\) labels the data point. We now consider a simple feedforward architecture without skip
or feedback connections. Without loss of generality, we assume that unit \(i\) is in the same layer
as unit \(n\), or in a lower layer. Due to the symmetry of the Hessian (see \ref{}), the remaining 
terms do not have to be computed. Utilizing the chain rule, we can can formulate this as
\begin{align}
\frac{\partial^2 E_p}{\partial \theta_{ij} \partial \theta_{nl}} = \frac{\partial z_i}{\partial \theta_{ij}} \frac{\partial}{\partial z_i} \left( \frac{\partial E_p}{\partial \theta_{nl}} \right) = a_j \frac{\partial}{\partial z_i} \left( \frac{\partial E_p}{\partial \theta_{nl}} \right).
\end{align}

We now introduce a set of quantities \(\sigma_n\) defined by
\begin{align}
\sigma_n \equiv \frac{\partial E_p}{\partial z_n}.
\end{align}

Using this we can write the second derivative as
\begin{align}
\frac{\partial^2 E_p}{\partial \theta_{ij} \partial \theta_{nl}} = a_j \frac{\partial}{\partial z_i} (\sigma_n a_l),
\end{align}

because of $\frac{\partial z_n}{\partial \theta_{nl}} = a_l$. Further we introduce some additional quantities
\begin{align}
g_{li} \equiv \frac{\partial z_l}{\partial z_i} \quad &\quad
b_{ni} \equiv \frac{\partial \sigma_n}{\partial z_i}.
\end{align}

Utilizing the product rule, the second derivatives can now be written in the following form
\begin{align}
\frac{\partial^2 E_p}{\partial \theta_{ij} \partial \theta_{nl}} = a_j \sigma_n \frac{\partial f}{\partial z_l}g_{li} + a_j a_l b_{ni},
\end{align}
where $ \frac{\partial f}{\partial z_l} =  \frac{\partial a_l}{\partial z_l}$ and $ \frac{\partial f}{\partial z_l}g_{li} =  \frac{\partial a_l}{\partial z_i}$
Using the chain rule for partial derivatives we can evaluate the \( g_{li} \) as follows
\begin{align}
g_{li} = \sum_r \frac{\partial z_l}{\partial z_r} \frac{\partial z_r}{\partial z_i} ,
\end{align}
where the sum runs over all units \( r \) which send connections to unit \( l \). The above equation can be recursivly defined as
\marginpar{\cite{Bishop1992}}
\begin{align}
g_{li} = \sum_r f'(z_r) \theta_{rl} g_{ri},
\end{align}
where $g_{ri} = \frac{\partial z_r}{\partial z_i}$ and $ \frac{\partial z_l}{\partial z_r}= \frac{\partial z_l}{\partial a_r}\frac{\partial a_r}{\partial z_r} = \theta_{rl} \frac{\partial f}{\partial z_r}$.
To obtain the initial conditions, we set \(g_{ii} = 1\) and \(g_{li} = 0\) for all units \(l \neq i\) in the same or
lower layers (i.e., layers nearer to the output). All the other \(g_{li}\) will be determinded during forward propagation
using the above recursive equation. We can obtain the values for $\sigma_n$ in a very similar fashion
\begin{align}
\sigma_n = \sum_r \frac{\partial E_p}{\partial z_r} \frac{\partial z_r}{\partial z_n},
\end{align}
where the sum runs over all units \(r\) to which unit \(n\) sends connections. Therefore we get
\begin{align}
    \sigma_n = \sum_r \frac{\partial E_p}{\partial z_r} \frac{\partial z_r}{\partial z_n} = \sum_r \sigma_r \frac{\partial z_r}{\partial z_n} = \sum_r \sigma_r \theta_{rn} \frac{\partial a_n}{\partial z_n} = f'(z_n) \sum_r \sigma_r \theta_{rn}.
\end{align}
To obtain the initial condition for $\sigma_m$, where $m$ is the label of an output unit, we derive
\begin{align}
\sigma_m =  \frac{\partial E_p}{\partial z_m} = \frac{\partial E_p}{\partial a_m} \frac{\partial a_m}{\partial z_m} = f'(z_m) \frac{\partial E_p}{\partial a_m}.
\end{align}
Next up, we derive a generalized back-propagation equation for the \(b_{ni}\). Substituting the back-propagation formula of \(\sigma_n\) into the definition of \(b_{ni}\), we get
\begin{align}
b_{ni} = \frac{\partial}{\partial z_i} \left( f'(z_n) \sum_r \theta_{rn} \sigma_r \right).
\end{align}
calcuting the derivative gives
\begin{align}
b_{ni} = f''(z_n) g_{ni} \sum_r \theta_{rn} \sigma_r + f'(z_n) \sum_r \theta_{rn} b_{ri}.
\end{align}

Using the above relationships, we can formulate the initial condition for $b_{mi}$
\begin{align}
b_{mi} = \frac{\partial \sigma_m}{\partial z_i} =  \frac{\partial }{\partial z_i}\left(\frac{\partial E_p}{\partial z_m} \right) = \frac{\partial^2 E_p}{\partial z_m^2} \frac{\partial z_m }{\partial z_i} = H_m g_{mi},
\end{align}
with
\begin{align}
H_m \equiv \frac{\partial^2 E_p}{\partial z_m^2} = f''(z_m) \frac{\partial E_p}{\partial a_m} + (f'(z_m))^2 \frac{\partial^2 E_p}{\partial a_m^2}.
\end{align}

Once we have obtained the initial values of all the $b_{mi}$, the
$b_{ni}$ of the remaining units, are determinded via back-propagation.
To summarize this process, for each pattern or data point \( p \):

\begin{enumerate}
    \item Perform forward-propagation to calculate all \( a_n \) and \( g_{li} \) using their respective equations.
    \item Execute back-propagation to determine \( \sigma_n \) and \( b_{ni} \).
    \item Finally, evaluate the value of \(\frac{\partial^2 E_p}{\partial \theta_{ij} \partial \theta_{nl}}\) using the determinded values.
\end{enumerate}
\marginpar{\cite{Bishop1992}}

The total number of distinct forward and backward propagations required per training pattern is equal to twice the number of hidden and output units in the network.
The number of operations for each propagation scales with \(N\), where \(N\) is the total number of weights in the network.














